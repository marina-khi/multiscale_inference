
\section*{Appendix}

\def\theequation{A.\arabic{equation}}
\setcounter{equation}{0}
\allowdisplaybreaks[4]


In what follows, we prove the theoretical results from Section \ref{sec-method}. The proofs of the results from Sections \ref{sec-test-shape} and \ref{sec-test-equality} are deferred to the Supplementary Material. Throughout the Appendix, we use the following notation: The symbol $C$ denotes a universal real constant which may take a different value on each occurrence. For $a,b \in \reals$, we write $a_+ = \max \{0,a\}$ and $a \vee b = \max\{a,b\}$. For any set $A$, the symbol $|A|$ denotes the cardinality of $A$. The notation $X \stackrel{\mathcal{D}}{=} Y$ means that the two random variables $X$ and $Y$ have the same distribution. Finally, $f_0(\cdot)$ and $F_0(\cdot)$ denote the density and distribution function of the standard normal distribution, respectively.



\subsection*{Auxiliary results using strong approximation theory}


The main purpose of this section is to prove that there is a version of the multiscale statistic $\widehat{\Phi}_T$ defined in \eqref{Phi-hat-statistic} which is close to a Gaussian statistic whose distribution is known. More specifically, we prove the following result. 
%
%
\begin{propA}\label{propA-strong-approx}
Under the conditions of Theorem \ref{theo-stat}, there exist statistics $\widetilde{\Phi}_T$ for $T = 1,2,\ldots$ with the following two properties: (i) $\widetilde{\Phi}_T$ has the same distribution as $\widehat{\Phi}_T$ for any $T$, and (ii)
\[ \big| \widetilde{\Phi}_T - \Phi_T \big| = o_p \Big( \frac{T^{1/q}}{\sqrt{T h_{\min}}} + \rho_T \sqrt{\log T} \Big), \]
where $\Phi_T$ is a Gaussian statistic as defined in \eqref{Phi-statistic}. 
\end{propA}
%
%
\begin{proof}[\textnormal{\textbf{Proof of Proposition \ref{propA-strong-approx}}}] 
For the proof, we draw on strong approximation theory for stationary processes $\{\varepsilon_t\}$ that fulfill the conditions \ref{C-err1}--\ref{C-err3}. By Theorem 2.1 and Corollary 2.1 in \cite{BerkesLiuWu2014}, the following strong approximation result holds true: On a richer probability space, there exist a standard Brownian motion $\mathbb{B}$ and a sequence $\{ \widetilde{\varepsilon}_t: t \in \naturals \}$ such that $[\widetilde{\varepsilon}_1,\ldots,\widetilde{\varepsilon}_T] \stackrel{\mathcal{D}}{=} [\varepsilon_1,\ldots,\varepsilon_T]$ for each $T$ and 
\begin{equation}\label{eq-strongapprox-dep}
\max_{1 \le t \le T} \Big| \sum\limits_{s=1}^t \widetilde{\varepsilon}_s - \sigma \mathbb{B}(t) \Big| = o\big( T^{1/q} \big) \quad \text{a.s.},  
\end{equation}
where $\sigma^2 = \sum_{k \in \integers} \cov(\varepsilon_0, \varepsilon_k)$ denotes the long-run error variance. To apply this result, we define 
\[ \widetilde{\Phi}_T = \max_{(u,h) \in \mathcal{G}_T} \Big\{ \Big|\frac{\widetilde{\phi}_T(u,h)}{\widetilde{\sigma}}\Big| - \lambda(h) \Big\}, \]
where $\widetilde{\phi}_T(u,h) = \sum\nolimits_{t=1}^T w_{t,T}(u,h) \widetilde{\varepsilon}_t$ and $\widetilde{\sigma}^2$ is the same estimator as $\widehat{\sigma}^2$ with $Y_t = m(t/T) + \varepsilon_t$ replaced by $\widetilde{Y}_t = m(t/T) + \widetilde{\varepsilon}_t$ for $1 \le t \le T$. In addition, we let
\begin{align*}
\Phi_T & = \max_{(u,h) \in \mathcal{G}_T} \Big\{ \Big|\frac{\phi_T(u,h)}{\sigma}\Big| - \lambda(h) \Big\} \\
\Phi_T^{\diamond} & = \max_{(u,h) \in \mathcal{G}_T} \Big\{ \Big|\frac{\phi_T(u,h)}{\widetilde{\sigma}}\Big| - \lambda(h) \Big\} 
\end{align*}
with $\phi_T(u,h) = \sum\nolimits_{t=1}^T w_{t,T}(u,h) \sigma Z_t$ and $Z_t = \mathbb{B}(t) - \mathbb{B}(t-1)$. With this notation, we can write 
\begin{equation}\label{eq-strongapprox-bound1}
\big| \widetilde{\Phi}_T - \Phi_T \big| \le \big| \widetilde{\Phi}_T - \Phi_T^{\diamond} \big| + \big| \Phi_T^{\diamond} - \Phi_T \big| = \big| \widetilde{\Phi}_T - \Phi_T^{\diamond} \big| + o_p \big( \rho_T \sqrt{\log T} \big), 
\end{equation}
where the last equality follows by taking into account that $\phi_T(u,h) \sim \normal(0,\sigma^2)$ for all $(u,h) \in \mathcal{G}_T$, $|\mathcal{G}_T| = O(T^\theta)$ for some large but fixed constant $\theta$ and $\widetilde{\sigma}^2 = \sigma^2 + o_p(\rho_T)$. Straightforward calculations yield that 
\[ \big| \widetilde{\Phi}_T - \Phi_T^{\diamond} \big| \le \widetilde{\sigma}^{-1} \max_{(u,h) \in \mathcal{G}_T} \big| \widetilde{\phi}_T(u,h) - \phi_T(u,h) \big|. \]
Using summation by parts,
%($\sum_{i=1}^n a_i b_i = \sum_{i=1}^{n-1} A_i (b_i - b_{i+1}) + A_n b_n$ with $A_j = \sum_{j=1}^i a_j$) 
we further obtain that 
\begin{align*}
\big| \widetilde{\phi}_T(u,h) - \phi_T(u,h) \big| 
 & \le W_T(u,h) \max_{1 \le t \le T} \Big| \sum\limits_{s=1}^t \widetilde{\varepsilon}_s - \sigma \sum\limits_{s=1}^t \big\{ \mathbb{B}(s) - \mathbb{B}(s-1) \big\} \Big| \\
 & = W_T(u,h) \max_{1 \le t \le T} \Big| \sum\limits_{s=1}^t \widetilde{\varepsilon}_s - \sigma \mathbb{B}(t) \Big|,
\end{align*}
where
\[ W_T(u,h) = \sum\limits_{t=1}^{T-1} |w_{t+1,T}(u,h) - w_{t,T}(u,h)| + |w_{T,T}(u,h)|. \]
Standard arguments show that $\max_{(u,h) \in \mathcal{G}_T} W_T(u,h) = O( 1/\sqrt{Th_{\min}} )$. Applying the strong approximation result \eqref{eq-strongapprox-dep}, we can thus infer that 
\begin{align}
\big| \widetilde{\Phi}_T - \Phi_T^{\diamond} \big| 
 & \le \widetilde{\sigma}^{-1} \max_{(u,h) \in \mathcal{G}_T} \big| \widetilde{\phi}_T(u,h) - \phi_T(u,h) \big| \nonumber \\
 & \le \widetilde{\sigma}^{-1} \max_{(u,h) \in \mathcal{G}_T} W_T(u,h) \max_{1 \le t \le T} \Big| \sum\limits_{s=1}^t \widetilde{\varepsilon}_s - \sigma \mathbb{B}(t) \Big| 
   = o_p \Big( \frac{T^{1/q}}{\sqrt{Th_{\min}}} \Big). \label{eq-strongapprox-bound2}
\end{align}
Plugging \eqref{eq-strongapprox-bound2} into \eqref{eq-strongapprox-bound1} completes the proof.
\end{proof}



\subsection*{Auxiliary results using anti-concentration bounds}


In this section, we establish some properties of the Gaussian statistic $\Phi_T$ defined in \eqref{Phi-statistic}. We in particular show that $\Phi_T$ does not concentrate too strongly in small regions of the form $[x-\delta_T,x+\delta_T]$ with $\delta_T$ converging to zero.  
%
%
\begin{propA}\label{propA-anticon}
Under the conditions of Theorem \ref{theo-stat}, it holds that 
\[ \sup_{x \in \reals} \pr \Big( | \Phi_T - x | \le \delta_T \Big) = o(1), \]
where $\delta_T = T^{1/q} / \sqrt{T h_{\min}} + \rho_T \sqrt{\log T}$.
\end{propA}
%
%
\begin{proof}[\textnormal{\textbf{Proof of Proposition \ref{propA-anticon}}}] 
The main technical tool for proving Proposition \ref{propA-anticon} are anti-concentration bounds for Gaussian random vectors. The following proposition slightly generalizes anti-concentration results derived in \cite{Chernozhukov2015}, in particular Theorem 3 therein. 
\begin{propA}\label{theo-anticon}
Let $(X_1,\ldots,X_p)^\top$ be a Gaussian random vector in $\reals^p$ with $\ex[X_j] = \mu_j$ and $\var(X_j) = \sigma_j^2 > 0$ for $1 \le j \le p$. Define $\overline{\mu} = \max_{1 \le j \le p} |\mu_j|$ together with $\underline{\sigma} = \min_{1 \le j \le p} \sigma_j$ and $\overline{\sigma} = \max_{1 \le j \le p} \sigma_j$. Moreover, set $a_p = \ex[ \max_{1 \le j \le p} (X_j-\mu_j)/\sigma_j ]$ and $b_p = \ex[ \max_{1 \le j \le p} (X_j-\mu_j) ]$. For every $\delta > 0$, it holds that
\[ \sup_{x \in \reals} \pr \Big( \big| \max_{1 \le j \le p} X_j - x \big| \le \delta \Big) \le C \delta \big\{ \overline{\mu} + a_p + b_p + \sqrt{1 \vee \log(\underline{\sigma}/\delta)} \big\}, \]
where $C > 0$ depends only on $\underline{\sigma}$ and $\overline{\sigma}$. 
\end{propA} 
%For the sake of completeness, 
The proof of Proposition \ref{theo-anticon} is provided in the Supplementary Material. To apply Proposition \ref{theo-anticon} to our setting at hand, we introduce the following notation: We write $x = (u,h)$ along with $\mathcal{G}_T = \{ x : x \in \mathcal{G}_T \} = \{x_1,\ldots,x_p\}$, where $p := |\mathcal{G}_T| \le O(T^\theta)$ for some large but fixed $\theta > 0$ by our assumptions. Moreover, for $j = 1,\ldots,p$, we set 
\begin{align*}
X_{2j-1} & = \frac{\phi_T(x_{j1},x_{j2})}{\sigma} - \lambda(x_{j2}) \\
X_{2j} & = -\frac{\phi_T(x_{j1},x_{j2})}{\sigma} - \lambda(x_{j2}) 
\end{align*}
with $x_j = (x_{j1},x_{j2})$. This notation allows us to write
\[ \Phi_T = \max_{1 \le j \le 2p} X_j, \]
where $(X_1,\ldots,X_{2p})^\top$ is a Gaussian random vector with the following properties: (i) $\mu_j := \ex[X_j] = - \lambda(x_{j2})$ and thus $\overline{\mu} = \max_{1 \le j \le 2p} |\mu_j| \le C \sqrt{\log T}$, and (ii) $\sigma_j^2 := \var(X_j) = 1$ for all $j$. Since $\sigma_j = 1$ for all $j$, it holds that $a_{2p} = b_{2p}$. Moreover, as the variables $(X_j - \mu_j)/\sigma_j$ are standard normal, we have that $a_{2p} = b_{2p} \le \sqrt{2 \log (2p)} \le C \sqrt{\log T}$. With this notation at hand, we can apply Proposition \ref{theo-anticon} to obtain that 
\[ \sup_{x \in \reals} \pr \Big( \big| \Phi_T - x \big| \le \delta_T \Big) \le C \delta_T \Big[ \sqrt{\log T} + \sqrt{ \log(1/\delta_T) } \Big] = o(1) \]
with $\delta_T = T^{1/q} / \sqrt{T h_{\min}} + \rho_T \sqrt{\log T}$, which is the statement of Proposition \ref{propA-anticon}.
\end{proof}



\subsection*{Proof of Theorem \ref{theo-stat}}


To prove Theorem \ref{theo-stat}, we make use of the two auxiliary results derived above. By Proposition \ref{propA-strong-approx}, there exist statistics $\widetilde{\Phi}_T$ for $T = 1,2,\ldots$ which are distributed as $\widehat{\Phi}_T$ for any $T \ge 1$ and which have the property that 
\begin{equation}\label{statement-propA-strong-approx}
\big| \widetilde{\Phi}_T - \Phi_T \big| = o_p \Big( \frac{T^{1/q}}{\sqrt{T h_{\min}}} + \rho_T \sqrt{\log T} \Big), 
\end{equation}
where $\Phi_T$ is a Gaussian statistic as defined in \eqref{Phi-statistic}. The approximation result \eqref{statement-propA-strong-approx} allows us to replace the multiscale statistic $\widehat{\Phi}_T$ by an identically distributed version $\widetilde{\Phi}_T$ which is close to the Gaussian statistic $\Phi_T$. In the next step, we show that  
\begin{equation}\label{eq-theo-stat-step2}
\sup_{x \in \reals} \big| \pr(\widetilde{\Phi}_T \le x) - \pr(\Phi_T \le x) \big| = o(1), 
\end{equation}
which immediately implies the statement of Theorem \ref{theo-stat}. For the proof of \eqref{eq-theo-stat-step2}, we use the following simple lemma: 
\begin{lemmaA}\label{lemma1-theo-stat}
Let $V_T$ and $W_T$ be real-valued random variables for $T = 1,2,\ldots$ such that $V_T - W_T = o_p(\delta_T)$ with some $\delta_T = o(1)$. If 
\begin{equation}\label{eq-lemma1-cond}
\sup_{x \in \reals} \pr(|V_T - x| \le \delta_T) = o(1), 
\end{equation}
then 
\begin{equation}\label{eq-lemma1-statement}
\sup_{x \in \reals} \big| \pr(V_T \le x) - \pr(W_T \le x) \big| = o(1). 
\end{equation}
\end{lemmaA}
The statement of Lemma \ref{lemma1-theo-stat} can be summarized as follows: If $W_T$ can be approximated by $V_T$ in the sense that $V_T - W_T = o_p(\delta_T)$ and if $V_T$ does not concentrate too strongly in small regions of the form $[x - \delta_T,x+\delta_T]$ as assumed in \eqref{eq-lemma1-cond}, then the distribution of $W_T$ can be approximated by that of $V_T$ in the sense of \eqref{eq-lemma1-statement}.
\begin{proof}[\textnormal{\textbf{Proof of Lemma \ref{lemma1-theo-stat}}}] 
It holds that 
\begin{align*}
 & \big| \pr(V_T \le x) - \pr(W_T \le x) \big| \\
 & = \big| \ex \big[ 1(V_T \le x) - 1(W_T \le x) \big] \big| \\
 & \le \big| \ex \big[ \big\{ 1(V_T \le x) - 1(W_T \le x) \big\} 1(|V_T - W_T| \le \delta_T) \big] \big| + \big| \ex \big[ 1(|V_T - W_T| > \delta_T) \big] \big| \\
 & \le \ex \big[ 1(|V_T - x| \le \delta_T, |V_T - W_T| \le \delta_T) \big] + o(1) \\
 & \le \pr (|V_T - x| \le \delta_T) + o(1). \qedhere
\end{align*}
\end{proof}
We now apply this lemma with $V_T = \Phi_T$, $W_T = \widetilde{\Phi}_T$ and $\delta_T = T^{1/q} / \sqrt{T h_{\min}} + \rho_T \sqrt{\log T}$: From \eqref{statement-propA-strong-approx}, we already know that $\widetilde{\Phi}_T - \Phi_T = o_p(\delta_T)$. Moreover, by Proposition \ref{propA-anticon}, it holds that 
\begin{equation}\label{statement-propA-anticon}
\sup_{x \in \reals} \pr \Big( | \Phi_T - x | \le \delta_T \Big) = o(1). 
\end{equation}
Hence, the conditions of Lemma \ref{lemma1-theo-stat} are satisfied. Applying the lemma, we obtain \eqref{eq-theo-stat-step2}, which completes the proof of Theorem \ref{theo-stat}.
%Note that with the help of Theorem 2.1 in \cite{DuembgenSpokoiny2001}, we can further show that $\Phi_T = O_p(1)$. Together with \eqref{statement-propA-anticon}, this says that the Gaussian multiscale statistic $\Phi_T$ is asymptotically tight and does not concentrate too strongly in small regions of the form $[x - \delta_T,x + \delta_T]$. Putting everything together, we are now in a position to apply Lemma \ref{lemma1-theo-stat}, which in turn yields \eqref{eq-theo-stat-step2}. This completes the proof of Theorem \ref{theo-stat}. 



\subsection*{Proof of Proposition \ref{prop-test-2}}


Write $\widehat{\psi}_T(u,h) = \widehat{\psi}_T^A(u,h) + \widehat{\psi}_T^B(u,h)$ with $\widehat{\psi}_T^A(u,h) = \sum\nolimits_{t=1}^T w_{t,T}(u,h) \varepsilon_t$ and $\widehat{\psi}_T^B(u,h) = \sum\nolimits_{t=1}^T w_{t,T}(u,h) m_T(\frac{t}{T})$. By assumption, there exists $(u_0,h_0) \in \mathcal{G}_T$ with $[u_0-h_0,u_0+h_0] \subseteq [0,1]$ such that $m_T(w) \ge c_T \sqrt{\log T/(Th_0)}$ for all $w \in [u_0-h_0,u_0+h_0]$. Since the kernel $K$ is symmetric and $u_0 = t/T$ for some $t$, it holds that $S_{T,1}(u_0,h_0) = 0$ and thus 
\[ w_{t,T}(u_0,h_0) = K\Big(\frac{\frac{t}{T}-u_0}{h_0}\Big) \Big/ \Big\{ \sum_{t=1}^T K^2\Big(\frac{\frac{t}{T}-u_0}{h_0}\Big) \Big\}^{1/2} \ge 0. \]
Together with the assumption that $m_T(w) \ge c_T \sqrt{\log T/(Th_0)}$ for all $w \in [u_0-h_0,u_0+h_0]$, this implies that 
\begin{equation}\label{eq1-proof-prop-test-power}
\widehat{\psi}_T^B(u_0,h_0) \ge c_T \sqrt{\frac{\log T}{Th_0}} \sum\limits_{t=1}^T w_{t,T}(u_0,h_0).
\end{equation}
Standard calculations exploiting the Lipschitz continuity of the kernel $K$ show that for any $(u,h) \in \mathcal{G}_T$ and any given natural number $\ell$, 
\begin{equation}\label{eq-riemann-sum}
\Big| \frac{1}{Th} \sum\limits_{t=1}^T K\Big(\frac{\frac{t}{T}-u}{h}\Big) \Big(\frac{\frac{t}{T}-u}{h}\Big)^\ell - \int_0^1 \frac{1}{h} K\Big(\frac{w-u}{h}\Big) \Big(\frac{w-u}{h}\Big)^\ell dw \Big| \le \frac{C}{Th}, 
\end{equation}
where the constant $C$ does not depend on $u$, $h$ and $T$. With the help of \eqref{eq-riemann-sum}, we obtain that for any $(u,h) \in \mathcal{G}_T$ with $[u-h,u+h] \subseteq [0,1]$, 
\begin{equation}\label{eq2-proof-prop-test-power}
\Big| \sum\limits_{t=1}^T w_{t,T}(u,h) - \frac{\sqrt{Th}}{\kappa} \Big| \le \frac{C}{\sqrt{Th}}, 
\end{equation}
where $\kappa = (\int K^2(\varphi)d\varphi)^{1/2}$ and the constant $C$ does once again not depend on $u$, $h$ and $T$. From \eqref{eq2-proof-prop-test-power}, it follows that $\sum\nolimits_{t=1}^T w_{t,T}(u,h) \ge \sqrt{Th} / (2\kappa)$ for sufficiently large $T$ and any $(u,h) \in \mathcal{G}_T$ with $[u-h,u+h] \subseteq [0,1]$. This together with \eqref{eq1-proof-prop-test-power} allows us to infer that 
\begin{equation}\label{eq3-proof-prop-test-power}
\widehat{\psi}_T^B(u_0,h_0) \ge \frac{c_T \sqrt{\log T}}{2 \kappa} 
\end{equation}
for sufficiently large $T$. Moreover, arguments very similar to those for the proof of Proposition \ref{propA-strong-approx} yield that
\begin{equation}\label{eq4-proof-prop-test-power}
\max_{(u,h) \in \mathcal{G}_T} |\widehat{\psi}_T^A(u,h)| = O_p(\sqrt{\log T}). 
\end{equation}
With the help of \eqref{eq3-proof-prop-test-power}, \eqref{eq4-proof-prop-test-power} and the fact that $\lambda(h) \le \lambda(h_{\min}) \le C \sqrt{\log T}$, we finally arrive at 
\begin{align}
\widehat{\Psi}_T 
 & \ge \max_{(u,h) \in \mathcal{G}_T} \frac{|\widehat{\psi}_T^B(u,h)|}{\widehat{\sigma}} - \max_{(u,h) \in \mathcal{G}_T} \Big\{ \frac{|\widehat{\psi}_T^A(u,h)|}{\widehat{\sigma}} + \lambda(h) \Big\} \nonumber \\
 & = \max_{(u,h) \in \mathcal{G}_T} \frac{|\widehat{\psi}_T^B(u,h)|}{\widehat{\sigma}} + O_p(\sqrt{\log T}) \nonumber \\
 & \ge \frac{c_T \sqrt{\log T}}{2 \kappa \widehat{\sigma}} + O_p(\sqrt{\log T}). \label{eq5-proof-prop-test-power}
\end{align}  
Since $q_T(\alpha) = O(\sqrt{\log T})$ for any fixed $\alpha \in (0,1)$, \eqref{eq5-proof-prop-test-power} immediately implies that $\pr(\widehat{\Psi}_T \le q_T(\alpha)) = o(1)$. 



\subsection*{Proof of Proposition \ref{prop-test-3}}

 
The statement of Proposition \ref{prop-test-3} is a consequence of the following observation: For all $(u,h) \in \mathcal{G}_T$ with 
\[ \Big|\frac{\widehat{\psi}_T(u,h) - \ex \widehat{\psi}_T(u,h)}{\widehat{\sigma}}\Big| - \lambda(h) \le q_T(\alpha) \quad \text{and} \quad \Big|\frac{\widehat{\psi}_T(u,h)}{\widehat{\sigma}}\Big| - \lambda(h) > q_T(\alpha), \]
it holds that $\ex[\widehat{\psi}_T(u,h)] \ne 0$, which in turn implies that $m(v) \ne 0$ for some $v \in I_{u,h}$. From this observation, we can infer the following: On the event 
\[ \big\{ \widehat{\Phi}_T \le q_T(\alpha) \big\} = \Big\{ \max_{(u,h) \in \mathcal{G}_T} \Big( \Big|\frac{\widehat{\psi}_T(u,h) - \ex \widehat{\psi}_T(u,h)}{\widehat{\sigma}}\Big| - \lambda(h) \Big) \le q_T(\alpha) \Big\}, \]
it holds that for all $(u,h) \in \mathcal{A}_T$, 
$m(v) \ne 0$ for some $v \in I_{u,h}$. Hence, we obtain that 
\[ \big\{ \widehat{\Phi}_T \le q_T(\alpha) \big\} \subseteq E_T. \]
As a result, we arrive at  
\[ \pr(E_T) \ge \pr \big(  \widehat{\Phi}_T \le q_T(\alpha) \big) = (1-\alpha) + o(1), \]
where the last equality holds by Theorem \ref{theo-stat}.



