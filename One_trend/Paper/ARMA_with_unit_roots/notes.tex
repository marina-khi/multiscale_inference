\documentclass[a4paper,12pt]{article}
\usepackage{amsmath}
\usepackage{amssymb,amsthm,graphicx}
%\usepackage{bbm} (Do not want to use these, may create conflicts)
%\usepackage{gensymb}
\usepackage{enumitem}
\usepackage{color}
\usepackage{epsfig}
\usepackage{graphics}
\usepackage{pdfpages}
\usepackage{subcaption}
\usepackage[font=small]{caption}
\usepackage[hang,flushmargin]{footmisc} 
\usepackage{float}
\usepackage{booktabs}
\usepackage[mathscr]{euscript}
\usepackage{natbib}
\usepackage{setspace}
\usepackage{mathrsfs}
\usepackage[left=2.7cm,right=2.7cm,bottom=2.7cm,top=2.7cm]{geometry}
\parindent0pt 


% General

\newcommand{\reals}{\mathbb{R}}
\newcommand{\integers}{\mathbb{Z}}
\newcommand{\naturals}{\mathbb{N}}

\newcommand{\pr}{\mathbb{P}}        % probability
\newcommand{\ex}{\mathbb{E}}        % expectation
\newcommand{\var}{\textnormal{Var}} % variance
\newcommand{\cov}{\textnormal{Cov}} % covariance

\newcommand{\law}{\mathcal{L}} % law of X
\newcommand{\normal}{N}        % normal distribution 

\newcommand{\argmax}{\textnormal{argmax}}
\newcommand{\argmin}{\textnormal{argmin}}

\newcommand{\ind}{\boldsymbol{1}} % indicator function
\renewcommand{\ker}{W} % kernel function

\newcommand{\X}{X}
\newcommand{\pairs}{\mathcal{S}}
\newcommand{\countries}{\mathcal{C}}
\newcommand{\intervals}{\mathcal{F}}
\newcommand{\indexset}{\mathcal{M}}

% Convergence

\newcommand{\convd}{\stackrel{d}{\longrightarrow}}              % convergence in distribution
\newcommand{\convp}{\stackrel{P}{\longrightarrow}}              % convergence in probability
\newcommand{\convas}{\stackrel{\textrm{a.s.}}{\longrightarrow}} % convergence almost surely
\newcommand{\convw}{\rightsquigarrow}                           % weak convergence


% Theorem-like declarations

\theoremstyle{plain}

\newtheorem{theorem}{Theorem}[section]
\newtheorem{prop}{Proposition}[section]
\newtheorem{corollary}{Corollary}[section]
\newtheorem{lemma}{Lemma}[section]
\newtheorem{definition}{Definition}[section]
\newtheorem{remark}{Remark}[section]
\newtheorem{algo}{Algorithm}
\newtheorem{theoremA}{Theorem}[section]
\newtheorem{propA}{Proposition}[section]
\newtheorem{corollaryA}{Corollary}[section]
\newtheorem{lemmaA}{Lemma}[section]
\renewcommand{\thetheoremA}{A.\arabic{theoremA}}
\renewcommand{\thepropA}{A.\arabic{propA}}
\renewcommand{\thecorollaryA}{A.\arabic{corollaryA}}
\renewcommand{\thelemmaA}{A.\arabic{lemmaA}}
\newtheorem{lemmaS}{Lemma}[section]
\renewcommand{\thelemmaS}{S.\arabic{lemmaS}}


% Theorem numbering to the left

\makeatletter
\newcommand{\lefteqno}{\let\veqno\@@leqno}
\makeatother


% Heading

\newcommand{\heading}[3]
{  \setcounter{page}{1}
   \begin{center}

   %\phantom{Distance to upper boundary}
   %\vspace{0.5cm}

   {\LARGE \textbf{#1}}
   \vspace{0.25cm}

   {\LARGE \textbf{#2}}
   \vspace{0.25cm}

   {\LARGE \textbf{#3}}
   \end{center}
}

\newcommand{\headingsupplement}[4]
{  \setcounter{page}{1}
   \begin{center}

   %\phantom{Distance to upper boundary}
   %\vspace{0.5cm}

   {\LARGE \textbf{#1}}
   \vspace{0.25cm}

   {\LARGE \textbf{#2}}
   \vspace{0.25cm}

   {\LARGE \textbf{#3}}
   \vspace{0.25cm}

   {\LARGE \textbf{#4}}
   \end{center}
}


% Authors

\newcommand{\authors}[4]
{  %\parindent0pt
   \begin{center}
      \phantom{-----------------}
      \begin{minipage}[c][1.5cm][c]{5.5cm}
      \begin{center} 
      {\large #1}  
      \vspace{0.1cm}      

      #2 
      \end{center}
      \end{minipage}
      \begin{minipage}[c][1.5cm][c]{5.5cm}
      \begin{center} 
      {\large #3} 
      \vspace{0.1cm}

      #4 \phantom{-}
      \end{center}
      \end{minipage}
      \phantom{---------}
   \end{center}
}

\newcommand{\authorssupplement}[4]
{  %\parindent0pt
   \begin{center}
      \phantom{-----------------}
      \begin{minipage}[c][1.5cm][c]{5.5cm}
      \begin{center} 
      {\large #1} \\[0.01cm]            
      #2 
      \end{center}
      \end{minipage}
      \begin{minipage}[c][1.5cm][c]{5.5cm}
      \begin{center} 
      {\large #3} \\[0.01cm]      
      #4 
      \end{center}
      \end{minipage}
      \phantom{---------}
   \end{center}
}


% Version

\newcommand{\version}[1]
{  \begin{center}
   {\large #1}
   \end{center}   
} 










\begin{document}
\renewcommand{\baselinestretch}{1.2}\normalsize



\section{Parameter estimation in ARMA models with a unit root in the MA polynomial}


Let $\{X_t\}$ be a stationary causal ARMA($p,1$) process of the form 
\begin{equation}
X_t - \sum\limits_{j=1}^p a_j^* X_{t-j} = \eta_t - \eta_{t-1},
\end{equation}
where $\eta_t$ are i.i.d.\ innovations with $\ex[\eta_t] = 0$ and $\ex[\eta^2] = \nu^*$. We use the notation $\boldsymbol{a}^* = (a_1^*,\ldots,a_p^*)$ and $\boldsymbol{\theta}^* = (\boldsymbol{a}^*,\nu^*)$. We now construct a maximum likelihood estimator of the parameters $a_1^*,\ldots,a_p^*$ and $\omega^*$. The construction proceeds in two steps: We first define an infeasible likelihood function which cannot be computed in practice and then approximate it by a feasible version. 
\vspace{10pt}


\textit{Step 1.} Let $\Pi_s Z_t$ be the orthogonal projection of a general (square-integrable) random variable $Z_t$ onto the linear space spanned by $X_1,\ldots,X_s$, denoted by $\text{span}\{X_1,\ldots,X_s\}$. The projection $\Pi_{t-1} X_t$ is the best linear predictor of $X_t$ based on $X_1,\ldots,X_{t-1}$. Let $\xi_t(\boldsymbol{\theta}^*) = X_t - \Pi_{t-1} X_t$ be the prediction innovations and $e_t(\boldsymbol{\theta}^*) = \ex[ \xi_t^2(\boldsymbol{\theta}^*)]$ the corresponding prediction error. Under the assumption that the innovations $\xi_t(\boldsymbol{\theta}^*)$ are i.i.d.\ Gaussian, the (infeasible) log-likelihood is given by  
\[ \mathcal{L}_T(\boldsymbol{\theta}^*) = -\frac{1}{2} \sum\limits_{t=1}^T \log \big(2 \pi e_t(\boldsymbol{\theta}^*)\big) - \frac{1}{2} \sum\limits_{t=1}^T \frac{\xi_t^2(\boldsymbol{\theta}^*)}{e_t(\boldsymbol{\theta}^*)}. \]
The prediction innovations $\xi_t(\boldsymbol{\theta}^*)$ and the prediction errors $e_t(\boldsymbol{\theta}^*)$ can be shown to have the following representations:  
\begin{align}
\xi_t(\boldsymbol{\theta}^*) & = V_t(\boldsymbol{a}^*) - \frac{1}{\beta(\nu^*) + t} \sum\limits_{s=p+1}^{t-1} V_s(\boldsymbol{a}^*) + \frac{\beta(\nu^*)+p+1}{\beta(\nu^*) + t} \Pi_p \eta_p \label{eq-pred-innovation} \\
e_t(\boldsymbol{\theta}^*)   & = \Big( 1 + \frac{1}{\beta(\nu^*)+t} \Big) \nu^* \label{eq-pred-error} 
\end{align}
for $t > p$, where $V_t(\boldsymbol{a}^*) = \sum_{k=p+1}^s (X_k - \sum_{j=1}^p a_j^* X_{k-j})$ and $\beta(\nu^*) = (\nu^*/\mu_p) - p - 1$ with $\mu_p = \ex[(\eta_p - \Pi_p \eta_p)^2]$.
\vspace{10pt}


\textit{Step 2.} For a general parameter vector $\boldsymbol{\theta} = (\boldsymbol{a},\nu) = (a_1,\ldots,a_p,\nu)$, we approximate the innovations $\xi_t(\boldsymbol{\theta})$ by 
\[ \widehat{\xi}_t(\boldsymbol{\theta}) = V_t(\boldsymbol{a}) - \frac{1}{t} \sum\limits_{s=p+1}^{t-1} V_s(\boldsymbol{a}) \]
and the prediction error $e_t(\boldsymbol{\theta}) = \ex[ \xi_t^2(\boldsymbol{\theta})]$ by $\nu$. A more convenient representation of $\widehat{\xi}_t(\boldsymbol{\theta})$ is given by 
\[ \widehat{\xi}_t(\boldsymbol{\theta}) = Q_{t,0} - \sum\limits_{j=1}^p a_j Q_{t,j} \quad \text{with} \quad Q_{t,j} = \sum\limits_{\ell=p+1}^t X_{\ell-j} - \frac{1}{t} \sum\limits_{s=p+1}^{t-1} \sum\limits_{\ell=p+1}^s X_{\ell-j}. \]
Replacing $\xi_t(\boldsymbol{\theta})$ and $e_t(\boldsymbol{\theta})$ by the approximations $ \widehat{\xi}_t(\boldsymbol{\theta})$ and $\nu$ in in $\mathcal{L}_T(\boldsymbol{\theta})$ yields the feasible likelihood 
\[ L_T(\boldsymbol{\theta}) =  -\frac{T-p}{2} \log(2 \pi \nu) - \frac{1}{2 \nu} \sum\limits_{t=p+1}^T \widehat{\xi}_t^2(\boldsymbol{\theta}). \]
Estimators $\widehat{\boldsymbol{\theta}} = (\widehat{\boldsymbol{a}},\widehat{\nu})$ of the parameters $\boldsymbol{\theta}^* = (\boldsymbol{a}^*,\nu^*)$ are defined as
\[ \widehat{\boldsymbol{\theta}} = (\widehat{\boldsymbol{a}},\widehat{\nu}) = \text{arg} \max_{\theta \in \Theta} L_T(\boldsymbol{\theta}). \] 
It is straightforward to solve this maximization problem and to show that
\begin{align*}
\widehat{\boldsymbol{a}} & = \widehat{\boldsymbol{\Gamma}}_Q^{-1} \widehat{\boldsymbol{\gamma}}_Q \\
\widehat{\nu} & = \frac{1}{T-p} \sum\limits_{t=p+1}^T \Big( Q_{t,0} - \sum\limits_{j=1}^p \widehat{a}_j Q_{t,j} \Big)^2,
\end{align*}
where $\widehat{\boldsymbol{\Gamma}}_Q = (\widehat{\gamma}_Q(i,j): 1 \le i,j \le p)$ is a $p \times p$ matrix and $\widehat{\boldsymbol{\gamma}}_Q = (\widehat{\gamma}_Q(0,1),\ldots,\widehat{\gamma}_Q(0,p))^\top$ is a vector in $\reals^p$ with the entries $\widehat{\gamma}_Q(i,j) = \sum_{t=p+1}^T Q_{t,i} Q_{t,j}$. 
\vspace{10pt}


The estimators $\widehat{\boldsymbol{a}}$ and $\widehat{\nu}$ have the following theoretical properties.
\begin{prop}\label{prop-ARMA}
Suppose that the process $\{ \eta_t\}$ has a finite fourth cumulant $\kappa$. Then 
\[ \sqrt{T} ( \widehat{\boldsymbol{\theta}} - \boldsymbol{\theta}^* ) \convd \normal \left( \begin{pmatrix} 0 \\ 0 \end{pmatrix}, \begin{pmatrix} \nu^* \Gamma^{-1} & 0 \\ 0 & 2 (\nu^*)^2 + \kappa \end{pmatrix} \right), \]
where $\boldsymbol{\Gamma} = (\gamma(i-j): 1 \le i,j \le p)$ is the autocovariance matrix of the AR($p$) process $\{ Y_t \}$ with $Y_t = \sum_{j=1}^p a_j^* Y_{t-j} + \eta_t$.  
\end{prop} 


\textbf{Derivation of (\ref{eq-pred-innovation}) and (\ref{eq-pred-error}).} Writing $W_t(\boldsymbol{a}^*) = X_t - \sum_{j=1}^p a_j^* X_{t-j}$, we have
\begin{equation}\label{eq-pred-1}
\boldsymbol{\xi}_t(\theta^*) = W_t(\boldsymbol{a}^*) + \Pi_{t-1} \eta_{t-1} 
\end{equation}
for $t > p$. By definition, $\Pi_t \eta_t$ belongs to the linear space spanned by $X_1,\ldots,X_{t-1}$. Moreover $\Pi_t \eta_t$ is orthogonal to the space spanned by $X_1,\ldots,X_{t-1}$ since $\Pi_{t-1} \Pi_t \eta_t = \Pi_{t-1} \eta_t = 0$. Noticing that $\text{span}\{\xi_t(\boldsymbol{\theta}^*)\} \oplus \text{span}\{X_1,\ldots,X_{t-1}\} = \text{span}\{X_1,\ldots,X_t\}$, we can infer that 
\begin{equation}\label{eq-pred-2}
\Pi_t \eta_t = \frac{\ex[\eta_t \xi_t(\boldsymbol{\theta}^*)]}{e_t(\boldsymbol{\theta}^*)} \xi_t(\boldsymbol{\theta}^*). 
\end{equation}
Since $\xi_t(\boldsymbol{\theta}^*) = \eta_t + (\Pi_{t-1} \eta_{t-1} - \eta_{t-1})$, it holds that $\ex[\eta_t \xi_t(\boldsymbol{\theta}^*)] = \nu^*$ and $e_t(\boldsymbol{\theta}^*) = \nu^* + \mu_{t-1}$ with $\mu_t = \ex[(\eta_t - \Pi_t \eta_t)^2]$. Plugging this into \eqref{eq-pred-2} yields 
\begin{equation}\label{eq-pred-3}
\xi_t(\boldsymbol{\theta}^*) = W_t(\boldsymbol{a}^*) + \frac{\nu^*}{\nu^* + \mu_{t-2}} \xi_{t-1}(\boldsymbol{\theta}^*). 
\end{equation}
The term $\mu_t$ can be rewritten as 
\[ \mu_t = \ex[(\eta_t - \Pi_t \eta_t)^2] = \nu^* - \ex(\Pi_t\eta_t)^2 = \nu^* - \frac{(\nu^*)^2}{\nu^* + \mu_{t-1}} = \frac{\nu^* \mu_{t-1}}{\nu^* + \mu_{t-1}}. \] 
This yields the recurrence equation $1/\mu_t = 1/\nu^* + 1/\mu_{t-1}$, which can be recursively applied to obtain that $1/\mu_t = (t-p)/\nu^* + 1/\mu_p$ for $t > p$. Using this in \eqref{eq-pred-3} gives that 
\[ \frac{\nu^*}{\nu^* + \mu_{t-2}} = \frac{\nu^*/\mu_{t-2}}{1 + \nu^*/\mu_{t-2}} =  \frac{\nu^*/\mu_p + t - 2 - p}{\nu^*/\mu_p + t - 1 - p} \]
and thus
\begin{equation}\label{eq-pred-4}
\xi_t(\boldsymbol{\theta}^*) = W_t(\boldsymbol{a}^*) + \frac{\beta(\nu^*) + t - 1}{\beta(\nu^*) + t} \xi_{t-1}(\boldsymbol{\theta}^*) 
\end{equation}
with $\beta(\nu^*) = \nu^*/\mu_p - p - 1$ for $t > p+1$. By iteratively applying \eqref{eq-pred-4}, we arrive at 
\[ \xi_t(\boldsymbol{\theta}^*) = \sum\limits_{s=p+1}^t \frac{\beta(\nu^*) + s}{\beta(\nu^*) + t} W_s(\boldsymbol{a}^*) + \frac{\beta(\nu^*) + p + 1}{\beta(\nu^*) + t} \Pi_p \eta_p, \] 
which can be equivalently written as
\[ \xi_t(\boldsymbol{\theta}^*) = V_t(\boldsymbol{a}^*) - \frac{1}{\beta(\nu^*) + t} \sum\limits_{s=p+1}^{t-1} V_s(\boldsymbol{a}^*) + \frac{\beta(\nu^*)+p+1}{\beta(\nu^*) + t} \Pi_p \eta_p. \]
Moreover, using the representation $e_t(\boldsymbol{\theta}^*) = \nu^* + \mu_{t-1}$ and the formulas on $\mu_t$ from above, it is easily seen that 
\[ e_t(\boldsymbol{\theta}^*) = \Big( 1 + \frac{1}{\beta(\nu^*)+t} \Big) \nu^*. \]


\textbf{Proof of Proposition \ref{prop-ARMA}.}
Let the process $\{Y_t\}$ be defined by the equations $Y_t = \sum_{j=1}^p a_j^* Y_{t-j} + \eta_t$. Since $X_t = Y_t - Y_{t-1}$, we obtain that 
\begin{align}
V_t(\boldsymbol{a}) 
 & = \sum\limits_{k=p+1}^t \Big( X_k - \sum\limits_{j=1}^p a_j X_{k-j} \Big) \label{eq-repres-V-1} \\
% & = \sum\limits_{k=p+1}^t \Big( \{Y_k - Y_{k-1}\} - \sum\limits_{j=1}^p a_j \{Y_{k-j}-Y_{k-j-1}\} \Big) \\
 & = \{ Y_t - Y_p \} - \sum\limits_{j=1}^p a_j \{Y_{t-j} - Y_{p-j}\}. \label{eq-repres-V-2}
\end{align} 
From \eqref{eq-repres-V-1}, it immediately follows that 
\begin{equation}\label{eq-repres-Xi-1}
\widehat{\xi}_t(\boldsymbol{\theta}) = \eta_t(\boldsymbol{a}) - \frac{1}{t} \sum\limits_{k=p+1}^{t-1} \eta_k(\boldsymbol{a}) - \frac{p+1}{t} \eta_p(\boldsymbol{a}) \quad \text{with} \quad \eta_t(\boldsymbol{a}) = Y_t - \sum_{j=1}^p a_j Y_{t-j}, 
\end{equation}
where $\eta_t(\boldsymbol{a})$ equals $\eta_t$ for $\boldsymbol{a} = \boldsymbol{a}^*$, that is, $\eta_t(\boldsymbol{a}^*) = \eta_t$. With the help of \eqref{eq-repres-V-2}, we can further write
\begin{equation}\label{eq-repres-Xi-2}
\widehat{\xi}_t(\boldsymbol{\theta}) = U_{t,0} - \sum\limits_{j=1}^p a_j U_{t,j} \quad \text{with} \quad U_{t,j} = Y_{t-j} - \frac{1}{t} \sum\limits_{k=p+1}^{t-1} Y_{k-j} - \frac{p+1}{t} Y_{p-j}. 
\end{equation}
Using \eqref{eq-repres-Xi-2} and taking the first derivatives of the likelihood $L_t(\boldsymbol{\theta})$, we obtain the first-order conditions
\begin{align} 
\frac{\partial L_T(\boldsymbol{\theta})}{\partial a_k} & = \frac{1}{\nu} \sum\limits_{p+1}^T \Big( U_{t,0} - \sum\limits_{j=1}^p a_j U_{t,j} \Big) U_{t,k} \stackrel{!}{=} 0 \quad \text{for } 1 \le k \le p \label{FOC1} \\
\frac{\partial L_T(\boldsymbol{\theta})}{\partial \nu} & = -\frac{T-p}{2\nu} + \frac{1}{2\nu^2} \sum\limits_{t=p+1}^T \widehat{\xi}_t^2(\boldsymbol{\theta}) \stackrel{!}{=} 0. \label{FOC2}  
\end{align}
From \eqref{FOC1} together with some straightforward calculations, we get that 
%\[ \sum_{j=1}^p \Big( \sum_{t=p+1}^T U_{t,j} U_{t,k} \Big) \widehat{a}_j = \sum\limits_{t=p+1}^T U_{t,0} U_{t,k} \] 
%for $1 \le k \le p$ and $\widehat{\omega} = (T-p)^{-1} \sum_{t=p+1}^T \widehat{\xi}_t^2(\widehat{\theta})$. Using these equations together with some straightforward calculations, we obtain that 
\begin{equation}\label{FOC1a}
\sum_{j=1}^p \Big( \frac{1}{T-p} \sum_{t=p+1}^T U_{t,j} U_{t,k} \Big) \big( \widehat{a}_j - a_j^* \big) = \frac{1}{T-p} \sum\limits_{t=p+1}^T \widehat{\xi}_t(\theta^*) U_{t,k} 
\end{equation}
for $1 \le k \le p$, or equivalently, 
\begin{equation}\label{FOC1b} 
\widehat{\boldsymbol{\Gamma}}_U ( \widehat{a} - a^*) = \widehat{\boldsymbol{\rho}}_U,
\end{equation}
where $\widehat{\boldsymbol{\rho}}_U = (\widehat{\rho}_U(1),\ldots,\widehat{\rho}_U(p))^\top$ with $\widehat{\rho}_U(k) = (T-p)^{-1} \sum_{t=p+1}^T \widehat{\xi}_t(\theta^*) U_{t,k}$ and  
\[ \widehat{\boldsymbol{\Gamma}}_U =
\begin{pmatrix} 
\widehat{\gamma}_U(1,1) & \dots  & \widehat{\gamma}_U(p,1) \\
\vdots      & \ddots & \vdots      \\
\widehat{\gamma}_U(1,p) & \dots  & \widehat{\gamma}_U(p,p) \\
\end{pmatrix} 
\]
with $\widehat{\gamma}_U(j,k) = (T-p)^{-1} \sum_{t=p+1}^T U_{t,j} U_{t,k}$. From \eqref{FOC2} and \eqref{FOC1a}, it further follows that
\begin{equation}\label{FOC2a}
\widehat{\nu} = \frac{1}{T-p} \sum\limits_{t=p+1}^T \widehat{\xi}_t^2(\boldsymbol{\theta}^*) - \sum\limits_{j=1}^p (\widehat{a}_j - a_j^*) \Big( \frac{1}{T-p} \sum\limits_{t=p+1}^T \widehat{\xi}_t(\boldsymbol{\theta}^*) U_{t,j} \Big).
\end{equation}
Noting that $\partial \widehat{\xi}_t(\boldsymbol{\theta})/ \partial \boldsymbol{\theta} = (-U_{t,1},\ldots,-U_{t,p})$, Lemmas 5 and 6 in Pham-Dinh (1978, Estimation of parameters in the ARMA model when the characteristic polynomial of the MA operator has a unit zero, AOS) yield that $\widehat{\boldsymbol{\Gamma}}_U = \boldsymbol{\Gamma} + o_p(1)$ and
\begin{equation}\label{CLT-ARMA-aux}
\sqrt{T} \begin{pmatrix} \widehat{\boldsymbol{\rho}}_U \\ \frac{1}{T-p} \sum_{t=p+1}^T \widehat{\xi}_t^2(\boldsymbol{\theta}^*) \end{pmatrix} \convd \normal \left( \begin{pmatrix} 0 \\ 0 \end{pmatrix}, \begin{pmatrix} \nu^* \boldsymbol{\Gamma} & 0 \\ 0 & 2 (\nu^*)^2 + \kappa \end{pmatrix} \right). 
\end{equation}
(To prove \eqref{CLT-ARMA-aux}, one uses that $U_{t,j} = Y_{t-j} - t^{-1} \sum_{k=p+1}^{t-1} Y_{k-j} - \{(p+1)/t\} Y_{p-j}$ with $\{Y_t\}$ being a stationary, causal AR($p$) process and $\widehat{\xi}_t(\boldsymbol{\theta}^*) = \eta_t - t^{-1} \sum_{k=p+1}^{t-1} \eta_k - \{(p+1)/t\} \eta_p$ with $\eta_t$ being i.i.d.\ variables.) Proposition \ref{prop-ARMA} follows upon applying these results to \eqref{FOC1b} and \eqref{FOC2a}. \qed



\end{document}
