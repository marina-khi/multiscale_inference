
\section{Proofs of the results from Section \ref{sec-method}}\label{sec-supp-proofs1}


In this section, we prove the theoretical results from Section \ref{sec-method}. We use the following notation: The symbol $C$ denotes a universal real constant which may take a different value on each occurrence. For $a,b \in \reals$, we write $a_+ = \max \{0,a\}$ and $a \vee b = \max\{a,b\}$. For any set $A$, the symbol $|A|$ denotes the cardinality of $A$. The notation $X \stackrel{\mathcal{D}}{=} Y$ means that the two random variables $X$ and $Y$ have the same distribution. Finally, $f_0(\cdot)$ and $F_0(\cdot)$ denote the density and distribution function of the standard normal distribution, respectively.



\subsection*{Auxiliary results using strong approximation theory}


The main purpose of this section is to prove that there is a version of the multiscale statistic $\widehat{\Phi}_T$ defined in \eqref{Phi-hat-statistic} which is close to a Gaussian statistic whose distribution is known. More specifically, we prove the following result. 
%
%
\begin{propA}\label{propA-strong-approx}
Under the conditions of Theorem \ref{theo-stat}, there exist statistics $\widetilde{\Phi}_T$ for $T = 1,2,\ldots$ with the following two properties: (i) $\widetilde{\Phi}_T$ has the same distribution as $\widehat{\Phi}_T$ for any $T$, and (ii)
\[ \big| \widetilde{\Phi}_T - \Phi_T \big| = o_p \Big( \frac{T^{1/q}}{\sqrt{T h_{\min}}} + \rho_T \sqrt{\log T} \Big), \]
where $\Phi_T$ is a Gaussian statistic as defined in \eqref{Phi-statistic}. 
\end{propA}
%
%
\begin{proof}[\textnormal{\textbf{Proof of Proposition \ref{propA-strong-approx}}}] 
For the proof, we draw on strong approximation theory for stationary processes $\{\varepsilon_t\}$ that fulfill the conditions \ref{C-err1}--\ref{C-err3}. By Theorem 2.1 and Corollary 2.1 in \cite{BerkesLiuWu2014}, the following strong approximation result holds true: On a richer probability space, there exist a standard Brownian motion $\mathbb{B}$ and a sequence $\{ \widetilde{\varepsilon}_t: t \in \naturals \}$ such that $[\widetilde{\varepsilon}_1,\ldots,\widetilde{\varepsilon}_T] \stackrel{\mathcal{D}}{=} [\varepsilon_1,\ldots,\varepsilon_T]$ for each $T$ and 
\begin{equation}\label{eq-strongapprox-dep}
\max_{1 \le t \le T} \Big| \sum\limits_{s=1}^t \widetilde{\varepsilon}_s - \sigma \mathbb{B}(t) \Big| = o\big( T^{1/q} \big) \quad \text{a.s.},  
\end{equation}
where $\sigma^2 = \sum_{k \in \integers} \cov(\varepsilon_0, \varepsilon_k)$ denotes the long-run error variance. To apply this result, we define 
\[ \widetilde{\Phi}_T = \max_{(u,h) \in \mathcal{G}_T} \Big\{ \Big|\frac{\widetilde{\phi}_T(u,h)}{\widetilde{\sigma}}\Big| - \lambda(h) \Big\}, \]
where $\widetilde{\phi}_T(u,h) = \sum\nolimits_{t=1}^T w_{t,T}(u,h) \widetilde{\varepsilon}_t$ and $\widetilde{\sigma}^2$ is the same estimator as $\widehat{\sigma}^2$ with $Y_{t,T} = m(t/T) + \varepsilon_t$ replaced by $\widetilde{Y}_{t,T} = m(t/T) + \widetilde{\varepsilon}_t$ for $1 \le t \le T$. In addition, we let
\begin{align*}
\Phi_T & = \max_{(u,h) \in \mathcal{G}_T} \Big\{ \Big|\frac{\phi_T(u,h)}{\sigma}\Big| - \lambda(h) \Big\} \\
\Phi_T^{\diamond} & = \max_{(u,h) \in \mathcal{G}_T} \Big\{ \Big|\frac{\phi_T(u,h)}{\widetilde{\sigma}}\Big| - \lambda(h) \Big\} 
\end{align*}
with $\phi_T(u,h) = \sum\nolimits_{t=1}^T w_{t,T}(u,h) \sigma Z_t$ and $Z_t = \mathbb{B}(t) - \mathbb{B}(t-1)$. With this notation, we can write 
\begin{equation}\label{eq-strongapprox-bound1}
\big| \widetilde{\Phi}_T - \Phi_T \big| \le \big| \widetilde{\Phi}_T - \Phi_T^{\diamond} \big| + \big| \Phi_T^{\diamond} - \Phi_T \big| = \big| \widetilde{\Phi}_T - \Phi_T^{\diamond} \big| + o_p \big( \rho_T \sqrt{\log T} \big), 
\end{equation}
where the last equality follows by taking into account that $\phi_T(u,h) \sim \normal(0,\sigma^2)$ for all $(u,h) \in \mathcal{G}_T$, $|\mathcal{G}_T| = O(T^\theta)$ for some large but fixed constant $\theta$ and $\widetilde{\sigma}^2 = \sigma^2 + o_p(\rho_T)$. Straightforward calculations yield that 
\[ \big| \widetilde{\Phi}_T - \Phi_T^{\diamond} \big| \le \widetilde{\sigma}^{-1} \max_{(u,h) \in \mathcal{G}_T} \big| \widetilde{\phi}_T(u,h) - \phi_T(u,h) \big|. \]
Using summation by parts,
%($\sum_{i=1}^n a_i b_i = \sum_{i=1}^{n-1} A_i (b_i - b_{i+1}) + A_n b_n$ with $A_j = \sum_{j=1}^i a_j$) 
we further obtain that 
\begin{align*}
\big| \widetilde{\phi}_T(u,h) - \phi_T(u,h) \big| 
 & \le W_T(u,h) \max_{1 \le t \le T} \Big| \sum\limits_{s=1}^t \widetilde{\varepsilon}_s - \sigma \sum\limits_{s=1}^t \big\{ \mathbb{B}(s) - \mathbb{B}(s-1) \big\} \Big| \\
 & = W_T(u,h) \max_{1 \le t \le T} \Big| \sum\limits_{s=1}^t \widetilde{\varepsilon}_s - \sigma \mathbb{B}(t) \Big|,
\end{align*}
where
\[ W_T(u,h) = \sum\limits_{t=1}^{T-1} |w_{t+1,T}(u,h) - w_{t,T}(u,h)| + |w_{T,T}(u,h)|. \]
Standard arguments show that $\max_{(u,h) \in \mathcal{G}_T} W_T(u,h) = O( 1/\sqrt{Th_{\min}} )$. Applying the strong approximation result \eqref{eq-strongapprox-dep}, we can thus infer that 
\begin{align}
\big| \widetilde{\Phi}_T - \Phi_T^{\diamond} \big| 
 & \le \widetilde{\sigma}^{-1} \max_{(u,h) \in \mathcal{G}_T} \big| \widetilde{\phi}_T(u,h) - \phi_T(u,h) \big| \nonumber \\
 & \le \widetilde{\sigma}^{-1} \max_{(u,h) \in \mathcal{G}_T} W_T(u,h) \max_{1 \le t \le T} \Big| \sum\limits_{s=1}^t \widetilde{\varepsilon}_s - \sigma \mathbb{B}(t) \Big| 
   = o_p \Big( \frac{T^{1/q}}{\sqrt{Th_{\min}}} \Big). \label{eq-strongapprox-bound2}
\end{align}
Plugging \eqref{eq-strongapprox-bound2} into \eqref{eq-strongapprox-bound1} completes the proof.
\end{proof}



\subsection*{Auxiliary results using anti-concentration bounds}


In this section, we establish some properties of the Gaussian statistic $\Phi_T$ defined in \eqref{Phi-statistic}. We in particular show that $\Phi_T$ does not concentrate too strongly in small regions of the form $[x-\delta_T,x+\delta_T]$ with $\delta_T$ converging to zero.  
%
%
\begin{propA}\label{propA-anticon}
Under the conditions of Theorem \ref{theo-stat}, it holds that 
\[ \sup_{x \in \reals} \pr \Big( | \Phi_T - x | \le \delta_T \Big) = o(1), \]
where $\delta_T = T^{1/q} / \sqrt{T h_{\min}} + \rho_T \sqrt{\log T}$.
\end{propA}
%
%
\begin{proof}[\textnormal{\textbf{Proof of Proposition \ref{propA-anticon}}}] 
The main technical tool for proving Proposition \ref{propA-anticon} are anti-concentration bounds for Gaussian random vectors. The following proposition slightly generalizes anti-concentration results derived in \cite{Chernozhukov2015}, in particular Theorem 3 therein. 
\begin{propA}\label{theo-anticon}
Let $(X_1,\ldots,X_p)^\top$ be a Gaussian random vector in $\reals^p$ with $\ex[X_j] = \mu_j$ and $\var(X_j) = \sigma_j^2 > 0$ for $1 \le j \le p$. Define $\overline{\mu} = \max_{1 \le j \le p} |\mu_j|$ together with $\underline{\sigma} = \min_{1 \le j \le p} \sigma_j$ and $\overline{\sigma} = \max_{1 \le j \le p} \sigma_j$. Moreover, set $a_p = \ex[ \max_{1 \le j \le p} (X_j-\mu_j)/\sigma_j ]$ and $b_p = \ex[ \max_{1 \le j \le p} (X_j-\mu_j) ]$. For every $\delta > 0$, it holds that
\[ \sup_{x \in \reals} \pr \Big( \big| \max_{1 \le j \le p} X_j - x \big| \le \delta \Big) \le C \delta \big\{ \overline{\mu} + a_p + b_p + \sqrt{1 \vee \log(\underline{\sigma}/\delta)} \big\}, \]
where $C > 0$ depends only on $\underline{\sigma}$ and $\overline{\sigma}$. 
\end{propA} 
%For the sake of completeness, 
The proof of Proposition \ref{theo-anticon} is provided at the end of this section for completeness. To apply Proposition \ref{theo-anticon} to our setting at hand, we introduce the following notation: We write $x = (u,h)$ along with $\mathcal{G}_T = \{ x : x \in \mathcal{G}_T \} = \{x_1,\ldots,x_p\}$, where $p := |\mathcal{G}_T| \le O(T^\theta)$ for some large but fixed $\theta > 0$ by our assumptions. Moreover, for $j = 1,\ldots,p$, we set 
\begin{align*}
X_{2j-1} & = \frac{\phi_T(x_{j1},x_{j2})}{\sigma} - \lambda(x_{j2}) \\
X_{2j} & = -\frac{\phi_T(x_{j1},x_{j2})}{\sigma} - \lambda(x_{j2}) 
\end{align*}
with $x_j = (x_{j1},x_{j2})$. This notation allows us to write
\[ \Phi_T = \max_{1 \le j \le 2p} X_j, \]
where $(X_1,\ldots,X_{2p})^\top$ is a Gaussian random vector with the following properties: (i) $\mu_j := \ex[X_j] = - \lambda(x_{j2})$ and thus $\overline{\mu} = \max_{1 \le j \le 2p} |\mu_j| \le C \sqrt{\log T}$, and (ii) $\sigma_j^2 := \var(X_j) = 1$ for all $j$. Since $\sigma_j = 1$ for all $j$, it holds that $a_{2p} = b_{2p}$. Moreover, as the variables $(X_j - \mu_j)/\sigma_j$ are standard normal, we have that $a_{2p} = b_{2p} \le \sqrt{2 \log (2p)} \le C \sqrt{\log T}$. With this notation at hand, we can apply Proposition \ref{theo-anticon} to obtain that 
\[ \sup_{x \in \reals} \pr \Big( \big| \Phi_T - x \big| \le \delta_T \Big) \le C \delta_T \Big[ \sqrt{\log T} + \sqrt{ \log(1/\delta_T) } \Big] = o(1) \]
with $\delta_T = T^{1/q} / \sqrt{T h_{\min}} + \rho_T \sqrt{\log T}$, which is the statement of Proposition \ref{propA-anticon}.
\end{proof}



\subsection*{Proof of Theorem \ref{theo-stat}}


To prove Theorem \ref{theo-stat}, we make use of the two auxiliary results derived above. By Proposition \ref{propA-strong-approx}, there exist statistics $\widetilde{\Phi}_T$ for $T = 1,2,\ldots$ which are distributed as $\widehat{\Phi}_T$ for any $T \ge 1$ and which have the property that 
\begin{equation}\label{statement-propA-strong-approx}
\big| \widetilde{\Phi}_T - \Phi_T \big| = o_p \Big( \frac{T^{1/q}}{\sqrt{T h_{\min}}} + \rho_T \sqrt{\log T} \Big), 
\end{equation}
where $\Phi_T$ is a Gaussian statistic as defined in \eqref{Phi-statistic}. The approximation result \eqref{statement-propA-strong-approx} allows us to replace the multiscale statistic $\widehat{\Phi}_T$ by an identically distributed version $\widetilde{\Phi}_T$ which is close to the Gaussian statistic $\Phi_T$. In the next step, we show that  
\begin{equation}\label{eq-theo-stat-step2}
\sup_{x \in \reals} \big| \pr(\widetilde{\Phi}_T \le x) - \pr(\Phi_T \le x) \big| = o(1), 
\end{equation}
which immediately implies the statement of Theorem \ref{theo-stat}. For the proof of \eqref{eq-theo-stat-step2}, we use the following simple lemma: 
\begin{lemmaA}\label{lemma1-theo-stat}
Let $V_T$ and $W_T$ be real-valued random variables for $T = 1,2,\ldots$ such that $V_T - W_T = o_p(\delta_T)$ with some $\delta_T = o(1)$. If 
\begin{equation}\label{eq-lemma1-cond}
\sup_{x \in \reals} \pr(|V_T - x| \le \delta_T) = o(1), 
\end{equation}
then 
\begin{equation}\label{eq-lemma1-statement}
\sup_{x \in \reals} \big| \pr(V_T \le x) - \pr(W_T \le x) \big| = o(1). 
\end{equation}
\end{lemmaA}
The statement of Lemma \ref{lemma1-theo-stat} can be summarized as follows: If $W_T$ can be approximated by $V_T$ in the sense that $V_T - W_T = o_p(\delta_T)$ and if $V_T$ does not concentrate too strongly in small regions of the form $[x - \delta_T,x+\delta_T]$ as assumed in \eqref{eq-lemma1-cond}, then the distribution of $W_T$ can be approximated by that of $V_T$ in the sense of \eqref{eq-lemma1-statement}.
\begin{proof}[\textnormal{\textbf{Proof of Lemma \ref{lemma1-theo-stat}}}] 
It holds that 
\begin{align*}
 & \big| \pr(V_T \le x) - \pr(W_T \le x) \big| \\
 & = \big| \ex \big[ 1(V_T \le x) - 1(W_T \le x) \big] \big| \\
 & \le \big| \ex \big[ \big\{ 1(V_T \le x) - 1(W_T \le x) \big\} 1(|V_T - W_T| \le \delta_T) \big] \big| + \big| \ex \big[ 1(|V_T - W_T| > \delta_T) \big] \big| \\
 & \le \ex \big[ 1(|V_T - x| \le \delta_T, |V_T - W_T| \le \delta_T) \big] + o(1) \\
 & \le \pr (|V_T - x| \le \delta_T) + o(1). \qedhere
\end{align*}
\end{proof}
We now apply this lemma with $V_T = \Phi_T$, $W_T = \widetilde{\Phi}_T$ and $\delta_T = T^{1/q} / \sqrt{T h_{\min}} + \rho_T \sqrt{\log T}$: From \eqref{statement-propA-strong-approx}, we already know that $\widetilde{\Phi}_T - \Phi_T = o_p(\delta_T)$. Moreover, by Proposition \ref{propA-anticon}, it holds that 
\begin{equation}\label{statement-propA-anticon}
\sup_{x \in \reals} \pr \Big( | \Phi_T - x | \le \delta_T \Big) = o(1). 
\end{equation}
Hence, the conditions of Lemma \ref{lemma1-theo-stat} are satisfied. Applying the lemma, we obtain \eqref{eq-theo-stat-step2}, which completes the proof of Theorem \ref{theo-stat}.
%Note that with the help of Theorem 2.1 in \cite{DuembgenSpokoiny2001}, we can further show that $\Phi_T = O_p(1)$. Together with \eqref{statement-propA-anticon}, this says that the Gaussian multiscale statistic $\Phi_T$ is asymptotically tight and does not concentrate too strongly in small regions of the form $[x - \delta_T,x + \delta_T]$. Putting everything together, we are now in a position to apply Lemma \ref{lemma1-theo-stat}, which in turn yields \eqref{eq-theo-stat-step2}. This completes the proof of Theorem \ref{theo-stat}. 



\subsection*{Proof of Proposition \ref{prop-test-2}}


To start with, we introduce the notation $\widehat{\psi}_T(u,h) = \widehat{\psi}_T^A(u,h) + \widehat{\psi}_T^B(u,h)$ with $\widehat{\psi}_T^A(u,h) = \sum\nolimits_{t=1}^T w_{t,T}(u,h) \varepsilon_t$ and $\widehat{\psi}_T^B(u,h) = \sum\nolimits_{t=1}^T w_{t,T}(u,h) m_T(\frac{t}{T})$. By assumption, there exists $(u_0,h_0) \in \mathcal{G}_T$ with $[u_0-h_0,u_0+h_0] \subseteq [0,1]$ such that $m_T^\prime(w) \ge c_T \sqrt{\log T/(Th_0^3)}$ for all $w \in [u_0-h_0,u_0+h_0]$. (The case that $-m_T^\prime(w) \ge c_T \sqrt{\log T/(Th_0^3)}$ for all $w$ can be treated analogously.) Below, we prove that under this assumption, 
\begin{equation}\label{eq-psiB}
\widehat{\psi}_T^B(u_0,h_0) \ge \frac{\kappa c_T \sqrt{\log T}}{2} 
\end{equation}
for sufficiently large $T$, where $\kappa = (\int K(\varphi) \varphi^2 d\varphi) / (\int K^2(\varphi) \varphi^2 d\varphi)^{1/2}$. Moreover, by arguments very similar to those for the proof of Proposition \ref{propA-strong-approx}, it follows that
\begin{equation}\label{eq-psiA}
\max_{(u,h) \in \mathcal{G}_T} |\widehat{\psi}_T^A(u,h)| = O_p(\sqrt{\log T}). 
\end{equation}
With the help of \eqref{eq-psiB}, \eqref{eq-psiA} and the fact that $\lambda(h) \le \lambda(h_{\min}) \le C \sqrt{\log T}$, we can infer that
\begin{align}
\widehat{\Psi}_T 
 & \ge \max_{(u,h) \in \mathcal{G}_T} \frac{|\widehat{\psi}_T^B(u,h)|}{\widehat{\sigma}} - \max_{(u,h) \in \mathcal{G}_T} \Big\{ \frac{|\widehat{\psi}_T^A(u,h)|}{\widehat{\sigma}} + \lambda(h) \Big\} \nonumber \\
 & = \max_{(u,h) \in \mathcal{G}_T} \frac{|\widehat{\psi}_T^B(u,h)|}{\widehat{\sigma}} + O_p(\sqrt{\log T}) \nonumber \\
 & \ge \frac{\kappa c_T \sqrt{\log T}}{2 \widehat{\sigma}} + O_p(\sqrt{\log T}) \label{eq-proof-prop-test-2-conclusion}
\end{align}  
for sufficiently large $T$. Since $q_T(\alpha) = O(\sqrt{\log T})$ for any fixed $\alpha \in (0,1)$, \eqref{eq-proof-prop-test-2-conclusion} immediately yields that $\pr(\widehat{\Psi}_T \le q_T(\alpha)) = o(1)$, which is the statement of Proposition \ref{prop-test-2}. 


\begin{proof}[\textnormal{\textbf{Proof of (\ref{eq-psiB})}}] 
Write $m_T(\frac{t}{T}) = m_T(u_0) + m_T^\prime(\xi_{u_0,t,T})(\frac{t}{T} - u_0)$, where $\xi_{u_0,t,T}$ is an intermediate point between $u_0$ and $t/T$. The local linear weights $w_{t,T}(u_0,h_0)$ are constructed such that $\sum_{t=1}^T w_{t,T}(u_0,h_0) = 0$. We thus obtain that 
\begin{equation}\label{eq-psiB-exp}
\widehat{\psi}_T^B(u_0,h_0) = \sum\limits_{t=1}^T w_{t,T}(u_0,h_0) \Big(\frac{\frac{t}{T} - u_0}{h_0}\Big) h_0 m_T^\prime(\xi_{u_0,t,T}).
\end{equation}
Moreover, since the kernel $K$ is symmetric and $u_0 = t/T$ for some $t$, it holds that $S_{T,1}(u_0,h_0) = 0$, which in turn implies that 
\begin{align}
w_{t,T} & (u_0,h_0) \Big( \frac{\frac{t}{T} - u_0}{h_0} \Big) \nonumber \\* & = K\Big(\frac{\frac{t}{T}-u_0}{h_0}\Big) \Big( \frac{\frac{t}{T} - u_0}{h_0} \Big)^2 \Big/ \Big\{ \sum_{t=1}^T K^2\Big(\frac{\frac{t}{T}-u_0}{h_0}\Big) \Big( \frac{\frac{t}{T} - u_0}{h_0} \Big)^2 \Big\}^{1/2} \ge 0. \label{weight-interior} 
\end{align}
From \eqref{eq-psiB-exp}, \eqref{weight-interior} and the assumption that $m_T^\prime(w) \ge c_T \sqrt{\log T/(Th_0^3)}$ for all $w \in [u_0-h_0,u_0+h_0]$, we get that 
\begin{equation}\label{eq-psiB-bound}
\widehat{\psi}_T^B(u_0,h_0) \ge c_T \sqrt{\frac{\log T}{T h_0}} \sum\limits_{t=1}^T w_{t,T}(u_0,h_0) \Big( \frac{\frac{t}{T} - u_0}{h_0} \Big).  
\end{equation}
Standard calculations exploiting the Lipschitz continuity of the kernel $K$ show that for any $(u,h) \in \mathcal{G}_T$ and any given natural number $\ell$, 
\begin{equation}\label{eq-riemann-sum}
\Big| \frac{1}{Th} \sum\limits_{t=1}^T K\Big(\frac{\frac{t}{T}-u}{h}\Big) \Big(\frac{\frac{t}{T}-u}{h}\Big)^\ell - \int_0^1 \frac{1}{h} K\Big(\frac{w-u}{h}\Big) \Big(\frac{w-u}{h}\Big)^\ell dw \Big| \le \frac{C}{Th}, 
\end{equation}
where the constant $C$ does not depend on $u$, $h$ and $T$. With the help of \eqref{weight-interior} and \eqref{eq-riemann-sum}, we obtain that for any $(u,h) \in \mathcal{G}_T$ with $[u-h,u+h] \subseteq [0,1]$, 
\begin{equation}\label{eq-psiB-weight-bound}
\Big| \sum\limits_{t=1}^T w_{t,T}(u,h) \Big(\frac{\frac{t}{T} - u}{h}\Big) - \kappa \sqrt{Th} \Big| \le \frac{C}{\sqrt{Th}}, 
\end{equation}
where the constant $C$ does once again not depend on $u$, $h$ and $T$. \eqref{eq-psiB-weight-bound} implies that $\sum\nolimits_{t=1}^T w_{t,T}(u,h) (\frac{t}{T} - u)/h \ge \kappa \sqrt{Th} / 2$ for sufficiently large $T$ and any $(u,h) \in \mathcal{G}_T$ with $[u-h,u+h] \subseteq [0,1]$. Using this together with \eqref{eq-psiB-bound}, we immediately obtain \eqref{eq-psiB}.
\end{proof}



\subsection*{Proof of Proposition \ref{prop-test-3}}

 
In what follows, we show that 
\begin{equation}\label{claim-prop-test-3}
\pr(E_T^+) \ge (1-\alpha) + o(1). 
\end{equation}
The other statements of Proposition \ref{prop-test-3} can be verified by analogous arguments. \eqref{claim-prop-test-3} is a consequence of the following two observations:  
\begin{enumerate}[label=(\roman*),leftmargin=0.75cm]

\item For all $(u,h) \in \mathcal{G}_T$ with   
\[ \Big|\frac{\widehat{\psi}_T(u,h) - \ex \widehat{\psi}_T(u,h)}{\widehat{\sigma}}\Big| - \lambda(h) \le q_T(\alpha) \quad \text{and} \quad \frac{\widehat{\psi}_T(u,h)}{\widehat{\sigma}} - \lambda(h) > q_T(\alpha), \]
it holds that $\ex[\widehat{\psi}_T(u,h)] > 0$. 

\item For all $(u,h) \in \mathcal{G}_T$ with $[u-h,u+h] \subseteq [0,1]$,  $\ex[\widehat{\psi}_T(u,h)] > 0$ implies that $m^\prime(v) > 0$ for some $v \in [u-h,u+h]$. 

\end{enumerate}
Observation (i) is trivial, (ii) can be seen as follows: Let $(u,h)$ be any point with $(u,h) \in \mathcal{G}_T$ and $[u-h,u+h] \subseteq [0,1]$. It holds that $\ex[\widehat{\psi}_T(u,h)] = \widehat{\psi}_T^B(u,h)$, where $\widehat{\psi}_T^B(u,h)$ has been defined in the proof of Proposition \ref{prop-test-2}. As already shown in \eqref{eq-psiB-exp},  
\[ \widehat{\psi}_T^B(u,h) = \sum\limits_{t=1}^T w_{t,T}(u,h) \Big( \frac{\frac{t}{T} - u}{h} \Big) \, h m^\prime(\xi_{u,t,T}), \]
where $\xi_{u,t,T}$ is some intermediate point between $u$ and $t/T$. Moreover, by \eqref{weight-interior}, it holds that $w_{t,T}(u,h) (\frac{t}{T} - u)/h \ge 0$ for any $t$. Hence, $\ex[\widehat{\psi}_T(u,h)] = \widehat{\psi}_T^B(u,h)$ can only take a positive value if $m^\prime(v) > 0$ for some $v \in [u-h,u+h]$. 


From observations (i) and (ii), we can draw the following conclusions: On the event 
\[ \big\{ \widehat{\Phi}_T \le q_T(\alpha) \big\} = \Big\{ \max_{(u,h) \in \mathcal{G}_T} \Big( \Big|\frac{\widehat{\psi}_T(u,h) - \ex \widehat{\psi}_T(u,h)}{\widehat{\sigma}}\Big| - \lambda(h) \Big) \le q_T(\alpha) \Big\}, \]
it holds that for all $(u,h) \in \mathcal{A}_T^+$ with $[u-h,u+h] \subseteq [0,1]$, $m^\prime(v) > 0$ for some $v \in I_{u,h} = [u-h,u+h]$. We thus obtain that $\{ \widehat{\Phi}_T \le q_T(\alpha) \} \subseteq E_T^+$. This in turn implies that 
\[ \pr(E_T^+) \ge \pr \big(  \widehat{\Phi}_T \le q_T(\alpha) \big) = (1-\alpha) + o(1), \]
where the last equality holds by Theorem \ref{theo-stat}. 



\subsection*{Proof of Corollary \ref{corollary-test-3}}


\textcolor{red}{
Let $\alpha = \alpha_T \rightarrow 0$ and let $\widetilde{\Phi}_T$ be defined as in the proof of Theorem \ref{theo-stat}. It holds that 
\begin{align*} 
\big| \pr(\widetilde{\Phi}_T \le q_T(\alpha_T)) - (1-\alpha_T) \big| 
 & = \big| \pr(\widetilde{\Phi}_T \le q_T(\alpha_T)) - \pr(\Phi_T \le q_T(\alpha_T)) \big| \\*
 & \le \sup_{x \in \reals} \big| \pr(\widetilde{\Phi}_T \le x) - \pr(\Phi_T \le x) \big| = o(1), 
\end{align*}
where the last equality is due to \eqref{eq-theo-stat-step2}. From this, it immediately follows that $\pr(\widetilde{\Phi}_T \le q_T(\alpha_T)) \rightarrow 1$. Moreover, since $\widetilde{\Phi}_T$ and $\widehat{\Phi}_T$ have the same distribution by construction, we obtain that 
\begin{equation}\label{eq-corollary-test-3}
\pr(\widehat{\Phi}_T \le q_T(\alpha_T)) \rightarrow 1. 
\end{equation}
Taking into account \eqref{eq-corollary-test-3}, Corollary \ref{corollary-test-3} can be proven in exactly the same way as Proposition  \ref{prop-test-3}. 
}



\subsection*{Proof of Proposition \ref{theo-anticon}}

 
The proof makes use of the following three lemmas, which correspond to Lemmas 5--7 in \cite{Chernozhukov2015}. 
\begin{lemmaA}\label{lemma1-anticon}
Let $(W_1,\ldots,W_p)^\top$ be a (not necessarily centred) Gaussian random vector in $\reals^p$ with $\var(W_j) = 1$ for all $1 \le j \le p$. Suppose that $\textnormal{Corr}(W_j,W_k) < 1$ whenever $j \ne k$. Then the distribution of $\max_{1 \le j \le p} W_j$ is absolutely continuous with respect to Lebesgue measure and a version of the density is given by 
\[ f(x) = f_0(x) \sum\limits_{j=1}^p e^{\ex[W_j]x - \ex[W_j]^2/2} \, \pr \big(W_k \le x \text{ for all } k \ne j \, \big| \, W_j = x \big). \]
\end{lemmaA}
\begin{lemmaA}\label{lemma2-anticon}
Let $(W_0,W_1,\ldots,W_p)^\top$ be a (not necessarily centred) Gaussian random vector with $\var(W_j) = 1$ for all $0 \le j \le p$. Suppose that $\ex[W_0] \ge 0$. Then the map 
\[ x \mapsto  e^{\ex[W_0]x - \ex[W_0]^2/2} \, \pr \big(W_j \le x \text{ for } 1 \le j \le p \, \big| \, W_0 = x \big) \]
is non-decreasing on $\reals$. 
\end{lemmaA}
\begin{lemmaA}\label{lemma3-anticon}
Let $(X_1,\ldots,X_p)^\top$ be a centred Gaussian random vector in $\reals^p$ with $\max_{1 \le j \le p} \ex[X_j^2] \le \sigma_X^2$ for some $\sigma_X^2 > 0$. Then for any $r > 0$, 
\[ \pr \Big( \max_{1 \le j \le p} X_j \ge \ex \Big[ \max_{1 \le j \le p} X_j \Big] + r \Big) \le e^{-r^2/(2\sigma_X^2)}. \]
\end{lemmaA} 
The proof of Lemmas \ref{lemma1-anticon} and \ref{lemma2-anticon} can be found in \cite{Chernozhukov2015}. Lemma \ref{lemma3-anticon} is a standard result on Gaussian concentration whose proof is given e.g.\ in \cite{Ledoux2001}; see Theorem 7.1 therein. We now closely follow the arguments for the proof of Theorem 3 in \cite{Chernozhukov2015}. The proof splits up into three steps. 
\vspace{7pt}


\textit{Step 1.} %To start with, we show that the analysis can be restricted to the unit variance case. To see this, 
Pick any $x \ge 0$ and set 
\[ W_j = \frac{X_j - x}{\sigma_j} + \frac{\overline{\mu} + x}{\underline{\sigma}}. \]
By construction, $\ex[W_j] \ge 0$ and $\var(W_j) = 1$. Defining $Z = \max_{1 \le j \le p} W_j$, it holds that  
\begin{align*}
\pr \Big( \Big| \max_{1 \le j \le p} X_j - x \Big| \le \delta \Big) 
 & \le \pr \Big( \Big| \max_{1 \le j \le p} \frac{X_j - x}{\sigma_j} \Big| \le \frac{\delta}{\underline{\sigma}} \Big) \\
 & \le \sup_{y \in \reals} \pr \Big( \Big| \max_{1 \le j \le p} \frac{X_j - x}{\sigma_j} + \frac{\overline{\mu} + x}{\underline{\sigma}} - y \Big| \le \frac{\delta}{\underline{\sigma}} \Big) \\
 & = \sup_{y \in \reals} \pr \Big( |Z - y| \le \frac{\delta}{\underline{\sigma}} \Big). 
\end{align*}


\textit{Step 2.} We now bound the density of $Z$. Without loss of generality, we assume that $\text{Corr}(W_j,W_k) < 1$ for $k \ne j$. The marginal distribution of $W_j$ is $\normal(\nu_j,1)$ with $\nu_j = \ex[W_j] = (\mu_j/\sigma_j + \overline{\mu}/{\underline{\sigma}}) + (x/\underline{\sigma} - x/\sigma_j) \ge 0$. Hence, by Lemmas \ref{lemma1-anticon} and \ref{lemma2-anticon}, the random variable $Z$ has a density of the form
\begin{equation}\label{eq-dens-Z}
f_p(z) = f_0(z) G_p(z), 
\end{equation}
where the map $z \mapsto G_p(z)$ is non-decreasing. Define $\overline{Z} = \max_{1 \le j \le p} (W_j - \ex[W_j])$ and set $\overline{z} = 2 \overline{\mu}/\underline{\sigma} + x(1/\underline{\sigma} - 1/\overline{\sigma})$ such that $\ex[W_j] \le \overline{z}$ for any $1 \le j \le p$. With these definitions at hand, we obtain that  
\begin{align*}
\int_z^{\infty} f_0(u)du \, G_p(z) & \le \int_z^{\infty} f_0(u) G_p(u) du = \pr(Z > z) \\ 
 & \le P(\overline{Z} > z - \overline{z}) \le \exp \Big( - \frac{(z - \overline{z} - \ex[\overline{Z}])^2_+}{2} \Big), 
\end{align*}
where the last inequality follows from Lemma \ref{lemma3-anticon}. Since $W_j - \ex[W_j] = (X_j - \mu_j)/\sigma_j$, it holds that 
\[ \ex[\overline{Z}] = \ex \Big[ \max_{1 \le j \le p} \Big\{ \frac{X_j-\mu_j}{\sigma_j} \Big\} \Big] =: a_p. \]
Hence, for every $z \in \reals$, 
\begin{equation}\label{eq-bound-Gp}
G_p(z) \le \frac{1}{1 - F_0(z)} \exp\Big( - \frac{(z - \overline{z} - a_p)_+^2}{2} \Big). 
\end{equation}
Mill's inequality states that for $z > 0$, 
\[ z \le \frac{f_0(z)}{1-F_0(z)} \le z \frac{1+z^2}{z^2}. \]
Since $(1+z^2)/z^2 \le 2$ for $z \ge 1$ and $f_0(z)/\{1-F_0(z)\} \le 1.53 \le 2$ for $z \in (-\infty,1)$, we can infer that
\[ \frac{f_0(z)}{1-F_0(z)} \le 2 (z \vee 1) \quad \text{for any } z \in \reals. \]
This together with \eqref{eq-dens-Z} and \eqref{eq-bound-Gp} yields that
\[ f_p(z) \le 2 (z \vee 1)  \exp\Big( - \frac{(z - \overline{z} - a_p)_+^2}{2} \Big) \quad \text{for any } z \in \reals. \]
 

\textit{Step 3.} By Step 2, we get that for any $y \in \reals$ and $u > 0$, 
\[ \pr( |Z - y| \le u) = \int_{y - u}^{y + u} f_p(z) dz \le 2u \max_{z \in [y-u,y+u]} f_p(z) \le 4u (\overline{z} + a_p + 1), \] 
where the last inequality follows from the fact that the map $z \mapsto z e^{-(z-a)^2/2}$ (with $a > 0$) is non-increasing on $[a+1,\infty)$. Combining this bound with Step 1, we further obtain that for any $x \ge 0$ and $\delta > 0$, 
\begin{equation}\label{eq-bound1-Levy}
\pr \Big( \Big| \max_{1 \le j \le p} X_j - x \Big| \le \delta \Big) \le 4\delta \Big\{ \frac{2\overline{\mu}}{\underline{\sigma}} + |x| \Big(\frac{1}{\underline{\sigma}} - \frac{1}{\overline{\sigma}}\Big) + a_p + 1 \Big\} \big/ \underline{\sigma}. 
\end{equation} 
This inequality also holds for $x < 0$ by an analogous argument, and hence for all $x \in \reals$. 


Now let $0 < \delta \le \underline{\sigma}$ and define $b_p = \ex \max_{1 \le j \le p} \{X_j - \mu_j\}$. For any $|x| \le \delta + \overline{\mu} + b_p + \overline{\sigma} \sqrt{2\log(\underline{\sigma}/\delta)}$, \eqref{eq-bound1-Levy} yields that 
\begin{align}
\pr \Big( \Big| \max_{1 \le j \le p} X_j - x \Big| \le \delta \Big) 
 & \le \frac{4 \delta}{\underline{\sigma}} \Big\{ \overline{\mu} \Big( \frac{3}{\underline{\sigma}} - \frac{1}{\overline{\sigma}} \Big) + a_p + \Big( \frac{1}{\underline{\sigma}} - \frac{1}{\overline{\sigma}} \Big) b_p \nonumber \\ & \phantom{\le \frac{4 \delta}{\underline{\sigma}} \Big\{} + \Big( \frac{\overline{\sigma}}{\underline{\sigma}} - 1 \Big) \sqrt{2\log\Big(\frac{\underline{\sigma}}{\delta}\Big)} + 2 - \frac{\underline{\sigma}}{\overline{\sigma}} \Big\} \nonumber \\[0.2cm]
 & \le C \delta \big\{ \overline{\mu} + a_p + b_p + \sqrt{1 \vee \log(\underline{\sigma}/\delta)} \big\} \label{eq-bound2-Levy}
\end{align}
with a sufficiently large constant $C > 0$ that depends only on $\underline{\sigma}$ and $\overline{\sigma}$. For $|x| \ge \delta + \overline{\mu} + b_p + \overline{\sigma}\sqrt{2\log(\underline{\sigma}/\delta)}$, we obtain that 
\begin{equation}\label{eq-bound3-Levy}
\pr \Big( \Big| \max_{1 \le j \le p} X_j - x \Big| \le \delta \Big) \le \frac{\delta}{\underline{\sigma}}, 
\end{equation}
which can be seen as follows: If $x > \delta + \overline{\mu}$, then $|\max_j X_j - x| \le \delta$ implies that $|x| - \delta \le \max_j X_j \le \max_j \{ X_j - \mu_j \} + \overline{\mu}$ and thus $\max_j \{ X_j - \mu_j \} \ge |x| - \delta - \overline{\mu}$. Hence, it holds that 
\begin{equation}\label{eq-bound3-Levy-prep1}
\pr \Big( \Big| \max_{1 \le j \le p} X_j - x \Big| \le \delta \Big) \le \pr \Big( \max_{1 \le j \le p} \big\{ X_j - \mu_j \} \ge |x| - \delta - \overline{\mu} \Big). 
\end{equation}
If $x < - (\delta + \overline{\mu})$, then $|\max_j X_j - x| \le \delta$ implies that $\max_j \{ X_j - \mu_j \} \le -|x| + \delta + \overline{\mu}$. Hence, in this case,
\begin{align}
\pr \Big( \Big| \max_{1 \le j \le p} X_j - x \Big| \le \delta \Big) 
 & \le \pr \Big( \max_{1 \le j \le p} \big\{ X_j - \mu_j \} \le -|x| + \delta + \overline{\mu} \Big) \nonumber \\
 & \le \pr \Big( \max_{1 \le j \le p} \big\{ X_j - \mu_j \} \ge |x| - \delta - \overline{\mu} \Big), \label{eq-bound3-Levy-prep2}
\end{align}
where the last inequality follows from the fact that for centred Gaussian random variables $V_j$ and $v > 0$, $\pr(\max_j V_j \le -v) \le \pr(V_1 \le -v) = P(V_1 \ge v) \le \pr(\max_j V_j \ge v)$. With \eqref{eq-bound3-Levy-prep1} and \eqref{eq-bound3-Levy-prep2}, we obtain that for any $|x| \ge \delta + \overline{\mu} + b_p + \overline{\sigma}\sqrt{2\log(\underline{\sigma}/\delta)}$,
\begin{align*} 
\pr \Big( & \Big| \max_{1 \le j \le p} X_j - x \Big| \le \delta \Big) \le \pr \Big( \max_{1 \le j \le p} \big\{ X_j - \mu_j \} \ge |x| - \delta - \overline{\mu} \Big) \\
 & \le \pr \Big( \max_{1 \le j \le p} \big\{ X_j - \mu_j \big\} \ge \ex \Big[ \max_{1 \le j \le p} \big\{ X_j-\mu_j \big\} \Big] + \overline{\sigma} \sqrt{2\log(\underline{\sigma}/\delta)} \Big) \le \frac{\delta}{\underline{\sigma}}, 
\end{align*}
the last inequality following from Lemma \ref{lemma3-anticon}. To sum up, we have established that for any $0 < \delta \le \underline{\sigma}$ and any $x \in \reals$, 
\begin{equation}\label{claim-prop-anticon}
\pr \Big( \Big| \max_{1 \le j \le p} X_j - x \Big| \le \delta \Big) \le C \delta \big\{ \overline{\mu} + a_p + b_p + \sqrt{1 \vee \log(\underline{\sigma}/\delta)} \big\} 
\end{equation}
with some constant $C > 0$ that does only depend on $\underline{\sigma}$ and $\overline{\sigma}$. For $\delta > \underline{\sigma}$, \eqref{claim-prop-anticon} trivially follows upon setting $C \ge 1/\underline{\sigma}$. This completes the proof. 



\section{Proofs of the results from Section \ref{sec-error-var}}\label{sec-supp-proofs2}


In what follows, we prove Proposition \ref{prop-lrv} from Section \ref{sec-error-var}. The notation is the same as in the previous section. In particular, the symbol $C$ is used to denote a generic constant which may take a different value on each occurrence. 


\subsection*{Auxiliary results}


To start with, we derive some auxiliary results needed for the proof of Proposition \ref{prop-lrv}. We assume throughout that the assumptions of Proposition \ref{prop-lrv} are satisfied. In particular, we suppose that $q = q_T$ and $p=p_T$ are such that 
\[ C \log T \le p \ll q \quad \text{and} \quad \log T \ll q \ll \sqrt{T} \]
for some sufficiently large constant $C$, where the symbol $v_T \ll w_T$ means that $v_T/w_T \rightarrow 0$ as $T \rightarrow \infty$. The first result bounds the $L_2$-distance between the $\ell$-th sample autocovariance
\[ \widehat{\gamma}_q^*(\ell) =  \frac{1}{T-q} \sum\limits_{t=q+\ell+1}^T \Delta_q \varepsilon_t \, \Delta_q \varepsilon_{t-\ell} \]
of the process $\{\Delta_q \varepsilon_t \}$ and its true counterpart $\gamma_q(\ell) = \cov(\Delta_q \varepsilon_t,\Delta_q \varepsilon_{t-\ell})$. 
\begin{lemmaA}\label{lemma-lrv-1}
For any $1 \le \ell \le p$,   
\[ \ex \Big[ \big( \widehat{\gamma}_q^*(\ell) - \gamma_q(\ell) \big)^2 \Big] \le \frac{C}{T-q}, \]
where $C$ is a fixed constant independent of $\ell$, $p$, $q$ and $T$. 
\end{lemmaA}


\begin{proof}[\textnormal{\textbf{Proof of Lemma \ref{lemma-lrv-1}}}] 
It holds that
\[ \widehat{\gamma}_q^*(\ell) - \gamma_q(\ell) = \Sigma_{q,1}(\ell) - \Sigma_{q,2}(\ell) - \Sigma_{q,3}(\ell) + \Sigma_{q,4}(\ell) - \Big(1 - \frac{T-q-\ell}{T-q}\Big) \gamma_q(\ell), \]
where
\begin{align*}
\Sigma_{q,1}(\ell) & = \frac{1}{T-q} \sum_{t=q+\ell+1}^T \big\{ \varepsilon_t \varepsilon_{t-\ell} - \ex \varepsilon_t \varepsilon_{t-\ell} \big\} \\
\Sigma_{q,2}(\ell) & = \frac{1}{T-q} \sum_{t=q+\ell+1}^T \big\{ \varepsilon_t \varepsilon_{t-\ell-q} - \ex \varepsilon_t \varepsilon_{t-\ell-q} \big\} \\
\Sigma_{q,3}(\ell) & = \frac{1}{T-q} \sum_{t=q+\ell+1}^T \big\{ \varepsilon_{t-q} \varepsilon_{t-\ell} - \ex \varepsilon_{t-q} \varepsilon_{t-\ell} \big\} \\
\Sigma_{q,4}(\ell) & = \frac{1}{T-q} \sum_{t=q+\ell+1}^T \big\{ \varepsilon_{t-q} \varepsilon_{t-\ell-q} - \ex \varepsilon_{t-q} \varepsilon_{t-\ell-q} \big\}. 
\end{align*}
In what follows, we prove that 
\begin{equation}\label{eq1-lemma-lrv-1}
\ex \Sigma_{q,k}^2(\ell) \le \frac{C}{T-q} 
\end{equation}
for $1 \le k \le 4$, where the constant $C$ only depends on the coefficients $c_0,c_1,c_2,\ldots$ of the MA($\infty$) representation of $\{\varepsilon_t\}$ and the innovation variance $\nu^2$. From this, it immediately follows that 
\begin{align*}
\ex \Big[ \big( \widehat{\gamma}_q^*(\ell) - \gamma_q(\ell) \big)^2 \Big] 
 & \le 4 \sum_{k=1}^4 \ex \Sigma_{q,k}^2(\ell) + \Big(1 - \frac{T-q-\ell}{T-q}\Big)^2 \gamma_q^2(\ell) \\
 & \le C \Big\{ \frac{1}{T-q} + \Big(\frac{p}{T-q}\Big)^2 \Big\} \le \frac{C}{T-q}  
\end{align*}
with $C$ independent of $\ell$, $p$, $q$ and $T$, which completes the proof. 


It remains to show \eqref{eq1-lemma-lrv-1} for $1 \le k \le 4$. We restrict attention to $k=1$, the proof for the other cases being completely analogous. Since the variables $\varepsilon_t$ have the MA($\infty$) expansion $\varepsilon_t = \sum_{k=0}^{\infty} c_k \eta_{t-k}$ and $\gamma_\varepsilon(\ell) = (\sum_{k=0}^\infty c_k c_{k+\ell}) \nu^2$, it holds that 
\begin{align*} 
 & \ex \big[ \varepsilon_{t} \varepsilon_{t-\ell} \varepsilon_{t^\prime} \varepsilon_{t^\prime-\ell} \big] \\*
 & = \Big( \sum_{k=0}^{\infty} c_k \, c_{k+t^\prime-t} \, c_{k+\ell} \, c_{k+\ell+t^\prime-t} \Big) \kappa +  \Big( \sum_{k=0}^{\infty} c_k \, c_{k+\ell} \Big)^2 \nu^4 \\*
 & \quad + \Big( \sum_{k=0}^{\infty} c_k \, c_{k+t^\prime-t} \Big)^2 \nu^4 + \Big( \sum_{k=0}^{\infty} c_k \, c_{k+\ell+t^\prime-t} \Big) \Big( \sum_{k=0}^{\infty} c_k \, c_{k+\ell+t-t^\prime} \Big) \nu^4 \\
 & = \Big( \sum_{k=0}^{\infty} c_k \, c_{k+t^\prime-t} \, c_{k+\ell} \, c_{k+\ell+t^\prime-t} \Big) \kappa + \gamma_\varepsilon^2(\ell) + \gamma_\varepsilon^2(t^\prime-t) + \gamma_\varepsilon(t^\prime-t+\ell) \gamma_\varepsilon(t^\prime-t-\ell)
\end{align*}
with $\kappa=\ex[\eta_0^4] - 3 \nu^4$ and $c_k=0$ for $k < 0$. From this, we obtain that $\ex \Sigma_{q,1}^2(\ell) = \sum_{k=1}^3 \ex \Sigma_{q,1,k}^2(\ell)$ with  
\begin{align*}
\ex \Sigma_{q,1,1}^2(\ell) & = \frac{\kappa}{(T-q)^2} \sum_{t,t^\prime=q+\ell+1}^T \Big( \sum_{k=0}^{\infty} c_k \, c_{k+t^\prime-t} \, c_{k+\ell} \, c_{k+\ell+t^\prime-t} \Big) \\
\ex \Sigma_{q,1,2}^2(\ell) & = \frac{1}{(T-q)^2} \sum_{t,t^\prime=q+\ell+1}^T \gamma_{\varepsilon}^2(t^\prime-t) \\
\ex \Sigma_{q,1,3}^2(\ell) & = \frac{1}{(T-q)^2} \sum_{t,t^\prime=q+\ell+1}^T \gamma_{\varepsilon}(t^\prime-t+\ell) \gamma_{\varepsilon}(t^\prime-t-\ell). 
\end{align*}
Let $\# \{ t^\prime - t = r \}$ be the number of pairs $(t,t^\prime)$ with $q+1 \le t,t^\prime \le T$ such that $t^\prime - t = r$ and note that $\# \{ t^\prime - t = r \} \le T-q$ for any $r$. It holds that 
\begin{align}
\ex \Sigma_{q,1,1}^2(\ell) & \le \frac{1}{(T-q)^2} \sum_{r=-T}^T \# \big\{ t^\prime - t = r \big\} \sum_{k=0}^{\infty} | c_k \, c_{k+r} \, c_{k+\ell} \, c_{k+\ell+r} | \nonumber \\
 & \le \frac{1}{T-q} \sum_{k=0}^{\infty} | c_k \, c_{k+\ell} | \sum_{r=-\infty}^\infty | c_{k+r} \, c_{k+\ell+r} | \nonumber \\
 & \le \frac{\{ \max_j|c_j| \}^2 \{ \sum_{k=0}^{\infty} |c_k| \}^2}{T-q}  \le \frac{C}{T-q}, \label{eq2-lemma-lrv-1} 
\end{align}
where $C$ only depends on the parameters $c_0,c_1,c_2,\ldots$ of the MA($\infty$) representation of $\{ \varepsilon_t\}$. Moreover, 
\begin{equation}\label{eq3-lemma-lrv-1}
\ex \Sigma_{q,1,2}^2(\ell) = \frac{1}{(T-q)^2} \sum_{r=-T}^T \# \big\{ t^\prime - t = r \big\} \gamma_{\varepsilon}^2(r) \le \frac{\{ \sum_{r=-\infty}^\infty |\gamma_{\varepsilon}^2(r)| \}}{T-q} \le \frac{C}{T-q} 
\end{equation}
and analogously
\begin{equation}\label{eq4-lemma-lrv-1}
\ex \Sigma_{q,1,3}^2(\ell) \le \frac{2 \gamma_{\varepsilon}(0) \{ \sum_{r=-\infty}^{\infty} |\gamma_{\varepsilon}(r)| \}}{T-q} \le \frac{C}{T-q}, 
\end{equation}
where $C$ only depends on the MA parameters $c_0,c_1,c_2,\ldots$ and the innovation variance $\nu^2$ (noting that $\gamma_\varepsilon(k) = \sum_{j=0}^\infty c_j c_{j+k} \nu^2$). Combining \eqref{eq2-lemma-lrv-1}--\eqref{eq4-lemma-lrv-1}, we arrive at \eqref{eq1-lemma-lrv-1} for $k = 1$. 
\end{proof}


The next result bounds the $L_2$-distance between the $\ell$-th sample autocovariance 
\[ \widehat{\gamma}_q(\ell) = \frac{1}{T-q} \sum_{t=q+\ell+1}^T \Delta_q Y_{t,T} \, \Delta_q Y_{t-\ell,T} \]
of the observed process $\{ \Delta_q Y_{t,T} \}$ and $\gamma_{\varepsilon}(\ell)$. 
\begin{lemmaA}\label{lemma-lrv-2}
For any $1 \le \ell \le p$,   
\[ \ex \Big[ \big( \widehat{\gamma}_q(\ell) - \gamma_q(\ell) \big)^2 \Big] \le \frac{C}{T-q}, \]
where $C$ is a fixed constant independent of $\ell$, $p$, $q$ and $T$. 
\end{lemmaA}


\begin{proof}[\textnormal{\textbf{Proof of Lemma \ref{lemma-lrv-2}}}] 
We decompose $\widehat{\gamma}_q(\ell)$ as $\widehat{\gamma}_q(\ell) = \widehat{\gamma}_q^*(\ell) + R_{q,A}(\ell) + R_{q,B}(\ell) + R_{q,C}(\ell)$, where 
\begin{align*} 
R_{q,A}(\ell) & = \frac{1}{T-q} \sum_{t=q+\ell+1}^T \Delta_q m_t \Delta_q \varepsilon_{t-\ell} \\
R_{q,B}(\ell) & = \frac{1}{T-q} \sum_{t=q+\ell+1}^T \Delta_q \varepsilon_t \Delta_q m_{t-\ell} \\
R_{q,C}(\ell) & = \frac{1}{T-q} \sum_{t=q+\ell+1}^T \Delta_q m_t \Delta_q m_{t-\ell} 
\end{align*}
with $\Delta_q m_t = m(\frac{t}{T}) - m(\frac{t-q}{T})$. 
As $m$ is Lipschitz continuous, we can apply the Cauchy-Schwarz inequality to obtain that 
\[ \ex R_{q,A}^2(\ell) \le  \Big( \frac{1}{T-q} \sum_{t=q+\ell+1}^T \{ \Delta_q m_t \}^2 \Big) \Big( \frac{1}{T-q} \sum_{t=q+\ell+1}^T \{ \Delta_q \varepsilon_{t-\ell} \}^2 \Big) \le C \Big( \frac{q}{T} \Big)^2. \]
where $C$ is independent of $\ell$, $p$, $q$ and $T$. Analogously, we get that $\ex R_{q,k}^2(\ell) \le C (q/T)^2$ for $K = B,C$. Using this together with Lemma \ref{lemma-lrv-1}, we arrive at 
\begin{align*}
\ex \Big[ \big( \widehat{\gamma}_q(\ell) - \gamma_q(\ell) \big)^2 \Big] 
 & \le 4 \Big\{ \ex \Big[ \big( \widehat{\gamma}_q^*(\ell) - \gamma_q(\ell) \big)^2 \Big] + \ex R_{q,A}^2(\ell) + \ex R_{q,B}^2(\ell) + \ex R_{q,C}^2(\ell) \Big\} \\
 & \le C \Big\{ \frac{1}{T-q} + \Big(\frac{q}{T}\Big)^2 \Big\}, 
\end{align*}
which completes the proof. 
\end{proof}


With the help of Lemma \ref{lemma-lrv-2}, we now obtain bounds on the terms $\ex \| \widehat{\boldsymbol{\gamma}}_q - \boldsymbol{\gamma}_q \|_2$ and $\ex \| \widehat{\boldsymbol{\Gamma}}_q - \boldsymbol{\Gamma}_q \|_2$, where $\boldsymbol{\gamma}_q = (\gamma_q(1),\dots,\gamma_q(p))^\top$ and $\boldsymbol{\Gamma}_q = (\gamma_q(i-j): 1 \le i,j \le p)$. 
\begin{lemmaA}\label{lemma-lrv-3}
It holds that 
\[ \ex \| \widehat{\boldsymbol{\gamma}}_q - \boldsymbol{\gamma}_q \|_2 \le C \sqrt{\frac{p}{T-q}} \qquad \text{and} \qquad \ex \| \widehat{\boldsymbol{\Gamma}}_q - \boldsymbol{\Gamma}_q \|_2 \le C \sqrt{\frac{p^2}{T-q}} \]
with some constant $C$ independent of $p$, $q$ and $T$. 
\end{lemmaA}


\begin{proof}[\textnormal{\textbf{Proof of Lemma \ref{lemma-lrv-3}}}] 
From Lemma \ref{lemma-lrv-2}, it immediately follows that $\ex \| \widehat{\boldsymbol{\gamma}}_q - \boldsymbol{\gamma}_q \|_2^2 \le C p / (T-q)$, which yields the first statement of the lemma. The second statement is obtained by using Lemma \ref{lemma-lrv-2} and the bound
\[ \| \widehat{\boldsymbol{\Gamma}}_q - \boldsymbol{\Gamma}_q \|_2 \le \max_{1 \le i \le p} \Big( \sum\limits_{j=1}^p \big| \widehat{\gamma}_q(i-j) - \gamma_q(i-j) \big| \Big) \le \sum\limits_{\ell=-p}^p \big| \widehat{\gamma}_q(\ell) - \gamma_q(\ell) \big|, \]
which follows from Gershgorin's theorem. 
\end{proof}


\begin{lemmaA}\label{lemma-lrv-4}
It holds that 
\[ \| \widetilde{\boldsymbol{a}}_q - \boldsymbol{a} \|_2 = O_p\Big( \frac{p}{\sqrt{T}} \Big). \]
\end{lemmaA}


\begin{proof}[\textnormal{\textbf{Proof of Lemma \ref{lemma-lrv-4}}}] It holds that   
\begin{align*}
\widetilde{\boldsymbol{a}}_q - \boldsymbol{a} = \widehat{\boldsymbol{\Gamma}}_q^{-1} \widehat{\boldsymbol{\gamma}}_q - \boldsymbol{a}   
 & = \widehat{\boldsymbol{\Gamma}}_q^{-1} \big[ \widehat{\boldsymbol{\gamma}}_q - \boldsymbol{\Gamma}_q \boldsymbol{a} + ( \boldsymbol{\Gamma}_q - \widehat{\boldsymbol{\Gamma}}_q )\boldsymbol{a} \big] \\
 & = \widehat{\boldsymbol{\Gamma}}_q^{-1} \big[ (\widehat{\boldsymbol{\gamma}}_q - \boldsymbol{\gamma}_q) + (\boldsymbol{\gamma}_q - \boldsymbol{\Gamma}_q \boldsymbol{a}) + (\boldsymbol{\Gamma}_q - \widehat{\boldsymbol{\Gamma}}_q)\boldsymbol{a} \big] 
\end{align*}
and thus 
\begin{equation}\label{eq1-lemma-lrv-4}
\| \widetilde{\boldsymbol{a}}_q - \boldsymbol{a} \|_2 \le \| \widehat{\boldsymbol{\Gamma}}_q^{-1} \|_2 \big[ \| \widehat{\boldsymbol{\gamma}}_q - \boldsymbol{\gamma}_q \|_2 + \| \boldsymbol{\gamma}_q - \boldsymbol{\Gamma}_q \boldsymbol{a} \|_2 + \| \boldsymbol{\Gamma}_q - \widehat{\boldsymbol{\Gamma}}_q \|_2 \| \boldsymbol{a} \|_2 \big]. 
\end{equation}
We now use the following results: (i) By Lemma \ref{lemma-lrv-3} and standard arguments, one can show that the inverse $\widehat{\boldsymbol{\Gamma}}_q^{-1}$ exists with probability tending to $1$ and $\| \widehat{\boldsymbol{\Gamma}}_q^{-1}  - \boldsymbol{\Gamma}_q^{-1} \|_2 = o_p(1)$, which implies that $\| \widehat{\boldsymbol{\Gamma}}_q^{-1} \|_2 = O_p(1)$. (ii) By Lemma \ref{lemma-lrv-3}, we immediately get that $\| \widehat{\boldsymbol{\gamma}}_q - \boldsymbol{\gamma}_q \|_2 = O_p(\sqrt{p/T})$. (iii) It holds that $\boldsymbol{\gamma}_q - \boldsymbol{\Gamma}_q \boldsymbol{a} =\nu^2 \boldsymbol{c}_q + \boldsymbol{\rho}_q$, where $\boldsymbol{c}_q = (c_{q-1},\ldots,c_{q-p})^\top$ and $\boldsymbol{\rho}_q = (\rho_q(1),\ldots,\rho_q(p))^\top$ with $\rho_q(\ell) = \sum_{j=p+1}^{\infty} a_j \gamma_q(\ell-j)$. Since $|c_j| \le C \xi^j$ and $|\rho_q(\ell)| \le C \sum_{j=p+1}^{\infty} \xi^j \le C \xi^p$, we can infer that $\| \boldsymbol{\gamma}_q - \boldsymbol{\Gamma}_q \boldsymbol{a} \|_2 \le C \sqrt{p} (\xi^{q-p} + \xi^p )$. (iv) By Lemma \ref{lemma-lrv-3}, $\| \boldsymbol{\Gamma}_q - \widehat{\boldsymbol{\Gamma}}_q \|_2 = O_p(p/\sqrt{T})$ and $\| \boldsymbol{a} \|_2 \le ( \sum_{j=1}^{\infty} a_j^2 )^{1/2} \le C$. Plugging (i)--(iv) into \eqref{eq1-lemma-lrv-4}, we arrive at 
\[ \| \widetilde{\boldsymbol{a}}_q - \boldsymbol{a} \|_2 = O_p \Big( \frac{p}{\sqrt{T}} + \sqrt{p} \xi^{q-p} + \sqrt{p} \xi^p \Big) = O_p\Big(\frac{p}{\sqrt{T}}\Big), \]
which is the statement of the lemma.
\end{proof}


\begin{lemmaA}\label{lemma-lrv-5}
It holds that 
\[ \widetilde{\nu}^2 = \nu^2 + O_p\Big( p \sqrt{\frac{p}{T}} \Big). \]
\end{lemmaA}


%\begin{proof}[\textnormal{\textbf{Proof of Lemma \ref{lemma-lrv-5}}}] 
%Straightforward calculations yield that $\widetilde{\nu}^2 = \widetilde{\nu}_A^2 + \widetilde{\nu}_B^2 + \widetilde{\nu}_C^2$, where
%\begin{align*}
%\widetilde{\nu}_A^2 & = \frac{1}{2T} \sum\limits_{t=p+1}^T \Big\{ \Delta_1 \varepsilon_t - \sum\limits_{j=1}^p \widetilde{a}_j \Delta_1 \varepsilon_{t-j} \Big\}^2 \\
%\widetilde{\nu}_B^2 & = \frac{1}{2T} \sum\limits_{t=p+1}^T \Big\{ \Delta_1 m_t - \sum\limits_{j=1}^p \widetilde{a}_j \Delta_1 m_{t-j} \Big\}^2 \\
%\widetilde{\nu}_C^2 & = \frac{1}{T} \sum\limits_{t=p+1}^T \Big\{ \Delta_1 \varepsilon_t - \sum\limits_{j=1}^p \widetilde{a}_j \Delta_1 \varepsilon_{t-j} \Big\} \Big\{ \Delta_1 m_t - \sum\limits_{j=1}^p \widetilde{a}_j \Delta_1 m_{t-j} \Big\}. 
%\end{align*}
%Moreover, $\widetilde{\nu}_A^2 = \widetilde{\nu}_{A,1}^2 + \widetilde{\nu}_{A,2}^2 + \widetilde{\nu}_{A,3}^2$, where
%\begin{align*}
%\widetilde{\nu}_{A,1}^2 & = \frac{1}{2T} \sum\limits_{t=p+1}^T \Big\{ \Delta_1 \varepsilon_t - \sum\limits_{j=1}^p a_j \Delta_1 \varepsilon_{t-j} \Big\}^2 \\
%\widetilde{\nu}_{A,2}^2 & = \frac{1}{2T} \sum\limits_{t=p+1}^T \Big\{ \sum\limits_{j=1}^p (a_j - \widetilde{a}_j) \Delta_1 \varepsilon_{t-j} \Big\}^2 \\
%\widetilde{\nu}_{A,3}^2 & = \frac{1}{T} \sum\limits_{t=p+1}^T \Big\{ \Delta_1 \varepsilon_t - \sum\limits_{j=1}^p a_j \Delta_1 \varepsilon_{t-j} \Big\} \Big\{ \sum\limits_{j=1}^p (a_j - \widetilde{a}_j) \Delta_1 \varepsilon_{t-j} \Big\}.
%\end{align*}
%\end{proof}



\newpage


\subsection*{Proof of Proposition \ref{prop-lrv}}


We first show that the pilot estimator $\widetilde{\boldsymbol{a}}_q$ converges to $\boldsymbol{a}$. In particular, we verify that $\widetilde{\boldsymbol{a}}_q - \boldsymbol{a} = O_p(T^{-1/2})$. By Lemma \ref{lemma-lrv-2}, it holds that $\widehat{\boldsymbol{\Gamma}}_q = \boldsymbol{\Gamma}_q + O_p(T^{-1/2})$ and $\widehat{\boldsymbol{\gamma}}_q = \boldsymbol{\gamma}_q + O_p(T^{-1/2})$. Since $\boldsymbol{\Gamma}_q$ is invertible, this implies that 
\[ \widetilde{\boldsymbol{a}}_q = \boldsymbol{\Gamma}_q^{-1} \boldsymbol{\gamma}_q + O_p(T^{-1/2}). \]
With the help of equation \eqref{YW-eq}, we can further infer that 
\[ \widetilde{\boldsymbol{a}}_q - \boldsymbol{a} = -\nu^2 \boldsymbol{\Gamma}_q^{-1} \boldsymbol{c}_q + O_p(T^{-1/2}). \]
As already noted in Section \ref{subsec-error-var-AR}, the entries of the vector $\boldsymbol{c}_q = (c_{q-1},\ldots,c_{c-p})^\top$ decay exponentially fast to zero, that is, $|c_k| \le C \rho^k$ for some $0 < \rho < 1$. Moreover, it holds that $\gamma_q(\ell) \rightarrow 2 \gamma_\varepsilon(\ell)$ for any fixed $\ell$ as $q \rightarrow \infty$. Consequently, $\| \nu^2 \boldsymbol{\Gamma}_q^{-1} \boldsymbol{c}_q \|_\infty = o(T^{-1/2})$, where $\| \cdot \|_\infty$ denotes the usual supremum norm for vectors. As a result, we obtain that $\widetilde{\boldsymbol{a}}_q - \boldsymbol{a} = O_p(T^{-1/2})$.  
  

We next show that $\widehat{\boldsymbol{a}}_r - \boldsymbol{a} = O_p(T^{-1/2})$, where $r \ge 1$ is any fixed integer that does not grow with the sample size $T$. By definition, it holds that $\widehat{\boldsymbol{a}}_r = \widehat{\boldsymbol{\Gamma}}_r^{-1} (\widehat{\boldsymbol{\gamma}}_r + \widetilde{\nu}^2 \widetilde{\boldsymbol{c}}_r)$.  From Lemma \ref{lemma-lrv-2}, it follows that $\widehat{\boldsymbol{\Gamma}}_r^{-1} = \boldsymbol{\Gamma}_r^{-1} + O_p(T^{-1/2})$ and $\widehat{\boldsymbol{\gamma}}_r = \boldsymbol{\gamma}_r + O_p(T^{-1/2})$. Moreover, with the help of the fact that $\widetilde{\boldsymbol{a}}_q - \boldsymbol{a} = O_p(T^{-1/2})$, it is straightforward to verify that $\widetilde{\nu}^2 - \nu^2 = O_p(T^{-1/2})$ and $\widetilde{\boldsymbol{c}}_r - \boldsymbol{c}_r =  O_p(T^{-1/2})$. Hence, we arrive at  
\begin{equation}\label{eq-conv-est-AR-SS}
\widehat{\boldsymbol{a}}_r = \boldsymbol{\Gamma}_r^{-1} (\boldsymbol{\gamma}_r + \nu^2 \boldsymbol{c}_r) + O_p(T^{-1/2}) = \boldsymbol{a} + O_p(T^{-1/2}), 
\end{equation}
where the last equality is due to equation \eqref{YW-eq}.


From \eqref{eq-conv-est-AR-SS}, it immediately follows that $\widehat{\boldsymbol{a}} - \boldsymbol{a} = O_p(T^{-1/2})$, which in turn allows us to infer that $\widehat{\nu}^2 - \nu^2 = O_p(T^{-1/2})$ and $\widehat{\sigma}^2 = \sigma^2 + O_p(T^{-1/2})$ by straightforward arguments. 


