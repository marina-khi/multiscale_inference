\documentclass[a4paper,12pt]{article}
\usepackage{amsmath, bm}
\usepackage{amssymb,amsthm,graphicx}
\usepackage{enumitem}
\usepackage{color}
\usepackage{epsfig}
\usepackage{graphics}
\usepackage{pdfpages}
\usepackage{subcaption}
\usepackage[font=small]{caption}
\usepackage[hang,flushmargin]{footmisc} 
\usepackage{float}
\usepackage{booktabs}
\usepackage[mathscr]{euscript}
\usepackage{natbib}
\usepackage{setspace}
\usepackage{mathrsfs}
\usepackage[left=2.7cm,right=2.7cm,bottom=2.7cm,top=2.7cm]{geometry}
\parindent0pt 


% General

\newcommand{\reals}{\mathbb{R}}
\newcommand{\integers}{\mathbb{Z}}
\newcommand{\naturals}{\mathbb{N}}

\newcommand{\pr}{\mathbb{P}}        % probability
\newcommand{\ex}{\mathbb{E}}        % expectation
\newcommand{\var}{\textnormal{Var}} % variance
\newcommand{\cov}{\textnormal{Cov}} % covariance

\newcommand{\law}{\mathcal{L}} % law of X
\newcommand{\normal}{N}        % normal distribution 

\newcommand{\argmax}{\textnormal{argmax}}
\newcommand{\argmin}{\textnormal{argmin}}

\newcommand{\ind}{\mathbbm{1}} % indicator function
\newcommand{\kernel}{K} % kernel function
\newcommand{\wght}{W} % kernel weight
\newcommand{\thres}{\pi} % threshold parameter


% Convergence

\newcommand{\convd}{\stackrel{d}{\longrightarrow}}              % convergence in distribution
\newcommand{\convp}{\stackrel{P}{\longrightarrow}}              % convergence in probability
\newcommand{\convas}{\stackrel{\textrm{a.s.}}{\longrightarrow}} % convergence almost surely
\newcommand{\convw}{\rightsquigarrow}                           % weak convergence


% Theorem-like declarations

\theoremstyle{plain}

\newtheorem{theorem}{Theorem}[section]
\newtheorem{prop}[theorem]{Proposition}
\newtheorem{lemma}[theorem]{Lemma}
\newtheorem{corollary}[theorem]{Corollary}
\newtheorem*{theo}{Theorem}
\newtheorem{propA}{Proposition}[section]
\newtheorem{lemmaA}[propA]{Lemma}
\newtheorem{definition}{Definition}[section]
\newtheorem{remark}{Remark}[section]
\renewcommand{\thelemmaA}{A.\arabic{lemmaA}}
\renewcommand{\thepropA}{A.\arabic{propA}}
\newtheorem*{algo}{Clustering Algorithm}


% Theorem numbering to the left

\makeatletter
\newcommand{\lefteqno}{\let\veqno\@@leqno}
\makeatother


% Heading

\newcommand{\heading}[2]
{  \setcounter{page}{1}
   \begin{center}

   \phantom{Distance to upper boundary}
   \vspace{0.5cm}

   {\LARGE \textbf{#1}}
   \vspace{0.4cm}
 
   {\LARGE \textbf{#2}}
   \end{center}
}


% Authors

\newcommand{\authors}[4]
{  \parindent0pt
   \begin{center}
      \begin{minipage}[c][2cm][c]{5cm}
      \begin{center} 
      {\large #1} 
      \vspace{0.05cm}
      
      #2 
      \end{center}
      \end{minipage}
      \begin{minipage}[c][2cm][c]{5cm}
      \begin{center} 
      {\large #3}
      \vspace{0.05cm}

      #4 
      \end{center}
      \end{minipage}
   \end{center}
}

%\newcommand{\authors}[2]
%{  \parindent0pt
%   \begin{center}
%   {\large #1} 
%   \vspace{0.1cm}
%      
%   #2 
%   \end{center}  
%}


% Version

\newcommand{\version}[1]
{  \begin{center}
   {\large #1}
   \end{center}
   \vspace{3pt}
} 










\begin{document}
\numberwithin{equation}{section}
\allowdisplaybreaks[1]

\subsection{Theoretical properties of the test}\label{subsec-test-theo}


To start with, we introduce the auxiliary statistic 
\begin{align}\label{Phi-hat-statistic}
\widehat{\Phi}_{n,T} = \max_{1 \le i < j \le n} \widehat{\Phi}_{ij,T},
\end{align}
where
\[ \widehat{\Phi}_{ij,T} = \max_{(u,h) \in \mathcal{G}_T} \Big\{ \Big| \frac{\widehat{\phi}_{ij,T}(u,h)} {\{ \widehat{\sigma}_i^2 + \widehat{\sigma}_j^2 \}^{1/2}} \Big| - \lambda(h) \Big \} \]
and $\widehat{\phi}_{ij,T}(u,h) = \sum_{t=1}^T w_{t,T}(u,h) \big\{ (\varepsilon_{it} - \bar{\varepsilon}_i) + \bm{\beta}_i^\top (\mathbf{X}_{it} - \bar{\mathbf{X}}_{i}) - (\varepsilon_{jt} - \bar{\varepsilon}_j) -\bm{\beta}_j^\top (\mathbf{X}_{jt} - \bar{\mathbf{X}}_{j}) \big\}$ with $\bar{\varepsilon}_i = \bar{\varepsilon}_{i,T} = T^{-1} \sum_{t=1}^T \varepsilon_{it}$ and $ \bar{\mathbf{X}}_{i} =  \bar{\mathbf{X}}_{i, T} = T^{-1}\sum_{t=1}^T  \mathbf{X}_{it}$ respectively. Our first theoretical result characterizes the asymptotic behaviour of the statistic $\widehat{\Phi}_{n,T}$. 
\begin{theorem}\label{theo-stat-equality}
Suppose that the error processes $\mathcal{E}_i = \{ \varepsilon_{it}: 1 \le t \le T \}$ are independent across $i$ and satisfy \ref{C-err1}--\ref{C-err3} for each $i$. Moreover, let \ref{C-ker}--\ref{C-h} be fulfilled and assume that $\widehat{\sigma}_i^2 = \sigma_i^2 + o_p(\rho_T)$ with $\rho_T = o(1/\log T)$ for each $i$. Then 
\[ \pr \big( \widehat{\Phi}_{n,T} \le q_{n,T}(\alpha)| \{ \mathbf{X}_{it}: 1 \le t \le T, 1 \leq i \leq n \} \big) = (1 - \alpha) + o(1) \text{ a.s.} \]
\end{theorem}


\section{Proof of the Theorem \ref{theo-stat-equality}}\label{sec-proof-1}


In this section, we prove the theoretical results from Section \ref{sec-method}. We use the following notation: The symbol $C$ denotes a universal real constant which may take a different value on each occurrence. For $a,b \in \reals$, we write $a_+ = \max \{0,a\}$ and $a \vee b = \max\{a,b\}$. For any set $A$, the symbol $|A|$ denotes the cardinality of $A$. The notation $X \stackrel{\mathcal{D}}{=} Y$ means that the two random variables $X$ and $Y$ have the same distribution. Finally, $f_0(\cdot)$ and $F_0(\cdot)$ denote the density and distribution function of the standard normal distribution, respectively.



\subsection*{Auxiliary results using strong approximation theory}


The main purpose of this section is to prove that there is a version of the multiscale statistic $\widehat{\Phi}_{n,T}$ defined in \eqref{Phi-hat-statistic} which is close to a Gaussian statistic whose distribution is known. More specifically, we prove the following result. 
\begin{propA}\label{propA-strong-approx-equality}
Under the conditions of Theorem \ref{theo-stat-equality}, there exist statistics $\widetilde{\Phi}_{n,T}$ for $T = 1,2,\ldots$ with the following two properties: (i) $\widetilde{\Phi}_{n, T}$ has the same distribution as $\widehat{\Phi}_{n, T}$ for any $T$, and (ii)
\[ \big| \widetilde{\Phi}_{n, T} - \Phi_{n,T} \big| = o_p \Big( \frac{T^{1/q}}{\sqrt{T h_{\min}}} + \rho_T \sqrt{\log T} \Big), \]
where $\Phi_{n,T}$ is a Gaussian statistic as defined in \eqref{Phi-statistic}. 
\end{propA}
\begin{proof}[\textnormal{\textbf{Proof of Proposition \ref{propA-strong-approx-equality}}}] 
For the proof, we draw on strong approximation theory for each stationary process $\mathcal{E}_i = \{\varepsilon_{it}: 1 \leq t \leq T\}$ that fulfill the conditions \ref{C-err1}--\ref{C-err3}. By Theorem 2.1 and Corollary 2.1 in \cite{BerkesLiuWu2014}, the following strong approximation result holds true: On a richer probability space, there exist a standard Brownian motion $\mathbb{B}$ and a sequence $\{ \widetilde{\varepsilon}_{t}: t \in \naturals \}$ such that $[\widetilde{\varepsilon}_{1},\ldots,\widetilde{\varepsilon}_{T}] \stackrel{\mathcal{D}}{=} [\varepsilon_{1},\ldots,\varepsilon_{T}]$ for each $T$ and 
\begin{equation}\label{eq-strongapprox-dep}
\max_{1 \le t \le T} \Big| \sum\limits_{s=1}^t \widetilde{\varepsilon}_{s} - \sigma \mathbb{B}(t) \Big| = o\big( T^{1/q} \big) \quad \text{a.s.},  
\end{equation}
where $\sigma^2 = \sum_{k \in \integers} \cov(\varepsilon_{0}, \varepsilon_{k})$ denotes the long-run error variance.

To apply this result, we define 
\begin{align}\label{Phi-tilde-statistic}
\widetilde{\Phi}_{n,T} = \max_{1 \le i < j \le n} \widetilde{\Phi}_{ij,T},
\end{align}
where
\[ \widetilde{\Phi}_{ij, T} = \max_{(u,h) \in \mathcal{G}_T} \Big\{ \Big|\frac{\widetilde{\phi}_{ij, T}(u,h)}{\{\widehat{\sigma}_i^2 + \widehat{\sigma}_j^2 \}^{1/2}} \Big| - \lambda(h) \Big\}, \]
where $\widetilde{\phi}_{ij, T}(u,h) = \sum\nolimits_{t=1}^T w_{t,T}(u,h) \big\{ (\widetilde{\varepsilon}_{it} - \bar{\widetilde{\varepsilon}}_i) + \bm{\beta}_i^\top (\mathbf{X}_{it} - \bar{\mathbf{X}}_{i}) - (\widetilde{\varepsilon}_{jt} - \bar{\widetilde{\varepsilon}}_j) -\bm{\beta}_j^\top (\mathbf{X}_{jt} - \bar{\mathbf{X}}_{j}) \big\}$ and $\widetilde{\sigma}^2_i$ are the same estimators as $\widehat{\sigma}^2_i$ with $Y_{it} = 
(\bm{\beta}_i - \widehat{\bm{\beta}}_i)^\top \mathbf{X}_{it} + m_i ( t/T) + \big( \alpha_i - \widehat{\alpha}_i \big) + \varepsilon_{it}$
replaced by $\widetilde{Y}_{t,T} = 
(\bm{\beta}_i - \widehat{\bm{\beta}}_i)^\top \mathbf{X}_{it} + m_i(t/T) + \big( \alpha_i - \widehat{\alpha}_i \big) + \widetilde{\varepsilon}_{it}$  for $1 \le t \le T$. In addition, we let
\begin{align*}
\Phi_{n, T} & = \max_{1\leq i < j \leq n} \Phi_{ij, T}= \max_{1\leq i < j \leq n} \max_{(u,h) \in \mathcal{G}_T} \Big\{ \Big|\frac{\phi_{ij, T}(u,h)}{\{\widehat{\sigma}_i^2 + \widehat{\sigma}_j^2 \}^{1/2}}\Big| - \lambda(h) \Big\} \\
\Phi_{n, T}^{\diamond} & =\max_{1\leq i < j \leq n} \Phi_{ij, T}^{\diamond} = \max_{1\leq i< j \leq n}\max_{(u,h) \in \mathcal{G}_T} \Big\{ \Big|\frac{\phi_{ij, T}(u,h)}{\{\widetilde{\sigma}_i^2 + \widetilde{\sigma}_j^2 \}^{1/2}}\Big| - \lambda(h) \Big\} 
\end{align*}
with $$\phi_{ij, T}(u,h) = \sum\nolimits_{t=1}^T w_{t,T}(u,h) \big\{ \widehat{\sigma}_i (Z_{it} - \bar{Z}_i) + \widehat{\bm{\beta}}_i^\top (\mathbf{X}_{it} - \bar{\mathbf{X}}_{i}) - \widehat{\sigma}_j (Z_{jt} - \bar{Z}_j) - \widehat{\bm{\beta}}_j^\top (\mathbf{X}_{jt} - \bar{\mathbf{X}}_{j}) \big\}$$ and $Z_{it} = \mathbb{B}_i(t) - \mathbb{B}_i(t-1)$. With this notation, we can write 
\begin{equation}\label{eq-strongapprox-bound1}
\big| \widetilde{\Phi}_{n, T} - \Phi_{n, T} \big| \le \big| \widetilde{\Phi}_{n, T} - \Phi_{n, T}^{\diamond} \big| + \big| \Phi_{n, T}^{\diamond} - \Phi_{n, T} \big|. 
\end{equation}
First consider $|\widetilde{\Phi}_{n, T} - \Phi_{n, T}^{\diamond}|$. Straightforward calculations yield that 
\begin{align}\label{eq-strongapprox-bound2}
\big| \widetilde{\Phi}_{n, T} - \Phi_{n, T}^{\diamond} \big| \le \{\widetilde{\sigma}_i^2 + \widetilde{\sigma}_j^2 \}^{-1/2} \max_{1\le i < j \le n} \max_{(u,h) \in \mathcal{G}_T} \big| \widetilde{\phi}_{ij, T}(u,h) - \phi_{ij, T}(u,h) \big|.
\end{align}
Using summation by parts,
($\sum_{i=1}^n a_i b_i = \sum_{i=1}^{n-1} A_i (b_i - b_{i+1}) + A_n b_n$ with $A_j = \sum_{j=1}^i a_j$) 
we further obtain that 
\begin{align*}
\big| \widetilde{\phi}_{ij, T}(u,h) &- \phi_{ij, T}(u,h) \big| = \\
=&\bigg|\sum_{t=1}^T w_{t,T}(u,h) \big\{ (\widetilde{\varepsilon}_{it} - \bar{\widetilde{\varepsilon}}_i) + \bm{\beta}_i^\top (\mathbf{X}_{it} - \bar{\mathbf{X}}_{i}) - (\widetilde{\varepsilon}_{jt} - \bar{\widetilde{\varepsilon}}_j) -\bm{\beta}_j^\top (\mathbf{X}_{jt} - \bar{\mathbf{X}}_{j}) -\\
&-\widehat{\sigma}_i (Z_{it} - \bar{Z}_i) - \widehat{\bm{\beta}}_i^\top (\mathbf{X}_{it} - \bar{\mathbf{X}}_{i}) - \widehat{\sigma}_j (Z_{jt} - \bar{Z}_j) + \widehat{\bm{\beta}}_j^\top (\mathbf{X}_{jt} - \bar{\mathbf{X}}_{j}) \big\}\bigg| = \\
=&\Big|\sum_{t=1}^{T-1} A_{ij, t} \big(w_{t,T}(u,h) -w_{t+1,T}(u,h)\big) + A_{ij, T} w_{t,T}(u,h)\Big|,
\end{align*}
where 
\begin{align*}
A_{ij, t} = \sum_{s=1}^t \big\{ (\widetilde{\varepsilon}_{is} - \bar{\widetilde{\varepsilon}}_i) &+ \bm{\beta}_i^\top (\mathbf{X}_{is} - \bar{\mathbf{X}}_{i}) - (\widetilde{\varepsilon}_{js} - \bar{\widetilde{\varepsilon}}_j) -\bm{\beta}_j^\top (\mathbf{X}_{js} - \bar{\mathbf{X}}_{j}) - \\
&-\widehat{\sigma}_i (Z_{it} - \bar{Z}_i) - \widehat{\bm{\beta}}_i^\top (\mathbf{X}_{it} - \bar{\mathbf{X}}_{i}) - \widehat{\sigma}_j (Z_{jt} - \bar{Z}_j) + \widehat{\bm{\beta}}_j^\top (\mathbf{X}_{jt} - \bar{\mathbf{X}}_{j}) \big\}.
\end{align*}
Note that by construction $A_{ij, T} = 0$ for all pairs $(i, j)$. Denoting 
\[ W_T(u,h) = \sum\limits_{t=1}^{T-1} |w_{t+1,T}(u,h) - w_{t,T}(u,h)|,\]
we have 
\begin{align}\label{eq-strongapprox-bound3}
\big| \widetilde{\phi}_{ij, T}(u,h) - \phi_{ij, T}(u,h) \big| =& \Big|\sum_{t=1}^{T-1} A_{ij, t} \big(w_{t,T}(u,h) -w_{t+1,T}(u,h)\big)\Big|\le W_T(u, h)\max_{1 \le t \le T} |A_{ij, t}|.
\end{align}
Now consider $\max_{1 \le t \le T} |A_{ij, t}|$:
\begin{align*}
\max_{1 \le t \le T} |A_{ij, t}|  \le & \max_{1 \le t \le T} \Big| \sum\limits_{s=1}^t (\widetilde{\varepsilon}_{is} - \bar{\widetilde{\varepsilon}}_{i}) - \sigma_i \sum\limits_{s=1}^t \big\{ Z_{is} - \bar{Z_i} \big\} \Big| + \max_{1 \le t \le T} \Big| \sum\limits_{s=1}^t (\widetilde{\varepsilon}_{js} - \bar{\widetilde{\varepsilon}}_{j}) - \sigma_j \sum\limits_{s=1}^t \big\{ Z_{js} - \bar{Z_j} \big\} \Big| + \\
& + \max_{1 \le t \le T} \Big|\sum\limits_{s=1}^t(\bm{\beta}_i - \widehat{\bm{\beta}_i})^\top (\mathbf{X}_{is} - \bar{\mathbf{X}}_{i}) \Big| 
+ \max_{1 \le t \le T} \Big|\sum\limits_{s=1}^t(\bm{\beta}_j - \widehat{\bm{\beta}}_j)^\top (\mathbf{X}_{js} - \bar{\mathbf{X}}_{j}) \Big| \le \\
\le & \max_{1 \le t \le T} \Big| \sum\limits_{s=1}^t \widetilde{\varepsilon}_{is} - \sigma_i \sum\limits_{s=1}^t Z_{is} \Big| + \max_{1 \le t \le T} \Big| \sum\limits_{s=1}^t \widetilde{\varepsilon}_{js} - \sigma_j \sum\limits_{s=1}^t Z_{js} \Big| +\\
& + \max_{1 \le t \le T} \Big| t (\bar{\widetilde{\varepsilon}}_{i} - \sigma_i \bar{Z_i}) \Big| + \max_{1 \le t \le T} \Big| t (\bar{\widetilde{\varepsilon}}_{j} - \sigma_j \bar{Z_j}) \Big| + \\
& + \max_{1 \le t \le T} \Big|\sum\limits_{s=1}^t(\bm{\beta}_i - \widehat{\bm{\beta}_i})^\top (\mathbf{X}_{is} - \bar{\mathbf{X}}_{i}) \Big| 
+ \max_{1 \le t \le T} \Big|\sum\limits_{s=1}^t(\bm{\beta}_j - \widehat{\bm{\beta}}_j)^\top (\mathbf{X}_{js} - \bar{\mathbf{X}}_{j}) \Big| \le \\
\le & 2 \max_{1 \le t \le T} \Big| \sum\limits_{s=1}^t \widetilde{\varepsilon}_{is} - \sigma_i \sum\limits_{s=1}^t Z_{is} \Big| + 2 \max_{1 \le t \le T} \Big| \sum\limits_{s=1}^t \widetilde{\varepsilon}_{js} - \sigma_j \sum\limits_{s=1}^t Z_{js} \Big| +\\
& + \max_{1 \le t \le T} \Big|\sum\limits_{s=1}^t(\bm{\beta}_i - \widehat{\bm{\beta}_i})^\top (\mathbf{X}_{is} - \bar{\mathbf{X}}_{i}) \Big| 
+ \max_{1 \le t \le T} \Big|\sum\limits_{s=1}^t(\bm{\beta}_j - \widehat{\bm{\beta}}_j)^\top (\mathbf{X}_{js} - \bar{\mathbf{X}}_{j}) \Big| = \\
= & 2 \max_{1 \le t \le T} \Big| \sum\limits_{s=1}^t \widetilde{\varepsilon}_{is} - \sigma_i \sum\limits_{s=1}^t \big(\mathbb{B}_{i}(s) - \mathbb{B}_{i}(s-1) \big) \Big| + 2 \max_{1 \le t \le T} \Big| \sum\limits_{s=1}^t \widetilde{\varepsilon}_{js} - \sigma_j \sum\limits_{s=1}^t \big(\mathbb{B}_{j}(s) - \mathbb{B}_{j}(s-1) \big) \Big| +\\
& + \max_{1 \le t \le T} \Big|\sum\limits_{s=1}^t(\bm{\beta}_i - \widehat{\bm{\beta}_i})^\top (\mathbf{X}_{is} - \bar{\mathbf{X}}_{i}) \Big| 
+ \max_{1 \le t \le T} \Big|\sum\limits_{s=1}^t(\bm{\beta}_j - \widehat{\bm{\beta}}_j)^\top (\mathbf{X}_{js} - \bar{\mathbf{X}}_{j}) \Big| =\\
= & 2 \max_{1 \le t \le T} \Big| \sum\limits_{s=1}^t \widetilde{\varepsilon}_{is} - \sigma_i \mathbb{B}_{i}(t) \Big| + 2 \max_{1 \le t \le T} \Big| \sum\limits_{s=1}^t \widetilde{\varepsilon}_{js} - \sigma_j \mathbb{B}_{j}(t) \Big| +\\
& + \max_{1 \le t \le T} \Big|(\bm{\beta}_i - \widehat{\bm{\beta}_i})^\top\sum\limits_{s=1}^t (\mathbf{X}_{is} - \bar{\mathbf{X}}_{i}) \Big| 
+ \max_{1 \le t \le T} \Big|(\bm{\beta}_j - \widehat{\bm{\beta}}_j)^\top\sum\limits_{s=1}^t (\mathbf{X}_{js} - \bar{\mathbf{X}}_{j}) \Big|
\end{align*}
Now suppose that \textcolor{red}{for all $1 \le i \le n$ we have $\bm{\beta}_i - \widehat{\bm{\beta}}_i = o(T^{-1/2})$ almost surely}. Then applying the strong approximation result \eqref{eq-strongapprox-dep}, we can infer that
\begin{align}\label{max_At}
\max_{1 \le t \le T} |A_{ij, t}|  = o(T^{1/q}) + o(T^{-1/2})\max_{1 \le t \le T}\Big|\sum\limits_{s=1}^t (\mathbf{X}_{is} - \bar{\mathbf{X}}_{i})\Big| +  o(T^{-1/2})\max_{1 \le t \le T}\Big|\sum\limits_{s=1}^t (\mathbf{X}_{js} - \bar{\mathbf{X}}_{j} )\Big|
\end{align}
Standard arguments show that $\max_{(u,h) \in \mathcal{G}_T} W_T(u,h) = O( 1/\sqrt{Th_{\min}} )$. Plugging \eqref{max_At} in \eqref{eq-strongapprox-bound3} and then in \eqref{eq-strongapprox-bound2}, we can thus infer that 
\begin{align}\label{eq-strongapprox-bound4}
\big| \widetilde{\Phi}_{n, T} - \Phi_{n, T}^{\diamond} \big| &\le \{\widetilde{\sigma}_i^2 + \widetilde{\sigma}_j^2 \}^{-1/2}  \max_{(u,h) \in \mathcal{G}_T} W_T(u, h) \max_{1\le i < j \le n}\max_{1\le t \le T} |A_{ij, t}|\le\nonumber \\
&\le o \Big( \frac{T^{1/q}}{\sqrt{Th_{\min}}} \Big) + o \Big( \frac{T^{-1/2}}{\sqrt{Th_{\min}}} \Big)\max_{1 \le i \le n}\max_{1\le t \le T}\Big|\sum\limits_{s=1}^t (\mathbf{X}_{js} - \bar{\mathbf{X}}_{j} )\Big|.
\end{align}

\color{red}{Now consider $|\Phi_{n, T}^{\diamond} - \Phi_{n, T}|$. Since $\phi_{ij, T}(u,h)$ conditionally on $\{\mathbf{X}_{it}\}$ is distributed as $ \normal(\widehat{\bm{\beta}}_i^\top (\mathbf{X}_{it} - \bar{\mathbf{X}}_{i}) -  \widehat{\bm{\beta}}_j^\top (\mathbf{X}_{jt} - \bar{\mathbf{X}}_{j}),\sigma^2_i + \sigma^2_j)$ for all $(u,h) \in \mathcal{G}_T$ and all $1\le i < j \le n$, $|\mathcal{G}_T| = O(T^\theta)$ for some large but fixed constant $\theta$ by Assumption (??), $n$ is fixed and $\widehat{\sigma}^2_i = \sigma^2_i + o_p(\rho_T)$ as well as $\widetilde{\sigma}^2_i = \sigma^2_i + o_p(\rho_T)$, we can establish that
\begin{align}\label{eq-strongapprox-bound5}
\big| \Phi_{n, T}^{\diamond} - \Phi_{n, T} \big| \le \max_{1\leq i< j \leq n}\max_{(u,h) \in \mathcal{G}_T} \Big|\frac{\phi_{ij, T}(u,h)}{\{\widetilde{\sigma}_i^2 + \widetilde{\sigma}_j^2 \}^{1/2}} - \frac{\phi_{ij, T}(u,h)}{\{\widehat{\sigma}_i^2 + \widehat{\sigma}_j^2 \}^{1/2}}\Big| = o_P(\rho_T \sqrt{\log T}).
\end{align}
Plugging \eqref{eq-strongapprox-bound5} in \eqref{eq-strongapprox-bound1}, we get 
\begin{equation}\label{eq-strongapprox-bound6}
\big| \widetilde{\Phi}_{n, T} - \Phi_{n, T} \big| \le \big| \widetilde{\Phi}_{n, T} - \Phi_{n, T}^{\diamond} \big| + o_P(\rho_T \sqrt{\log T}). 
\end{equation}
Plugging \eqref{eq-strongapprox-bound6} into \eqref{eq-strongapprox-bound1} completes the proof.}
\end{proof}



\subsection*{Auxiliary results using anti-concentration bounds}


In this section, we establish some properties of the Gaussian statistic $\Phi_{n,T}$ defined in \eqref{Phi-statistic}. We in particular show that $\Phi_{n,T}$ does not concentrate too strongly in small regions of the form $[x-\delta_T,x+\delta_T]$ with $\delta_T$ converging to zero.  
%
%
\begin{propA}\label{propA-anticon-equality}
Under the conditions of Theorem \ref{theo-stat-equality}, it holds that 
\[ \sup_{x \in \reals} \pr \Big( | \Phi_{n,T} - x | \le \delta_T \big| \{\mathbf{X}_{it}: 1\le i \le n, 1 \le t \le T \} \Big) = o(1) \text{ a.s.}, \]
where $\delta_T = T^{1/q} / \sqrt{T h_{\min}} + \rho_T \sqrt{\log T}$.
\end{propA}
%
%
\color{red}{\begin{proof}[\textnormal{\textbf{Proof of Proposition \ref{propA-anticon-equality}}}] 
The main technical tool for proving Proposition \ref{propA-anticon-equality} are anti-concentration bounds for Gaussian random vectors. The following proposition slightly generalizes anti-concentration results derived in \cite{Chernozhukov2015}, in particular Theorem 3 therein. 
\begin{propA}\label{theo-anticon}
Let $(X_1,\ldots,X_p)^\top$ be a Gaussian random vector in $\reals^p$ with $\ex[X_j] = \mu_j$ and $\var(X_j) = \sigma_j^2 > 0$ for $1 \le j \le p$. Define $\overline{\mu} = \max_{1 \le j \le p} |\mu_j|$ together with $\underline{\sigma} = \min_{1 \le j \le p} \sigma_j$ and $\overline{\sigma} = \max_{1 \le j \le p} \sigma_j$. Moreover, set $a_p = \ex[ \max_{1 \le j \le p} (X_j-\mu_j)/\sigma_j ]$ and $b_p = \ex[ \max_{1 \le j \le p} (X_j-\mu_j) ]$. For every $\delta > 0$, it holds that
\[ \sup_{x \in \reals} \pr \Big( \big| \max_{1 \le j \le p} X_j - x \big| \le \delta \Big) \le C \delta \big\{ \overline{\mu} + a_p + b_p + \sqrt{1 \vee \log(\underline{\sigma}/\delta)} \big\}, \]
where $C > 0$ depends only on $\underline{\sigma}$ and $\overline{\sigma}$. 
\end{propA} 
The proof of Proposition \ref{theo-anticon} is provided in \cite{KhismatullinaVogt2018}. To apply Proposition \ref{theo-anticon} to our setting at hand, we introduce the following notation: We write $x = (u,h)$ along with $\mathcal{G}_T = \{ x : x \in \mathcal{G}_T \} = \{x_1,\ldots,x_p\}$, where $p := |\mathcal{G}_T| \le O(T^\theta)$ for some large but fixed $\theta > 0$ by our assumptions. Moreover, for $j = 1,\ldots,p$, we set 
\begin{align*}
X_{2j-1} & = \frac{\phi_T(x_{j1},x_{j2})}{\sigma} - \lambda(x_{j2}) \\
X_{2j} & = -\frac{\phi_T(x_{j1},x_{j2})}{\sigma} - \lambda(x_{j2}) 
\end{align*}
with $x_j = (x_{j1},x_{j2})$. This notation allows us to write
\[ \Phi_T = \max_{1 \le j \le 2p} X_j, \]
where $(X_1,\ldots,X_{2p})^\top$ is a Gaussian random vector with the following properties: (i) $\mu_j := \ex[X_j] = - \lambda(x_{j2})$ and thus $\overline{\mu} = \max_{1 \le j \le 2p} |\mu_j| \le C \sqrt{\log T}$, and (ii) $\sigma_j^2 := \var(X_j) = 1$ for all $j$. Since $\sigma_j = 1$ for all $j$, it holds that $a_{2p} = b_{2p}$. Moreover, as the variables $(X_j - \mu_j)/\sigma_j$ are standard normal, we have that $a_{2p} = b_{2p} \le \sqrt{2 \log (2p)} \le C \sqrt{\log T}$. With this notation at hand, we can apply Proposition \ref{theo-anticon} to obtain that 
\[ \sup_{x \in \reals} \pr \Big( \big| \Phi_T - x \big| \le \delta_T \Big) \le C \delta_T \Big[ \sqrt{\log T} + \sqrt{ \log(1/\delta_T) } \Big] = o(1) \]
with $\delta_T = T^{1/q} / \sqrt{T h_{\min}} + \rho_T \sqrt{\log T}$, which is the statement of Proposition \ref{propA-anticon}.
\end{proof}}



\subsection*{Proof of Theorem \ref{theo-stat-equality}}


To prove Theorem \ref{theo-stat-equality}, we make use of the two auxiliary results derived above. By Proposition \ref{propA-strong-approx-equality}, there exist statistics $\widetilde{\Phi}_{n, T}$ for $T = 1,2,\ldots$ which are distributed as $\widehat{\Phi}_T$ for any $T \ge 1$ and which have the property that 
\begin{equation}\label{statement-propA-strong-approx-equality}
\big| \widetilde{\Phi}_{n, T} - \Phi_{n, T} \big| = o_p \Big( \frac{T^{1/q}}{\sqrt{T h_{\min}}} + \rho_T \sqrt{\log T} \Big), 
\end{equation}
where $\Phi_{n, T}$ is a Gaussian statistic as defined in \eqref{Phi-statistic}. The approximation result \eqref{statement-propA-strong-approx} allows us to replace the multiscale statistic $\widehat{\Phi}_{n, T}$ by an identically distributed version $\widetilde{\Phi}_{n, T}$ which is close to the Gaussian statistic $\Phi_{n, T}$. \color{red}{In the next step, we show that  
\begin{equation}\label{eq-theo-stat-equality-step2}
\sup_{x \in \reals} \big| \pr(\widetilde{\Phi}_T \le x) - \pr(\Phi_T \le x) \big| = o(1), 
\end{equation}
which immediately implies the statement of Theorem \ref{theo-stat}. For the proof of \eqref{eq-theo-stat-step2}, we use the following simple lemma: 
\begin{lemmaA}\label{lemma1-theo-stat}
Let $V_T$ and $W_T$ be real-valued random variables for $T = 1,2,\ldots$ such that $V_T - W_T = o_p(\delta_T)$ with some $\delta_T = o(1)$. If 
\begin{equation}\label{eq-lemma1-cond}
\sup_{x \in \reals} \pr(|V_T - x| \le \delta_T) = o(1), 
\end{equation}
then 
\begin{equation}\label{eq-lemma1-statement}
\sup_{x \in \reals} \big| \pr(V_T \le x) - \pr(W_T \le x) \big| = o(1). 
\end{equation}
\end{lemmaA}
The statement of Lemma \ref{lemma1-theo-stat} can be summarized as follows: If $W_T$ can be approximated by $V_T$ in the sense that $V_T - W_T = o_p(\delta_T)$ and if $V_T$ does not concentrate too strongly in small regions of the form $[x - \delta_T,x+\delta_T]$ as assumed in \eqref{eq-lemma1-cond}, then the distribution of $W_T$ can be approximated by that of $V_T$ in the sense of \eqref{eq-lemma1-statement}.
We now apply this lemma with $V_T = \Phi_T$, $W_T = \widetilde{\Phi}_T$ and $\delta_T = T^{1/q} / \sqrt{T h_{\min}} + \rho_T \sqrt{\log T}$: From \eqref{statement-propA-strong-approx}, we already know that $\widetilde{\Phi}_T - \Phi_T = o_p(\delta_T)$. Moreover, by Proposition \ref{propA-anticon}, it holds that 
\begin{equation}\label{statement-propA-anticon}
\sup_{x \in \reals} \pr \Big( | \Phi_T - x | \le \delta_T \Big) = o(1). 
\end{equation}
Hence, the conditions of Lemma \ref{lemma1-theo-stat} are satisfied. Applying the lemma, we obtain \eqref{eq-theo-stat-step2}, which completes the proof of Theorem \ref{theo-stat-equality}.}


\newpage
\bibliographystyle{ims}
{\small
\setlength{\bibsep}{0.55em}
\bibliography{bibliography}}

\end{document}
