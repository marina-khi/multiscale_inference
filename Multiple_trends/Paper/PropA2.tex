\documentclass[a4paper,12pt]{article}
\usepackage{amsmath, bm}
\usepackage{amssymb,amsthm,graphicx}
\usepackage{enumitem}
\usepackage{color}
%\usepackage{epsfig}
%\usepackage{graphics}
%\usepackage{pdfpages}
%\usepackage{subcaption}
%\usepackage[font=small]{caption}
%\usepackage[hang,flushmargin]{footmisc} 
\usepackage{float}
%\usepackage{booktabs}
\usepackage[mathscr]{euscript}
\usepackage{natbib}
%\usepackage{setspace}
%\usepackage{mathrsfs}
\usepackage{bibentry}
\usepackage[left=2.7cm,right=2.7cm,bottom=2.7cm,top=2.7cm]{geometry}
\parindent0pt 

\newcommand{\doublehat}[1]{\skew{5.5}\widehat{\widehat{#1}}}
\newcommand{\doublehattwo}[1]{\widehat{\widehat{#1}}}
\newcommand{\gaussianstat}{\Phi^\prime}
\newcommand{\gaussiankernel}{\phi^\prime}
\newcommand{\pseudogaussianstat}{\Phi}
\newcommand{\pseudogaussiankernel}{\phi}




% General

\newcommand{\reals}{\mathbb{R}}
\newcommand{\integers}{\mathbb{Z}}
\newcommand{\naturals}{\mathbb{N}}

\newcommand{\pr}{\mathbb{P}}        % probability
\newcommand{\ex}{\mathbb{E}}        % expectation
\newcommand{\var}{\textnormal{Var}} % variance
\newcommand{\cov}{\textnormal{Cov}} % covariance

\newcommand{\law}{\mathcal{L}} % law of X
\newcommand{\normal}{N}        % normal distribution 

\newcommand{\argmax}{\textnormal{argmax}}
\newcommand{\argmin}{\textnormal{argmin}}

\newcommand{\ind}{\mathbbm{1}} % indicator function
\newcommand{\kernel}{K} % kernel function
\newcommand{\wght}{W} % kernel weight
\newcommand{\thres}{\pi} % threshold parameter


% Convergence

\newcommand{\convd}{\stackrel{d}{\longrightarrow}}              % convergence in distribution
\newcommand{\convp}{\stackrel{P}{\longrightarrow}}              % convergence in probability
\newcommand{\convas}{\stackrel{\textrm{a.s.}}{\longrightarrow}} % convergence almost surely
\newcommand{\convw}{\rightsquigarrow}                           % weak convergence


% Theorem-like declarations

\theoremstyle{plain}

\newtheorem{theorem}{Theorem}[section]
\newtheorem{prop}[theorem]{Proposition}
\newtheorem{lemma}[theorem]{Lemma}
\newtheorem{corollary}[theorem]{Corollary}
\newtheorem*{theo}{Theorem}
\newtheorem{propA}{Proposition}[section]
\newtheorem{lemmaA}[propA]{Lemma}
\newtheorem{definition}{Definition}[section]
\newtheorem{remark}{Remark}[section]
\renewcommand{\thelemmaA}{A.\arabic{lemmaA}}
\renewcommand{\thepropA}{A.\arabic{propA}}
\newtheorem*{algo}{Clustering Algorithm}


% Theorem numbering to the left

\makeatletter
\newcommand{\lefteqno}{\let\veqno\@@leqno}
\makeatother


% Heading

\newcommand{\heading}[2]
{  \setcounter{page}{1}
   \begin{center}

   \phantom{Distance to upper boundary}
   \vspace{0.5cm}

   {\LARGE \textbf{#1}}
   \vspace{0.4cm}
 
   {\LARGE \textbf{#2}}
   \end{center}
}


% Authors

\newcommand{\authors}[4]
{  \parindent0pt
   \begin{center}
      \begin{minipage}[c][2cm][c]{5cm}
      \begin{center} 
      {\large #1} 
      \vspace{0.05cm}
      
      #2 
      \end{center}
      \end{minipage}
      \begin{minipage}[c][2cm][c]{5cm}
      \begin{center} 
      {\large #3}
      \vspace{0.05cm}

      #4 
      \end{center}
      \end{minipage}
   \end{center}
}

%\newcommand{\authors}[2]
%{  \parindent0pt
%   \begin{center}
%   {\large #1} 
%   \vspace{0.1cm}
%      
%   #2 
%   \end{center}  
%}


% Version

\newcommand{\version}[1]
{  \begin{center}
   {\large #1}
   \end{center}
   \vspace{3pt}
} 










\begin{document}
\begin{propA}\label{propA-intermediate-relation-2}
There exists a sequence of random numbers $\{\gamma_{n, T}\}_T$, that converges to $0$ as $T\to \infty$, such that
\begin{align}\label{eq-intermediate-relation-2}
\pr \Big(\big|\doublehattwo{\Phi}_{n,T} &- \widehat{\Phi}_{n,T}\big| > \gamma_{n, T} \Big) = o(1).
\end{align}
\end{propA}
\begin{proof}[\textnormal{\textbf{Proof of Proposition \ref{propA-intermediate-relation-2}}}] 

Straightforward calculations yield that
\begin{align*}
\begin{split}
\big| \doublehattwo{\Phi}_{n, T} - \widehat{\Phi}_{n, T} \big| &\le \max_{1 \le i < j \le n} \max_{(u,h) \in \mathcal{G}_T} \left( \left|\frac{\doublehat{\phi}_{ij,T}(u,h)}{(\doublehattwo{\sigma}_i^2 + \doublehattwo{\sigma}_j^2)^{1/2}} - \frac{\doublehat{\phi}_{ij,T}(u,h)}{(\widehat{\sigma}_i^2 + \widehat{\sigma}_j^2)^{1/2}}\right| + \left|\frac{\doublehat{\phi}_{ij,T}(u,h)}{(\widehat{\sigma}_i^2 + \widehat{\sigma}_j^2)^{1/2}} - \frac{\widehat{\phi}_{ij,T}(u,h)} {\{ \widehat{\sigma}_i^2 + \widehat{\sigma}_j^2 \}^{1/2}} \right|\right)\le \\
&\le \max_{1 \le i < j \le n} \left( \big|\{\doublehattwo{\sigma}_i^2 + \doublehattwo{\sigma}_j^2 \}^{-1/2} - \{\widehat{\sigma}_i^2 + \widehat{\sigma}_j^2 \}^{-1/2}\big| \max_{(u,h) \in \mathcal{G}_T} \left|\doublehat{\phi}_{ij,T}(u,h)\right|\right) +\\
&\quad +\max_{1\le i < j \le n} \bigg( \{\widehat{\sigma}_i^2 + \widehat{\sigma}_j^2 \}^{-1/2}\max_{(u,h) \in \mathcal{G}_T} \Big| \doublehat{\phi}_{ij, T}(u,h) - \widehat{\phi}_{ij, T}(u,h) \Big|\bigg).
\end{split}
\end{align*}

%First, consider the maximum of the kernel averages $\left|\doublehat{\phi}_{ij,T}(u,h)\right|$: 
%\begin{align*}
%\max_{1 \le i < j \le n} \max_{(u,h) \in \mathcal{G}_T}\left|\doublehat{\phi}_{ij,T}(u,h)\right| &= \max_{1 \le i <j\le n} \max_{(u,h) \in \mathcal{G}_T}\left| \sum_{t=1}^T w_{t,T}(u,h) \big\{ (\varepsilon_{it} - \bar{\varepsilon}_i) - (\varepsilon_{jt} - \bar{\varepsilon}_j)  \big\}\right| \le \\
%& \le 2 \max_{1 \le i \le n} \max_{(u,h) \in \mathcal{G}_T}\left| \sum_{t=1}^T w_{t,T}(u,h) (\varepsilon_{it} - \bar{\varepsilon}_i) \right|
%\end{align*}

Then, consider the difference of the kernel averages:
\begin{align*}
\Big| \doublehat{\phi}_{ij, T}(u,h) &- \widehat{\phi}_{ij, T}(u,h) \Big|  =\Big| \sum_{t=1}^T w_{t,T}(u,h) \big\{ (\bm{\beta}_i - \widehat{\bm{\beta}}_i)^\top (\mathbf{X}_{it} - \bar{\mathbf{X}}_{i}) - (\bm{\beta}_j - \widehat{\bm{\beta}}_j)^\top (\mathbf{X}_{jt} - \bar{\mathbf{X}}_{j}) \big\} \Big| \le \\
&\le \Big| \sum_{t=1}^T w_{t,T}(u,h) (\bm{\beta}_i - \widehat{\bm{\beta}}_i)^\top (\mathbf{X}_{it} - \bar{\mathbf{X}}_{i}) \Big| +  \Big| \sum_{t=1}^T w_{t,T}(u,h)(\bm{\beta}_j - \widehat{\bm{\beta}}_j)^\top (\mathbf{X}_{jt} - \bar{\mathbf{X}}_{j})  \Big| =\\
&= |\bm{\beta}_i - \widehat{\bm{\beta}}_i|^\top \Big| \sum_{t=1}^T w_{t,T}(u,h)  (\mathbf{X}_{it} - \bar{\mathbf{X}}_{i}) \Big| + |\bm{\beta}_j - \widehat{\bm{\beta}}_j|^\top \Big| \sum_{t=1}^T w_{t,T}(u,h) (\mathbf{X}_{jt} - \bar{\mathbf{X}}_{j})  \Big| =\\
&= |\bm{\beta}_i - \widehat{\bm{\beta}}_i|^\top \Big| \sum_{t=1}^T w_{t,T}(u,h)  (\mathbf{X}_{it} - \bar{\mathbf{X}}_{i}) \Big| + |\bm{\beta}_j - \widehat{\bm{\beta}}_j|^\top \Big| \sum_{t=1}^T w_{t,T}(u,h) (\mathbf{X}_{jt} - \bar{\mathbf{X}}_{j})  \Big| \leq\\
&\leq |\bm{\beta}_i - \widehat{\bm{\beta}}_i|^\top \Big| \sum_{t=1}^T w_{t,T}(u,h) \mathbf{X}_{it} \Big| +  \big|(\bm{\beta}_i - \widehat{\bm{\beta}}_i)^\top\bar{\mathbf{X}}_{i}\big| \Big| \sum_{t=1}^T w_{t,T}(u,h)  \Big| +\\
&\quad +|\bm{\beta}_j - \widehat{\bm{\beta}}_j|^\top \Big| \sum_{t=1}^T w_{t,T}(u,h) \mathbf{X}_{jt}  \Big| + \big|(\bm{\beta}_j - \widehat{\bm{\beta}}_j)^\top\bar{\mathbf{X}}_{j}\big| \Big| \sum_{t=1}^T w_{t,T}(u,h)  \Big| 
 \end{align*}
Hence,
\begin{align}\label{ineq-diff-1}
\begin{split}
\big| \doublehattwo{\Phi}_{n, T} - \widehat{\Phi}_{n, T} \big| \le  & \max_{1 \le i < j \le n} \big|\{\doublehattwo{\sigma}_i^2 + \doublehattwo{\sigma}_j^2 \}^{-1/2} - \{\widehat{\sigma}_i^2 + \widehat{\sigma}_j^2 \}^{-1/2}\big|\max_{1 \le i< j \le n} \max_{(u,h) \in \mathcal{G}_T} \left|\doublehat{\phi}_{ij,T}(u,h)\right| + \\
&+ 2\max_{1\le i < j \le n} \{\widehat{\sigma}_i^2+ \widehat{\sigma}_j^2 \}^{-1/2} \max_{1 \le i \le n}\left(|\bm{\beta}_i - \widehat{\bm{\beta}}_i|^\top \max_{(u,h) \in \mathcal{G}_T} \Big| \sum_{t=1}^T w_{t,T}(u,h) \mathbf{X}_{it} \Big|\right) + \\
&+ 2\max_{1\le i < j \le n} \{\widehat{\sigma}_i^2+ \widehat{\sigma}_j^2 \}^{-1/2}\max_{1 \le i \le n}\big|(\bm{\beta}_i - \widehat{\bm{\beta}}_i)^\top\bar{\mathbf{X}}_{i}\big| \max_{(u,h) \in \mathcal{G}_T}  \Big| \sum_{t=1}^T w_{t,T}(u,h)  \Big|
\end{split}
\end{align}

We start by evaluating the second summand in \eqref{ineq-diff-1}.

First, by our assumptions $\widehat{\sigma}_i^2 = \sigma_i^2 + o_P(\rho_T)$. Moreover, for all $i \in \{1, \ldots, n\}$ we know $\sigma_i^2 \neq 0$. Hence, 
\begin{align}\label{ineq-diff-2}
\max_{1\le i < j \le n}\{\widehat{\sigma}_i^2+ \widehat{\sigma}_j^2 \}^{-1/2}  = O_P(1).
\end{align}

Then, by Theorem \ref{theo-regs}, we know that 
\begin{align}\label{ineq-diff-3}
|\bm{\beta}_i - \widehat{\bm{\beta}}_i| = O_P(1/\sqrt{T}).
\end{align}

Now consider the term $\big| \sum_{t=1}^T w_{t,T}(u,h) \mathbf{X}_{it}  \big|$. Without loss of generality, we can regard the covariates $\mathbf{X}_{it}$ to be scalars $X_{it}$, not vectors. The proof in case of vectors proceeds analogously.

 
By construction the weights $w_{t, T}(u, h)$ are not equal to $0$ if and only if \linebreak $T(u-h) \le t \le T(u+h)$. We can use this fact to rewrite
\begin{align*}
\Big| \sum_{t=1}^T w_{t,T}(u,h) X_{it}   \Big|  = \Big| \sum_{t=\lfloor T(u-h) \rfloor}^{\lceil T(u+h) \rceil} w_{t,T}(u,h)X_{it}   \Big|.
\end{align*}
%Note that 
%\begin{align*}
%E\big[w_{t,T}(u,h)X_{it} \big] = w_{t,T}(u,h)  E [X_{it}]  = 0
%\end{align*}
%and, according to \ref{C-reg3},
%\begin{align*}
%E\big[w^2_{t,T}(u,h)X_{it}^2\big] =w^2_{t,T}(u,h) E\big[X_{it}^2\big] = w^2_{t,T}(u,h) E\big[X_{i0}^2\big] < \infty.
%\end{align*}
%Analogously, according to \ref{C-reg3},
%\begin{align*}
%E\big[w^{q^\prime}_{t,T}(u,h)X_{it}^{q^\prime}\big] =w^{q^\prime}_{t,T}(u,h) E\big[X_{it}^{q^\prime}\big] = w^{q^\prime}_{t,T}(u,h) E\big[X_{i0}^{q^\prime}\big] < \infty.
%\end{align*}

Note that
\begin{align*}
\sum_{t=\lfloor T(u-h) \rfloor}^{\lceil T(u+h) \rceil} w^2_{t,T}(u,h) &= \sum_{t=1}^T w^2_{t,T}(u,h) =\\
&= \sum_{t=1}^T\frac{K^2\Big(\frac{\frac{t}{T}-u}{h}\Big) \Big[ S_{T,2}(u,h) - \Big(\frac{\frac{t}{T}-u}{h}\Big) S_{T,1}(u,h) \Big]^2}{ \bigg\{\sum\nolimits_{s=1}^T  K^2\Big(\frac{\frac{s}{T}-u}{h}\Big) \Big[ S_{T,2}(u,h) - \Big(\frac{\frac{s}{T}-u}{h}\Big) S_{T,1}(u,h) \Big]^2 \bigg\} } =\\
&= 1.
\end{align*}
Denoting by $D_{T, u, h}$ the number of integers between $\lfloor T(u-h) \rfloor$ and $\lceil T(u+h) \rceil$ incl. (with obvious bounds $2Th \leq D_{T, u, h} \leq 2Th + 2$), we can normalize the weights as follows:
\begin{align*}
\sum_{t=\lfloor T(u-h) \rfloor}^{\lceil T(u+h) \rceil} \big(\sqrt{D_{T, u, h}}\cdot w_{t,T}(u,h)\big)^2 = D_{T, u, h}.
\end{align*}

According to Theorem \ref{theo-wu} (Theorem 2(ii) in \cite{Wu2016}), if we denote the weights from the theorem as $a_t = \sqrt{D_{T, u, h}}\cdot w_{t,T}(u,h)$, we can bound the following probability:
\begin{align*}
\pr&\left(\Big| \sum_{t=\lfloor T(u-h) \rfloor}^{\lceil T(u+h) \rceil} \sqrt{D_{T, u, h}}\cdot w_{t,T}(u,h)X_{it}  \Big| \geq x\right) \leq\\
&\leq C_1 \frac{\big( \sum_{t=\lfloor T(u-h) \rfloor}^{\lceil T(u+h) \rceil} |\sqrt{D_{T, u, h}}\cdot w_{t,T}(u,h)|^{q^\prime}\big) ||X_{i\cdot}||^{q^\prime}_{q^\prime, \alpha}}{ x^{q^\prime}} + C_2 \exp \left(-\frac{C_3  x^2}{D_{T, u, h}||X_{i\cdot}||^{2}_{2, \alpha}}\right)=\\
&= C_1 \frac{(\sqrt{D_{T, u, h}})^{q^\prime}\big( \sum_{t=\lfloor T(u-h) \rfloor}^{\lceil T(u+h) \rceil} |w_{t,T}(u,h)|^{q^\prime}\big) ||X_{i\cdot}||^{q^\prime}_{q^\prime, \alpha}}{ x^{q^\prime}} + C_2 \exp \left(-\frac{C_3  x^2}{D_{T, u, h}||X_{i\cdot}||^{2}_{2, \alpha}}\right)  
\end{align*}

Now take any $\delta>0$:
\begin{align*}
\pr &\left(\frac{\max_{(u, h) \in \mathcal{G}_T} \Big| \sum_{t=\lfloor T(u-h) \rfloor}^{\lceil T(u+h) \rceil} w_{t,T}(u,h)X_{it}  \Big|}{\sqrt{T}} \geq \delta \right) =\\
&=\pr\left(\max_{(u, h) \in \mathcal{G}_T} \Big| \sum_{t=\lfloor T(u-h) \rfloor}^{\lceil T(u+h) \rceil} w_{t,T}(u,h)X_{it}  \Big| \geq \delta \sqrt{T} \right) \leq \\
&\overset{\text{Boole's inequality}}{\leq} \sum_{(u, h) \in \mathcal{G}_T} \pr \left( \Big| \sum_{t=\lfloor T(u-h) \rfloor}^{\lceil T(u+h) \rceil} w_{t,T}(u,h)X_{it}  \Big| \geq \delta \sqrt{T} \right) = \\
&\overset{\text{``normalisation''}}{=} \sum_{(u, h) \in \mathcal{G}_T} \pr \left( \Big| \sum_{t=\lfloor T(u-h) \rfloor}^{\lceil T(u+h) \rceil} \sqrt{D_{T, u, h}}\cdot w_{t,T}(u,h)X_{it}  \Big| \geq \delta\sqrt{D_{T, u, h}T}  \right) \leq \\
&\overset{\text{Wu's Theorem}}{\leq} \sum_{(u, h) \in \mathcal{G}_T} \left[C_1 \frac{(\sqrt{D_{T, u, h}})^{q^\prime}\big( \sum |w_{t,T}(u,h)|^{q^\prime}\big) ||X_{i\cdot}||^{q^\prime}_{q^\prime, \alpha}}{ \big(\delta\sqrt{D_{T, u, h}T}\big)^{q^\prime}} + C_2 \exp \left(-\frac{C_3 \big(\delta\sqrt{D_{T, u, h}T}\big)^2}{D_{T, u, h}||X_{i\cdot}||^{2}_{2, \alpha}}\right) \right] =\\
&\overset{\text{simplification}}{=} \sum_{(u, h) \in \mathcal{G}_T} \left[C_1 \frac{\big( \sum |w_{t,T}(u,h)|^{q^\prime}\big) ||X_{i\cdot}||^{q^\prime}_{q^\prime, \alpha}}{ \big(\delta \sqrt{T}\big)^{q^\prime}} + C_2 \exp \left(-\frac{C_3 \delta^2 T }{||X_{i\cdot}||^{2}_{2, \alpha}}\right) \right] \leq\\
&\leq C_1 \frac{ T^\theta ||X_{i\cdot}||^{q^\prime}_{q^\prime, \alpha}}{T^{q^\prime/2}\cdot \delta^{q^\prime}} \max_{(u, h) \in \mathcal{G}_T} \left( \sum_{t=\lfloor T(u-h) \rfloor}^{\lceil T(u+h) \rceil} |w_{t,T}(u,h)|^{q^\prime}\right)+ C_2 T^\theta \exp \left(-\frac{C_3 \delta^2 T}{||X_{i\cdot}||^{2}_{2, \alpha}}\right) = \\
&= C \frac{ T^{\theta - q^\prime/2}}{\delta^{q^\prime}} + C T^\theta \exp \left(-C T \delta^2\right).
\end{align*}
where the symbol $C$ denotes a universal real constant that does not depend neither on $T$ nor on $\delta$ and that takes a different value on each occurrence. Here in the last equality we used the following facts:
\begin{enumerate}
	\item $||X_{i\cdot}||^{q^\prime}_{q^\prime, \alpha} = \sup_{t\geq 0} (t+1)^{\alpha} \sum_{s=t}^{\infty} \delta_{q^\prime}(H_{i}, s)  < \infty$ holds true since $\sum_{s=t}^{\infty}\delta_{q^\prime}(H_{i}, s) = O(t^{-\alpha})$ by Assumption \ref{C-reg5};
	\item $\max_{(u, h) \in \mathcal{G}_T} \left( \sum_{t=\lfloor T(u-h) \rfloor}^{\lceil T(u+h) \rceil} |w_{t,T}(u,h)|^{q^\prime}\right) < \infty$ because for every $x \in [0, 1]$ we have $ 0 \leq |x|^{q^\prime/2} \leq x \leq 1$. Thus, since $\sum_{t=\lfloor T(u-h) \rfloor}^{\lceil T(u+h) \rceil} w^2_{t,T}(u,h) = 1$, we have \linebreak $0 \leq w^2_{t,T}(u,h) \leq 1$ for all $t$ and
$$ 0 \leq |w_{t,T}(u,h)|^{q^\prime} =  |w^2_{t,T}(u,h)|^{q^\prime/2} \leq w^2_{t,T}(u,h) \leq 1.$$
This leads us to a bound:  
\begin{align*}
\max_{(u, h) \in \mathcal{G}_T} \left( \sum_{t=\lfloor T(u-h) \rfloor}^{\lceil T(u+h) \rceil} |w_{t,T}(u,h)|^{q^\prime}\right) \leq
\max_{(u, h) \in \mathcal{G}_T} \left( \sum_{t=\lfloor T(u-h) \rfloor}^{\lceil T(u+h) \rceil} |w_{t,T}(u,h)|^{2}\right) =1 < \infty.
\end{align*}
	\item $||X_{i\cdot}||^{2}_{2, \alpha} < \infty$ (follows from $1$).
\end{enumerate}


By Assumption \ref{C-reg3}, $\theta - q^\prime/2 <0$ and the term on the RHS of the above inequality is converging to $0$ as $T \to \infty$ for any fixed $\delta >0$. Hence, 
\begin{align*}
\max_{(u, h) \in \mathcal{G}_T} \Big| \sum_{t=\lfloor T(u-h) \rfloor}^{\lceil T(u+h) \rceil} w_{t,T}(u,h)X_{it}  \Big| = o_P(\sqrt{T}),
\end{align*}
and similarly
\begin{align}\label{ineq-diff-4}
\max_{(u, h) \in \mathcal{G}_T} \Big| \sum_{t=\lfloor T(u-h) \rfloor}^{\lceil T(u+h) \rceil} w_{t,T}(u,h)\mathbf{X}_{it}  \Big| = o_P(\sqrt{T}).
\end{align}
Combining \eqref{ineq-diff-2}, \eqref{ineq-diff-3} and \eqref{ineq-diff-4}, we get the following:
\begin{align}\label{ineq-diff-5}
\begin{split}
2&\max_{1\le i < j \le n} \{\widehat{\sigma}_i^2+ \widehat{\sigma}_j^2 \}^{-1/2}\max_{1\le i \le n} \bigg(|\bm{\beta}_i - \widehat{\bm{\beta}}_i|^\top \max_{(u,h) \in \mathcal{G}_T} \Big| \sum_{t=1}^T w_{t,T}(u,h) \mathbf{X}_{it} \Big|\bigg)  =  \\
&=O_P(1) \cdot O_P(1/\sqrt{T}) \cdot o_P(\sqrt{T}) = o_P(1).
\end{split}
\end{align}
%We can rewrite them as follows.
%\begin{align*}
%w_{t,T}(u,h) = \frac{\frac{1}{\sqrt{Th}} K\Big(\frac{\frac{t}{T}-u}{h}\Big) \Big[ S_{T,2}(u,h) - \Big(\frac{\frac{t}{T}-u}{h}\Big) S_{T,1}(u,h) \Big]}{ \bigg\{\frac{1}{Th}\sum\nolimits_{t=1}^T  K^2\Big(\frac{\frac{t}{T}-u}{h}\Big) \Big[ S_{T,2}(u,h) - \Big(\frac{\frac{t}{T}-u}{h}\Big) S_{T,1}(u,h) \Big]^2 \bigg\}^{1/2} }, 
%\end{align*}
%where
%$S_{T,\ell}(u,h) = (Th)^{-1} \sum\nolimits_{t=1}^T K(\frac{\frac{t}{T}-u}{h}) (\frac{\frac{t}{T}-u}{h})^\ell$ for $\ell = 1,2$. For the sake of simplicity, denote the part of the denominator inside the curly brackets by $\nu_T(u, h)$. Then we can prove the following result:
%\begin{lemmaA}\label{lemma-weights}
%Let 
%\begin{align*}
%\nu(u, h) = \int_{-u/h}^{(1- u)/h} K^2(v)\big[S_{2}(u,h) -v S_{1}(u,h)\big]^2 dv,
%\end{align*}
%with $S_j(u, h) = \int_{-u/h}^{(1-u)/h} K(\rho)\rho^j d\rho$. It holds that 
%\begin{align*}
%|\nu_T(u, h) - \nu(u, h)| \leq \frac{C}{Th}
%\end{align*}
%for all $u\in [0, 1]$ and $h\in (0, 0.5)$, where $C$ does not depend on $u, h$ and $T$. Moreover, 
%\begin{align*}
%\inf_{u\in [0, 1], h\in (0, 0.5)} \nu(u, h) > 0.
%\end{align*}
%\end{lemmaA}
%\begin{proof}
%Proof is ommitted (for now).
%\end{proof}


Now, consider the third summand in \eqref{ineq-diff-1}. 

Similarly as before, 
\begin{align}\label{ineq-diff-6}
\max_{1\le i < j \le n}\{\widehat{\sigma}_i^2+ \widehat{\sigma}_j^2 \}^{-1/2}  = O_P(1)
\end{align}
and
\begin{align}\label{ineq-diff-7}
|\bm{\beta}_i - \widehat{\bm{\beta}}_i| = O_P(1/\sqrt{T}).
\end{align}

Then, by Proposition \ref{propA-reg-5} $\bar{\X}_i = o_P(1)$.

Finally, consider the local linear kernel weights $w_{t,T}(u,h)$ defined in \eqref{weights}. Again, by construction the weights $w_{t, T}(u, h)$ are not equal to $0$ if and only if \linebreak $T(u-h) \le t \le T(u+h)$. We can use this fact to bound  $\left| \sum_{t=1}^T w_{t,T}(u,h)  \right|$ for all $(u, h) \in \mathcal{G}_T$ using the Cauchy-Schwarz inequality:
\begin{align*}
\Big| \sum_{t=1}^T w_{t,T}(u,h)   \Big| & = \left| \sum_{t=\lfloor T(u-h) \rfloor}^{\lceil T(u+h) \rceil} w_{t,T}(u,h)   \right| = \left| \sum_{t=\lfloor T(u-h) \rfloor}^{\lceil T(u+h) \rceil} w_{t,T}(u,h)\cdot  1 \right| \leq\\
&\leq \sqrt{\sum_{t=\lfloor T(u-h) \rfloor}^{\lceil T(u+h) \rceil} w^2_{t,T}(u,h)}\sqrt{\sum_{t=\lfloor T(u-h) \rfloor}^{\lceil T(u+h) \rceil} 1^2} =\\
&=\sqrt{1}\cdot\sqrt{D_{T, u, h}}   \leq \sqrt{2Th + 2} \leq \sqrt{2Th_{\max} +2} \leq \sqrt{T+2}.
\end{align*}
Hence, 
\begin{align}\label{ineq-diff-8}
\max_{(u,h) \in \mathcal{G}_T}  \Big| \sum_{t=1}^T w_{t,T}(u,h)  \Big| = O(\sqrt{T}).
\end{align}

Combining \eqref{ineq-diff-6}, \eqref{ineq-diff-7}, Proposition \ref{propA-reg-5} and \eqref{ineq-diff-8}, we get the following:
\begin{align}\label{ineq-diff-9}
\begin{split}
2&\max_{1\le i < j \le n} \{\widehat{\sigma}_i^2+ \widehat{\sigma}_j^2 \}^{-1/2}\max_{1\le i  \le n}\big|(\bm{\beta}_i - \widehat{\bm{\beta}}_i)^\top\bar{\mathbf{X}}_{i}\big| \max_{(u,h) \in \mathcal{G}_T}  \Big| \sum_{t=1}^T w_{t,T}(u,h)  \Big|  =  \\
&=O_P(1) \cdot O_P(1/\sqrt{T}) \cdot o_P(1) \cdot O(\sqrt{T}) = o_P(1).
\end{split}
\end{align}

Lastly, we look at the first summand in \eqref{ineq-diff-1}. Since $\doublehattwo{\sigma}_i^2 = \sigma_i^2 + o_P(\rho_T)$ and $\widehat{\sigma}_i^2 = \sigma_i^2 + o_P(\rho_T)$ by our assumptions, we have that 
\begin{align}\label{ineq-diff-10}
\max_{1 \le i < j \le n} \big|\{\doublehattwo{\sigma}_i^2 + \doublehattwo{\sigma}_j^2 \}^{-1/2} - \{\widehat{\sigma}_i^2 + \widehat{\sigma}_j^2 \}^{-1/2}\big| = o_P(\rho_T).
\end{align}
Then since $\doublehat{\phi}_{ij,T}(u,h)$ has the same distribution as $\widetilde{\phi}_{ij, T}(u, h)$ for each $1\le i < j \le n$ and each $(u, h) \in \mathcal{G}_T$, we can look at $ \max_{(u,h) \in \mathcal{G}_T}\left|\widetilde{\phi}_{ij, T}(u, h)\right|$ instead:
\begin{align}\label{ineq-diff-11}
\begin{split}
\pr&\left(\left|\max_{(u,h) \in \mathcal{G}_T}\left|\doublehat{\phi}_{ij, T}(u, h)\right|\right| \geq c_T\right) = \pr\left(\left|\max_{(u,h) \in \mathcal{G}_T}\left|\widetilde{\phi}_{ij, T}(u, h)\right| \right|\geq c_T\right) =\\
&= \pr\left(\left|\max_{(u,h) \in \mathcal{G}_T}\left|\widetilde{\phi}_{ij, T}(u, h)\right| - \max_{(u,h) \in \mathcal{G}_T}\left|\gaussiankernel_{ij, T}(u, h)\right| + \max_{(u,h) \in \mathcal{G}_T}\left|\gaussiankernel_{ij, T}(u, h)\right| \right|\geq c_T\right) \leq \\
& \leq \pr\left(\left|\max_{(u,h) \in \mathcal{G}_T}\left|\widetilde{\phi}_{ij, T}(u, h)\right| - \max_{(u,h) \in \mathcal{G}_T}\left|\gaussiankernel_{ij, T}(u, h)\right|\right| \geq c_T/2 \right) + \pr \left(\left|\max_{(u,h) \in \mathcal{G}_T}\left|\gaussiankernel_{ij, T}(u, h)\right| \right|\geq c_T/2\right) \leq\\
&\leq \pr\left(\max_{(u,h) \in \mathcal{G}_T}\left|\widetilde{\phi}_{ij, T}(u, h) - \gaussiankernel_{ij, T}(u, h)\right|\geq c_T/2 \right) + \pr \left(\max_{(u,h) \in \mathcal{G}_T}\left|\gaussiankernel_{ij, T}(u, h)\right| \geq c_T/2\right).
\end{split}
\end{align}
Here we will need one result that we will prove further: by \eqref{eq-strongapprox-bound5} we have
\begin{align*}
\max_{(u,h) \in \mathcal{G}_T}\left|\widetilde{\phi}_{ij, T}(u, h) - \gaussiankernel_{ij, T}(u, h)\right| = o_P\left(\frac{T^{1/q}}{\sqrt{Th_{\min}}}\right).
\end{align*}
Furthermore, $\gaussiankernel_{ij, T}(u,h)$ is distributed as $ \normal(0,{\sigma}^2_i + {\sigma}^2_j)$ for all $(u,h) \in \mathcal{G}_T$ and all $1\le i < j \le n$ and $|\mathcal{G}_T| = O(T^\theta)$ for some large but fixed constant $\theta$ by Assumption \ref{C-grid}. By the standard results from the probability theory, we know that
\begin{align*}
\max_{(u,h) \in \mathcal{G}_T}\left|\gaussiankernel_{ij, T}(u, h)\right| = O_P(\sqrt{\log{T}}).
\end{align*}
Hence, if we take $c_T = o(\sqrt{\log{T}})$ in \eqref{ineq-diff-11}, we will get the following:
\begin{align*}
\pr&\left(\left|\max_{(u,h) \in \mathcal{G}_T}\left|\doublehat{\phi}_{ij, T}(u, h)\right|\right| \geq c_T\right) \leq\\
&\leq \pr\left(\max_{(u,h) \in \mathcal{G}_T}\left|\widetilde{\phi}_{ij, T}(u, h) - \gaussiankernel_{ij, T}(u, h)\right|\geq c_T/2 \right) + \pr \left(\max_{(u,h) \in \mathcal{G}_T}\left|\gaussiankernel_{ij, T}(u, h)\right| \geq c_T/2\right) = \\
&= o(1) + o (1) = o(1),
\end{align*}
which means that
\begin{align}\label{ineq-diff-12}
\left|\max_{(u,h) \in \mathcal{G}_T}\left|\doublehat{\phi}_{ij, T}(u, h)\right|\right| =o_P(\sqrt{\log{T}})
\end{align}

Combining \eqref{ineq-diff-10} and \eqref{ineq-diff-12}, we get the following:
\begin{align}\label{ineq-diff-13}
\begin{split}
&\max_{1 \le i < j \le n} \big|\{\doublehattwo{\sigma}_i^2 + \doublehattwo{\sigma}_j^2 \}^{-1/2} - \{\widehat{\sigma}_i^2 + \widehat{\sigma}_j^2 \}^{-1/2}\big|\max_{1 \le i< j \le n} \max_{(u,h) \in \mathcal{G}_T} \left|\doublehat{\phi}_{ij,T}(u,h)\right| = \\
&\quad= o_P(\rho_T) \cdot o_P(\sqrt{\log{T}}) = \\
&\quad = o_P(1).
\end{split}
\end{align}

Plugging \eqref{ineq-diff-5}, \eqref{ineq-diff-9} and \eqref{ineq-diff-13} in \eqref{ineq-diff-1}, we get that $\big|\doublehattwo{\Phi}_{n,T} - \widehat{\Phi}_{n,T}\big| = o_P(1)$ and the statement of the theorem follows.
\end{proof}


\end{document}