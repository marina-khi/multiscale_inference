\documentclass[a4paper,12pt]{article}
\usepackage{amsmath, bm}
\usepackage{amssymb,amsthm,graphicx}
\usepackage{enumitem}
\usepackage{color}
\usepackage{float}
\usepackage{subcaption}
\usepackage[mathscr]{euscript}
\usepackage{dsfont}
\usepackage{natbib}
\usepackage{amsmath}
\makeatletter
\renewcommand{\eqref}[1]{\tagform@{\ref{#1}}}
\def\maketag@@@#1{\hbox{#1}}
\makeatother
\usepackage{bibentry}
\usepackage[left=2.7cm,right=2.7cm,bottom=2.7cm,top=2.7cm]{geometry}
%\parindent0pt 
\newcommand{\doublehat}[1]{\skew{5.5}\widehat{\widehat{#1}}}
\newcommand{\doublehattwo}[1]{\widehat{\widehat{#1}}}



% General

\newcommand{\reals}{\mathbb{R}}
\newcommand{\integers}{\mathbb{Z}}
\newcommand{\naturals}{\mathbb{N}}

\newcommand{\pr}{\mathbb{P}}        % probability
\newcommand{\ex}{\mathbb{E}}        % expectation
\newcommand{\var}{\textnormal{Var}} % variance
\newcommand{\cov}{\textnormal{Cov}} % covariance

\newcommand{\law}{\mathcal{L}} % law of X
\newcommand{\normal}{N}        % normal distribution 

\newcommand{\argmax}{\textnormal{argmax}}
\newcommand{\argmin}{\textnormal{argmin}}

\newcommand{\ind}{\boldsymbol{1}} % indicator function
\renewcommand{\ker}{W} % kernel function

\newcommand{\X}{X}
\newcommand{\pairs}{\mathcal{S}}
\newcommand{\countries}{\mathcal{C}}
\newcommand{\intervals}{\mathcal{F}}
\newcommand{\indexset}{\mathcal{M}}

% Convergence

\newcommand{\convd}{\stackrel{d}{\longrightarrow}}              % convergence in distribution
\newcommand{\convp}{\stackrel{P}{\longrightarrow}}              % convergence in probability
\newcommand{\convas}{\stackrel{\textrm{a.s.}}{\longrightarrow}} % convergence almost surely
\newcommand{\convw}{\rightsquigarrow}                           % weak convergence


% Theorem-like declarations

\theoremstyle{plain}

\newtheorem{theorem}{Theorem}[section]
\newtheorem{prop}{Proposition}[section]
\newtheorem{corollary}{Corollary}[section]
\newtheorem{lemma}{Lemma}[section]
\newtheorem{definition}{Definition}[section]
\newtheorem{remark}{Remark}[section]
\newtheorem{algo}{Algorithm}
\newtheorem{theoremA}{Theorem}[section]
\newtheorem{propA}{Proposition}[section]
\newtheorem{corollaryA}{Corollary}[section]
\newtheorem{lemmaA}{Lemma}[section]
\renewcommand{\thetheoremA}{A.\arabic{theoremA}}
\renewcommand{\thepropA}{A.\arabic{propA}}
\renewcommand{\thecorollaryA}{A.\arabic{corollaryA}}
\renewcommand{\thelemmaA}{A.\arabic{lemmaA}}
\newtheorem{lemmaS}{Lemma}[section]
\renewcommand{\thelemmaS}{S.\arabic{lemmaS}}


% Theorem numbering to the left

\makeatletter
\newcommand{\lefteqno}{\let\veqno\@@leqno}
\makeatother


% Heading

\newcommand{\heading}[3]
{  \setcounter{page}{1}
   \begin{center}

   %\phantom{Distance to upper boundary}
   %\vspace{0.5cm}

   {\LARGE \textbf{#1}}
   \vspace{0.25cm}

   {\LARGE \textbf{#2}}
   \vspace{0.25cm}

   {\LARGE \textbf{#3}}
   \end{center}
}

\newcommand{\headingsupplement}[4]
{  \setcounter{page}{1}
   \begin{center}

   %\phantom{Distance to upper boundary}
   %\vspace{0.5cm}

   {\LARGE \textbf{#1}}
   \vspace{0.25cm}

   {\LARGE \textbf{#2}}
   \vspace{0.25cm}

   {\LARGE \textbf{#3}}
   \vspace{0.25cm}

   {\LARGE \textbf{#4}}
   \end{center}
}


% Authors

\newcommand{\authors}[4]
{  %\parindent0pt
   \begin{center}
      \phantom{-----------------}
      \begin{minipage}[c][1.5cm][c]{5.5cm}
      \begin{center} 
      {\large #1}  
      \vspace{0.1cm}      

      #2 
      \end{center}
      \end{minipage}
      \begin{minipage}[c][1.5cm][c]{5.5cm}
      \begin{center} 
      {\large #3} 
      \vspace{0.1cm}

      #4 \phantom{-}
      \end{center}
      \end{minipage}
      \phantom{---------}
   \end{center}
}

\newcommand{\authorssupplement}[4]
{  %\parindent0pt
   \begin{center}
      \phantom{-----------------}
      \begin{minipage}[c][1.5cm][c]{5.5cm}
      \begin{center} 
      {\large #1} \\[0.01cm]            
      #2 
      \end{center}
      \end{minipage}
      \begin{minipage}[c][1.5cm][c]{5.5cm}
      \begin{center} 
      {\large #3} \\[0.01cm]      
      #4 
      \end{center}
      \end{minipage}
      \phantom{---------}
   \end{center}
}


% Version

\newcommand{\version}[1]
{  \begin{center}
   {\large #1}
   \end{center}   
} 










\begin{document}



\heading{Multiscale Testing for Equality}{of Nonparametric Trend Curves}

\vspace{-0.5cm}

\authors{Marina Khismatullina\renewcommand{\thefootnote}{1}\footnotemark[1]}{Erasmus University Rotterdam}{Michael Vogt\renewcommand{\thefootnote}{2}\footnotemark[2]}{Ulm University} 
\footnotetext[1]{Corresponding author. Address: Erasmus School of Economics, Erasmus University Rotterdam, 3062 PA Rotterdam, Netherlands. Email: \texttt{khismatullina@ese.eur.nl}.}
\renewcommand{\thefootnote}{2}
\footnotetext[2]{Address: Institute of Statistics, Department of Mathematics and Economics, Ulm University, 89081 Ulm, Germany. Email: \texttt{m.vogt@uni-ulm.de}.}
\renewcommand{\thefootnote}{\arabic{footnote}}
\setcounter{footnote}{0}

\vspace{-1cm}



\renewcommand{\abstractname}{}
\begin{abstract}
{\noindent We develop multiscale methods to test qualitative hypotheses about nonparametric time trends in the presence of covariates. In many applications, practitioners are interested whether the observed time series all have the same time trend. Moreover, when some of the trends are different, it may be useful to know exactly which of the time trends are different. In addition, when two trends are not the same, it may also be relevant to know in which time regions they differ from each other. We design multiscale tests to formally approach these questions. We derive asymptotic theory for the proposed tests and show that the proposed test has asymptotic power of one against a certain class of local alternatives.}
\end{abstract}

\vspace{-0.1cm}

\enlargethispage{0.25cm}
\renewcommand{\baselinestretch}{1.2}\normalsize

\textbf{Key words:} Multiscale statistics; nonparametric regression; time series errors; shape constraints; strong approximations; anti-concentration bounds.

\textbf{AMS 2010 subject classifications:} 62E20; 62G10; 62G20; 62M10. 

\vspace{-0.25cm}

\numberwithin{equation}{section}
\allowdisplaybreaks[1]

%\section{State of the art and preliminary work}\label{sec-intro}


The comparison of nonparametric curves is a classic topic in econometrics and statistics. Depending on the specific application, the curves of interest are densities, distribution functions, time trends or regression curves. The problem of testing for equality of densities has been studied in \cite{Mammen1992}, \cite{Anderson1994} and \cite{Li2009} among others. Tests for equality of distribution functions can be found for example in \cite{Kiefer1959}, \cite{Anderson1962} and \cite{Finner2018}. Tests for equality of trend or regression curves have been developed in \cite{HaerdleMarron1990}, \cite{Hall1990}, \cite{Delgado1993}, \cite{DegrasWu2012}, \cite{Zhang2012} and \cite{Hidalgo2014} among many others. In the proposed project, we focus on the comparison of nonparametric trend curves.


The statistical problem of comparing trends has a wide range of applications in economics, finance and other fields such as climatology and biology. In economics, one may wish is to compare trends in real gross domestic product (GDP) across different countries \citep[cp.][]{Grier1989}. Another example concerns the dynamics of long-term interest rates. To better understand these dynamics, researchers aim to compare the yields of US Treasury bills at different maturities over time \citep[cp.][]{Park2009}. In finance, it is of interest to compare the volatility trends of different stocks \citep[cp.][]{Nyblom2000}. Finally, in climatology, researchers are interested in comparing the trending behaviour of temperature time series across different spatial locations \citep[cp.][]{KarolyWu2005}. 


Classically, time trends are modelled stochastically in econometrics; see e.g.\ \cite{Stock1988}. Recently, however, there has been a growing interest in econometric models with deterministic time trends; see \cite{Cai2007}, \cite{Atak2011}, \cite{Robinson2012} and \cite{ChenGaoLi2012} among others. Non- and semiparametric trend modelling has attracted particular interest in a panel data context. \cite{LiChenGao2010}, \cite{Atak2011}, \cite{Robinson2012} and \cite{ChenGaoLi2012} considered panel models where the observed time series have a common time trend. In many applications, however, the assumption of a common time trend is quite harsh. In particular when the number of observed time series is large, it is quite natural to suppose that the time trend may differ across time series. More flexible panel settings with heterogeneous trends have been studied, for example, in \cite{Zhang2012} and \cite{Hidalgo2014}. 

 
In what follows, we consider a general panel framework with heterogeneous trends which is useful for a number of economic and financial applications: Suppose we observe a panel of $n$ time series $\mathcal{Z}_i = \{ (Y_{it},\mathbf{X}_{it}): 1 \le t \le T \}$ for $1 \le i \le n$, where $Y_{it}$ are real-valued random variables and $\mathbf{X}_{it} = (X_{it,1},\ldots,X_{it,d})^\top$ are $d$-dimensional random vectors. Each time series $\mathcal{Z}_i$ is modelled by the equation
\begin{equation}\label{model}
Y_{it} = m_i \Big( \frac{t}{T} \Big) + \bm{\beta}_i^\top \mathbf{X}_{it} + \alpha_i + \varepsilon_{it}
\end{equation}
for $1 \le t \le T$, where $m_i: [0,1] \rightarrow \mathbb{R}$ is a nonparametric (deterministic) trend function, $\mathbf{X}_{it}$ is a vector of regressors or controls and $\bm{\beta}_i$ is the corresponding parameter vector. Moreover, $\alpha_i$ are so-called fixed effect error terms and $\varepsilon_{it}$ are standard regression errors with $\ex[\varepsilon_{it}|\mathbf{X}_{it}] = 0$ for all $t$. Model \eqref{model} nests a number of panel models which have recently been considered in the literature. Special cases of model \eqref{model} with a nonparametric trend specification are for example considered in \cite{Atak2011}, \cite{Zhang2012} and \cite{Hidalgo2014}. Versions of model \eqref{model} with a parametric trend are studied in \cite{Vogelsang2005}, \cite{Sun2011} and \cite{Xu2012} among others.

Within the general framework of model \eqref{model}, we can formulate a number of interesting statistical questions concerning the set of trend functions $\{ m_i:  1 \le i \le n \}$. 

\vspace{10pt}


\noindent \textbf{(a) Testing for equality of nonparametric trend curves} 
\vspace{10pt} 

 
\noindent In many application contexts, an important question is whether the time trends $m_i$ in model \eqref{model} are all the same. Put differently, the question is whether the observed time series
have a common trend. This question can formally be addressed by a statistical test of the null hypothesis 
\[ H_0: \text{There exists a function } m: [0,1] \rightarrow \mathbb{R} \text{ such that } m_i = m  \text{ for all } 1 \le i \le n. \]
A closely related question is whether all time trends have the same parametric form. To formulate the corresponding null hypothesis, let $m(\theta,\cdot): [0,1] \rightarrow \mathbb{R}$ be a function which is known up to the finite-dimensional parameter $\theta \in \Theta$, where $\Theta$ denotes the parameter space. The null hypothesis of interest now reads as follows:  
\[ H_{0,\text{para}}: \text{ There exists } \theta \in \Theta \text{ such that } m_i(\cdot) = m(\theta,\cdot) \text{ for all } 1 \le i \le n. \]  
If $m(\theta,w) = a + b w$ with $\theta = (a,b)$, for example, then $H_0$ is the hypothesis that all trends $m_i$ are linear with the same intercept $a$ and slope $b$. A somewhat simpler but yet important hypothesis is given by 
\[ H_{0,\text{const}}: m_i \equiv 0 \text{ for all } 1 \le i \le n. \]
Under this hypothesis, there is no time trend at all in the observed time series. Put differently, all the time trends $m_i$ are constant. (Note that under the normalization constraint $\int_0^1 m_i(w) dw = 0$, $m_i$ must be equal to zero if it is a constant function.) A major goal of our project is to develop new tests for the hypotheses $H_0$, $H_{0,\text{para}}$ and $H_{0,\text{const}}$ in model \eqref{model}. In order to keep the exposition as clear as possible, we focus attention on the hypothesis $H_0$ in what follows. Tests of $H_{0,\text{para}}$, $H_{0,\text{const}}$ and related hypotheses have for example been studied in \cite{Lyubchich2016} and \cite{ChenWu2018}. 


In recent years, a number of different approaches have been developed to test the hypothesis $H_0$. \cite{DegrasWu2012} consider the problem of testing $H_0$ within the model framework
\begin{equation}\label{model-degras}
Y_{it} = m_i \Big( \frac{t}{T} \Big) + \alpha_i + \varepsilon_{it} \qquad (1 \le t \le T, \, 1 \le i \le n), 
\end{equation}
where $\mathbb{E}[\varepsilon_{it}] = 0$ for all $i$ and $t$ and the terms $\alpha_i$ are assumed to be deterministic. Obviously, \eqref{model-degras} is a special case of \eqref{model} which does not include additional regressors. \cite{DegrasWu2012} construct an $L_2$-type statistic to test $H_0$. The statistic is based on the difference between estimators of the trend with and without imposing $H_0$. Let $\hat{m}_{i,h}$ be the estimator of $m_i$ and $\hat{m}_h$ the estimator of the common trend $m$ under $H_0$, where $h$ denotes the bandwidth parameter. With these estimators, the authors define the statistic
\begin{equation}\label{stat-degras}
\Delta_{n,T} = \sum_{i=1}^n \int_0^1 \big(\hat{m}_{i,h}(u) - \hat{m}_h(u)\big)^2 du, 
\end{equation} 
which measures the $L_2$-distance between $\hat{m}_{i, h}$ and $\hat{m}_h$. In the theoretical part of their paper, they derive the limit distribution of $\Delta_{n,T}$. \cite{ChenWu2018} develop theory for test statistics closely related to those from \cite{DegrasWu2012}, but under more general conditions on the error terms. 


\cite{Zhang2012} investigate the problem of testing the hypothesis $H_0$ in a slightly restricted version of model \eqref{model}, where $\beta_i = \beta$ for all $i$. The regression coefficients $\beta_i$ are thus assumed to be homogeneous in their setting. They construct a residual-based test statistic as follows: First, they obtain profile least squares estimators $\hat{\beta}$ and $\hat{m}_h(t/T)$ of the parameter vector $\beta$ and the common trend $m$ under $H_0$, where $h$ denotes the bandwidth. With these estimators, they compute the residuals $\hat{u}_{it} = Y_{it} - \hat{\beta}^T X_{it} - \hat{m}_h(t/T)$. These residuals are shown to have the form $\hat{u}_{it} = \Delta_i(t/T) + \eta_{it}$, where $\Delta_i$ is a deterministic function with the property that $\Delta_i \equiv 0$ under $H_0$ and $\eta_{it}$ denotes the error term. Testing $H_0$ is thus equivalent to testing the hypothesis $H_0^\prime: \Delta_i \equiv 0$ for all $1 \le i \le n$. The authors construct a test statistic for the hypothesis $H_0^\prime$ on the basis of nonparametric kernel estimators of the functions $\Delta_i$ and derive its limit distribution.  


The tests of \cite{Zhang2012}, \cite{DegrasWu2012} and \cite{ChenWu2018} are based on nonparametric estimators of the trend functions $m_i$ that depend on one or several bandwidth parameters. Unfortunately, it is far from clear how to choose these bandwidths in an appropriate way. This is a general problem concerning essentially all tests based on nonparametric curve estimators. There are of course many theoretical results on optimal bandwidth choice for estimation purposes. However, the optimal bandwidth for curve estimation is usually not optimal for testing. Optimal bandwidth choice for tests is indeed an open problem, and only little theory for simple cases is available \citep[cp.][]{GaoGijbels2008}. Since tests based on nonparametric curve estimators are commonly quite sensitive to the choice of bandwidth and theory for optimal bandwidth selection is not available, it appears preferable to work with bandwidth-free tests. 


A classical way to obtain a bandwidth-free test of the hypothesis $H_0$ is to use CUSUM-type statistics which are based on partial sum processes. This approach is taken in \cite{Hidalgo2014}. A more modern approach to obtain a bandwidth-free test is to employ multiscale methods. These methods avoid the need to choose a bandwidth by considering a large collection of bandwidths simultaneously. More specifically, the basic idea is as follows: Let $S_h$ be a test statistic for the null hypothesis of interest, which depends on the bandwidth $h$. Rather than considering only a single statistic $S_h$ for a specific bandwidth $h$, a multiscale approach simultaneously considers a whole family of statistics $\{S_h: h \in \mathcal{H} \}$, where $\mathcal{H}$ is a set of bandwidth values. The multiscale test then proceeds as follows: For each bandwidth or scale $h$, one checks whether $S_h > q_h(\alpha)$, where $q_h(\alpha)$ is a bandwidth-dependent critical value (for given significance level $\alpha$). The multiscale test rejects if $S_h > q_h(\alpha)$ for at least one scale $h$. The main theoretical difficulty in this approach is of course to derive appropriate critical values $q_h(\alpha)$. Specifically, the critical values $q_h(\alpha)$ need to be determined such that the multiscale test has the correct (asymptotic) level, that is, such that $\pr (S_h > q_h(\alpha) \text{ for some } h \in \mathcal{H} ) = (1-\alpha) + o(1)$. 


Multiscale methods have been developed for a variety of different test problems in recent years. \cite{ChaudhuriMarron1999, ChaudhuriMarron2000} introduced the so-called SiZer method which has been extended in various directions; see for example \cite{HannigMarron2006} and \cite{Rondonotti2007}. \cite{HorowitzSpokoiny2001} proposed a multiscale test for the parametric form of a regression function. \cite{DuembgenSpokoiny2001} constructed a multiscale approach which works with additively corrected supremum statistics. This general approach has been very influential in recent years and has been further developed in numerous ways; see for example \cite{Duembgen2002}, \cite{Rohde2008} and \cite{ProkschWernerMunk2018} for multiscale methods in the regression context and \cite{DuembgenWalther2008}, \cite{RufibachWalther2010}, \cite{SchmidtHieber2013} and \cite{EckleBissantzDette2017} for methods in the context of density estimation. Importantly, all of these studies are restricted to the case of independent data. It turns out that it is highly non-trivial to extend the multiscale approach of \cite{DuembgenSpokoiny2001} to the case of dependent data. A first step into this direction has recently been made in \cite{KhismatullinaVogt2018}. They developed multiscale methods to test for local increases/decreases of the nonparametric trend function $m$ in the univariate time series model $Y_t = m(t/T) + \varepsilon_t$.  


To the best of our knowledge, multiscale tests of the hypotheses $H_0$, $H_{0,\text{para}}$ and $H_{0,\text{const}}$ in model \eqref{model} are not available in the literature. The only exception is \cite{Park2009} who developed SiZer methods for the comparison of nonparametric trend curves in a strongly simplified version of model \eqref{model}. Their analysis, however, is mainly methodological and not fully backed up by theory. Indeed, theory has only been derived for the special case $n=2$, that is, for the case that only two time series are observed. 
\vspace{10pt}


\noindent \textbf{(b) Clustering of nonparametric trend curves} 
\vspace{10pt} 


\noindent Consider the situation that the null hypothesis $H_0: m_1 = \ldots = m_n$ is violated in the general panel data model \eqref{model}. Even though some of the trend functions $m_i$ are different in this case, there may still be groups of time series with the same time trend. Formally, a group stucture can be defined as follows within the framework of model \eqref{model}: There exist sets or groups of time series $G_1,\ldots,G_{K_0}$ with $\{1,\ldots,n\} = \dot\bigcup_{k=1}^{K_0} G_k$ such that for each $1 \le k \le K_0$, 
\begin{equation}\label{model-groups}
m_i = m_j \quad \text{for all } i,j \in G_k. 
\end{equation}
According to \eqref{model-groups}, the time series of a given group $G_k$ all have the same time trend. In many applications, it is very natural to suppose that there is such a group structure in the data. An interesting statistical problem which we aim to investigate in our project is how to estimate the unknown groups $G_1,\ldots,G_{K_0}$ and their unknown number $K_0$ from the data. 


Several approaches to this problem have been proposed in the context of models closely related to \eqref{model}. \cite{DegrasWu2012} used a repeated testing procedure based on $L_2$-type test statistics of the form \eqref{stat-degras} in order to estimate the unknown group structure in model \eqref{model-degras}. \cite{Zhang2013} developed a clustering method within the same model framework which makes use of an extended Bayesian information criterion. \cite{VogtLinton2017} constructed a thresholding method to estimate the unknown group structure in the panel model $Y_{it} = m_i(X_{it}) + u_{it}$, where $X_{it}$ are random regressors and $u_{it}$ are general error terms that may include fixed effects. Their approach can also be adapted to the case of fixed regressors $X_{it} = t/T$.  As an alternative to a group structure, factor-type structures have been imposed on the trend and regression functions in panel models. Such factor-type structures are studied in \cite{Kneip2012}, \cite{LintonVogt2015} and \cite{BonevaLintonVogt2016} among others. 


The problem of estimating the unknown groups $G_1,\ldots,G_{K_0}$ and their unknown number $K_0$ in model \eqref{model} has close connections to functional data clustering. There, the aim is to cluster smooth random curves that are functions of (rescaled) time and that are observed with or without noise. A number of different clustering approaches have been proposed in the context of functional data models; see for example \cite{Abraham2003}, \cite{Tarpey2003} and \cite{Tarpey2007} for procedures based on $k$-means clustering, \cite{James2003} and \cite{Chiou2007} for model-based clustering approaches and \cite{Jacques2014} for a recent survey. 


The problem of finding the unknown group structure in model \eqref{model} is also closely related to a developing literature in econometrics which aims to identify unknown group structures in parametric panel regression models. In its simplest form, the panel regression model under consideration is given by the equation $Y_{it} = \beta_i^\top X_{it} + u_{it}$ for $1 \le t \le T$ and $1 \le i \le n$, where the coefficient vectors $\beta_i$ are allowed to vary across individuals $i$ and the error terms $u_{it}$ may include fixed effects. Similar to the trend functions in model \eqref{model}, the coefficients $\beta_i$ are assumed to belong to a number of groups: there are $K_0$ groups $G_1,\ldots,G_{K_0}$ such that $\beta_i = \beta_j$ for all $i,j \in G_k$ and all $1\le k \le K_0$. The problem of estimating the unknown groups and their unknown number has been studied in different versions of this modelling framework; cp.\ \cite{Su2016}, \cite{Su2018} and \cite{Wang2018} among others. \cite{Bonhomme2015} considered a related model where the group structure is not imposed on the regression coefficients but rather on some unobserved time-varying fixed effect components of the panel model. 


Virtually all the proposed procedures to cluster nonparametric curves in panel and functional data models related to \eqref{model} depend on a number of bandwidth or smoothing parameters required to estimate the nonparametric functions $m_i$. In general, nonparametric curve estimators are strongly affected by the chosen bandwidth parameters. A clustering procedure which is based on such estimators can be expected to be strongly influenced by the choice of bandwidths as well. Moreover, as in the context of statistical testing, there is no theory available on how to pick the bandwidths optimally for the clustering problem. Hence, as in the context of testing, it is desirable to construct a clustering procedure which is free of bandwidth or smoothing parameters that need to be selected. 


One way to obtain a clustering method which does not require to select any bandwidth parameter is to use multiscale methods. This approach has recently been taken in \cite{VogtLinton2018}. They develop a clustering approach in the context of the panel model $Y_{it} = m_i(X_{it}) + u_{it}$, where $X_{it}$ are random regressors and $u_{it}$ are general error terms that may include fixed effects. Imposing the same group structure as in \eqref{model-groups} on their  model, they construct estimators of the unknown groups and their unknown number as follows: In a first step, they develop bandwidth-free multiscale statistics $\hat{d}_{ij}$ which measure the distance between pairs of functions $m_i$ and $m_j$. To construct them, they make use of the multiscale testing methods described in part (a) of this section. In a second step, the statistics $\hat{d}_{ij}$ are employed as dissimilarity measures in a hierarchical clustering algorithm. 


%\setlength{\parindent}{10ex} 

\section{Introduction}\label{sec:intro}

Comparison of several regression curves is a classical topic in econometrics and statistics. In many cases of practical interest, the functional forms of the objective regression curves are unknown, hence, the parametric approach is not applicable. In this paper, we propose a novel approach that addresses this particular problem in a nonparametric context. Specifically, we present a new testing procedure for detecting differences between the nonparametric trends curves. 

In what follows, we consider a general panel framework with heterogeneous trends. Suppose we observe a panel of $n$ time series $\mathcal{T}_i = \{ (Y_{it},\X_{it}): 1 \le t \le T \}$ for\linebreak $1 \le i \le n$, where $Y_{it}$ are real-valued random variables and $\X_{it} = (X_{it,1},\ldots,X_{it,d})^\top$ are $d$-dimensional random vectors. Each time series $\mathcal{W}_i$ is modelled by the equation
\begin{equation}\label{eq:model}
Y_{it} = m_i \Big( \frac{t}{T} \Big) + \bfbeta_i^\top \X_{it} + \alpha_i + \varepsilon_{it}
\end{equation}
for $1 \le t \le T$, where $\bfbeta_i$ is a $d \times 1$ vector of unknown parameters, $\X_{it}$ is a $d\times 1$ vector of individual covariates or controls, $m_i$ is an unknown nonparametric (deterministic) trend function defined on $[0,1]$, $\alpha_i$ are so-called fixed effect error terms and \linebreak $\mathcal{E}_i = \{ \varepsilon_{it}: 1 \le t \le T \}$ is a zero-mean stationary error process. 


An important question in many applications is whether the observed time series have a common trend. In other words, the researchers would like to know if $m_i$ are the same for all $i$. Moreover, when there is evidence that this is not the case, one of the major related statistical problems is to determine which of the trends are different. % and whether we can group the time series with the similar trends together.
In addition, when two trends $m_i$ and $m_j$ are not the same, it may also be relevant to know in which time regions they differ from each other. In this paper, we introduce new statistical methods to approach these questions. In particular, we develop a test of the hypothesis that all time trends in model \eqref{eq:model} are the same. In this setting, the null hypothesis is formulated as 
\begin{align}\label{eq:null}
H_0: m_1 = m_2 = \ldots = m_n,
\end{align}
whereas the alternative hypothesis is 
$$H_1: \text{ there exists } x\in [0, 1] \text{ such that } m_i (x) \neq m_j(x) \text{ for some } 1\leq i < j \leq n.$$

The method that we propose does not only allow to test whether the null hypothesis is violated. It also allows to detect, with a given statistical confidence, which time trends are different and in which time regions they differ. More specifically, for any given interval $[u-h,u+h] \subseteq [0,1]$, consider the hypothesis
\[ H_0^{[i,j]}(u,h): m_i(w) = m_j(w) \text{ for all } w \in [u-h,u+h]. \] 
Here, we can regard $h$ as a bandwidth, a common tuning parameter in nonparametric estimation. The given interval $\interval_{(u, h)} = [u-h,u+h] \subseteq [0,1]$ is then fully characterized by $u$, its center (a location parameter), and $h$, the bandwidth. In order to determine the regions where the time trends are different, we consider a broad range of pairs $(u, h)$ with the property that they fully cover the unit interval $[0, 1]$. Formally, let \linebreak $\grid := \{(u, h): \interval_{(u, h)} = [u-h, u+h] \subseteq [0,1]\}$ be a grid of location-bandwidth points such that 
\begin{align*}
\bigcup_{(u, h) \in \grid}  \interval_{(u, h)} = [0,1].
\end{align*}
We then reformulate our null hypothesis \eqref{eq:null} as
\begin{align*}
H_0: \ & \text{The hypotheses } H_0^{[i,j]}(u,h) \text{ hold true for all intervals }  \interval_{(u, h)}, (u, h) \in \grid, \\ & \text{ and for all } 1 \le i < j \le n. 
\end{align*} 
$H_0^{[i,j]}(u,h)$ can thus be viewed as a local null hypothesis that characterizes the behavior of two trend functions only locally, whereas $H_0$ specified in \eqref{eq:null} is the global null hypothesis that is concerned with the comparison of all of the trends on the whole unit interval.

In this paper, we introduce a method that allows us to test the hypotheses $H_0^{[i,j]}(u,h)$  simultaneously for all pairs $(i, j)$ and for all intervals $\interval_{(u, h)}$ under consideration. Specifically, we develop a multiscale test for the model \eqref{eq:model}. The underlying idea of any multiscale test is to consider a number of test statistics (each corresponding to a different set of values of some tuning parameters) all at once rather than to perform a separate test for each single test statistics. In our case, this means testing many local null hypotheses $H_0^{[i,j]}(u,h)$ simultaneously which leads to a well-known multiple testing problem. Our method accounts for this problem by using appropriate critical values that depend on the scale of the problem, i.e. on the number of hypotheses tested simultaneously and the relationship between them. In the paper, we show that the suggested procedure for obtaining critical values leads to good theoretical properties of the proposed test: it has the correct (asymptotic) level and an (asymptotic) power of one against a certain class of local alternatives. %Moreover, we prove that when testing at a given level $\alpha$, the probability of rejecting even one local null hypothesis is no more than this level $\alpha$.

Trend comparison is a common statistical problem that arises in various contexts. For example, in economics the researchers compare trends in real gross domestic product across several countries \citep[][]{Grier1989}, in yield over time of US Treasury bills at different maturities \citep[][]{Park2009}, or the evolution of long-term interest rates in a number of countries \citep[][]{Christiansen1997}. In finance, comparison and subsequent classification of the trends of market fragmentation can be used to assess the market quality in the European stock market (\citeauthor{VogtLinton2017}, \citeyear{VogtLinton2017}, \citeyear{VogtLinton2020}). In climatology, the temperature time series in different geographical areas are investigated in the context of the regional and global warming trends \citep[][]{KarolyWu2005}. Finally, in industry, mobile phone providers are interested in finding the differences between the cell phone download activity in various locations \citep[][]{DegrasWu2012}.


In the statistical literature, the problem of testing whether the observed time series all have the same trend  has been widely studied, and tests for equality of trends or regression curves have been developed in \cite{HaerdleMarron1990}, \cite{Hall1990}, \linebreak \cite{Delgado1993} and \cite{DegrasWu2012} among many others. Versions of model \eqref{eq:model} with a parametric trend are considered in \cite{Vogelsang2005}, \cite{Sun2011} and \cite{Xu2012} among others. In the nonparametric context, \cite{LiChenGao2010}, \cite{Atak2011}, \cite{Robinson2012} and \cite{ChenGaoLi2012} studied panel models under the assumption that the observed time series have a common time trend. However, in many applications the restriction of including a common time trend in the model is questionable at best. For instance, when we observe a large number of time series it is reasonable to expect that at least some of the trends are different from the others. Consequently, it often makes sense to relax the assumption of a common trend, which leads to more flexible panel settings with heterogeneous trends. Such models have been studied, for example, in \cite{DegrasWu2012},  \cite{Zhang2012} and \cite{Hidalgo2014}. \cite{DegrasWu2012} consider the problem of testing $H_0$ in a model that is a special case of \eqref{eq:model} and does not include additional regressors. \cite{ChenWu2018} develop theory for a very similar model framework but under more general conditions on the error terms. \linebreak \cite{Zhang2012} investigate the problem of testing the hypothesis $H_0$ in a slightly restricted version of model \eqref{eq:model}, where $\bfbeta_i = \bfbeta$ for all $i$. All of these tests have an important drawback: they involve classical nonparametric estimation of the trend functions that depends on one or several bandwidth parameters, which imposes a certain limit on the applicability of such tests since in most cases it is far from clear how to choose bandwidth parameters in an appropriate way. Contrary to the aforementioned methods, our multiscale testing procedure allows us to consider a large collection of bandwidths simultaneously avoiding the problem of choosing only one bandwidth altogether.

Recently, \cite{KhismatullinaVogt2021} proposed a new inference method that allows researchers to detect differences between epidemic time trends in the context of the COVID-19 pandemic. In their paper, the authors present a statistically rigorous procedure that, similarly to ours, not only allows to compare trends across different countries, but to pinpoint the time intervals where the differences occur as well. Moreover, they also circumvent the need to pick a bandwidth parameter by using a multiscale testing approach. However, the model in \cite{KhismatullinaVogt2021} is only a special case of the model \eqref{eq:model} which includes neither the covariates $\X_{it}$, nor the fixed effects $\alpha_i$. Furthermore, the authors place major restriction on the error terms: in their model, $\varepsilon_{it}$ are independent across $t$. In contrast, our model \eqref{eq:model} can be regarded as a generalized version of theirs that allows for a wider range of economic and financial applications.

%This is a general problem concerning essentially all tests based on nonparametric curve estimators. There are of course many theoretical results on optimal bandwidth choice for estimation purposes. However, the optimal bandwidth for curve estimation is usually not optimal for testing. Optimal bandwidth choice for tests is indeed an open problem, and only little theory for simple cases is available \citep[][]{GaoGijbels2008}. Since tests based on nonparametric curve estimators are commonly quite sensitive to the choice of bandwidth and theory for optimal bandwidth selection is not available, it appears preferable to work with bandwidth-free tests. A classical way to obtain a bandwidth-free test of the hypothesis $H_0$ is to use CUSUM-type statistics which are based on partial sum processes. This approach is taken in \cite{Hidalgo2014}. A more modern approach to obtain a bandwidth-free test is to employ multiscale methods.

%More specifically, the basic idea is as follows: Let $S_h$ be a test statistic for the null hypothesis of interest, which depends on the bandwidth $h$. Rather than considering only a single statistic $S_h$ for a specific bandwidth $h$, a multiscale approach simultaneously considers a whole family of statistics $\{S_h: h \in \mathcal{H} \}$, where $\mathcal{H}$ is a set of bandwidth values. The multiscale test then proceeds as follows: For each bandwidth or scale $h$, one checks whether $S_h > q_h(\alpha)$, where $q_h(\alpha)$ is a bandwidth-dependent critical value (for given significance level $\alpha$). The multiscale test rejects if $S_h > q_h(\alpha)$ for at least one scale $h$. The main theoretical difficulty in this approach is of course to derive appropriate critical values $q_h(\alpha)$. Specifically, the critical values $q_h(\alpha)$ need to be determined such that the multiscale test has the correct (asymptotic) level, that is, such that $\pr (S_h > q_h(\alpha) \text{ for some } h \in \mathcal{H} ) = (1-\alpha) + o(1)$. 


%Multiscale methods have been developed for a variety of different test problems in recent years. \cite{ChaudhuriMarron1999, ChaudhuriMarron2000} introduced the so-called SiZer method which has been extended in various directions; see for example \cite{HannigMarron2006} and \cite{Rondonotti2007}. \cite{HorowitzSpokoiny2001} proposed a multiscale test for the parametric form of a regression function. \cite{DuembgenSpokoiny2001} constructed a multiscale approach which works with additively corrected supremum statistics. This general approach has been very influential in recent years and has been further developed in numerous ways; see for example \cite{Duembgen2002}, \cite{Rohde2008} and \cite{ProkschWernerMunk2018} for multiscale methods in the regression context and \cite{DuembgenWalther2008}, \cite{RufibachWalther2010}, \cite{SchmidtHieber2013} and \cite{EckleBissantzDette2017} for methods in the context of density estimation. Importantly, all of these studies are restricted to the case of independent data. It turns out that it is highly non-trivial to extend the multiscale approach of \cite{DuembgenSpokoiny2001} to the case of dependent data. A first step into this direction has recently been made in \cite{KhismatullinaVogt2020}. They developed multiscale methods to test for local increases/decreases of the nonparametric trend function $m$ in the univariate time series model $Y_t = m(t/T) + \varepsilon_t$.  


To sum up, the main theoretical contribution of the current paper is the multiscale testing method that allows to make simultaneous confidence statements about which of the time trends are distinct and the regions where they differ. We believe that currently there are no equivalent statistical methods. Even though tests for equality of the trends have been developed already for a while, most existing procedures allow only to test whether the trend curves are all the same or not, but they almost never allow to infer which curves are different and where. To the best of our knowledge, the only two exceptions are \cite{KhismatullinaVogt2021}, whose contribution is briefly discussed above, and \cite{Park2009} who developed SiZer methods for the comparison of nonparametric trend curves in a significantly simplified version of the model \eqref{eq:model}. In addition to restricted model, \cite{Park2009} derive theoretical results for their analysis only for the special case of observing only two time series, whereas in other cases, the algorithm is provided without detailed proof.

The structure of the paper is as follows. Section \ref{sec:model} introduces the model setting and the necessary technical assumptions that are required for the theory. The multiscale test is developed step by step in Section \ref{sec:test}. The main theoretical results are presented in Section \ref{sec:theo}. To keep the discussion as clear as possible, we include in the main text of the paper only the essential parts of the theoretical arguments, and the technical details and extended proofs are deferred to the Appendix. \textcolor{red}{Section \ref{sec:app} presents two empirical application to illustrate the usage of our method: testing for common trend in the GDP growth data and cross-country trend comparison of the housing prices.} Section \ref{sec:conclusion} concludes.



\section{The model framework}\label{sec:model}


\subsection{Notation}


%Throughout the paper, we adopt the following notation. For a vector $\mathbf{v} = (v_1, \ldots, v_m)\in\reals^m$, we write $|\mathbf{v}|_q = \big(\sum_{i=1}^m v_i^q\big)^{1/q}$ to denote its $\ell_q$-norm and use the shorthand $|\mathbf{v}| = |\mathbf{v}|_2$ in the special case $q = 2$. For a random vector $\mathbf{V}$, we define its $\mathcal{L}^q$-norm by $\|\mathbf{V}\|_q = (\ex |\mathbf{V}|^q)^{1/q}$ and write $\|\mathbf{V}\| := \|\mathbf{V}\|_2$ in the case $q = 2$.


Throughout the paper, we adopt the following notation. For a vector $\mathbf{v} = (v_1, \ldots, v_m)\in\reals^m$, we write $\|\mathbf{v}\|_q = \big(\sum_{i=1}^m v_i^q\big)^{1/q}$ to denote its $\ell_q$-norm and use the shorthand $\|\mathbf{v}\| = \|\mathbf{v}\|_2$ in the special case $q = 2$. For a random variable $V$, we define its $\mathcal{L}^q$-norm by $\|V\|_q = (\ex |V|^q)^{1/q}$ and write $\|V\| := \|V\|_2$ in the case $q = 2$.


Let $\eta_t$ ($t \in \integers$) be independent and identically distributed ($\text{i.i.d.}$) random variables, write $\mathcal{F}_t  = (\ldots, \eta_{t-1}, \eta_t)$ and let $g: \reals^\infty \to \reals$ be a measurable function such that $g(\mathcal{F}_t) = g(\ldots, \eta_{t-1}, \eta_t)$ is a properly defined random variable. Following \cite{Wu2005}, we define the \textit{physical dependence measure} of the process $\{g(\mathcal{F}_t)\}_{t=-\infty}^\infty$ by
\begin{align}\label{eq:physical_dep}
\delta_q(g, t) = \| g(\mathcal{F}_t) - g(\mathcal{F}_t^\prime) \|_q,
\end{align}
where $\mathcal{F}_t^\prime  = (\ldots, \eta_{-1}, \eta^\prime_0, \eta_1, \ldots, \eta_t)$ is a coupled version of $\mathcal{F}_t$ with $\eta_0^\prime$ being an i.i.d.\ copy of $\eta_0$. Evidently, $\delta_q(g, t)$ measures the dependency of the random variable $g(\mathcal{F}_t)$ on the innovation term $\eta_0$. %, i.e., how replacing $\epsilon_0$ by an i.i.d. copy while keeping all other innovations in place affects the output $\mathbf{L}(\mathcal{F}_t)$.


\subsection{Model}\label{subsec:model_setting}


We observe a panel of $n$ time series $\mathcal{T}_i = \{(Y_{it}, \X_{it}): 1 \le t \le T \}$ of length $T$ for $1 \le i \le n$. Each time series $\mathcal{T}_i$ satisfies the model equation 
\begin{equation}\label{eq:model_full}
Y_{it} = \bfbeta^\top_i \X_{it} + m_i \Big( \frac{t}{T} \Big) + \alpha_i + \varepsilon_{it} 
\end{equation}
for $1 \le t \le T$, where $\bfbeta_i$ is a $d \times 1$ vector of unknown parameters, $\X_{it}$ is a $d\times 1$ vector of individual covariates, $m_i$ is an unknown nonparametric trend function defined on the unit interval $[0,1]$ with $\int_0^1 m_i(u) du = 0$ for all $i$, $\alpha_i$ is a (deterministic or random) intercept term and $\mathcal{E}_i = \{ \varepsilon_{it}: 1 \le t \le T \}$ is a zero-mean stationary error process. As common in nonparametric regression, the trend functions $m_i$ in model \eqref{eq:model_full} depend on rescaled time $t/T$ rather than on real time $t$; 
%Using rescaled time is equivalent to restricting the domain of the functions to the unit interval which in turn allows us to apply the usual asymptotic arguments. 
see e.g.\ \cite{Robinson1989}, \cite{Dahlhaus1997} and \cite{VogtLinton2014} for a discussion of rescaled time in nonparametric estimation. The condition $\int_0^1 m_i(u) du = 0$ is required for identification in the presence of the intercept terms $\alpha_i$. Without imposing this condition, one can freely shift the functions $m_i$ by any (positive or negative) constant $c_i$ while simultaneously subtracting this constant from $\alpha_i$:
\[ Y_{it} = [m_i(t/T) + c_i] + \bfbeta_i^\top \X_{it} + [\alpha_i - c_i] + \varepsilon_{it}. \]
The term $\alpha_i$ can be regarded as a fixed effect error term which captures unobserved characteristics of the time series $\mathcal{T}_i$ that remain constant over time. We allow the error terms $\alpha_i$ to be dependent across $i$ in an arbitrary way. Hence, by including them in \textcolor{red}{the} model equation \eqref{eq:model_full}, we allow the $n$ time series $\mathcal{T}_i$ in our panel to be correlated with each other. \textcolor{red}{Moreover, we allow for an arbitrary dependence of the individual covariates $\X_{it}$ across $i$. Whereas the variables $\alpha_i$ and $\X_{it}$ may be correlated across $i$, the error processes $\mathcal{E}_i$ are assumed to be independent across $i$.} Technical conditions regarding the model are discussed below. Throughout the paper we restrict attention to the case where the number of time series $n$ in model \eqref{eq:model_full} is fixed. Extending our theoretical results to the case where $n$ slowly grows with the sample size $T$ is a possible topic for further research.


\subsection{Assumptions}\label{subsec:model_assumptions}


The error processes $\mathcal{E}_i = \{ \varepsilon_{it}: 1 \le t \le T\}$ satisfy the following conditions. 
\begin{enumerate}[label=(C\arabic*),leftmargin=1.05cm]
\item \label{C-err1} 
The variables $\varepsilon_{it}$ allow for the representation $\varepsilon_{it} = g_i(\mathcal{F}_{it})$, where $\mathcal{F}_{it} = (\ldots,\eta_{it-2}, \linebreak \eta_{it-1},\eta_{it})$, the variables $\eta_{it}$ are i.i.d.\ across $t$, and $g_i: \reals^\infty \rightarrow \reals$ is a measurable function. 
It holds that $\ex[\varepsilon_{it}] = 0$ and $\| \varepsilon_{it} \|_q \le C < \infty$ for some $q > 4$ and a sufficiently large constant $C$. 
\item \label{C-err2} The processes $\mathcal{E}_i = \{ \varepsilon_{it}: 1 \le t \le T\}$ are independent across $i$.
\end{enumerate}
Assumption \ref{C-err1} implies that the error processes $\mathcal{E}_i$ are stationary and causal (in the sense that $\varepsilon_{it}$ does not depend on future innovations $\eta_{is}$ with $s>t$). The class of error processes that satisfy condition \ref{C-err1} is very large. It includes linear processes, nonlinear transformations thereof, as well as a large variety of nonlinear processes such as Markov chain models and nonlinear autoregressive models \citep[][]{Wu2016}. Following \cite{Wu2005}, we impose conditions on the dependence structure of the error processes $\mathcal{E}_i$ in terms of the physical dependence measure $\delta_q(g_i, t)$ defined in \eqref{eq:physical_dep}. In particular, we assume the following: 
\begin{enumerate}[label=(C\arabic*),leftmargin=1.05cm]
\setcounter{enumi}{2}
\item \label{C-err3} For each $i$, it holds that $\sum\nolimits_{s \ge t} \delta_q(g_i, s) = O ( t^{-\gamma} (\log t)^{-A})$ with $q$ from \ref{C-err1}, where $A > \frac{2}{3} (1/q + 1 + \gamma)$ and $\gamma = \{q^2 - 4 + (q-2) \sqrt{q^2 + 20q + 4}\} / 8q$. 
\end{enumerate}
For fixed $i$ and $t$, the expression $\sum\nolimits_{s \ge t} \delta_q(g_i, s)$ measures the cumulative effect of the innovation $\eta_0$ on the variables $\varepsilon_{it}, \varepsilon_{it+1},\ldots$ in terms of the $\mathcal{L}^q$-norm. Condition \ref{C-err3} puts some restrictions on the decay of $\sum\nolimits_{s \ge t} \delta_q(g_i, s)$ (as a function of $t$) and in particular implies that $\sum\nolimits_{s \ge 0} \delta_q(g_i, s)$ is finite. It is fulfilled by a wide range of stationary processes $\mathcal{E}_i$. For a detailed discussion of \ref{C-err1}--\ref{C-err3} and some examples of error processes that satisfy these conditions, see \cite{KhismatullinaVogt2020}.


The covariates $\X_{it} = (X_{it,1},\ldots,X_{it,d})^\top$ are assumed to have the following properties. 
\begin{enumerate}[label=(C\arabic*),leftmargin=1.05cm]
\setcounter{enumi}{3}
\item \label{C-reg1} The variables $X_{it,j}$ allow for the representation $X_{it,j} = h_{ij}(\mathcal{G}_{it,j})$, where $\mathcal{G}_{it,j} = (\ldots, \xi_{it-1,j}, \xi_{it,j})$, the random variables $\xi_{it,j}$ are i.i.d.\ across $t$ and $h_{ij}: \reals^\infty \rightarrow \reals$ is a measurable function such that $X_{it,j}$ is well-defined. We use the notation $\X_{it} = \boldsymbol{h}_{i}(\mathcal{G}_{it})$ with $\boldsymbol{h}_i = (h_{i1}, \ldots, h_{id})^\top$ and $\mathcal{G}_{it} = (\mathcal{G}_{it,1}, \ldots, \mathcal{G}_{it,d})^\top$. It holds that $\ex [X_{it,j}]=0$ and $\| X_{it,j} \|_{q^\prime} <\infty$ for all $i$ and $j$, where $q^\prime > \max \{ 2, \theta q \}$ with $q$ from \ref{C-err1} and $\theta$ specified in \ref{C-grid} below..
%$q^\prime > \max\{ 2\theta, 4\}$ and all $j$, where $\theta$ is specified in Assumption \ref{C-grid} below.
\item \label{C-reg2} The matrix $\ex[\X_{it} \X_{it}^\top]$ is invertible for each $i$. \textcolor{blue}{What about $t$?}
\item \label{C-reg3} For each $i$ and $j$, it holds that $\sum_{s=t}^{\infty} \delta_{q^\prime}(h_{ij}, s)= O(t^{-\alpha})$ for some $\alpha > 1/2 - 1/{q^\prime}$ with $q^\prime$ from \ref{C-reg1}.
\end{enumerate}
Assumption \ref{C-reg1} guarantees that the process $\{ \X_{it}: 1 \le t \le T \}$ is stationary and causal for each $i$. Similar to the restrictions on the error processes, we employ the definition of the physical dependence measure $\delta_{q^\prime}(h_{ij}, s)$ in Assumption \ref{C-reg3}, thus ensuring that the cumulative effect of the innovation $\xi_{i0}$ on the variables $\X_{i0}, \X_{i1}, \X_{i2},\ldots$ is finite. 


We finally impose some assumptions on the relationship between the covariates and the errors and on the trend functions $m_i$.
\begin{enumerate}[label=(C\arabic*),leftmargin=1.05cm]
\setcounter{enumi}{6}
\item \label{C-reg-err} The random variables $\Delta \X_{it} = \X_{it} - \X_{it-1}$ and $\Delta \varepsilon_{it} = \varepsilon_{it} - \varepsilon_{it-1}$ are uncorrelated, that is, $\cov(\Delta \X_{it}, \Delta \varepsilon_{it}) = \ex[\Delta \X_{it} \Delta \varepsilon_{it}] = 0$. 
\item \label{C-trend} The trend functions $m_i$ are continuously differentiable on $[0, 1]$. \textcolor{blue}{Do we need this as an assumption? We have this requirement in the Theorem 4.1.}
%The innovation processes $\{ \eta_{it}: t \in \integers \}$ and $\{\xi_{it}: t \in \integers \}$ are independent for each $i$. This implies that the processes $\{\varepsilon_{it}: t \in \integers\}$ and $\{ \X_{it}: t \in \integers \}$ are independent for each $i$ as well. \textcolor{red}{Weaken this!}
%\item \label{C-reg-err2} Let $\zeta_{i, t} = (u_{it}, \eta_{it})^\top$. Define $\mathcal{I}_{it} = (\ldots, \zeta_{i, t-1}, \zeta_{i, t})$ and $\mathbf{U}_i(\mathcal{I}_{it}) =  \boldsymbol{h}_i(\mathcal{U}_{it})G_i(\mathcal{J}_{it})$. With this notation at hand, we assume that $\sum_{s=0}^\infty \delta_2(\mathbf{U}_i, s)<\infty$.
%\textcolor{red}{This assumption is not needed. See the proof of ??}
\end{enumerate}
%\textcolor{red}{Assumption \ref{C-reg-err1} is a slightly relaxed independence assumption: even though we do not require the covariates $\X_{it}$ to be completely independent with the error terms $\varepsilon_{it}$, our theoretical results depend upon them being uncorrelated. We in particular need this restriction in order to prove asymptotic consistency for the differencing estimator $\widehat{\bfbeta}_i$ of $\bfbeta_i$ proposed in Section \ref{subsec:para:beta}. In principle, it would be possible to relax this assumption even further, but that would involve much more complicated estimation procedure of $\bfbeta_i$ and more arduous technical arguments. Assumption \ref{C-reg-err2} ensures short-range dependence among the variables in our model. Again, we can interpret this as the fact that the cumulative effect of a single error on all future values is bounded.}
%We employ these assumptions to prove the main theoretical results in our paper. For detailed proofs, we refer the reader to the Appendix.


\begin{remark}
The conditions \ref{C-reg1}--\ref{C-reg3} can be relaxed to cover nonstationary regressors as well as stationary ones. For example, \ref{C-reg1} may then be replaced by
\begin{enumerate}[label=(C\arabic*$^\ast$),leftmargin=1.15cm]
\setcounter{enumi}{3}
\item \label{C-reg1-star} The covariates $\X_{it}$ allow for the representation $\X_{it} = \boldsymbol{h}_i(t; \ldots,\xi_{it-1},\xi_{it})$ with $\xi_{it}$ being i.i.d.\ random variables and $\boldsymbol{h}_i := (h_{i1}, h_{i2}, \ldots, h_{id})^\top: \reals^\integers \rightarrow \reals^d$ a measurable function such that $\boldsymbol{h}_i(t;\mathcal{G}_{it})$ is well defined. 
\end{enumerate} 
The other assumptions can be adjusted accordingly. Our main theoretical results will in principle still hold in this case, however, the complexity of the technical arguments will increase drastically. Hence, for the sake of clarity, we restrict our attention only to stationary  covariates $\X_{it}$. 
\end{remark}


\section{Testing procedure}\label{sec:test}

In this section, we develop a multiscale testing procedure for the problem of comparison of the trend curves $m_i$ in model \eqref{eq:model_full}.  As we will see, the proposed multiscale method does not only allow to test whether the null hypothesis is violated. It also provides information on where violations occur. More specifically, it allows to identify, with a pre-specified confidence, (i) trend functions which are different from each other and (ii) time intervals where these trend functions differ.

\subsection{Preliminary steps}\label{subsec:test:prep}

Testing the null hypothesis $H_0: m_1 = m_2 = \ldots = m_n$ in model \eqref{eq:model_full} is a challenging task not only because it involves nonparametric estimation of the functions $m_i(\cdot)$, but also due to the presence of an unknown fixed term $\alpha_i$ and a vector of unknown parameters $\bfbeta_i$. It is clear that if $\alpha_i$ and $\bfbeta_i$ are known, the problem of testing for the common time trend would be greatly simplified. That is, we would test $H_0: m_1 = m_2 = \ldots = m_n$ in the model
\begin{align*}
Y_{it} - \alpha_i - \bfbeta_i^\top \X_{it} & =: Y_{it}^\circ\\
					& = m_i \Big( \frac{t}{T} \Big) + \varepsilon_{it}, 
\end{align*}
which is a standard nonparametric regression equation. However, in reality the variables $Y_{it}^\circ$ are not observed since the intercept $\alpha_i$ and the coefficients $\bfbeta_i$ are not known. Nevertheless, given appropriate estimators $\widehat{\alpha}_i$ and $\widehat{\bfbeta}_i$, we can consider
\begin{align*}
\widehat{Y}_{it} := Y_{it} -\widehat{\alpha}_i - \widehat{\bfbeta}_i^\top \X_{it} =(\bfbeta_i - \widehat{\bfbeta}_i)^\top \X_{it} + m_i \Big( \frac{t}{T} \Big) + \big( \alpha_i - \widehat{\alpha}_i \big) + \varepsilon_{it}. 
\end{align*}
Thus, the unobserved variables $Y_{it}^\circ$ can be approximated by $\widehat{Y}_{it}$, and in what follows we show that under some mild conditions on $\widehat{\alpha}_i$ and $\widehat{\bfbeta}_i$, this approximation is indeed sufficient for our analysis. 

But before we proceed further, we show how to construct consistent estimates $\widehat{\alpha}_i$ and $\widehat{\bfbeta}_i$. To begin with, we focus on the estimation of the vector of unknown parameters $\bfbeta_i$. We construct the estimator  $\widehat{\bfbeta}_i$ in the following way.

For each $i$, we consider the time series $\{\Delta Y_{it}: 2 \leq t \leq T\}$ of the differences $\Delta Y_{it} = Y_{it} - Y_{i t-1}$. We can write
\begin{align*}
	\Delta Y_{it} = Y_{it} - Y_{i t-1} =\bfbeta_i^\top \Delta \X_{it} + \bigg(m_i \Big( \frac{t}{T} \Big) - m_i \Big(\frac{t-1}{T}\Big)\bigg) + \Delta \varepsilon_{it},
\end{align*}
where $\Delta  \X_{it} =  \X_{it} -  \X_{it-1}$ and $ \Delta \varepsilon_{it} = \varepsilon_{it} - \varepsilon_{i t-1}$. Since $m_i(\cdot)$ is Lipschitz (by our assumption that $m_i(\cdot)$ is continuously differentiable on $[0, 1]$), we can use the fact that $ \big|m_i \big( \frac{t}{T} \big) - m_i \big(\frac{t-1}{T}\big) \big| = O\big(\frac{1}{T}\big)$ and rewrite 
\begin{align}\label{model_with_regs}
	\Delta Y_{it} = \bfbeta_i^\top \Delta \X_{it} + \Delta \varepsilon_{it} + O\Big(\frac{1}{T}\Big).
\end{align}

Now, for each $i$ we employ the least squares estimation method to estimate $\bfbeta_i$ in \eqref{model_with_regs}, treating $\Delta \X_{it}$ as the regressors and $\Delta Y_{it}$ as the response variable. That is, we propose the following differencing estimator:
\begin{align}\label{eq:beta:est}
\widehat{\bfbeta}_i = \Big( \sum_{t=2}^T \Delta \X_{it} \Delta \X_{it}^\top \Big)^{-1} \sum_{t=2}^T \Delta \X_{it} \Delta Y_{it}
\end{align}
In Lemma \ref{lemma-beta-rate} in the Appendix, we show that $\widehat{\bfbeta}_i$ is a consistent estimator of $\bfbeta_i$ with the property $\bfbeta_i - \widehat{\bfbeta}_i = O_p(T^{-1/2})$ under our assumptions. 

Next, given $\widehat{\bfbeta}_i$, consider an appropriate estimator $\widehat{\alpha}_{i}$ for the intercept $\alpha_i$ calculated by
\begin{align}
\widehat{\alpha}_i &= \frac{1}{T}\sum_{t=1}^T \big(Y_{it} - \widehat{\bfbeta}_i^\top \X_{it}\big) = \frac{1}{T}\sum_{t=1}^T \big(\bfbeta_i^\top \X_{it} - \widehat{\bfbeta}_i^\top \X_{it} + \alpha_i + m_i(t/T) + \varepsilon_{it}\big) \nonumber \\
&= \big(\bfbeta_i - \widehat{\bfbeta}_i \big)^\top\frac{1}{T}\sum_{t=1}^T  \X_{it} + \alpha_i + \frac{1}{T}\sum_{i=1}^T m_i(t/T) + \frac{1}{T}\sum_{i=1}^T \varepsilon_{it}. \label{eq:alpha:est}
\end{align}
Note that $\frac{1}{T}\sum_{i=1}^T \varepsilon_{it} = O_p(T^{-1/2})$ and $\frac{1}{T}\sum_{i=1}^T m_i(t/T) = O(T^{-1})$ due to Lipschitz continuity of $m_i$ and normalization $\int_{0}^1 m_i(u)du = 0$. Furthermore, $\frac{1}{T}\sum_{t=1}^T  \X_{it} = O_p(1)$ by Chebyshev's inequality and $\widehat{\bfbeta}_i - \bfbeta_i = O_p (T^{-1/2})$. Plugging all these results together in \eqref{eq:alpha:est}, we get that $\widehat{\alpha}_i - \alpha_i = O_p(T^{-1/2})$. Thus, the unobserved variables \linebreak $Y_{it}^\circ := Y_{it} - \bfbeta_i^\top \X_{it} - \alpha_i = m_i(t/T) + \varepsilon_{it}$ can be well approximated by $\widehat{Y}_{it} $ since \linebreak $\widehat{Y}_{it} = Y_{it} -\widehat{\alpha}_i - \widehat{\bfbeta}_i^\top \X_{it} = Y_{it}^\circ + O_p(T^{-1/2})$.

We now turn to the estimator of the long-run error variance $\sigma_i^2 = \sum\nolimits_{\ell=-\infty}^{\infty} \cov(\varepsilon_{i0}, \varepsilon_{i\ell})$ which is necessary for the construction of the test statistics later on. For the moment, we assume that the long-run variance does not depend on $i$, that is $\sigma_i^2 = \sigma^2$ for all $i$. We will need this further for conducting the testing procedure properly. Nevertheless, we keep the indices throughout the paper in order to be congruous in notation. We further let $\widehat{\sigma}_i^2$ be an estimator of $\sigma_i^2$ which is computed from the constructed sample $\{ \widehat{Y}_{it}: 1 \le t \le T \}$. We thus regard $\widehat{\sigma}_i^2 = \widehat{\sigma}_i^2(\widehat{Y}_{i1},\ldots,\widehat{Y}_{iT})$ as a function of the variables $\widehat{Y}_{it}$ for $1 \le t \le T$. Hence, whereas the true long-run variance is the same for all time series, the estimators are different. Throughout the paper, we assume that $\widehat{\sigma}_i^2 = \sigma_i^2 + o_p(\rho_T)$, where the conditions on $\rho_T$ will be provided further in Section \ref{sec:theo}. Our theory works with any estimator $\widehat{\sigma}_i^2$ that has this property. 

We now discuss some possible choices of $\widehat{\sigma}_i^2$. Following \cite{Kim2016}, we can estimate $\sigma_i$ for each $i$ by a variant of the subseries variance estimator proposed first by \cite{Carlstein1986} and then extended by \cite{WuZhao2007}. Formally, we set
\begin{align}
\widehat{\sigma}_i^2 = \frac{1}{2(M-1)s_T}\sum_{m=1}^M \bigg[ \sum_{t = 1}^{s_T} & \Big(Y_{i(t + ms_T)} - Y_{i(t + (m-1)s_T)} \nonumber \\[-0.2cm] & - \widehat{\bfbeta}_i^\top\big(\X_{i(t+ms_T)} - \X_{i(t+(m-1)s_T)}\big) \Big)\bigg]^2, \label{eq:lrv}
\end{align}
where $s_T$ is the length of subseries and $M = \lfloor T/s_T\rfloor$ is the largest integer not exceeding $T/s_T$. As per the optimality result in \cite{Carlstein1986}, we set $s_T \asymp T^{1/3}$. For a finite sample, we choose $s_T = \lfloor T^{1/3}\rfloor$. According to  Lemma \ref{lemmaA:lrv} in Appendix, $\widehat{\sigma}_i^2$ is an asymptotically consistent estimator of $\sigma_i^2$ with the rate of convergence $O_p(T^{-1/3})$.

\textcolor{red}{Subseries variance estimator described above can be used without imposing any additional assumptions on the error processes $\mathcal{E}_i$. However, as noted in \cite{KhismatullinaVogt2020}, estimating the long-run error variance in such models under general weak dependence conditions is a particularly difficult problem. Estimators often tend to be quite imprecise. Hence, in practice we opt for imposing certain conditions on the error processes $\mathcal{E}_i = \{\varepsilon_{it}: 1 \le t \le T\}$. For example, we can assume that for each $i$ the error process $\mathcal{E}_i$ has the AR($\infty$) structure. In this case, it is possible to obtain more precise estimates $\hat{\sigma}_i^2$ using the difference-based method described in \cite{KhismatullinaVogt2020}. In our application examples, we choose to impose this restriction on the error terms. The detailed discussion of the behaviour of such an estimator in the presence of a trend and comparison with other long-run variance estimators can be found in \cite{KhismatullinaVogt2020}.}

%According to  Lemma \ref{lemmaA:lrv} in Appendix, $\widehat{\sigma}_i^2$ is an asymptotically consistent estimator of $\sigma_i^2$ with the rate of convergence $O_p(T^{-1/3})$. Recall that the rate of convergence of $\widehat{\sigma}^2_i$ necessary for proving our theoretical results is $o_p(\rho_T)$ with $\rho_T = o(\sqrt{h_{\min}}/\log T)$. {\color{green}Hence, for our theory to work with this estimator, we need to put some restrictions on $\rho_T$. Specifically, we need to have \linebreak $T^{-1/3}/\rho_T = o(1)$ which can be translated to the following condition on the minimal bandwidth:
%\begin{align}\label{eq:hmin}
%h_{\min} \gg T^{-2/3}\log^{2+2\iota} T \text{ for some} \,\,\, \iota >0.
%\end{align} Remember that under Assumption \ref{C-h}, we have that $h_{\min} \gg T^{- \left(1 - \frac{2}{q}\right)}\log T$. Taking into account \eqref{eq:hmin}, we can rewrite Assumption \ref{C-h} as follows:

%\begin{enumerate}[label=(C\arabic*$^\ast$),leftmargin=1.40cm]
%\setcounter{enumi}{12}
%\item \label{C-h-star} $h_{\min} \gg \max\{ T^{-(1-\frac{2}{q})} \log T, T^{-2/3}\log^{2+2\iota}  T \}$, that is, $h_{\min} / \{ T^{-(1-\frac{2}{q})} \log T \} \rightarrow \infty$ with $q > 4$ defined in \ref{C-err2} and $h_{\min} / \{ T^{-2/3}\log^{2+2\iota}  T \} \rightarrow \infty$ for some $\iota >0$, and $h_{\max} < 1/2$.
%\end{enumerate}

%This means that we can for example have $\rho_T = T^{-1/3}\log^\iota T$, which is still a slower rate of convergence than $T^{-1/3}$. To sum up, under the Assumptions \ref{C-err1} - \ref{C-grid}, \ref{C-h-star}, the subseries variance estimator provided by \eqref{eq:lrv} satisfies the necessary conditions for Theorem \ref{theo:stat:global}, and thus can be used for the construction of our multiscale statistics $\widehat{\Psi}_{n, T}$.}


\subsection{Construction of the test statistics}\label{subsec:test:stat}

We are now ready to introduce the multiscale statistic for testing the hypothesis \linebreak $H_0: m_1 = m_2 = \ldots = m_n$. For any pair of time series $i$ and $j$ and for any location-bandwidth pair $(u, h)$, we define the kernel averages
\begin{align}\label{eq:psi_hat_ij}
 \widehat{\psi}_{ij,T}(u,h) = \sum\limits_{t=1}^T w_{t,T}(u,h)(\widehat{Y}_{it} - \widehat{Y}_{jt}),
 \end{align}
where $w_{t,T}(u,h)$ are local linear kernel weights calculated by the following formula:
\begin{equation}\label{eq:weights}
w_{t,T}(u,h) = \frac{\Lambda_{t,T}(u,h)}{ \{\sum\nolimits_{t=1}^T \Lambda_{t,T}(u,h)^2 \}^{1/2} }, 
\end{equation}
where
\[ \Lambda_{t,T}(u,h) = K\Big(\frac{\frac{t}{T}-u}{h}\Big) \Big[ S_{T,2}(u,h) - \Big(\frac{\frac{t}{T}-u}{h}\Big) S_{T,1}(u,h) \Big], \]
$S_{T,\ell}(u,h) = (Th)^{-1} \sum\nolimits_{t=1}^T K(\frac{\frac{t}{T}-u}{h}) (\frac{\frac{t}{T}-u}{h})^\ell$ for $\ell = 1,2$ and $K$ is a kernel function. As common in nonparametric estimation, we assume that $K$ has the following properties: 
\begin{enumerate}[label=(C\arabic*),leftmargin=1.05cm]
\setcounter{enumi}{8}
\item \label{C-ker} The kernel $K$ is non-negative, symmetric about zero and integrates to one. Moreover, it has compact support $[-1,1]$ and is Lipschitz continuous, that is,\linebreak $|K(v) - K(w)| \le C |v-w|$ for any $v,w \in \reals$ and some constant $C > 0$. 
\end{enumerate} 
Assumption \ref{C-ker} allows us to use usual kernel functions such as rectangular, Epanechnikov and Gaussian kernels.

We regard the kernel average $\widehat{\psi}_{ij,T}(u,h)$ as a measure of the distance between the two trend curves $m_i$ and $m_j$ on the interval $\interval_{(u, h)} = [u-h,u+h]$. However, instead of working directly with the kernel averages $\widehat{\psi}_{ij,T}(u,h)$, we replace them by their normalized and corrected version:
\begin{align}\label{eq:psi_zero_ij}
\hat{\psi}^0_{ij,T}(u, h) =  \Big|\frac{\widehat{\psi}_{ij,T}(u,h)}{(\widehat{\sigma}_i^2 + \widehat{\sigma}_j^2)^{1/2}}\Big| - \lambda(h).
\end{align}
Here, $\lambda(h) = \sqrt{2 \log \{ 1/(2h) \}}$ is an additive correction term that balances the significance of many test statistics that correspond to different values of bandwidth parameters (see the discussion on this topic and comparison between multiscale testing procedures with and without this correction term in \citet{KhismatullinaVogt2020}).

We now aggregate the test statistics $\hat{\psi}^0_{ij, T}(u,h)$ for all $i$ and $j$ and a wide range of different locations $u$ and bandwidths (scales) $h$:
\begin{align}\label{eq:Psi_hat}
	\widehat{\Psi}_{n,T} = \max_{1 \le i < j \le n}\max_{(u,h) \in \mathcal{G}_T} \hat{\psi}^0_{ij,T}(u, h),
\end{align}

In \eqref{eq:Psi_hat}, $\mathcal{G}_T$ stands for the set of location-bandwidth pairs $(u, h)$ that was mentioned in Section \ref{sec:intro}. We use the subscript $T$ in $\grid_T$ to point out that the choice of the grid depends on the sample size $T$. Specifically, throughout the paper, we suppose that $\mathcal{G}_T$ is some subset of $\mathcal{G}_T^{\text{full}} = \{ (u,h): u = t/T \text{ and } h = s/T \text{ for some } 1 \le t, s \le T \text{ such that } \linebreak h \in [h_{\min},h_{\max}] \}$, where $h_{\min}$ and $h_{\max}$ denote some minimal and maximal bandwidth value, respectively. As was already discussed in Section \ref{sec:intro}, we assume that the set of intervals $\{\interval_{(u, h)} = [u-h, u+h]: (u, h) \in \grid_T\}$ covers the whole unit interval. Furthermore, for our theoretical results, we require the following additional conditions to hold:
\begin{enumerate}[label=(C\arabic*),leftmargin=1.05cm]
\setcounter{enumi}{9}

\item \label{C-grid} $|\mathcal{G}_T| = O(T^\theta)$ for some arbitrarily large but fixed constant $\theta > 0$, where $|\mathcal{G}_T|$ denotes the cardinality of $\mathcal{G}_T$. 

\item \label{C-h} $h_{\min} \gg T^{-(1-\frac{2}{q})} \log T$, that is, $h_{\min} / \{ T^{-(1-\frac{2}{q})} \log T \} \rightarrow \infty$ with $q > 4$ defined in \ref{C-err2} and $h_{\max} = o(1)$.

\end{enumerate}
Assumption \ref{C-grid} places relatively mild restrictions on the grid $\grid_T$: we allow the grid to grow with the sample size but only at a polynomial rate $T^\theta$ with fixed $\theta$. This is not a severe constraint because under this limitation, we can still work with the full set of location-bandwidth points $\grid_T = \grid_T^{\text{full}}$ which is more than enough for most applied problems. \textcolor{blue}{(I have changed the definition of $\mathcal{G}_T^{\text{full}}$ in the last iteration, it should have been enough. Now $|\mathcal{G}_T^{\text{full}}| \leq T^2$, does not it work?)} Assumption \ref{C-h} is concerned with the minimal and the maximal bandwidths that we use for our analysis. Specifically, according to Assumption \ref{C-h}, we can choose the minimal bandwidth $h_{\min}$ that converges to zero slower than $T^{-(1-\frac{2}{q})} \log T$ as the sample size $T$ goes to infinity. The maximal bandwidth $h_{\max}$ can be picked very large.

Note that the value $\max_{(u,h) \in \mathcal{G}_T} \hat{\psi}^0_{ij,T}(u, h)$ simultaneously takes into account all intervals $\interval_{(u, h)} = [u- h, u+h]$ with $(u,h) \in \mathcal{G}_T$. Thus, it can be interpreted as a global distance measure between the two curves $m_i$ and $m_j$, and the test statistics $\widehat{\Psi}_{n,T}$ is then defined as the maximal distance between any pair of curves $m_i$ and $m_j$ with $i \ne j$.

In Section \ref{subsec:test:test}, we show how to test the null hypothesis $H_0: m_1 =m_2 = \ldots = m_n$ using the multiscale test statistics $\widehat{\Psi}_{n,T}$.

\subsection{The testing procedure}\label{subsec:test:test}


Let $Z_{it}$ for $1 \le t \le T$ and $1 \le i \le n$ be independent standard normal random variables which are independent of the error terms $\varepsilon_{js}$ and the covariates $\X_{js}$ for all $1 \leq s \leq T $ and $1 \leq j \leq n$. Denote the empirical average of the variables $Z_{i1},\ldots,Z_{iT}$ by $\bar{Z}_{i,T} = T^{-1} \sum_{t=1}^T Z_{it}$. To simplify the notation, we will omit the subscript $T$ in $\bar{Z}_{i,T}$ in what follows. Similarly as with $\hat{\psi}^0_{ij,T}(u, h)$, for each $i$ and $j$, we introduce the normalized and corrected Gaussian kernel averages 
\begin{align}\label{eq:phi_zero_ij}
\phi^0_{ij,T}(u, h) =  \bigg|\frac{\phi_{ij,T}(u,h)}{(\sigma_i^2 + \sigma_j^2)^{1/2}}\bigg| - \lambda(h),
\end{align}
where 
\begin{align}\label{eq:phi_ij}
\phi_{ij,T}(u,h) = \sum\nolimits_{t=1}^T w_{t,T}(u,h) \, \big\{ \sigma_i (Z_{it} - \bar{Z}_i) - \sigma_j (Z_{jt} - \bar{Z}_j) \big\}
\end{align}
with $w_{t, T}(u, h)$ defined in \eqref{eq:weights}. 

Next, in the same way as in \eqref{eq:Psi_hat}, we define the global Gaussian test statistic
\begin{align}\label{eq:Phi}
\Phi_{n,T} = \max_{1 \le i < j \le n}\max_{(u,h) \in \mathcal{G}_T} \phi^0_{ij,T}(u, h)
\end{align}
and denote its $(1-\alpha)$-quantile by $q_{n,T}(\alpha)$.

Our multiscale test of the hypothesis $H_0: m_1 = m_2 = \ldots = m_n$ is defined as follows: 
\begin{center}
\begin{minipage}[c][0.75cm][c]{13cm}
\textit{For a given significance level $\alpha \in (0,1)$, we reject $H_0$ if $\widehat{\Psi}_{n,T} > q_{n,T}(\alpha)$.}
\end{minipage}
\end{center}

\begin{remark}
To prove the theoretical results in Section \ref{sec:theo}, we will use the following fact. By our assumption that the long-run variance $\sigma_i^2$ does not depend on $i$ \linebreak (i.e. $\sigma_i^2 = \sigma^2_j = \sigma^2$), we can rewrite the Gaussian normalized kernel averages \eqref{eq:phi_zero_ij} as
\[\phi^0_{ij,T}(u, h) = \frac{1}{\sqrt{2}} \Big|\sum\nolimits_{t=1}^T w_{t,T}(u,h) \, \big\{ (Z_{it} - \bar{Z}_i) - (Z_{jt} - \bar{Z}_j) \big\}\Big| - \lambda(h), \] 
which means that the distribution of the Gaussian test statistics does not depend neither on the data $\mathcal{T}_i =\{ (Y_{it}, \X_{it}) : 1\leq t \leq T)\}, \mathcal{T}_j =\{ (Y_{jt}, \X_{jt}) : 1\leq t \leq T)\}$, nor on any unknown quantities such as $\sigma^2_i$ or $\sigma_j^2$, and thus can be regarded as known. In addition to exploiting this fact while proving the theoretical results, we will also use it for calculating (approximately) the quantiles of $\Phi_{n, T}$ by the Monte Carlo simulations in Section \ref{subsec:test:impl}. However, for the sake of similarity to $\hat{\psi}^0_{ij,T}(u, h)$, in what follows, we will stick to the definition \eqref{eq:phi_zero_ij}, which involves the long-run variances $\sigma_i$ and $\sigma_j$.
\end{remark}

\begin{remark}
By construction, the $(1-\alpha)$ Gaussian quantile $q_{n, T}(\alpha)$ depends not only on the number of times series considered $n$ and the sample size $T$, but on the choice of the set of location-bandwidth pairs $\grid_T$ as well. However, we do not explicitly include this dependence since we believe it will only lead to the unnecessary complication of the notation. 
\end{remark}

\subsection{Locating the differences}\label{subsec:test:loc}

Suppose we reject the null hypothesis $H_0$. This fact does not provide us with a lot of information about the behaviour of the trend functions $m_i(\cdot)$. After performing the test described in Section \ref{subsec:test:test}, we can only make confidence statements that some of the trend functions are not equal somewhere on $[0, 1]$ (with a given statistical confidence), but we can not tell which of the functions are different and where they differ. Hence, we need an additional step in the testing procedure in order to locate those differences.

Formally, for a given pair of time series $(i, j)$ and for any given interval \linebreak $\interval_{(u, h)} = [u-h, u+h]$ such that $(u, h) \in \grid_T$ we consider the hypothesis 
\[ H_0^{[i,j]}(u,h): m_i(w) = m_j(w) \text{ for all } w \in [u-h,u+h]. \] 
We view $H_0^{[i,j]}(u,h)$ as the 'local' null hypothesis because it is concerned with only two trend functions $m_i(\cdot)$ and $m_j(\cdot)$ and their equality on a small, 'local', interval $\interval_{(u, h)} = [u-h, u+h]$. In contrast, we refer to $H_0$ introduced in \eqref{eq:null} as the global null hypothesis.

We define the multiscale test of the hypothesis $H_0^{[i,j]}(u,h)$ as follows: 
\begin{center}
\begin{minipage}[c][1.25cm][c]{13cm}
\textit{For a given significance level $\alpha \in (0,1)$, we reject $H_0^{[i,j]}(u,h)$ if \linebreak $\hat{\psi}^0_{ij,T}(u, h) > q_{n,T}(\alpha)$.}
\end{minipage}
\end{center}

For each pair of time series $(i, j)$, denote the set of intervals $\interval_{(u, h)}$ that consists of the intervals where we reject $H_0^{[i,j]}(u,h)$ at a significance level $\alpha$ by $\mathcal{S}^{[i, j]}(\alpha)$. We will prove later in Section \ref{sec:theo}, that we can make the following confidence statements:

\begin{center}
\begin{minipage}[c][1.25cm][c]{13cm}
\textit{We can state with (asymptotic) probability at least $1-\alpha$ that for all $i, j, \linebreak1\leq i < j \leq n$, we have that $m_i(\cdot)$ and $m_j(\cdot)$ differ on all of the intervals $\interval_{(u, h)} \in \mathcal{S}^{[i, j]}(\alpha)$.}
\end{minipage}
\end{center}


\subsection{Implementation of the test in practice}\label{subsec:test:impl}

In practice, we implement the test procedure described in Sections \ref{subsec:test:test} and \ref{subsec:test:loc} in the following way. 
\begin{enumerate}[label=\textit{Step \arabic*.}, leftmargin=1.45cm]
\item Fix a significance level $\alpha \in (0, 1)$. 
\item Compute the (approximated) quantile $q_{n, T}(\alpha)$ by Monte Carlo simulations. Specifically, draw a large number $N$ (say $N=5000$) of samples of independent standard normal random variables $\{Z_{it}^{(\ell)} : 1 \le t \le T, \, 1 \le i \le n \}$ for $1 \le \ell \le N$. For each sample $\ell$, compute the value $\Phi_{n,T}^{(\ell)}$ of the Gaussian test statistics $\Phi_{n, T}$ and store them. Calculate the empirical $(1-\alpha)$-quantile $\hat{q}_{n, T}(\alpha)$ from the stored values $\{ \Phi_{n, T}^{(\ell)}: 1 \le \ell \le N \}$. Use $\hat{q}_{n, T}(\alpha)$ as an approximated value of the quantile $q_{n, T}(\alpha)$.
\item Carry out the test for the global hypothesis $H_0$ by calculating $\widehat{\Psi}_{n, T}$ and checking if $\widehat{\Psi}_{n, T} > \hat{q}_{n, T}(\alpha)$. Reject the null if this is the case.
\item For each $i, j,\, 1 \le i < j \le n$, and each $(u, h) \in \grid_T$, carry out the test for the local null hypothesis $H_0^{[i, j]}(u, h)$ by checking if $\hat{\psi}^0_{ij, T}(u, h)> \hat{q}_{n, T}(\alpha)$. For each pair of time series $(i, j)$, find the set of intervals $\mathcal{S}^{[i, j]}(\alpha)$ that consists of the intervals where we reject $H_0^{[i,j]}(u,h)$. 
%Store the test results in the variable $r_{ij,T}(u, h) = \mathds{1}( |\hat{\psi}_{ij,T}^0(u, h)| > q_{n, T}(\alpha))$, where $\mathds{1}(\cdot)$ is an indicator function that is equal to $1$ if the condition inside the brackets is met and $0$ otherwise.
\item Display the results. One of the possible ways to do that is to produce a separate plot for each of the pairwise comparisons and draw only the intervals where we reject the corresponding local null. Formally, on each of the plots that present the results of the comparison of time series $i$ and $j$, we display the intervals $\interval_{(u, h)} = [u- h, u+h] \in \mathcal{S}^{[i, j]}(\alpha)$, i.e.\ the (rescaled) time intervals where we reject $H_0^{[i, j]}(u, h)$. 
\end{enumerate}



\section{Theoretical properties of the test}\label{sec:theo}


In order to investigate the theoretical properties of our multiscale test, we introduce the auxiliary statistic
\begin{align}\label{eq:Phi_hat}
\widehat{\Phi}_{n,T} = \max_{1 \le i < j \le n}  \max_{(u,h) \in \mathcal{G}_T} \widehat{\phi}^0_{ij,T}(u,h),
\end{align}
where
\begin{equation*}%\label{eq:phi_hat_ij_zero}
\widehat{\phi}^0_{ij,T}(u,h) =\bigg| \frac{\widehat{\phi}_{ij,T}(u,h)} {\{ \widehat{\sigma}_i^2 + \widehat{\sigma}_j^2 \}^{1/2}} \bigg| - \lambda(h)
\end{equation*}
and
\begin{align*}
\widehat{\phi}_{ij,T}(u,h) 
 & = \sum_{t=1}^T w_{t,T}(u,h) \big\{ (\varepsilon_{it} - \bar{\varepsilon}_i) + (\bfbeta_i - \widehat{\bfbeta}_i)^\top (\X_{it} - \bar{\X}_{i}) \\[-0.4cm]
 & \phantom{= \sum_{t=1}^T w_{t,T}(u,h) \big\{} - (\varepsilon_{jt} - \bar{\varepsilon}_j) -  (\bfbeta_j - \widehat{\bfbeta}_j)^\top (\X_{jt} - \bar{\X}_{j}) \big\}.
\end{align*}
We here use the notation $\bar{\varepsilon}_i = \bar{\varepsilon}_{i,T} := T^{-1} \sum_{t=1}^T \varepsilon_{it}$ and $\bar{\X}_{i} =  \bar{\X}_{i, T} := T^{-1}\sum_{t=1}^T  \X_{it}$. By construction, it holds that $\widehat{\phi}_{ij,T}(u,h) = \widehat{\psi}_{ij,T}(u,h)$ under $H_0^{[i, j]}(u, h)$. This implies that $\widehat{\Phi}_{n,T}$ is identical to $\widehat{\Psi}_{n,T}$ under the global null $H_0$. Hence, in order to determine the distribution of our main test statistic $\widehat{\Psi}_{n,T}$ under $H_0$, we can study the auxiliary statistic $\widehat{\Phi}_{n,T}$. The following theorem shows that the distribution of $\widehat{\Phi}_{n,T}$ is close to the distribution of the Gaussian statistic $\Phi_{n,T}$ introduced in \eqref{eq:Phi}. 


%However, $\widehat{\Phi}_{n,T}$ depends on the covariates $\X_{it}$ whereas the Gaussian version $\Phi_{n,T}$ that is used to calculate critical values for our test (defined in \eqref{eq:Phi}) is independent of them. This is the reason why we need to introduce additional intermediate test statistic that does not include the covariates, therefore, connecting $\widehat{\Phi}_{n,T}$ and $\Phi_{n,T}$. This intermediate test statistics will play an important role in the proof of our main theoretical result.

%Formally, for each $i,j$ we construct the kernel averages as 
%\[\doublehat{\phi}_{ij,T}(u,h) = \sum_{t=1}^T w_{t,T}(u,h) \big\{ (\varepsilon_{it} - \bar{\varepsilon}_i) - (\varepsilon_{jt} - \bar{\varepsilon}_j)  \big\}. \]
%We can view these kernel averages as constructed under the null from the unobserved variables $\doublehattwo{Y}_{it}$ and  $\doublehattwo{Y}_{jt}$ given by the following formula: 
%\begin{align*}
%\doublehattwo{Y}_{it} :&= Y_{it} - \bfbeta_i^\top \X_{it} -  \frac{1}{T}\sum_{t=1}^T \big(Y_{it} - \bfbeta_i^\top \X_{it}\big) =\\
%&=m_i \Big( \frac{t}{T} \Big)  - \frac{1}{T}\sum_{t=1}^T  m_i \Big( \frac{t}{T} \Big) + \varepsilon_{it} - \frac{1}{T}\sum_{t=1}^T \varepsilon_{it}.
%\end{align*}

%The intermediate statistic $\doublehattwo{\Phi}_{n, T}$ is then defined as 
%\begin{align}\label{eq:Phi_doublehat}
%\doublehattwo{\Phi}_{n,T} &= \max_{1 \le i < j \le n}\max_{(u,h) \in \mathcal{G}_T} \Bigg\{\bigg|\frac{\doublehat{\phi}_{ij,T}(u,h)}{\big(\doublehattwo{\sigma}_i^2 + \doublehattwo{\sigma}_j^2\big)^{1/2}}\bigg| - \lambda(h)\Bigg\}
%\end{align}
%with $\doublehattwo{\sigma}_i^2$ being an estimator of the long-run error variance $\sigma_i^2 = \sum\nolimits_{\ell=-\infty}^{\infty} \cov(\varepsilon_{i0}, \varepsilon_{i\ell})$ which is computed from the unobserved sample $\{ \doublehattwo{Y}_{it} : 1 \le t \le T \}$. We thus regard $\doublehattwo{\sigma}_i^2 = \doublehattwo{\sigma}_i^2(\doublehattwo{Y}_{i1},\ldots,\doublehattwo{Y}_{iT})$ as a function of the variables $\doublehat{Y}_{it}$ for $1 \le t \le T$. As with the estimator $\widehat{\sigma}_i^2$, we assume that $\doublehattwo{\sigma}_i^2 = \sigma_i^2 + o_p(\rho_T)$ with $\rho_T = o(\sqrt{h_{\min}}/\log T)$. 

%The statistics $\doublehattwo{\Phi}_{n,T}$ can thus be viewed as a version of the statistic $\widehat{\Phi}_{n,T}$ without the covariates. We formally prove that these two statistics are close in Proposition \ref{propA:intermediate2}.

%Now we can formally state our main theoretical result which characterizes the asymptotic behaviour of the statistic $\widehat{\Phi}_{n,T}$. 

\begin{theorem}\label{theo:stat:global}
Let \ref{C-err1}--\ref{C-h} be fulfilled and assume that $m_i$ is a continuously differentiable function on $[0, 1]$ satisfying the property $\int_0^1 m_i(u) du = 0$ for all $i$. Furthermore, assume that $\sigma_i^2 = \sigma^2$, $\widehat{\sigma}_i^2 = \sigma^2_i + o_p(\rho_T)$ with $\rho_T = o(\sqrt{h_{\min}}/\log T)$ for all $i$. Then  
\begin{equation*}
\sup_{x \in \reals} \big| \pr(\widehat{\Phi}_{n, T} \le x) - \pr(\Phi_{n,T} \le x) \big| = o(1),
\end{equation*}
%\[ \pr \big( \widehat{\Phi}_{n,T} \le q_{n,T}(\alpha) \big) = (1 - \alpha) + o(1). \]
\end{theorem}
Theorem \ref{theo:stat:global} is the principal instrument for deriving theoretical properties of our multiscale test. Its proof is provided in the Appendix. 

%Here, we briefly present the main arguments.

%First, we show that the distribution of the intermediate statistics $\doublehattwo{\Phi}_{n, T}$ introduced in \eqref{eq:Phi_doublehat} is indeed close to the distribution of $\widehat{\Phi}_{n, T}$, and therefore, we can approximate the distribution of $\widehat{\Phi}_{n, T}$ with the help of $\doublehattwo{\Phi}_{n, T}$.

%Second, we show that we can replace $\doublehattwo{\Phi}_{n, T}$ by an identically distributed version $\widetilde{\Phi}_{n, T}$ which is close to the Gaussian statistics $\Phi_{n, T}$ defined in \eqref{eq:Phi}. Formally, by the means of strong approximation theory derived in \cite{BerkesLiuWu2014} we prove that there exist statistics $\widetilde{\Phi}_{n, T}$ which are distributed as $\doublehattwo{\Phi}_{n, T}$ for any $T \ge 1$ and which have the property that 
%\begin{align}\label{eq:proof1}
%\big| \widetilde{\Phi}_{n, T} - \Phi_{n,T} \big| = o_p( \delta_T),
%\end{align}
%where $\delta_T = o(1)$.

%Then, we employ the anti-concentration results derived in \cite{Chernozhukov2015} in order to show that $\Phi_{n,T}$ does not concentrate too strongly in small regions of the form $[x-\delta_T,x+\delta_T]$. Or, in other words, it holds that 
%\begin{align}\label{eq:proof2}
%\sup_{x \in \reals} \pr \big( | \Phi_{n,T} - x | \le \delta_T \big) = o(1)
%\end{align}
%Taking \eqref{eq:proof1} together with \eqref{eq:proof2} and the fact that $\widetilde{\Phi}_{n, T}$ has the same distribution as $\doublehattwo{\Phi}_{n, T}$ yields that 
%\begin{equation*}
%\sup_{x \in \reals} \big| \pr(\doublehattwo{\Phi}_{n, T} \le x) - \pr(\Phi_{n,T} \le x) \big| = o(1). 
%\end{equation*}
 
%And finally, by the fact mentioned in the beginning of this proof that the distribution of the intermediate statistics $\doublehattwo{\Phi}_{n, T}$ is close to the distribution of $\widehat{\Phi}_{n, T}$, we conclude that 
%\begin{equation*}
%\sup_{x \in \reals} \big| \pr(\widehat{\Phi}_{n, T} \le x) - \pr(\Phi_{n,T} \le x) \big| = o(1),
%\end{equation*}
%which immediately implies the statement of Theorem \ref{theo:stat:global}.

\begin{remark}
The proof of Theorem \ref{theo:stat:global} builds on two important theoretical results: strong approximation theory developed in \cite{BerkesLiuWu2014} and anti-concentration results proved in \cite{Chernozhukov2015}. These results were already combined together for the purpose of developing the multiscale test for dependent data in \cite{KhismatullinaVogt2020}. We can say that our proof can be regarded as a further development of the proof strategy in \cite{KhismatullinaVogt2020} where they proposed a similar testing procedure for investigating properties of the trend function in one time series. We extend their theoretical result not only by working with multiple time series, but also by including the covariate terms in the model \eqref{eq:model}. Hence, our proof strategy builds on the similar stones but is much more technically involved.
\end{remark}

Now we examine the theoretical properties of the testing procedure proposed in \linebreak Sections \ref{subsec:test:test} and \ref{subsec:test:loc} with the help of Theorem \ref{theo:stat:global}. The following proposition 
%(which is a direct consequence of Theorem \ref{theo:stat:global}) 
states that our test has correct (asymptotic) size.

\begin{prop}\label{prop:test}
Suppose that the conditions of Theorem \ref{theo:stat:global} are satisfied. Then under $H_0$, we have
\[ \pr \big( \widehat{\Psi}_{n,T} \le q_{n,T}(\alpha) \big) = (1 - \alpha) + o(1). \]
\end{prop}

The next proposition characterizes the behaviour of our multiscale test under a certain class of local alternatives. To formulate it, we consider a sequence of pairs of functions $m_ i := m_{i,T}$ and $m_ j := m_{j,T}$ that depend on the sample size and that are locally sufficiently far from each other.

\begin{prop}\label{prop:test:power}
Let the conditions  of Theorem \ref{theo:stat:global} be satisfied. Moreover, assume that for some pair of indices $i$ and $j$, the functions $m_ i = m_{i,T}$ and $m_ j = m_{j,T}$ have the following property: There exists $(u,h) \in \mathcal{G}_T$ with $[u-h,u+h] \subseteq [0,1]$ such that $m_{i,T}(w) - m_{j,T}(w) \ge c_T \sqrt{\log T/(Th)}$ for all $w \in [u-h,u+h]$ or $m_{j,T}(w) - m_{i,T}(w) \ge c_T \sqrt{\log T/(Th)}$ for all $w \in [u-h,u+h]$, where $\{c_T\}$ is any sequence of positive numbers with $c_T \rightarrow \infty$. Then 
\[ \pr \big( \widehat{\Psi}_{n,T} \le q_{n,T}(\alpha) \big) = o(1). \]
\end{prop}

%Proof of Proposition \ref{prop:test:power} is provided in the Appendix.

Finally, we turn our attention to the local null hypotheses $H_0^{[i, j]}(u, h)$. Since we are testing many hypotheses at the same time, we would like to bound the probability of making even one false discovery. For this purpose, we employ the notion of the family-wise error rate (FWER) which is equal to the probability of making one or more type I errors. Formally, the FWER is defined as
\begin{align*} 
\text{FWER}(\alpha) = \pr \Big(\exists \,  i, j \in \{1, \ldots, n \}, (u, h) \in \grid_T : & \, \interval_{(u, h)} \in \mathcal{S}^{[i, j]}(\alpha) \\ & \, \text{and } H_0^{[i, j]}(u, h) \text { is true}\Big).
\end{align*}

We say that the FWER is controlled at level $\alpha$ if $\text{FWER}(\alpha) \leq \alpha$. The following result assures that for our testing procedure, this is indeed the case.

\begin{prop}\label{prop:test:fwer}
Suppose that the conditions  of Theorem \ref{theo:stat:global} are satisfied. Then 
\[ \textnormal{FWER}(\alpha) \leq \alpha. \]
\end{prop}

%Proposition \ref{prop:test:fwer} is a direct consequence of Theorem \ref{theo:stat:global}. Nevertheless, the detailed proof of this proposition is provided in the Appendix. 

The following corollary is an immediate consequence of Proposition \ref{prop:test:fwer} and gives the theoretical justification necessary for making simultaneous confidence statements about the locations of the differences between the trends.

\begin{corollary}\label{corollary1}
Under the conditions of Theorem \ref{theo:stat:global}, for any given $\alpha \in (0,1)$, we have
%\[ \pr\Big( \forall (i,j,k) \in \indexset: \text{ If } |\hat{\psi}_{ijk,T}| > c_{T,\textnormal{Gauss}}(\alpha,h_k), \text{ then } (i,j,k) \notin \indexset_0 \Big) \ge 1 - \alpha + o(1) \]
\begin{align*}
\pr\Big( \forall \, i,j \in \{1, \ldots, n\}, (u, h) \in \grid_T \text{ s.t. } H_0^{[i, j]}(u, h) \text{ is true}: & \, \hat{\psi}^0_{ij,T}(u, h) \le q_{n,T}(\alpha) \Big) \\ & \qquad \quad \quad \ge 1 - \alpha + o(1).
\end{align*}
%Hence, with asymptotic probability at least $1-\alpha$, the two intensity functions $\lambda_i$ and $\lambda_j$ differ on the interval $\mathcal{I}_k$ for all $(i,j,k) \in \indexset$ for which the test rejects $H_0^{(ijk)}$. 
\end{corollary} 


With the help of Corollary \ref{corollary1}, we are able to make simultaneous confidence statements about which of the trends are different and where:

\begin{center}
\begin{minipage}[c][1.75cm][c]{13cm}
\textit{We can state with (asymptotic) probability at least $1-\alpha$ that for all \linebreak $i, j \in \{1,\ldots, n\}$, $m_i(\cdot)$ and $m_j(\cdot)$ differ on all of the intervals $\interval_{(u, h)} \in \mathcal{S}^{[i, j]}(\alpha)$.}
\end{minipage}
\end{center}



\section{Clustering}
\subsection{Clustering of time trends}\label{subsec-test-equality-clustering}


Consider a situation in which the null hypothesis $H_0: m_1 = m_2 = \ldots = m_n$ is violated. Even though some of the trend functions are different in this case, part of them may still be the same. Put differently, there may be groups of time series which have the same time trend. Formally speaking, we define a group structure as follows: There exist sets or groups of time series $G_1,\ldots,G_N$ with $N \le n$ and $\{1,\ldots,n\} = \mathbin{\dot{\bigcup}}_{\ell=1}^{N} G_\ell$ such that for each $1 \le \ell \le N$,
\[ m_i = f_\ell \quad \text{for all } i \in G_\ell, \]
where $f_\ell$ are group-specific trend functions. Hence, the time series which belong to the group $G_\ell$ all have the same time trend $g_\ell$. Throughout the section, we suppose that the group-specific trend functions $f_\ell$ have the following properties: For each $\ell$, $f_\ell = f_{\ell,T}$ is a Lipschitz continuous function with $\int_0^1 f_{\ell,T}(w) dw = 0$. In particular, it holds that $|f_{\ell,T}(v) - f_{\ell,T}(w)| \le L |v-w|$ for all $v,w \in [0,1]$ and some constant $L < \infty$ that does not depend on $T$. Moreover, for any $\ell \ne \ell^\prime$, the trends $f_{\ell,T}$ and $f_{\ell^\prime,T}$ are assumed to differ in the following sense: There exists $(u,h) \in \mathcal{G}_T$ with $[u-h,u+h] \subseteq [0,1]$ such that $f_{\ell,T}(w) - f_{\ell^\prime,T}(w) \ge c_T \sqrt{\log T/(Th)}$ for all $w \in [u-h,u+h]$ or $f_{\ell^\prime,T}(w) - f_{\ell,T}(w) \ge c_T \sqrt{\log T/(Th)}$ for all $w \in [u-h,u+h]$, where $0 < c_T \rightarrow \infty$.


In many applications, it is natural to suppose that there is a group structure in the data. In this case, a particular interest lies in estimating the unknown groups from the data at hand. In what follows, we combine our multiscale methods with a clustering algorithm to achieve this. {\color{red}More specifically, we use the aggregated multiscale statistics $\max_{(u, h) \in \mathcal{G}_T}\hat{\psi}^0_{ij, T}(u, h)$} as distance measures which are fed into a hierarchical clustering algorithm. To describe the algorithm, we first need to introduce the notion of a dissimilarity measure: Let $S \subseteq \{1,\ldots,n\}$ and $S^\prime \subseteq \{1,\ldots,n\}$ be two sets of time series from our sample. We define a dissimilarity measure between $S$ and $S^\prime$ by setting 
{\color{red}\begin{equation}\label{dissimilarity}
\widehat{\Delta}(S,S^\prime) = \max_{\substack{i \in S, \\ j \in S^\prime}} \Big(\max_{(u, h) \in \mathcal{G}_T}\hat{\psi}^0_{ij, T}(u, h)\Big). 
\end{equation}}
This is commonly called a complete linkage measure of dissimilarity. Alternatively, we may work with an average or a single linkage measure. We now combine the dissimilarity measure $\widehat{\Delta}$ with a hierarchical agglomerative clustering (HAC) algorithm which proceeds as follows: 
\vspace{10pt}

\noindent \textit{Step $0$ (Initialization):} Let $\widehat{G}_i^{[0]} = \{ i \}$ denote the $i$-th singleton cluster for $1 \le i \le n$ and define $\{\widehat{G}_1^{[0]},\ldots,\widehat{G}_n^{[0]} \}$ to be the initial partition of time series into clusters. 
\vspace{5pt}

\noindent \textit{Step $r$ (Iteration):} Let $\widehat{G}_1^{[r-1]},\ldots,\widehat{G}_{n-(r-1)}^{[r-1]}$ be the $n-(r-1)$ clusters from the previous step. Determine the pair of clusters $\widehat{G}_{\ell}^{[r-1]}$ and $\widehat{G}_{{\ell}^\prime}^{[r-1]}$ for which 

\[ \widehat{\Delta}(\widehat{G}_{\ell}^{[r-1]},\widehat{G}_{{\ell}^\prime}^{[r-1]}) = \min_{1 \le k < k^\prime \le n-(r-1)} \widehat{\Delta}(\widehat{G}_{k}^{[r-1]},\widehat{G}_{k^\prime}^{[r-1]}) \]  
and merge them into a new cluster. 
\vspace{10pt}

\noindent Iterating this procedure for $r = 1,\ldots,n-1$ yields a tree of nested partitions \linebreak $\{\widehat{G}_1^{[r]},\ldots$ $\ldots,\widehat{G}_{n-r}^{[r]}\}$, which can be graphically represented by a dendrogram. Roughly speaking, the HAC algorithm merges the $n$ singleton clusters $\widehat{G}_i^{[0]} = \{ i \}$ step by step until we end up with the cluster $\{1,\ldots,n\}$. In each step of the algorithm, the closest two clusters are merged, where the distance between clusters is measured in terms of the dissimilarity $\widehat{\Delta}$. We refer the reader to Section 14.3.12 in \cite{HastieTibshiraniFriedman2009} for an overview of hierarchical clustering methods. 


When the number of groups $N$ is known, we estimate the group structure $\{G_1,\ldots, G_N\}$ by the $N$-partition $\{\widehat{G}_1^{[n-N]},\ldots,\widehat{G}_{N}^{[n-N]}\}$ produced by the HAC algorithm. When $N$ is unknown, we estimate it by the $\widehat{N}$-partition $\{\widehat{G}_1^{[n-\widehat{N}]},\ldots,\widehat{G}_{\widehat{N}}^{[n-\widehat{N}]}\}$, where $\widehat{N}$ is an estimator of $N$. The latter is defined as 
\[ \widehat{N} = \min \Big\{ r = 1,2,\ldots \Big| \max_{1 \le \ell \le r} \widehat{\Delta} \big( \widehat{G}_\ell^{[n-r]} \big) \le q_{n,T}(\alpha) \Big\}, \]
where we write $\widehat{\Delta}(S) = \widehat{\Delta}(S,S)$ for short and $q_{n,T}(\alpha)$ is the $(1-\alpha)$-quantile of $\Phi_{n,T}$ defined in Section \ref{subsec:test:test}. 


\newpage
The following proposition summarizes the theoretical properties of the estimators $\widehat{N}$ and $\{ \widehat{G}_1,\ldots,\widehat{G}_{\widehat{N}} \}$, where we use the shorthand $\widehat{G}_\ell = \widehat{G}_\ell^{[n-\widehat{N}]}$ for $1 \le \ell \le \widehat{N}$. 
\begin{prop}\label{prop:clustering:1}
Let the conditions of Theorem \ref{theo:stat:global} be satisfied. Then 
\[ \pr \Big( \big\{ \widehat{G}_1,\ldots,\widehat{G}_{\widehat{N}} \big\} = \{ G_1,\ldots,G_N \} \Big) \ge (1-\alpha) + o(1) \]
and 
\[ \pr \big( \widehat{N} = N \big) \ge (1-\alpha) + o(1). \]
\end{prop}
This result allows us to make statistical confidence statements about the estimated clusters $\{ \widehat{G}_1,\ldots,\widehat{G}_{\widehat{N}} \}$ and their number $\widehat{N}$. In particular, we can claim with asymptotic confidence $\ge 1 - \alpha$ that the estimated group structure is identical to the true group structure. Note that it is possible to let the significance level $\alpha$ depend on the sample size $T$ in Proposition \ref{prop:clustering:1}. In particular, we can allow $\alpha = \alpha_T$ to converge slowly to zero as $T \rightarrow \infty$, in which case we obtain that $\pr ( \{ \widehat{G}_1,\ldots,\widehat{G}_{\widehat{N}} \} = \{ G_1,\ldots,G_N \} ) \rightarrow 1$ and $\pr ( \widehat{N} = N ) \rightarrow 1$. The proof of Proposition \ref{prop:clustering:1} can be found in the Appendix.

Our multiscale methods do not only allow us to compute estimators of the unknown groups $G_1,\ldots,G_N$. They also provide information on the locations where two group-specific trend functions $f_\ell$ and $f_{\ell^\prime}$ differ from each other. To turn this claim into a mathematically precise statement, we need to introduce some notation. First of all, note that the indexing of the estimators $\widehat{G}_1,\ldots,\widehat{G}_{\widehat{N}}$ is completely arbitrary. We could, for example, change the indexing according to the rule $\ell \mapsto \widehat{N} - \ell + 1$. In what follows, we suppose that the estimated groups are indexed such that $P( \widehat{G}_\ell = G_\ell \text{ for all } \ell ) \ge (1-\alpha) + o(1)$. Proposition \ref{prop:clustering:1} implies that this is possible without loss of generality. Keeping this convention in mind, we define the sets 
\[ \mathcal{A}_{n,T}(\ell,\ell^\prime) = \Big\{ (u,h) \in \mathcal{G}_T: \Big| \frac{\widehat{\psi}_{ij,T}(u,h)}{(\widehat{\sigma}_i^2 + \widehat{\sigma}_j^2)^{1/2}} \Big| > q_{n,T}(\alpha) + \lambda(h) \text{ for some } i \in \widehat{G}_\ell, j \in \widehat{G}_{\ell^\prime} \Big\} \] 
and  
\[ \Pi_{n,T}(\ell,\ell^\prime) = \big\{ I_{u,h} = [u-h,u+h]: (u,h) \in \mathcal{A}_{n,T}(\ell,\ell^\prime) \big\} \]
for $1 \le \ell < \ell^\prime \le \widehat{N}$. An interval $I_{u,h}$ is contained in $\Pi_{n,T}(\ell,\ell^\prime)$ if our multiscale test indicates a significant difference between the trends $m_i$ and $m_j$ on the interval $I_{u,h}$ for some $i \in \widehat{G}_\ell$ and $j \in \widehat{G}_{\ell^\prime}$. Put differently,  $I_{u,h} \in \Pi_{n,T}(\ell,\ell^\prime)$ if the test suggests a significant difference between the trends of the $\ell$-th and the $\ell^\prime$-th group on the interval $I_{u,h}$. We further let
\[ E_{n,T}(\ell,\ell^\prime) = \Big\{ \forall I_{u,h} \in \Pi_{n,T}(\ell,\ell^\prime): f_\ell(v) \ne f_{\ell^\prime}(v) \text{ for some } v \in I_{u,h} = [u-h,u+h] \Big\} \]
be the event that the group-specific time trends $f_\ell$ and $f_{\ell^\prime}$ differ on all intervals $I_{u,h} \in \Pi_{n,T}(\ell,\ell^\prime)$. With this notation at hand, we can make the following formal statement whose proof is given in the Appendix. \textcolor{blue}{(add proof)} 
\begin{prop}\label{prop:clustering:2}
Under the conditions of Theorem \ref{theo:stat:global}, the event 
\[ E_{n,T} = \Big\{ \bigcap_{1 \le \ell < \ell^\prime \le \widehat{N}} E_{n,T}(\ell,\ell^\prime) \Big\} \cap \Big\{ \widehat{N} = N \text{ and } \widehat{G}_\ell = G_\ell \text{ for all } \ell \Big\} \]
asymptotically occurs with probability $\ge 1-\alpha$, that is, 
\[ \pr \big( E_{n,T} \big) \ge (1 - \alpha) + o(1). \]
\end{prop}
The statement of Proposition \ref{prop:clustering:2} remains to hold true when the sets of intervals $\Pi_{n,T}(\ell,\ell^\prime)$ are replaced by the corresponding sets of minimal intervals. \textcolor{blue}{(add definition of the minimal intervals)}  According to Proposition \ref{prop:clustering:2}, the sets $\Pi_{n,T}(\ell,\ell^\prime)$ allow us to locate, with a pre-specified confidence, time regions where the group-specific trend functions $f_\ell$ and $f_{\ell^\prime}$ differ from each other. In particular, we can claim with asymptotic confidence $\ge 1 - \alpha$ that the trend functions $f_\ell$ and $f_{\ell^\prime}$ differ on all intervals $I_{u,h} \in \Pi_{n,T}(\ell,\ell^\prime)$.


\section{\textcolor{red}{Applications}}\label{sec:app}
\subsection{Analysis of the GDP growth}\label{subsec:app:gdp}


To illustrate our test method from Section \ref{sec:test}, we repeat an application example from \cite{Zhang2012}, where the authors test the hypothesis of a common trend in the GDP growth data for $16$ OECD countries. Since we do not have access neither to the original dataset from \cite{Zhang2012}, nor to the exact data specifications, we perform our analysis on the data available from the most commonly used sources: Barro-Lee Educational Attainment dataset \citep*{Barro2013}, Refinitiv Datastream, OECD.Stat database, and Federal Reserve Economics Data (FRED). In our illsutration example, we consider the specification of the data that is as close as possible to the one in \linebreak \cite{Zhang2012} with one important distinction. In the original study, the authors examine 16 OECD countries (not specifying which ones), whereas we consider only $11$ OECD countries: Australia, Austria, Canada, Finland, France, Germany, Japan, Norway, Switzerland, UK, and USA. The reason for focusing our attention on $11$ countries lies in the fact that only for these countries there exist quarterly data of good quality covering the time period used in the original study: from the fourth quarter of 1975 up to and including the third quarter of 2010. In the appendix, we repeat the analysis for these $11$ countries plus additional $5$ countries for which some of the time series contain missing values. We use linear extrapolation to impute the missing values for these countries. Details how this extrapolation is done are deferred to the Appendix.

For this example, we collect the data from multiple sources. In the following list, we provide the specifications for the variables that we use in our analysis.

\begin{enumerate}
\item \textbf{GDP:} We use data on the Gross Domestic Product - Expenditure Approach ($GDP$) from the OECD.Stat \nocite{OECD} database (https://stats.oecd.org/Index.aspx). The data are freely available and were accessed on 7 December 2021. To be as close as possible to the specification of the data in \cite{Zhang2012}, we use the seasonally adjusted quarterly data on the GDP expressed in millions of \linebreak 2015 US dollars.\footnote{Since the publication of the original paper in 2013, the OECD reference year has changed from 2005 to 2015. We have decided to analyse the latest version of the data in order to be able to make more accurate and up-to-date conclusions.} The data span from 1960 to 2021 which fully covers the time period necessary for our analysis (1975Q4 to 2010Q3). 
%\item \textbf{Capital:} We use data on Capital Stock at Constant National Prices ($K$) from the Penn World Table \citep{Feenstra2015} retrieved from FRED \linebreak (https://fred.stlouisfed.org/, accessed on 7~December 2021). The data are given in the millions of the $2017$ US dollars. Since the observations are at an annual frequency, we linear intrapolate the data to obtain the quarterly values. Our approach is different from the one in \cite{Zhang2012} where they use the quarterly data from the beginning. The reason for such discrepancy is the fact that we have not found the quarterly time series of sufficient quality on Capital Stock at Constant National Prices in the common sources. We repeat the analysis using the quarterly data on Gross Fixed Capital Formation instead of Capital Stock in the Appendix.
\item \textbf{Capital:} We use data on Gross Fixed Capital Formation ($K$) from the OECD.Stat \nocite{OECD} database (https://stats.oecd.org/Index.aspx). The data are freely available and were accessed on 7~December 2021. The values are at a quarterly frequency, seasonally adjusted, and expressed in the millions of the $2015$ US dollars.  Our approach is different from the one in \cite{Zhang2012} where they use the data on Capital Stock at Constant National Prices as the capital variable. The reason for this discrepancy is the fact that we have not found the quarterly time series of sufficient quality on capital stock in the common sources. It is worth noting that since accurate data on capital stock is notoriously difficult to collect, the use of gross fixed capital formation as a measure of capital while explaining economic growth stock is standard in the literature (\cite{Sharma1994}, \cite{Lee2002}, \cite{Lee2005}).
\item \textbf{Labour:} We collect the data on the Number of Employed People ($L$) from various sources. For most of the countries (Austria, Australia, Canada, Germany, Japan, UK and the USA) we download the OECD data on Employed Population: Aged 15 and Over retrieved from FRED (https://fred.stlouisfed.org/, accessed on 7~December 2021). \nocite{OECDempl} The data for France and Switzerland were downloaded from Refinitiv Datastream on 7~December~2021. For all of the aforementioned countries the observations are at quarterly frequency and seasonally adjusted. The data for Finland and Norway were also obtained via Refinitiv Datastream on 7~December~2021, however, the only quarterly time series that are long enough to fully cover the time period under consideration are not seasonally adjusted. Hence, for these two countries we perform the seasonal adjustment ourselves. We do that using the default method of the function \verb|seas| from an \verb|R| package \verb|seasonal| \citep*{Sax2018} which is an interface to X-13-ARIMA-SEATS, the seasonal adjustment software used by the US Census Bureau. We repeat the analysis using not seasonally adjusted data for all of the $11$ countries as a robustness check and we report the results in the Appendix.

For all of the countries, the observations are given in thousands of persons.
\item \textbf{Human capital:} We use Educational Attainment for Population Aged 25 and Over ($H$) collected from http://www.barrolee.com (accessed on 7 December 2021) as a measure of human capital. Since the only available data is five-year census data, we follow \cite{Zhang2012} and use linear intropolation between the observation and constant extrapolation on the boundaries (second and third quarters of 2010) to obtain the quarterly observations.
\end{enumerate}


\begin{figure}[t]
\centering
\includegraphics[width=0.8\textwidth]{Plots/smoothed_gdp_data.pdf}
\vspace{0.2cm}
\caption{Local linear kernel estimates of the $n=11$ original time trends from the application of Section \ref{subsec:app:gdp}. Each panel shows the estimates for a different bandwidth $h$.}\label{plot:app:gdp}
\end{figure}

\begin{figure}[t]
\centering
\includegraphics[width=0.8\textwidth]{Plots/smoothed_gdp_data_augmented.pdf}
\vspace{0.2cm}
\caption{Local linear kernel estimates of the $n=11$ augmented time trends from the application of Section \ref{subsec:app:gdp}. Each panel shows the estimates for a different bandwidth $h$.}\label{plot:app:gdp_augm}
\end{figure}


We thus observe a panel of $n = 11$ time series $\mathcal{T}_i = \{(Y_{it}, \X_{it}): 1 \le t \le T \}$ of length $T = 140$ for each country $i \in \{1,\ldots,11\}$, where $Y_{it} = \Delta \ln GDP_{it} := \ln GDP_{it} - \ln GDP_{i(t-1)}$, $\X_{it} = (\Delta \ln L_{it}, \Delta \ln K_{it}, \Delta \ln H_{it})^\top$ with $\Delta \ln L_{it} := \ln L_{it} - \ln L_{i(t-1)}$, $\Delta \ln K_{it} := \ln K_{it} - \ln K_{i(t-1)}$ and $\Delta \ln H_{it} := \ln H_{it} - \ln H_{i(t-1)}$. Without loss of generality, we let $\Delta \ln GDP_{i1} = \Delta \ln L_{i1} = \Delta \ln K_{i1} = \Delta \ln H_{i1} = 0$. The time series $\mathcal{T}_i$ is assumed to follow the model 
\begin{equation}\label{eq:model:app}
Y_{it} = \bfbeta^\top_i \X_{it} + m_i \Big( \frac{t}{T} \Big) + \alpha_i + \varepsilon_{it} 
\end{equation}
for $1 \le t \le T$, where $\bfbeta_i = (\beta_{i, 1}, \beta_{i, 2}, \beta_{i, 3})^\top$ is a vector of unknown parameters, $m_i$ is a country-specific unknown nonparametric time trend, and $\alpha_i$ is a fixed-effect term. Similarly to \cite{Zhang2012}, we rewrite the model \eqref{eq:model:app} as
\begin{equation}\label{eq:model:app2}
\Delta \ln GDP_{it} = \beta_{i, 1} \Delta \ln L_{it} + \beta_{i, 2} \Delta \ln K_{it} + \beta_{i, 3} \Delta \ln H_{it} + m_i \Big( \frac{t}{T} \Big) + \alpha_i + \varepsilon_{it},
\end{equation}

for $i \in \{1, \ldots, 11\}$ and $t \in \{1, \ldots, 140\}$.

\textcolor{red}{Our test procedure depends on the estimator of the long-run variance $\sigma_i^2$ for the error process $\mathcal{E}_i = \{\varepsilon_{it}: 1 \le t \le T\}$. As was mentioned in Section \ref{subsec:test:prep}, without imposing any assumptions on the error process, the estimators tend to be imprecise. In order to obtain an estimator with a better finite sample properties, we impose a time series structure on the error process. Specically, we assume that for each $i$ the error process $\mathcal{E}_i$ has the AR($p_i$) structure $\varepsilon_{it} = \sum_{j=1}^{p_i} a_{i, j} \varepsilon_{i(t-j)} + \eta_{it}$, where the order of the process $p_i$ is country-specific and not known and $\eta_{it}$ are i.i.d.\ innovations with mean zero. We choose $p_i$ for each country separately based on the values of the information criterion.} \textcolor{blue}{(need to elaborate here)}. \textcolor{red}{This assumption allows us to use a difference-based long-run variance estimator proposed in \cite{KhismatullinaVogt2020} with the choice of the tuning parameters described below.}

We aim to test whether the time trend $m_i$ is the same at each of the $11$ countries. In other words, we want to test the null hypothesis $H_0: m_1 = \ldots = m_n$ with $n = 11$ in model \eqref{eq:model:app2}. To do so, we implement the multiscale test from Section \ref{sec:test} in the following way. 

\begin{enumerate}
\item We choose $K$ to be an Epanechnikov kernel.
\item We let $\mathcal{G}_T = U_T \times H_T$ with 
\begin{align*}
U_T & = \big\{ u \in [0,1]: u = \textstyle{\frac{4t}{T}} \text{ for some } t \in \naturals \big\} \\
H_T & = \big\{ h \in \big[ \textstyle{\frac{\log T}{T}}, \textstyle{\frac{1}{4}} \big]:  h = \textstyle{\frac{4\ell}{T}} \text{ for some } \ell \in \naturals \big\}. 
\end{align*}
We thus take into account all locations $u$ on an equidistant grid $U_T$ with step length $4/T$ and all bandwidths $h=4/T, 8/T, 12/T,\ldots$ with $\log T /T \le h \le 1/4$. Note that the lower bound $\log T / T$ is motivated by \ref{C-h} which requires that $\log T /T \ll h_{\min}$ (given that all moments of $\varepsilon_{it}$ exist).
\item We estimate the unknown parameters $\bfbeta_i = (\beta_{i, 1}, \beta_{i, 2}, \beta_{i, 3})^\top$ for each country $i$ separately using the first-differencing approach described in Section \ref{subsec:test:prep}.
\item We compute the estimator of the fixed-effect term $\alpha_i$ using the formula \eqref{eq:alpha:est}. We then work with the augmented time series $\widehat{Y}_{it} = Y_{it} - \widehat{\bfbeta}_i^\top \X_{it} - \widehat{\alpha}_{i}$ instead of the original data on the GDP growth rate $Y_{it}$.
\item To obtain the estimator $\hat{\sigma}_i^2$ for the long-run error variance $\sigma^2_i$, for each $i$ we apply the procedure from \cite{KhismatullinaVogt2020} to the augmented values \linebreak $\{\widehat{Y}_{it}: 1\leq t \leq T\}$ with the following specification of the parameters: $\underline{r}=1$, $\overline{r}=10$ and $q = 15$. We use $\hat{\sigma}^2_i$ for calculating the value of our main test statistic $\widehat{\Psi}_{n, T}$. %In the Appendix, we repeat the analysis using the averaged estimator $\hat{\sigma}^2 = \frac{1}{n}\sum_{i=1}^n \hat{\sigma}_i^2$.
\item The significance levels are taken to be $\alpha = 0.05$ and $\alpha = 0.1$ (\citet{Zhang2012} report the results for testing at a significance level $\alpha = 0.1$).
\item To obtain the (approximate) critical values $q_{n, T}(\alpha)$ of the multiscale test, we simulate $5000$ values of the statistic $\Phi_{n, T}$ defined in Section \ref{subsec:test:test} and compute their empirical $(1-\alpha)$ quantile $\hat{q}_{n, T}(\alpha)$.
\end{enumerate}

 %We implement the test in the same way as in the simulations of Section \ref{sec:sim}. 


Figure \ref{plot:app:gdp} depicts smoothed versiosn of the original time series on the GDP growth rate $\{Y_{it} = \Delta \ln GDP_{it}: 1 \le t \le T\}$ for each country $i$ of the $n=11$ countries under consideration. Figure \ref{plot:app:gdp_augm} presents local linear estimates of the trend functions $m_i$ for these countries after factoring out the effects of the covariates and the fixed-effect terms (i.e. calculated from the augmented time series~$\widehat{Y}_{it}$). In both figures, each panel corresponds to a different value of the bandwidth~$h$.

As can be seen in Figure \ref{plot:app:gdp}, in the original data on the GDP growth there are some notable diffrences between the countries for bandwidths $h = 0.05$ and $h = 0.1$. For example, while most of the countries experience the increase in the growth rate in the last two years, the data for one of the countries (Norway) suggests a decrease in the same period of time. Moreover, using the smallest bandwidth ($h= 0.05$) allows us to notice considerable differences in the behaviour of the time series in the middle of our time region. In constrast, the value of the bandwidth $h = 0.15$ is too big to detect any heterogeneity in the behaviour of the trends. As we see, the choice of the bandwidth is crucial in making conclusions in this example.

\begin{figure}[p!]
\begin{minipage}[t]{0.49\textwidth}
\includegraphics[width=\textwidth]{Plots/AUT_vs_NOR}
\caption{Test results for the comparison of Austria and Norway.}\label{fig:Austria:Norway}
\end{minipage}
\hspace{0.25cm}
\begin{minipage}[t]{0.49\textwidth}
\includegraphics[width=\textwidth]{Plots/CAN_vs_NOR}
\caption{Test results for the comparison of Canada and Norway.}\label{fig:Canada:Norway}
\end{minipage}
\caption*{Note: In each figure, panel (a) shows the two augmented time series, panel (b) presents smoothed versions of the augmented time series, and panel (c) depicts the set of intervals $\mathcal{S}^{[i,j]}(\alpha)$ in grey and the subset of minimal intervals $\mathcal{S}^{[i,j]}_{min}(\alpha)$ in black.}
\end{figure}

\begin{figure}[p!]
\begin{minipage}[t]{0.49\textwidth}
\includegraphics[width=\textwidth]{Plots/CHE_vs_NOR}
\caption{Test results for the comparison of Switzerland and Norway.}\label{fig:Switzerland:Norway}
\end{minipage}
\hspace{0.25cm}
\begin{minipage}[t]{0.49\textwidth}
\includegraphics[width=\textwidth]{Plots/DEU_vs_NOR}
\caption{Test results for the comparison of Germany and Norway.}\label{fig:Germany:Norway}
\end{minipage}
\caption*{Note: In each figure, panel (a) shows the two augmented time series, panel (b) presents smoothed versions of the augmented time series, and panel (c) depicts the set of intervals $\mathcal{S}^{[i,j]}(\alpha)$ in grey and the subset of minimal intervals $\mathcal{S}^{[i,j]}_{min}(\alpha)$ in black.}
\end{figure}

\begin{figure}[p!]

\begin{minipage}[t]{0.49\textwidth}
\includegraphics[width=\textwidth]{Plots/FRA_vs_NOR}
\caption{Test results for the comparison of France and Norway.}\label{fig:France:Norway}
\end{minipage}
\hspace{0.25cm}
\begin{minipage}[t]{0.49\textwidth}
\includegraphics[width=\textwidth]{Plots/USA_vs_NOR}
\caption{Test results for the comparison of the USA and Norway.}\label{fig:USA:Norway}
\end{minipage}

\caption*{Note: In each figure, panel (a) shows the two augmented time series, panel (b) presents smoothed versions of the augmented time series, and panel (c) depicts the set of intervals $\mathcal{S}^{[i,j]}(\alpha)$ in grey and the subset of minimal intervals $\mathcal{S}^{[i,j]}_{min}(\alpha)$ in black.}
\end{figure}

\begin{figure}[p!]
\begin{minipage}[t]{0.49\textwidth}
\includegraphics[width=\textwidth]{Plots/JPN_vs_NOR}
\caption{Test results for the comparison of Japan and Norway.}\label{fig:Japan:Norway}
\end{minipage}
\hspace{0.25cm}
\begin{minipage}[t]{0.49\textwidth}
\includegraphics[width=\textwidth]{Plots/FRA_vs_USA}
\caption{Test results for the comparison of France and the USA.}\label{fig:France:USA}
\end{minipage}
\caption*{Note: In each figure, panel (a) shows the two augmented time series, panel (b) presents smoothed versions of the augmented time series, and panel (c) depicts the set of intervals $\mathcal{S}^{[i,j]}(\alpha)$ in grey and the subset of minimal intervals $\mathcal{S}^{[i,j]}_{min}(\alpha)$ in black.}
\end{figure}


Now we look at the local linear estimates of the trend functions $m_i$ after excluding the effects of the covariates (Figure \ref{plot:app:gdp_augm}). We can see that the differences between the trends are much more pronounced and some heterogeneity between the countries is notable even for large values of $h$.

%We are now ready to apply the test procedure to the data. 
The results of applying our test are completely in line with the conclusions from the visual inspection. The test rejects the null hypothesis $H_0$ at both levels $\alpha =0.05$ and $\alpha = 0.1$ which is consistent with the findings of \cite{Zhang2012} where the authors report a rejection of the null hypothesis of a common trend at the level $\alpha = 0.1$. However, in contrast to \cite{Zhang2012}, we can say much more about the comparison of the trends between these $11$ countries. Specifically, we can make simultaneous confidence statements abouth which of the countries have different trends and where they differ. With the help of our multiscale method, we simultaneously test all of the local null hypotheses $H_0^{[i,j]}(u, h)$ that $m_i = m_j$ on the interval $\mathcal{I}_{(u, h)} = [u-h, u+h]$ for each $i, j, 1 \le i < j \le n$, and each $(u, h) \in \mathcal{G}_T$. The results are presented in Figures \ref{fig:Austria:Norway}--\ref{fig:France:USA}, with each figure corresponing to a specific pair of countries $(i,j)$ from our sample. Each figure is divided into three panels (a)--(c).  Panel (a) shows the augmented time series $\widehat{Y}_{it}$ and $\widehat{Y}_{jt}$ for the two countries $i$ and $j$ that are being compared. Panel (b) presents smoothed versions of the time series from (a), that is, it shows nonparametric kernel estimates (specifically, Nadaraya-Watson estimates) of the two trend functions $m_i$ and $m_j$, where the bandwidth is set to $14$ quarters (which corresponds to $h = 0.1$) and a rectangular kernel is used. Finally, panel (c) presents the results produced by our test for a significance level $\alpha = 0.05$: it depicts in grey the set $\mathcal{S}^{[i,j]}(\alpha)$ of all the intervals for which the test rejects the local null $H_0^{[i, j]}(u, h)$. The minimal intervals in the subset $\mathcal{S}^{[i,j]}(\alpha)$ are highlighted in black. Note that according to \eqref{corollary1}, we can make the following simultaneous confidence statement about the intervals plotted in panels (c) of Figures \ref{fig:Austria:Norway}--\ref{fig:France:USA}: we can claim, with confidence of about $95\%$, that there is a difference between the functions $m_i$ and $m_j$ on each of these intervals. 

Now we briefly comment on the results. Out of $55$ pairwise comparisons, our test detects the differences in the trends $8$ times. In $7$ such cases, one of the countries involved in the comparison is Norway. First, we look closely at these $7$ cases. It is visible from Figures \ref{fig:Austria:Norway} -- \ref{fig:Japan:Norway} that the differences in the trends are more pronounced towards the end of the considered period: the set of intervals where our test rejects always covers last $11$ years from last quarter in $1998$ up to the third quarter in $2010$. This coincides with tour visual observation that in the last $10$ years, only Norway exhibited downward movement of the trend in the GDP growth which became even more pronounced after factoring out the effect of the individual covariates. All of the other countries appear to have a slightly increasing trend function in the same time period. Hence, our test finds significant differences between the trends of Norway and several other countries in this time region.

Figure \ref{fig:France:USA} presents the results of the pairwise comparison between France and the USA. In this case, our test detects the differences between the underlying trends in the GDP growth only in the beginning of the considered time period. Specifically, we can claim with confidence no less than 95\%, that there are differences between the trends for France and the USA up to the fourth quarter 1986 but there is no evidence of any significant differences between the trends in the period of 1987 onwards.

\subsection{Analysis of the house prices}\label{subsec:app:house}
We next analyse the historical dataset on housing prices for OECD countries that covers time period 1870 -- 2012. We restrict our attention to the years 1890 -- 2012 since it implies almost no additional manipulations with the data.

We focus our attention on 8 countries (Australia, Belgium, Denmark, France, the Netherlands, Norway, Sweden and the USA) for which the imputation of the data is minimal. Specifically, for all of the countries except Belgium the time series cover the whole time period under consideration (1890 - 2012), and for Belgium there are only five missing values\footnote{The missing values in the time series for Belgium span five years during the World War I.} which we fill using linear interpolation from the nearest years. Time series for other countries in the dataset \cite{Knoll2017}



Data for consumer price index (CPI), real GDP ($GDP$), population size ($POP$) and long-term interest rate ($I$) are taken from the Jordà-Schularick-Taylor Macrohistory Database,\footnote{See \cite{Jorda2017} for the detailed descriprion of the variable construction.} which is freely available at \linebreak \url{http://www.macrohistory.net/data/} (accessed on 13 January 2022).

We deflate the nominal house prices from  \cite{Knoll2017} with the corresponding CPI to obtain real house prices ($HP$). 

Inflation ($INFL$) is measured as change in CPI.

We thus observe a panel of $n = 8$ time series $\mathcal{T}_i = \{(Y_{it}, \X_{it}): 1 \le t \le T \}$ of length \linebreak$T = 123$ for each country $i \in \{1,\ldots,8\}$, where $Y_{it} = \ln HP_{it}$, \linebreak $\X_{it} = (\ln GDP_{it}, \ln POP_{it}, I_{it}, INFL_{it})^\top$. The time series $\mathcal{T}_i$ is assumed to follow the model 
\begin{equation}\label{eq:model:app3}
Y_{it} = \bfbeta^\top_i \X_{it} + m_i \Big( \frac{t}{T} \Big) + \alpha_i + \varepsilon_{it} 
\end{equation}
for $1 \le t \le T$, where $\bfbeta_i = (\beta_{i, 1}, \beta_{i, 2}, \beta_{i, 3}, \beta_{i, 4})^\top$ is a vector of unknown parameters, $m_i$ is a country-specific unknown nonparametric time trend, and $\alpha_i$ is a fixed-effect term. As in Section \ref{subsec:app:gdp}, we rewrite the model \eqref{eq:model:app2} as
\begin{equation}\label{eq:model:app4}
\ln HP_{it} = \beta_{i, 1} \ln GDP_{it} + \beta_{i, 2} \ln POP_{it} + \beta_{i, 3} I_{it} + \beta_{i, 4} INFL_{it} + m_i \Big( \frac{t}{T} \Big) + \alpha_i + \varepsilon_{it}.
\end{equation}

Including nonparametric trend function is justified by the evidence found in \cite{Zhang2016}, where the authors claim that the time series for the logarithmic US house prices are trend stationary, i.e. it can be transformed into a stationary series by subtracting the deterministic trend. They find the evidence for this claim in a 120-year national dataset and metro area level and state level panel data sets. 

We implement the multiscale test in exactly the same way as in the previous application example.

Our findings coincide with the observations in \cite{Knoll2017}, where the authors find considerable cross-country heterogeneity in the trends in house prices.

\begin{figure}[p!]
\begin{minipage}[t]{0.49\textwidth}
\includegraphics[width=\textwidth]{Plots/hp_AUS_vs_NLD}
\caption{Test results for the comparison of the housing prices in Australia and the Netherlands.}\label{fig:hp:Australia:Netherlands}
\end{minipage}
\hspace{0.25cm}
\begin{minipage}[t]{0.49\textwidth}
\includegraphics[width=\textwidth]{Plots/hp_BEL_vs_DNK}
\caption{Test results for the comparison of the housing prices in Belgium and Denmark.}\label{fig:hp:Belgium:Denmark}
\end{minipage}
\caption*{Note: In each figure, panel (a) shows the two augmented time series of the house prices, panel (b) presents smoothed versions of the augmented time series, and panel (c) depicts the set of intervals $\mathcal{S}^{[i,j]}(\alpha)$ in grey and the subset of minimal intervals $\mathcal{S}^{[i,j]}_{min}(\alpha)$ in black.}
\end{figure}

\begin{figure}[p!]
\begin{minipage}[t]{0.49\textwidth}
\includegraphics[width=\textwidth]{Plots/hp_BEL_vs_NLD}
\caption{Test results for the comparison of the housing prices in Belgium and the Netherlands.}\label{fig:hp:Belgium:Netherlands}
\end{minipage}
\hspace{0.25cm}
\begin{minipage}[t]{0.49\textwidth}
\includegraphics[width=\textwidth]{Plots/hp_BEL_vs_USA}
\caption{Test results for the comparison of the housing prices in Belgium and the USA.}\label{fig:hp:Belgium:USA}
\end{minipage}
\caption*{Note: In each figure, panel (a) shows the two augmented time series of the house prices, panel (b) presents smoothed versions of the augmented time series, and panel (c) depicts the set of intervals $\mathcal{S}^{[i,j]}(\alpha)$ in grey and the subset of minimal intervals $\mathcal{S}^{[i,j]}_{min}(\alpha)$ in black.}
\end{figure}

\section{Conclusion}\label{sec:conclusion}

In this paper, we develop a new multiscale testing procedure for multiple time series for testing hypotheses about nonparametric time trends in the presence of covariates. This procedure addresses two important statistical problems about comparison of the time trends. First and foremost, with the help of the proposed method, we are able to test if all the time trends in the observed time series are the same or not. We prove the main theoretical results of the paper that the test has (asymptotically) the correct size and has an (asymptotic) power of one against a specific class of local alternatives. Second, our multiscale procedure allows us to tell which of the time trends are different and where the differences are located. For the purpose of pinpointing the differences, we consider many local null hypotheses at the same time, each corresponding to only a pair of time trends and a specific time interval. Our method allows us to test all of these hypotheses simultaneously controling the family-wise error rate, i.e. the probability of wrongly rejecting at least one true null hypothesis (making at least one type I error), at a desired level $\alpha$. This result allows us to make simultaneous confidence statements as follows: 

\begin{center}
\begin{minipage}[c][1.75cm][c]{14cm}
\textit{We can state with (asymptotic) probability at least $1-\alpha$ that for every pair of time series and every interval where our test rejects the local null, the trends of these time series differ at least somewhere on this particular interval.}
\end{minipage}
\end{center}

For the proof of the theoretical results, the main tools that are used are strong approximation theory developed in \cite{BerkesLiuWu2014} and the anti-concentration bounds for Gaussian random vectors verified in \cite{Chernozhukov2015}. The proof strategy that we employ in our paper has already been used in \cite{KhismatullinaVogt2020}, however, in that paper the authors proposed a multiscale method for testing qualitative hypotheses only about one time series. Our method can be regarded as a generalized version of the test developed in \cite{KhismatullinaVogt2020} where we not only consider comparison between various time series, but also add the covariates to the model and propose an estimation procedure for the unknown parameters.

Regarding future research, this project suggests some interesting issues and topics for consideration. \textcolor{red}{First, as was already mentioned, it should be possible to extend our theoretical results to the case where the number of time series slowly grows with the sample size. Second, the theory in this paper is developed under the assumption that the first differences of the covariates and of error terms are uncorrelated. This restriction limits possible applications of our method. Further insight can be gained by broadening the current work in these and possibly other directions.}

\newpage
\bibliographystyle{ims}
{\small
\setlength{\bibsep}{0.55em}
\bibliography{bibliography}}

\allowdisplaybreaks[3]
\section{Appendix}\label{sec-appendix}

\subsection{Proof of Theorem \ref{theo-regs}}\label{subsec-appendix-estimators}

We define the first-differenced regressors as follows.
\[ \Delta \mathbf{X}_{it} =\mathbf{H}_i(\mathcal{U}_{it}) - \mathbf{H}_i(\mathcal{U}_{it-1}) := \Delta \mathbf{H}_i(\mathcal{U}_{it}). \]
Similarly, 
\[\Delta \varepsilon_{it} = \varepsilon_{it} - \varepsilon_{it-1} = G_i(\mathcal{J}_{it}) - G_i(\mathcal{J}_{it-1}) = \Delta G_i(\mathcal{J}_{it}).
\]
 
With these assumptions we can prove the following propositions.
\begin{prop}\label{prop-reg-1}
Under Assumptions \ref{C-reg1} and \ref{C-reg3}, $|| \Delta \mathbf{H}_i(\mathcal{U}_{it})||_4 < \infty$.
\end{prop}

\begin{proof}[\textnormal{\textbf{Proof of Proposition \ref{prop-reg-1}}}]
By Assumption \ref{C-reg3},
\[
 || \Delta \mathbf{H}_i(\mathcal{U}_{it})||_4 \leq  ||\mathbf{H}_i(\mathcal{U}_{it})||_4 +  || \mathbf{H}_i(\mathcal{U}_{it-1})||_4 < \infty.
\]
\end{proof} 

\begin{prop}\label{prop-reg-2}
Under Assumption \ref{C-reg4}, $\Delta \mathbf{X}_{it}$ (elementwise) and $\Delta \varepsilon_{it}$ are uncorrelated for each $t\in \{1, \ldots, T\}$.
\end{prop}

\begin{proof}[\textnormal{\textbf{Proof of Proposition \ref{prop-reg-2}}}]
By Assumption \ref{C-reg4},
\begin{align*}
\ex [\Delta \mathbf{X}_{it}\Delta \varepsilon_{it}] &= \ex \big[(\mathbf{X}_{it} - \mathbf{X}_{it-1}) (\varepsilon_{it} - \varepsilon_{it-1})\big] =\\
=&\ex[\mathbf{X}_{it}  \varepsilon_{it}] - \ex[\mathbf{X}_{it-1}  \varepsilon_{it}]- \ex[\mathbf{X}_{it}  \varepsilon_{it-1}] + \ex[\mathbf{X}_{it-1}  \varepsilon_{it-1}] =\\
=& \ex[\mathbf{X}_{it}]\ex[  \varepsilon_{it}] - \ex[\mathbf{X}_{it-1}]\ex[  \varepsilon_{it}]- \ex[\mathbf{X}_{it}]\ex[  \varepsilon_{it-1}] + \ex[\mathbf{X}_{it-1}]\ex[  \varepsilon_{it-1}] =\\
=& \big( \ex[\mathbf{X}_{it}] - \ex[\mathbf{X}_{it-1}]\big)\big(\ex[  \varepsilon_{it}]  - \ex[\varepsilon_{it-1}]\big) = \ex [\Delta \mathbf{X}_{it}]\ex[\Delta \varepsilon_{it}]
\end{align*}
\end{proof} 


\begin{prop}\label{prop-reg-3}
Define 
\[ \Delta \mathbf{U}_i(\mathcal{I}_{i, t}) := \Delta \mathbf{H}_i(\mathcal{U}_{it}) \Delta G_i(\mathcal{J}_{it}).
\]
Under Assumptions \ref{C-err2} - \ref{C-err3}, \ref{C-reg5} - \ref{C-reg6}, we have that $\sum_{s=0}^\infty \delta_2(\Delta \mathbf{U}_i, s) < \infty$.
\end{prop}

\begin{proof}[\textnormal{\textbf{Proof of Proposition \ref{prop-reg-3}}}]
Note the following
\begin{align*}
 &\delta_2(\Delta \mathbf{U}_i, t) = || \Delta\mathbf{U}_i(\mathcal{I}_{i, t}) - \Delta \mathbf{U}_i(\mathcal{I}_{i,t}^\prime) ||_2 =\\
 &= || \Delta \mathbf{H}_i(\mathcal{U}_{it}) \Delta G_i(\mathcal{J}_{it}) -  \Delta \mathbf{H}_i(\mathcal{U}_{it}^\prime) \Delta G_i(\mathcal{J}_{it}^\prime) ||_2 =\\
 & = ||\big(\mathbf{H}_i(\mathcal{U}_{it}) - \mathbf{H}_i(\mathcal{U}_{it-1})\big)\big(G_i(\mathcal{J}_{it}) - G_i(\mathcal{J}_{it-1})\big) - \big(\mathbf{H}_i(\mathcal{U}_{it}^\prime) - \mathbf{H}_i(\mathcal{U}_{it-1}^\prime)\big)\big(G_i(\mathcal{J}_{it}^\prime) - G_i(\mathcal{J}_{it-1}^\prime)\big)||_2 =\\
 &= ||\mathbf{H}_i(\mathcal{U}_{it})G_i(\mathcal{J}_{it}) - \mathbf{H}_i(\mathcal{U}_{it-1})G_i(\mathcal{J}_{it}) - \mathbf{H}_i(\mathcal{U}_{it})G_i(\mathcal{J}_{it-1})  + \mathbf{H}_i(\mathcal{U}_{it-1})G_i(\mathcal{J}_{it-1}) - \\
 &\quad - \mathbf{H}_i(\mathcal{U}_{it}^\prime)G_i(\mathcal{J}_{it}^\prime) + \mathbf{H}_i(\mathcal{U}_{it-1}^\prime)G_i(\mathcal{J}_{it}^\prime) + \mathbf{H}_i(\mathcal{U}_{it}^\prime)G_i(\mathcal{J}_{it-1}^\prime)  - \mathbf{H}_i(\mathcal{U}_{it-1}^\prime)G_i(\mathcal{J}_{it-1}^\prime) ||_2\leq \\
 &\leq ||\mathbf{H}_i(\mathcal{U}_{it})G_i(\mathcal{J}_{it}) - \mathbf{H}_i(\mathcal{U}_{it}^\prime)G_i(\mathcal{J}_{it}^\prime) ||_2 + ||\mathbf{H}_i(\mathcal{U}_{it-1})G_i(\mathcal{J}_{it-1}) -  \mathbf{H}_i(\mathcal{U}_{it-1}^\prime)G_i(\mathcal{J}_{it-1}^\prime)||_2 + \\
 &\quad + ||\mathbf{H}_i(\mathcal{U}_{it-1})G_i(\mathcal{J}_{it}) -\mathbf{H}_i(\mathcal{U}_{it-1}^\prime)G_i(\mathcal{J}_{it}^\prime)    ||_2
 + ||\mathbf{H}_i(\mathcal{U}_{it})G_i(\mathcal{J}_{it-1}) -  \mathbf{H}_i(\mathcal{U}_{it}^\prime)G_i(\mathcal{J}_{it-1}^\prime) ||_2 = \\
 & = \delta_2(\mathbf{U}_i, t) + \delta_2(\mathbf{U}_i, t-1)  +\\
 &\quad + ||\mathbf{H}_i(\mathcal{U}_{it-1})G_i(\mathcal{J}_{it}) - \mathbf{H}_i(\mathcal{U}_{it-1}^\prime)G_i(\mathcal{J}_{it}) + \mathbf{H}_i(\mathcal{U}_{it-1}^\prime)G_i(\mathcal{J}_{it}) - \mathbf{H}_i(\mathcal{U}_{it-1}^\prime)G_i(\mathcal{J}_{it}^\prime)    ||_2+\\
 &\quad + ||\mathbf{H}_i(\mathcal{U}_{it})G_i(\mathcal{J}_{it-1}) -\mathbf{H}_i(\mathcal{U}_{it}^\prime)G_i(\mathcal{J}_{it-1})+ \mathbf{H}_i(\mathcal{U}_{it}^\prime)G_i(\mathcal{J}_{it-1})-  \mathbf{H}_i(\mathcal{U}_{it}^\prime)G_i(\mathcal{J}_{it-1}^\prime) ||_2 \leq\\
  &\leq \delta_2(\mathbf{U}_i, t) + \delta_2(\mathbf{U}_i, t-1)  + \\
  &\quad +||\big(\mathbf{H}_i(\mathcal{U}_{it-1}) - \mathbf{H}_i(\mathcal{U}_{it-1}^\prime)\big) G_i(\mathcal{J}_{it})||_2 +  ||\mathbf{H}_i(\mathcal{U}_{it-1}^\prime)\big(G_i(\mathcal{J}_{it}) - G_i(\mathcal{J}_{it}^\prime)\big)    ||_2+\\
 &\quad + ||\big(\mathbf{H}_i(\mathcal{U}_{it}) -\mathbf{H}_i(\mathcal{U}_{it}^\prime)\big)G_i(\mathcal{J}_{it-1})||_2 + ||\mathbf{H}_i(\mathcal{U}_{it}^\prime)\big(G_i(\mathcal{J}_{it-1}) -G_i(\mathcal{J}_{it-1}^\prime)\big) ||_2 \leq \\
 &\leq \delta_2(\mathbf{U}_i, t) + \delta_2(\mathbf{U}_i, t-1)  + \big(\delta_2(\mathbf{H}_i, t-1) +  \delta_2(\mathbf{H}_i, t)\big) ||G_i ||_2+ \big( \delta_2(G_i, t-1) +  \delta_2(G_i, t)\big)||\mathbf{H}_i ||_2  
\end{align*}

Here $\mathcal{U}_{it}^\prime  = (\ldots, u_{i(-1)}, u^\prime_{i0}, u_{i1}, \ldots, u_{it-1}, u_{it})$, $\mathcal{U}_{it-1}^\prime  = (\ldots, u_{i(-1)}, u^\prime_{i0}, u_{i1}, \ldots, u_{it-1})$, $\mathcal{J}_{it}^\prime  = (\ldots, \eta_{i(-1)}, \eta^\prime_{i0}, \eta_{i1}, \ldots, \eta_{it-1}, \eta_{it})$, $\mathcal{J}_{it-1}^\prime  = (\ldots, \eta_{i(-1)}, \eta^\prime_{i0}, \eta_{i1}, \ldots, \eta_{it-1})$ are coupled processes with $u_{i0}^\prime$ being an i.i.d. copy of $u_{i0}$ and $\eta_{i0}^\prime$ being an i.i.d. copy of $\eta_{i0}$.

This leads us to 
\begin{align*}
 &\sum_{s=0}^\infty \delta_2(\Delta \mathbf{U}_i, s) \leq \sum_{s=0}^\infty \delta_2(\mathbf{U}_i, s) + \sum_{s=1}^\infty\delta_2(\mathbf{U}_i, s-1)  +\\
 &\quad + \sum_{s=1}^\infty\big(\delta_2(\mathbf{H}_i, s-1) +  \delta_2(\mathbf{H}_i, s)\big) ||G_i ||_2 + \sum_{s=1}^\infty\big( \delta_2(G_i, s-1) +  \delta_2(G_i, s)\big)||\mathbf{H}_i ||_2 <\infty 
\end{align*}


\end{proof}


\begin{prop}\label{prop-reg-4}
Under Assumptions \ref{C-err1} - \ref{C-reg6},
\[ \Big| \frac{1}{\sqrt{T}}\sum_{t=1}^T \Delta \mathbf{X}_{it}\Delta \varepsilon_{it} \Big| = O_P(1).
\]
\end{prop}

\begin{proof}[\textnormal{\textbf{Proof of Proposition \ref{prop-reg-4}}}]
We need the following notation:
\begin{alignat*}{2}
&\mathcal{P}_{i,t}(\cdot) &&:= \ex[\cdot|\mathcal{I}_{i, t}] -\ex[\cdot|\mathcal{I}_{i, t-1}], \\
&\kappa_{i} && := \frac{1}{T}\sum_{t=1}^T  \Delta \mathbf{X}_{it}\Delta \varepsilon_{it}, \\
&\kappa_{i, s}^{\mathcal{P}} && := \frac{1}{T}\sum_{t=1}^T \mathcal{P}_{i, t-s}\big( \Delta \mathbf{X}_{it}\Delta \varepsilon_{it}\big).
\end{alignat*}
Then,
\begin{align*}
||\kappa_{i, s}^{\mathcal{P}}||^2 &= \Big|\Big| \frac{1}{T}\sum_{t=1}^T \mathcal{P}_{i, t-s}\big( \Delta \mathbf{X}_{it}\Delta \varepsilon_{it}\big) \Big|\Big|^2 \leq\\
&\leq \frac{1}{T^2} \sum_{t=1}^T \Big|\Big| \ex (\Delta \mathbf{X}_{it}\Delta \varepsilon_{it}|\mathcal{I}_{i, t-s}) -\ex (\Delta \mathbf{X}_{it}\Delta \varepsilon_{it}|\mathcal{I}_{i, t-s-1}) \Big|\Big|^2 =\\
&= \frac{1}{T^2} \sum_{t=1}^T \Big|\Big| \ex (\Delta \mathbf{X}_{it}\Delta \varepsilon_{it}|\mathcal{I}_{i, t-s}) -\ex (\Delta \mathbf{X}_{it, s}^\prime\Delta \varepsilon_{it, s}^\prime|\mathcal{I}_{i, t-s}) \Big|\Big|^2,
\end{align*}
where $\Delta \mathbf{X}_{it, s}^\prime\Delta \varepsilon_{it, s}^\prime$ denotes $\Delta \mathbf{X}_{it}\Delta \varepsilon_{it}$ with $\{\zeta_{i, t-s}\}$ replaced by its i.i.d. copy $\{\zeta_{i, t-s}^\prime\}$. In this case $\ex (\Delta \mathbf{X}_{it, s}^\prime\Delta \varepsilon_{it, s}^\prime|\mathcal{I}_{i, t-s -1}) = \ex (\Delta \mathbf{X}_{it, s}^\prime\Delta \varepsilon_{it, s}^\prime|\mathcal{I}_{i, t-s})$. Furthermore, by linearity of the expectation and Jensen's inequality, we have 
\begin{align*}
||\kappa_{i, s}^{\mathcal{P}}||^2 &\leq \frac{1}{T^2} \sum_{t=1}^T \Big|\Big| \ex (\Delta \mathbf{X}_{it}\Delta \varepsilon_{it}|\mathcal{I}_{i, t-s}) -\ex (\Delta \mathbf{X}_{it, s}^\prime\Delta \varepsilon_{it, s}^\prime|\mathcal{I}_{i, t-s}) \Big|\Big|^2 \leq \\
&\leq \frac{1}{T^2} \sum_{t=1}^T \Big|\Big| \Delta \mathbf{X}_{it}\Delta \varepsilon_{it} -\Delta \mathbf{X}_{it, s}^\prime\Delta \varepsilon_{it, s}^\prime\Big|\Big|^2 =\\
& = \frac{1}{T^2} \sum_{t=1}^T \Big|\Big| \Delta \mathbf{H}_i(\mathcal{U}_{it})  \Delta G_i(\mathcal{J}_{it}) - \Delta \mathbf{H}_i(\mathcal{U}_{it, s}^\prime)  \Delta G_i(\mathcal{J}_{it, s}^\prime)\Big|\Big|^2 =\\
& = \frac{1}{T^2} \sum_{t=1}^T \Big|\Big| \Delta \mathbf{U}_i(\mathcal{I}_{i,t})  - \Delta \mathbf{U}_i(\mathcal{I}_{i,t, s}^\prime) \Big|\Big|^2 \leq \frac{1}{T^2} \sum_{t=1}^T \delta_2^2(\Delta \mathbf{U}_i, s) = \frac{1}{T}\delta_2^2(\Delta \mathbf{U}_i, s)
\end{align*}
with $\mathcal{U}_{it, s}^\prime = (\ldots, u_{it-s-1}, u^\prime_{it-s}, u_{it-s+1}, \ldots, u_{it})$, $\mathcal{J}_{it, s}^\prime = (\ldots, \eta_{it-s-1}, \eta^\prime_{it-s}, \eta_{it-s+1}, \ldots, \eta_{it})$, $\zeta^\prime_{it} = (u_{it}^\prime, \eta_{it}^\prime)^\top$ and $\mathcal{I}_{i,t,s}^\prime =(\ldots, \zeta_{it-s-1}, \zeta^\prime_{it-s}, \zeta_{it-s+1}, \ldots, \zeta_{it})$.

Moreover,
\begin{align*}
\kappa_i - \ex \kappa_i &= \frac{1}{T}\sum_{t=1}^T \Delta \mathbf{X}_{it}\Delta \varepsilon_{it} - \ex \kappa_i = \frac{1}{T}\sum_{t=1}^T \ex ( \Delta \mathbf{X}_{it}\Delta \varepsilon_{it}|\mathcal{I}_{i,t}) - \ex \kappa_i =\\
&= \frac{1}{T}\sum_{t=1}^T \big( \ex ( \Delta \mathbf{X}_{it}\Delta \varepsilon_{it}|\mathcal{I}_{i,t} ) - \ex ( \mathbf{X}_{it}\Delta \varepsilon_{it}) \big) = \\
&= \frac{1}{T}\sum_{t=1}^T \sum_{s=0}^\infty \big( \ex ( \Delta \mathbf{X}_{it}\Delta \varepsilon_{it}|\mathcal{I}_{i,t-s} ) - \ex ( \Delta \mathbf{X}_{it}\Delta \varepsilon_{it}|\mathcal{I}_{i,t-s-1} )  \big) =\\
&= \frac{1}{T}\sum_{t=1}^T \sum_{s=0}^\infty \mathcal{P}_{i, t-s} (\Delta \mathbf{X}_{it}\Delta \varepsilon_{it}) = \sum_{s=0}^\infty \kappa_{i, s}^{\mathcal{P}}.
\end{align*}
Thus, by Proposition \ref{prop-reg-3},
\[ || \kappa_i - \ex \kappa_i || \leq \sum_{s=0}^\infty ||\kappa_{i, s}^{\mathcal{P}} || \leq \frac{1}{\sqrt{T}}\sum_{s=0}^\infty \delta_2(\Delta \mathbf{U}_i, s) = O\Big(\frac{1}{\sqrt{T}}\Big)
\]
Since $\ex \kappa_i = 0$ by Proposition \ref{prop-reg-2}, we conclude that
\[  \Big|\Big| \frac{1}{T}\sum_{t=1}^T  \Delta \mathbf{X}_{it}\Delta \varepsilon_{it} \Big|\Big| = O\Big(\frac{1}{\sqrt{T}}\Big).
\]
Therefore, the proposition follows.
\end{proof}



\begin{proof}[\textnormal{\textbf{Proof of Theorem \ref{theo-regs}}}]
Recall the differencing estimator $\widehat{\bm{\beta}}_i$:
\begin{align*}
\widehat{\bm{\beta}}_i &= \Big( \sum_{t=1}^T \Delta \mathbf{X}_{it} \Delta \mathbf{X}_{it}^\top \Big)^{-1} \sum_{t=1}^T \Delta \mathbf{X}_{it} \Delta Y_{it} =\\
& =  \Big( \sum_{t=1}^T \Delta \mathbf{X}_{it} \Delta \mathbf{X}_{it}^\top \Big)^{-1} \sum_{t=1}^T \Delta \mathbf{X}_{it} \bigg(\Delta \mathbf{X}_{it}^\top \bm{\beta}_i + \Delta \varepsilon_{it} + O\Big(\frac{1}{T}\Big) \bigg) =\\
&= \bm{\beta}_i +  \Big( \sum_{t=1}^T \Delta \mathbf{X}_{it} \Delta \mathbf{X}_{it}^\top \Big)^{-1} \sum_{t=1}^T \Delta \mathbf{X}_{it} \Delta \varepsilon_{it}
+  O\Big(\frac{1}{T}\Big) \Big( \sum_{t=1}^T \Delta \mathbf{X}_{it} \Delta \mathbf{X}_{it}^\top \Big)^{-1} \sum_{t=1}^T \Delta \mathbf{X}_{it}. 
\end{align*}
This leads to
\begin{align*}
\sqrt{T}( \widehat{\bm{\beta}}_i - \bm{\beta}_i) &=  \Big(\frac{1}{T} \sum_{t=1}^T \Delta \mathbf{X}_{it} \Delta \mathbf{X}_{it}^\top \Big)^{-1}\frac{1}{\sqrt{T}} \sum_{t=1}^T \Delta \mathbf{X}_{it} \Delta \varepsilon_{it} +\\
&+  O\Big(\frac{1}{\sqrt{T}}\Big) \Big( \frac{1}{T}\sum_{t=1}^T \Delta \mathbf{X}_{it} \Delta \mathbf{X}_{it}^\top \Big)^{-1} \frac{1}{T}\sum_{t=1}^T \Delta \mathbf{X}_{it}.
\end{align*}

Since 
\[\ex \Big[ \frac{1}{T}\sum_{t=1}^T \Delta H_{ij}(\mathcal{U}_{it})\Big] = 0\]
and 
\[\var \Big[ \frac{1}{T}\sum_{t=1}^T \Delta H_{ij}(\mathcal{U}_{it})\Big] \leq \frac{4}{T^2} \ex\big[H_{ij}^2(\mathcal{U}_{it})\big],\]
by Chebyshev's inequality we have that $\Big|\frac{1}{T}\sum_{t=1}^T \Delta H_{ij}(\mathcal{U}_{it})\Big| = O_P(1)$ for each $j\in\{1, \ldots, d\} $. And this in turn implies that
\begin{align}\label{proof-1}
\Big|\frac{1}{T}\sum_{t=1}^T \Delta \mathbf{H}_i (\mathcal{U}_{it}) \Big| = \Big| \frac{1}{T}\sum_{t=1}^T \Delta \mathbf{X}_{it}\Big| = O_P(1).
\end{align}


Similarly, by Proposition \ref{prop-reg-1} and Chebyshev's inequality, we have that for each $j, k\in\{1, \ldots, d\}$
\[  \Big|\frac{1}{T}\sum_{t=1}^T \Delta H_{ij}(\mathcal{U}_{it}) \Delta H_{ik}(\mathcal{U}_{it})\Big| = O_P(1),
\]
which leads to 
\begin{align}\label{proof-2}
\Big|\Big| \frac{1}{T}\sum_{t=1}^T \Delta \mathbf{H}_i (\mathcal{U}_{it})\Delta \mathbf{H}_i (\mathcal{U}_{it})^\top \Big|\Big| =\Big|\Big|\frac{1}{T}\sum_{t=1}^T\Delta \mathbf{X}_{it} \Delta \mathbf{X}_{it}^\top\Big|\Big| = O_P(1),
\end{align}
where $||A||$ with $A$ being a matrix is any matrix norm.

By Assumption \ref{C-reg2}, we know that $\ex [\Delta \mathbf{X}_{it} \Delta \mathbf{X}_{it}^\top] = \ex [\Delta \mathbf{X}_{i0} \Delta \mathbf{X}_{i0}^\top]$ is invertible, thus, 
\[\Bigg| \Bigg| \Big(\frac{1}{T}\sum_{t=1}^T\Delta \mathbf{X}_{it} \Delta \mathbf{X}_{it}^\top\Big)^{-1}\Bigg|\Bigg| = O_P(1).\]
By applying Proposition \ref{prop-reg-4}, \eqref{proof-1} and \eqref{proof-2}, the statement of the theorem follows.
\end{proof}

\subsection{Proof of Theorem \ref{theo-stat-equality-with-regs}}\label{subsec-appendix-stat-equaility}


\end{document}
