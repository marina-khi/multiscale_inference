\documentclass[a4paper,12pt]{article}
\usepackage{amsmath, bm}
\usepackage{amssymb,amsthm,graphicx}
\usepackage{enumitem}
\usepackage{color}
\usepackage{epsfig}
\usepackage{graphics}
\usepackage{pdfpages}
\usepackage{subcaption}
\usepackage[font=small]{caption}
\usepackage[hang,flushmargin]{footmisc} 
\usepackage{float}
\usepackage{booktabs}
\usepackage[mathscr]{euscript}
\usepackage{natbib}
\usepackage{setspace}
\usepackage{mathrsfs}
\usepackage[left=2.7cm,right=2.7cm,bottom=2.7cm,top=2.7cm]{geometry}
\parindent0pt 


% General

\newcommand{\reals}{\mathbb{R}}
\newcommand{\integers}{\mathbb{Z}}
\newcommand{\naturals}{\mathbb{N}}

\newcommand{\pr}{\mathbb{P}}        % probability
\newcommand{\ex}{\mathbb{E}}        % expectation
\newcommand{\var}{\textnormal{Var}} % variance
\newcommand{\cov}{\textnormal{Cov}} % covariance

\newcommand{\law}{\mathcal{L}} % law of X
\newcommand{\normal}{N}        % normal distribution 

\newcommand{\argmax}{\textnormal{argmax}}
\newcommand{\argmin}{\textnormal{argmin}}

\newcommand{\ind}{\mathbbm{1}} % indicator function
\newcommand{\kernel}{K} % kernel function
\newcommand{\wght}{W} % kernel weight
\newcommand{\thres}{\pi} % threshold parameter


% Convergence

\newcommand{\convd}{\stackrel{d}{\longrightarrow}}              % convergence in distribution
\newcommand{\convp}{\stackrel{P}{\longrightarrow}}              % convergence in probability
\newcommand{\convas}{\stackrel{\textrm{a.s.}}{\longrightarrow}} % convergence almost surely
\newcommand{\convw}{\rightsquigarrow}                           % weak convergence


% Theorem-like declarations

\theoremstyle{plain}

\newtheorem{theorem}{Theorem}[section]
\newtheorem{prop}[theorem]{Proposition}
\newtheorem{lemma}[theorem]{Lemma}
\newtheorem{corollary}[theorem]{Corollary}
\newtheorem*{theo}{Theorem}
\newtheorem{propA}{Proposition}[section]
\newtheorem{lemmaA}[propA]{Lemma}
\newtheorem{definition}{Definition}[section]
\newtheorem{remark}{Remark}[section]
\renewcommand{\thelemmaA}{A.\arabic{lemmaA}}
\renewcommand{\thepropA}{A.\arabic{propA}}
\newtheorem*{algo}{Clustering Algorithm}


% Theorem numbering to the left

\makeatletter
\newcommand{\lefteqno}{\let\veqno\@@leqno}
\makeatother


% Heading

\newcommand{\heading}[2]
{  \setcounter{page}{1}
   \begin{center}

   \phantom{Distance to upper boundary}
   \vspace{0.5cm}

   {\LARGE \textbf{#1}}
   \vspace{0.4cm}
 
   {\LARGE \textbf{#2}}
   \end{center}
}


% Authors

\newcommand{\authors}[4]
{  \parindent0pt
   \begin{center}
      \begin{minipage}[c][2cm][c]{5cm}
      \begin{center} 
      {\large #1} 
      \vspace{0.05cm}
      
      #2 
      \end{center}
      \end{minipage}
      \begin{minipage}[c][2cm][c]{5cm}
      \begin{center} 
      {\large #3}
      \vspace{0.05cm}

      #4 
      \end{center}
      \end{minipage}
   \end{center}
}

%\newcommand{\authors}[2]
%{  \parindent0pt
%   \begin{center}
%   {\large #1} 
%   \vspace{0.1cm}
%      
%   #2 
%   \end{center}  
%}


% Version

\newcommand{\version}[1]
{  \begin{center}
   {\large #1}
   \end{center}
   \vspace{3pt}
} 










\begin{document}



\heading{Multiscale Testing for Equality}{of Nonparametric Trend Curves}

\vspace{-0.5cm}

\authors{Marina Khismatullina\renewcommand{\thefootnote}{1}\footnotemark[1]}{University of Bonn}{Michael Vogt\renewcommand{\thefootnote}{2}\footnotemark[2]}{University of Bonn} 
\footnotetext[1]{Address: Bonn Graduate School of Economics, University of Bonn, 53113 Bonn, Germany. Email: \texttt{marina.k@uni-bonn.de}.}
\renewcommand{\thefootnote}{2}
\footnotetext[2]{Corresponding author. Address: Department of Economics and Hausdorff Center for Mathematics, University of Bonn, 53113 Bonn, Germany. Email: \texttt{michael.vogt@uni-bonn.de}.}
\renewcommand{\thefootnote}{\arabic{footnote}}
\setcounter{footnote}{0}

%\vspace{-0.5cm}

%\version{\today}

\vspace{-1cm}



\renewcommand{\abstractname}{}
\begin{abstract}
\noindent We develop multiscale methods to test qualitative hypotheses about nonparametric time trends. In many applications, practitioners are interested in whether the observed time series has a time trend at all, that is, whether the trend function is non-constant. Moreover, they would like to get further information about the shape of the trend function. Among other things, they would like to know in which time regions there is an upward/downward movement in the trend. When multiple time series are observed, another important question is whether the observed time series all have the same time trend. We design multiscale tests to formally approach these questions. We derive asymptotic theory for the proposed tests and investigate their finite sample performance by means of simulations. In addition, we illustrate the methods by two applications to temperature data. 
\end{abstract}

\vspace{-0.1cm}

\enlargethispage{0.25cm}
\renewcommand{\baselinestretch}{1.2}\normalsize

\textbf{Key words:} Multiscale statistics; nonparametric regression; time series errors; shape constraints; strong approximations; anti-concentration bounds.

\textbf{AMS 2010 subject classifications:} 62E20; 62G10; 62G20; 62M10. 

\vspace{-0.25cm}

\numberwithin{equation}{section}
\allowdisplaybreaks[1]


\section{The model}\label{sec-model}

The model setting is as follows. We observe time series $\mathcal{Y}_i = \{Y_{it}: 1 \le t \le T \}$ of length $T$ for $1 \le i \le n$. Each time series $\mathcal{Y}_i$ satisfies the model equation \begin{equation}\label{model_full}
Y_{it} = \beta^\prime X_{it} + m_i \Big( \frac{t}{T} \Big) + \alpha_i + \varepsilon_{it} 
\end{equation}
for $1 \le t \le T$, where $\beta$ is a $d \times 1$ vector of unknown parameters, $X_{it}$ is a $d\times 1$ vector of individual covariates, $m_i$ is an unknown nonparametric trend function defined on $[0,1]$, $\alpha_i$ is a (deterministic or random) intercept term and $\mathcal{E}_i = \{ \varepsilon_{it}: 1 \le t \le T \}$ is a zero-mean stationary error process. For identification, we normalize the functions $m_i$ such that $\int_0^1 m_i(u) du = 0$ for all $1 \le i \le n$. The term $\alpha_i$ can also be regarded as an additional error component. In the econometrics literature, it is commonly called a fixed effect error term. It can be interpreted as capturing unobserved characteristics of the time series $\mathcal{Y}_i$ which remain constant over time. We allow the error terms $\alpha_i$ to be dependent across $i$ in an arbitrary way. Hence, by including them in model equation \eqref{model_full}, we allow the $n$ time series $\mathcal{Y}_i$ in our sample to be correlated with each other. Whereas the terms $\alpha_i$ may be correlated, the error processes $\mathcal{E}_i$ are assumed to be independent across $i$. In addition, each process $\mathcal{E}_i$ is supposed to satisfy the conditions \ref{C-err1}--\ref{C-err3}. Finally note that throughout the paper, we restrict attention to the case where the number of time series $n$ in model \eqref{model} is fixed. Extending our theoretical results to the case where $n$ slowly grows with the sample size $T$ is a possible topic for further research.

The stationary error processes $\{\varepsilon_{it}\}_{t\in\integers}$ are assumed to have the following properties: 
\begin{enumerate}[label=(C\arabic*),leftmargin=1.05cm]

\item \label{C-err1} For each $i$ the variables $\varepsilon_{it}$ allow for the representation $\varepsilon_{it} = G_i(\ldots,\eta_{it-1},\eta_{it},\eta_{it+1},\ldots)$, where $\eta_{it}$ are i.i.d.\ random variables across $t$ and $G_i: \reals^\integers \rightarrow \reals$ is a measurable function. Denote $\mathcal{Z}_{it} = (\ldots,\eta_{it-2},\eta_{it-1},\eta_{it})$.

\item \label{C-err2} It holds that $\ex[\varepsilon_{it}] =0$ and $\| \varepsilon_{it} \|_q < \infty$ for some $q > 4$, where $\| \varepsilon_{it} \|_q = (\ex|\varepsilon_t|^q)^{1/q}$. 

\end{enumerate}

Following \cite{Wu2005}, we impose conditions on the dependence structure of the error process $\{\varepsilon_{it}\}_{t\in\integers}$ in terms of the physical dependence measure $d_{i, t,q} = \| \varepsilon_{it} - \varepsilon_{it}^\prime \|_q$, where $\varepsilon_{it}^\prime = G(\ldots,\eta_{i(-1)},\eta_{i0}^\prime,\eta_{i1},\ldots,\eta_{it-1},\eta_{it},\eta_{it+1},\ldots)$ with $\{\eta_{it}^\prime\}$ being an i.i.d.\ copy of $\{\eta_{it}\}$. In particular, we assume the following: 
\begin{enumerate}[label=(C\arabic*),leftmargin=1.05cm]
\setcounter{enumi}{2}

\item \label{C-err3} Define $\Theta_{i, t,q} = \sum\nolimits_{|s| \ge t} d_{i, s,q}$ for $t \ge 0$. It holds that 
$\Theta_{i, t,q} = O ( t^{-\tau_q} (\log t)^{-A} )$,  
where $A > \frac{2}{3} (1/q + 1 + \tau_q)$ and $\tau_q = \{q^2 - 4 + (q-2) \sqrt{q^2 + 20q + 4}\} / 8q$. 

\end{enumerate}
The conditions \ref{C-err1}--\ref{C-err3} are fulfilled by a wide range of stationary processes $\{\varepsilon_{it}\}_{t\in\integers}$.%As a first example, consider linear processes of the form $\varepsilon_t = \sum\nolimits_{i=0}^{\infty} c_i \eta_{t-i}$ with $\| \varepsilon_t \|_q < \infty$, where $c_i$ are absolutely summable coefficients and $\eta_t$ are i.i.d.\ innovations with $\ex[\eta_t] = 0$ and $\| \eta_t\|_q < \infty$. Trivially, \ref{C-err1} and \ref{C-err2} are fulfilled in this case. Moreover, if $|c_i| = O(\rho^i)$ for some $\rho \in (0,1)$, then \ref{C-err3} is easily seen to be satisfied as well. As a special case, consider an ARMA process $\{\varepsilon_t\}$ of the form $\varepsilon_t - \sum\nolimits_{i=1}^p a_i \varepsilon_{t-i} = \eta_t + \sum\nolimits_{j=1}^r b_j \eta_{t-j}$  with $\| \varepsilon_t \|_q < \infty$, where $a_1,\ldots,a_p$ and $b_1,\ldots,b_r$ are real-valued parameters. As before, we let $\eta_t$ be i.i.d.\ innovations with $\ex[\eta_t] = 0$ and $\| \eta_t\|_q < \infty$. Moreover, as usual, we suppose that the complex polynomials $A(z) = 1 - \sum\nolimits_{j=1}^p a_jz^j$ and $B(z) = 1 + \sum\nolimits_{j=1}^r b_jz^j$ do not have any roots in common. If $A(z)$ does not have any roots inside the unit disc, then the ARMA process $\{ \varepsilon_t \}$ is stationary and causal. Specifically, it has the representation $\varepsilon_t = \sum\nolimits_{i=0}^{\infty} c_i \eta_{t-i}$ with $|c_i| = O(\rho^i)$ for some $\rho \in (0,1)$, implying that \ref{C-err1}--\ref{C-err3} are fulfilled. The results in \cite{WuShao2004} show that condition \ref{C-err3} (as well as the other two conditions) is not only fulfilled for linear time series processes but also for a variety of non-linear processes. 


\section{Testing for equality of time trends}\label{sec-test-equality}

In this section, we adapt the multiscale method developed in Section \ref{sec-method} to test the hypothesis that the trend functions in model \eqref{model} are all the same. More formally, we test the null hypothesis $H_0: m_1 = m_2 = \ldots = m_n$ in model \eqref{model_full}. As we will see, the proposed multiscale method does not only allow to test whether the null hypothesis is violated. It also provides information on where violations occur. More specifically, it allows to identify, with a pre-specified confidence, (i) trend functions which are different from each other and (ii) time intervals where these trend functions differ.


%\subsection{Construction of the test statistic in the absence of exogenous regressors}\label{subsec-test-equality-stat-without-regs}
%
%As a starting point we will first provide a test for common trends in the absence of exogenous covariates $X_{it}$. So we begin with a model
%\begin{equation}\label{model}
%Y_{it} = m_i \Big( \frac{t}{T} \Big) + \alpha_i + \varepsilon_{it},
%\end{equation} 
%where $\varepsilon_{it}$ are zero-mean error terms and $\alpha_i$ are (random or deterministic) intercepts. Defining $Y_{it}^\circ = Y_{it} - \alpha_i$, this equation can be rewritten as $Y_{it}^\circ = m_i(\frac{t}{T}) + \varepsilon_{it}$, which is a standard nonparametric regression equation. The variables $Y_{it}^\circ$ are not observed, but they can be approximated by $\widehat{Y}_{it} = Y_{it} - \widehat{\alpha}_i$, where $\widehat{\alpha}_i = T^{-1} \sum_{t=1}^T Y_{it}$ is an estimator of the intercept $\alpha_i$. By construction, $\widehat{\alpha}_i - \alpha_i = T^{-1} \sum_{t=1}^T \varepsilon_{it} + T^{-1} \sum_{t=1}^T m_i(\frac{t}{T}) = O_p(T^{-1/2}) + T^{-1} \sum_{t=1}^T m_i(\frac{t}{T})$. Hence, $\widehat{\alpha}_i$ is a reasonable estimator of $\alpha_i$ if $T^{-1} \sum_{t=1}^T m_i(\frac{t}{T})$ converges to zero as $T \rightarrow \infty$. To ensure this, we suppose throughout the section that the functions $m_i$ are Lipschitz continuous, that is, $|m_i(v) - m_i(w)| \le L|v - w|$ for all $v,w \in [0,1]$ and some constant $L < \infty$. Since $\int_0^1 m_i(u) du = 0$ by normalization, this implies that $T^{-1} \sum_{t=1}^T m_i(\frac{t}{T}) = O(T^{-1})$. We further let $\widehat{\sigma}_i^2$ be an estimator of the long-run error variance $\sigma_i^2 = \sum\nolimits_{\ell=-\infty}^{\infty} \cov(\varepsilon_{i0}, \varepsilon_{i\ell})$ which is computed from the constructed sample $\{ \widehat{Y}_{it}: 1 \le t \le T \}$. We thus regard $\widehat{\sigma}_i^2 = \widehat{\sigma}_i^2(\widehat{Y}_{i1},\ldots,\widehat{Y}_{iT})$ as a function of the variables $\widehat{Y}_{it}$ for $1 \le t \le T$. Throughout the section, we assume that $\widehat{\sigma}_i^2 = \sigma_i^2 + o_p(\rho_T)$ with $\rho_T = o(1/\log T)$. Details on how to construct estimators of $\sigma_i^2$ are deferred to Section \ref{sec-error-var}. 
%
%
%We are now ready to introduce the multiscale statistic for testing the hypothesis $H_0: m_1 = m_2 = \ldots = m_n$. For any pair of time series $i$ and $j$, we define the kernel averages
%\[ \widehat{\psi}_{ij,T}(u,h) = \sum\limits_{t=1}^T w_{t,T}(u,h)(\widehat{Y}_{it} - \widehat{Y}_{jt}), \]
%where the kernel weights $w_{t,T}(u,h)$ are defined as in \eqref{weights}. The kernel average $\widehat{\psi}_{ij,T}(u,h)$ can be regarded as measuring the distance between the two trend curves $m_i$ and $m_j$ on the interval $[u-h,u+h]$. Similar as in Section \ref{subsec-method-stat}, we aggregate the kernel averages $\widehat{\psi}_{ij,T}(u,h)$ for all $(u,h) \in \mathcal{G}_T$ by the multiscale statistic 
%\[ \widehat{\Psi}_{ij,T} = \max_{(u,h) \in \mathcal{G}_T} \Big\{ \Big|\frac{\widehat{\psi}_{ij,T}(u,h)}{(\widehat{\sigma}_i^2 + \widehat{\sigma}_j^2)^{1/2}}\Big| - \lambda(h) \Big\}, \] 
%where $\lambda(h) = \sqrt{2 \log \{ 1/(2h) \}}$ and the set $\mathcal{G}_T$ has been introduced in Section \ref{subsec-method-stat}. The statistic $\widehat{\Psi}_{ij,T}$ can be interpreted as a distance measure between the two curves $m_i$ and $m_j$. We finally define the multiscale statistic for testing the null hypothesis $H_0: m_1 =m_2 = \ldots = m_n$ as
%\[ \widehat{\Psi}_{n,T} = \max_{1 \le i < j \le n} \widehat{\Psi}_{ij,T}, \]
%that is, we define it as the maximal distance $\widehat{\Psi}_{ij,T}$ between any pair of curves $m_i$ and $m_j$ with $i \ne j$. 
%
%
%\subsection{The test procedure}\label{subsec-test-equality-test}
%
%
%Let $Z_{it}$ for $1 \le t \le T$ and $1 \le i \le n$ be independent standard normal random variables which are independent of the error terms $\varepsilon_{it}$. Denote the empirical average of the variables $Z_{i1},\ldots,Z_{iT}$ by $\bar{Z}_{i,T} = T^{-1} \sum_{t=1}^T Z_{it}$. To simplify notation, we write $\bar{Z}_i = \bar{Z}_{i,T}$ in what follows. For each $i$ and $j$, we introduce the Gaussian statistic 
%\[ \Phi_{ij,T} = \max_{(u,h) \in \mathcal{G}_T} \Big\{ \Big|\frac{\phi_{ij,T}(u,h)}{(\widehat{\sigma}_i^2 + \widehat{\sigma}_j^2)^{1/2}}\Big| - \lambda(h) \Big\}, \] 
%where $\phi_{ij,T}(u,h) = \sum\nolimits_{t=1}^T w_{t,T}(u,h) \, \{ \widehat{\sigma}_i (Z_{it} - \bar{Z}_i) - \widehat{\sigma}_j (Z_{jt} - \bar{Z}_j) \}$. Moreover, we define the statistic
%\[ \Phi_{n,T} = \max_{1 \le i < j \le n} \Phi_{ij,T} \]
%and denote its $(1-\alpha)$-quantile by $q_{n,T}(\alpha)$. Our multiscale test of the hypothesis $H_0: m_1 = m_2 = \ldots = m_n$ is defined as follows: For a given significance level $\alpha \in (0,1)$, we reject $H_0$ if $\widehat{\Psi}_{n,T} > q_{n,T}(\alpha)$. 
%
%
%\subsection{Theoretical properties of the test}\label{subsec-test-equality-theo}
%
%
%To start with, we introduce the auxiliary statistic 
%\[ \widehat{\Phi}_{n,T} = \max_{1 \le i < j \le n} \widehat{\Phi}_{ij,T}, \]
%where
%\[ \widehat{\Phi}_{ij,T} = \max_{(u,h) \in \mathcal{G}_T} \Big\{ \Big| \frac{\widehat{\phi}_{ij,T}(u,h)} {\{ \widehat{\sigma}_i^2 + \widehat{\sigma}_j^2 \}^{1/2}} \Big| - \lambda(h) \Big \} \]
%and $\widehat{\phi}_{ij,T}(u,h) = \sum_{t=1}^T w_{t,T}(u,h) \{ (\varepsilon_{it} - \bar{\varepsilon}_i) - (\varepsilon_{jt} - \bar{\varepsilon}_j) \}$ with $\bar{\varepsilon}_i = \bar{\varepsilon}_{i,T} = T^{-1} \sum_{t=1}^T \varepsilon_{it}$. Our first theoretical result characterizes the asymptotic behaviour of the statistic $\widehat{\Phi}_{n,T}$ and parallels Theorem \ref{theo-stat} from Section \ref{sec-method}. 
%\begin{theorem}\label{theo-stat-equality}
%Suppose that the error processes $\mathcal{E}_i = \{ \varepsilon_{it}: 1 \le t \le T \}$ are independent across $i$ and satisfy \ref{C-err1}--\ref{C-err3} for each $i$. Moreover, let \ref{C-ker}--\ref{C-h} be fulfilled and assume that $\widehat{\sigma}_i^2 = \sigma_i^2 + o_p(\rho_T)$ with $\rho_T = o(1/\log T)$ for each $i$. Then 
%\[ \pr \big( \widehat{\Phi}_{n,T} \le q_{n,T}(\alpha) \big) = (1 - \alpha) + o(1). \]
%\end{theorem}
%Theorem \ref{theo-stat-equality} is the main stepping stone to derive the theoretical properties of our multiscale test. It can be proven by slightly modifying the arguments for Theorem \ref{theo-stat}. The details are provided in the Supplementary Material. The following proposition characterizes the behaviour of our multiscale test under the null hypothesis and under local alternatives. 
%\begin{prop}\label{prop-test-equality}
%Let the conditions  of Theorem \ref{theo-stat-equality} be satisfied. 
%\begin{enumerate}[label=(\alph*),leftmargin=0.75cm]
%\item Under the null hypothesis $H_0: m_1 = m_2 = \ldots = m_n$, it holds that 
%\[ \pr \big( \widehat{\Psi}_{n,T} \le q_{n,T}(\alpha) \big) = (1 - \alpha) + o(1). \]
%\item Let $m_i = m_{i,T}$ be a Lipschitz continuous function with $\int_0^1 m_{i,T}(w) dw = 0$ for any $i$. In particular, suppose that $|m_{i,T}(v) - m_{i,T}(w)| \le L |v - w|$ for all $v,w \in [0,1]$ and some fixed constant $L < \infty$ which does not depend on $T$. Moreover, assume that for some pair of indices $i$ and $j$, the functions $m_{i,T}$ and $m_{j,T}$ have the following property: There exists $(u,h) \in \mathcal{G}_T$ with $[u-h,u+h] \subseteq [0,1]$ such that $m_{i,T}(w) - m_{j,T}(w) \ge c_T \sqrt{\log T/(Th)}$ for all $w \in [u-h,u+h]$ or $m_{j,T}(w) - m_{i,T}(w) \ge c_T \sqrt{\log T/(Th)}$ for all $w \in [u-h,u+h]$, where $\{c_T\}$ is any sequence of positive numbers with $c_T \rightarrow \infty$. Then 
%\[ \pr \big( \widehat{\Psi}_{n,T} \le q_{n,T}(\alpha) \big) = o(1). \]
%\end{enumerate}
%\end{prop}
%Part (a) of Proposition \ref{prop-test-equality} is a direct consequence of Theorem \ref{theo-stat-equality}. The proof of part (b) is very similar to that of Proposition \ref{prop-test-2} and thus omitted. 
%
%

\subsection{Construction of the test statistic in the presence of exogenous regressors}\label{subsec-test-equality-stat-with-regs}

We now extend the model \eqref{model} to include the exogenous regressors:
\begin{equation}
Y_{it} = \beta_i^\prime X_{it} + m_i \Big( \frac{t}{T} \Big) + \alpha_i + \varepsilon_{it} 
\end{equation}
It is obvious that if $\beta_i$ is known, the problem of testing for the common time trend would be reduced to the one discussed before. That is, we would test $H_0: m_1 = m_2 = \ldots = m_n$ in the model
\begin{align*}
Y_{it} - \beta_i^\prime X_{it} & =: V_{it}\\
					& = m_i \Big( \frac{t}{T} \Big) + \alpha_i + \varepsilon_{it} 
\end{align*}
replacing $Y_{it}$ by $V_{it}$ in the construction of the test statistic. However, $\beta_i$ is not known so we need to estimate it first. Given an estimator $\widehat{\beta}_i$, we then consider
\begin{align*}
	\widehat{V}_{it} := Y_{it} - \widehat{\beta}_i^\prime X_{it} =(\widehat{\beta}_i - \beta_i)^\prime X_{it} + m_i \Big( \frac{t}{T} \Big) + \alpha_i + \varepsilon_{it} 
\end{align*}
and construct the kernel averages $\widehat{\psi}_{ij, T}(u, h)$ based on $\widehat{V}_{it}$ instead of $\widehat{Y}_{it}$. Specifically, for any pair of time series $i$ and $j$ we define the kernel averages
\begin{align*}
	\widehat{\psi}^{\prime}_{ij, T}(u, h) = \sum_{t=1}^T w_{t, T}(u, h)(\widehat{V}_{it} - \widehat{V}_{jt})
\end{align*}
with the kernel weights defined in ??. Similar as in Section \ref{subsec-test-equality-stat-without-regs}, we aggregate the kernel averages $\widehat{\psi}^{\prime}_{ij, T}(u, h)$ for all $(u, h)\in \mathcal{G}_T$ by the multiscale test statistic
\[ \widehat{\Psi}^\prime_{ij,T} = \max_{(u,h) \in \mathcal{G}_T} \Big\{ \Big|\frac{\widehat{\psi}^\prime_{ij,T}(u,h)}{(\widehat{\sigma}_i^2 + \widehat{\sigma}_j^2)^{1/2}}\Big| - \lambda(h) \Big\}. \]

We now focus on finding an appropriate estimator $\widehat{\beta}$ of $\beta$. For that purpose, we consider the time series $\{\Delta Y_{it}\}$ of the differences $\Delta Y_{it} = Y_{it} - Y_{i t-1}$ for each $i$. We then have
\begin{align*}
	\Delta Y_{it} = Y_{it} - Y_{i t-1} =\beta_i^\prime \Delta X_{it} + \bigg(m_i \Big( \frac{t}{T} \Big) - m_i \Big(\frac{t-1}{T}\Big)\bigg) + \Delta \varepsilon_{it},
\end{align*}
where $\Delta X_{it} = X_{it} - X_{it-1}$ and $ \Delta \varepsilon_{it} = \varepsilon_{it} - \varepsilon_{i t-1}$. Since $m_i(\cdot)$ is Lipschitz, we can use the fact that $ \big|m_i \big( \frac{t}{T} \big) - m_i \big(\frac{t-1}{T}\big) \big| = O\big(\frac{1}{T}\big)$ and rewrite 
\begin{align}\label{model_with_regs}
	\Delta Y_{it} = \beta_i^\prime \Delta X_{it} + \Delta \varepsilon_{it} + O\Big(\frac{1}{T}\Big).
\end{align}

In particular, for each $i$ we employ the least squares estimation method to estimate $\bm{\beta}_i$ in \eqref{model_with_regs}, treating $\Delta \mathbf{X}_{it}$ as the regressors and $\Delta Y_{it}$ as the response variable. That is, we propose the following differencing estimator:
\begin{align}\label{dif-est}
\widehat{\bm{\beta}}_i = \Big( \sum_{t=1}^T \Delta \mathbf{X}_{it} \Delta \mathbf{X}_{it}^T \Big)^{-1} \sum_{t=1}^T \Delta \mathbf{X}_{it} \Delta Y_{it}
\end{align}
The asymptotic consistency for this differencing estimator is given by the following theorem:

\begin{theorem}\label{thm-reg-1}
Under Assumptions \ref{C-err1} - \ref{C-reg3}, we have
\[\widehat{\bm{\beta}}_i - \bm{\beta}_i = O_P \Big(\frac{1}{\sqrt{T}}\Big),
\]
where $\widehat{\bm{\beta}}_i$ is the differencing estimator given by \eqref{dif-est}.
\end{theorem}



\subsection{Proof}
For a class of stochastic processes $\{\mathbf{L}(v, \mathcal{F}_t)\}_{t\in\integers}$, we say that the process is $\mathcal{L}^q$ \textit{stochastic Lipschitz-continuous} over $[0,1]$ if
\[ \sup_{0\leq v_1 < v_2 \leq 1} \frac{||\mathbf{L}(v_2, \mathcal{F}_0) - \mathbf{L}(v_1, \mathcal{F}_0)||_q}{|v_2 - v_1|} < \infty.
\]
We denote the collection of $\mathcal{L}^q$ \textit{stochastic Lipschitz-continuous} over $[0,1]$ classes by $Lip_q$.

Define the \textit{physical dependence measure} for the process $\mathbf{L}(v, \mathcal{F}_t)$ as the following:
\[ \delta_q(\mathbf{L}, t) = \sup_{v\in[0,1]} || \mathbf{L}(v, \mathcal{F}_t) - \mathbf{L}(v, \mathcal{F}_t^\prime) ||_q,
\]
where $\mathcal{F}_t^\prime  = (\ldots, \epsilon_{-1}, \epsilon^\prime_0, \epsilon_1, \ldots, \epsilon_{t-1}, \epsilon_t)$ is a coupled process with $\epsilon_0^\prime$ being an i.i.d. copy of $\epsilon_0$.


We need the following assumptions on the independent variables $X_{it}$ for each $i$:

\begin{enumerate}[label=(C\arabic*),leftmargin=1.05cm]
\setcounter{enumi}{3}

\item \label{C-reg1} The covariates $X_{it}$ allow for the representation $X_{it} = \mathbf{H}_i(t/T, \mathcal{U}_{it})$, where $\mathcal{U}_{it} =(\ldots,u_{it-1},u_{it})$ with $u_{it}$ being i.i.d.\ random variables and $\mathbf{H}_i := (H_{i1}, H_{i2}, \ldots, H_{id})^T: [0, 1] \times \reals^\integers \rightarrow \reals^d$ is a measurable function such that $\mathbf{H}_i(v, \mathcal{U}_{it})$ is well defined for each $v\in [0,1]$. 

\item \label{C-reg2} Let $N_i(v)$ be the $d\times d$ matrix with $kl$-th entry $n_{i, kl}(v) = \ex[H_{ik}(v, \mathcal{U}_{i0}), H_{il}(v, \mathcal{U}_{i0})]$. We assume that the smallest eigenvalue of $N_i(v)$ is bounded away from $0$ on $v\in[0,1]$.

\item \label{C-reg3} Let $\mathbf{H}_{i}(v, \mathcal{U}_{it}) \in Lip_2$ and $\sup_{0\leq v\leq 1}||\mathbf{H}_{i}(v, \mathcal{U}_{it})||_4 <\infty$.
\end{enumerate}

We define the first-differenced regressors as follows.
\[ \Delta \mathbf{X}_{it} =\mathbf{H}_i(v_t, \mathcal{U}_{it}) - \mathbf{H}_i(v_{t-1}, \mathcal{U}_{it-1}) := \Delta \mathbf{H}_i(v_t, \mathcal{U}_{it}) \]
where $v_t = t/T$. Similarly, 
\[\Delta \varepsilon_{it} = \varepsilon_{it} - \varepsilon_{it-1} = G_i(\mathcal{Z}_{it}) - G_i(\mathcal{Z}_{it-1}) = \Delta G_i(\mathcal{Z}_{it}).
\]
 
With these assumptions we can prove the following proposition.
\begin{prop}\label{prop-reg-1}
Under Assumptions \ref{C-reg1} - \ref{C-reg3},
\[ \Delta \mathbf{H}_i(v_t, \mathcal{U}_{it}) \in Lip_2 \text{ and } \sup_{0\leq v \leq 1} || \Delta \mathbf{H}_i(v, \mathcal{U}_{it})||_4 < \infty
\]
\end{prop}

\begin{proof}[\textnormal{\textbf{Proof of Proposition \ref{prop-reg-1}}}]
Note the following 
\begin{align*}
&\sup_{0\leq v_t < v_s \leq 1} \frac{||\Delta \mathbf{H}_i(v_s, \mathcal{U}_{i0}) - \Delta \mathbf{H}_i(v_t, \mathcal{U}_{i0}) ||_2}{|v_s - v_t|} \leq \\
\leq &\sup_{0\leq v_t < v_s \leq 1} \frac{||\mathbf{H}_i(v_s, \mathcal{U}_{i0}) - \mathbf{H}_i(v_{s-1}, \mathcal{U}_{i0}) - |\mathbf{H}_i(v_t, \mathcal{U}_{i0}) - \mathbf{H}_i(v_{t-1}, \mathcal{U}_{i0})| ||_2}{|v_s - v_t|} \leq\\
\leq &\sup_{0\leq v_t < v_s \leq 1} \frac{||\mathbf{H}_i(v_s, \mathcal{U}_{i0}) - \mathbf{H}_i(v_t, \mathcal{U}_{i0}) ||_2}{|v_s - v_t|} + \sup_{0\leq v_{t-1} < v_{s-1} \leq 1} \frac{||\mathbf{H}_i(v_{s-1}, \mathcal{U}_{i0}) - \mathbf{H}_i(v_{t-1}, \mathcal{U}_{i0}) ||_2}{|v_{s-1} - v_{t-1}|},
\end{align*}
where $v_s - v_t = v_{s-1} - v_{t-1}$ by definition. By Assumption \ref{C-reg3},
\[\sup_{0\leq v_t < v_s \leq 1} \frac{||\mathbf{H}_i(v_s, \mathcal{U}_{i0}) - \mathbf{H}_i(v_t, \mathcal{U}_{i0}) ||_2}{|v_s - v_t|} < \infty
\]
and 
\[\sup_{0\leq v_{t-1} < v_{s-1} \leq 1} \frac{||\mathbf{H}_i(v_{s-1}, \mathcal{U}_{i0}) - \mathbf{H}_i(v_{t-1}, \mathcal{U}_{i0}) ||_2}{|v_{s-1} - v_{t-1}|} < \infty.
\]
Thus, 
\[
\sup_{0\leq v_t < v_s \leq 1} \frac{||\Delta \mathbf{H}_i(v_s, \mathcal{U}_{i0}) - \Delta \mathbf{H}_i(v_t, \mathcal{U}_{i0}) ||_2}{|v_s - v_t|} < \infty
\]
and $\Delta \mathbf{H}_i(v_t, \mathcal{U}_{it}) \in Lip_2$. Moreover, by Assumption \ref{C-reg3},
\[
 \sup_{0\leq v_t \leq 1} || \Delta \mathbf{H}_i(v_t, \mathcal{U}_{it})||_4 \leq  \sup_{0\leq v_t \leq 1} ||\mathbf{H}_i(v, \mathcal{U}_{it})||_4 +  \sup_{0\leq v_{t-1} \leq 1} || \mathbf{H}_i(v_{t-1}, \mathcal{U}_{it-1})||_4 < \infty,
\]
which completes the proof.
\end{proof} 

To be able to prove the next proposition, we need an additional assumption on the relationship between the covariates and the error process.

\begin{enumerate}[label=(C\arabic*),leftmargin=1.05cm]
\setcounter{enumi}{6}
\item \label{C-reg4} For each $i$ $\{\eta_{it}\}_{t\in\integers}$ from Assumption \ref{C-err1} and $\{u_{it}\}_{t\in\integers}$ from Assumption \ref{C-reg1} are independent of each other.
\end{enumerate}


\begin{prop}\label{prop-reg-2}
Under Assumptions \ref{C-err1} - \ref{C-reg4},
\[ \Big| \frac{1}{\sqrt{T}}\sum_{t=1}^T \Delta \mathbf{X}_{it}\Delta \varepsilon_{it} \Big| = O_P(1)
\]
\end{prop}
\begin{proof}[\textnormal{\textbf{Proof of Proposition \ref{prop-reg-2}}}]
Let $\zeta_{i,t} = (u_{it}, \eta_{it})^T$. We need the following notation:
\begin{alignat*}{2}
&\mathcal{I}_{i,t} &&:= (\ldots, \zeta_{i, t-2}, \zeta_{i, t-1}, \zeta_{i,t}),\\
&\Delta \mathbf{U}_i(v, \mathcal{I}_{i, t}) &&:= \Delta \mathbf{H}_i(v, \mathcal{U}_{it}) \Delta G_i(\mathcal{Z}_{it}),\\
&\mathcal{P}_{i,t}(\cdot) &&:= \ex[\cdot|\mathcal{I}_{i, t}] -\ex[\cdot|\mathcal{I}_{i, t-1}], \\
&\kappa_{i} && := \frac{1}{T}\sum_{t=1}^T  \Delta \mathbf{X}_{it}\Delta \varepsilon_{it}, \\
&\kappa_{i, s}^{\mathcal{P}} && := \frac{1}{T}\sum_{t=1}^T \mathcal{P}_{i, t-s}\big( \Delta \mathbf{X}_{it}\Delta \varepsilon_{it}\big).
\end{alignat*}
Then,
\begin{align*}
||\kappa_{i, s}^{\mathcal{P}}||^2 &= \Big|\Big| \frac{1}{T}\sum_{t=1}^T \mathcal{P}_{i, t-s}\big( \Delta \mathbf{X}_{it}\Delta \varepsilon_{it}\big) \Big|\Big|^2 \leq\\
&\leq \frac{1}{T^2} \sum_{t=1}^T \Big|\Big| \ex (\Delta \mathbf{X}_{it}\Delta \varepsilon_{it}|\mathcal{I}_{i, t-s}) -\ex (\Delta \mathbf{X}_{it}\Delta \varepsilon_{it}|\mathcal{I}_{i, t-s-1}) \Big|\Big|^2 =\\
&= \frac{1}{T^2} \sum_{t=1}^T \Big|\Big| \ex (\Delta \mathbf{X}_{it}\Delta \varepsilon_{it}|\mathcal{I}_{i, t-s}) -\ex (\Delta \mathbf{X}_{it, s}^\prime\Delta \varepsilon_{it, s}^\prime|\mathcal{I}_{i, t-s}) \Big|\Big|^2,
\end{align*}
where $\Delta \mathbf{X}_{it, s}^\prime\Delta \varepsilon_{it, s}^\prime$ denotes $\Delta \mathbf{X}_{it}\Delta \varepsilon_{it}$ with $\{\zeta_{i, t-s}\}$ replaced by its i.i.d. copy $\{\zeta_{i, t-s}^\prime\}$. In this case $\ex (\Delta \mathbf{X}_{it, s}^\prime\Delta \varepsilon_{it, s}^\prime|\mathcal{I}_{i, t-s -1}) = \ex (\Delta \mathbf{X}_{it, s}^\prime\Delta \varepsilon_{it, s}^\prime|\mathcal{I}_{i, t-s})$. Furthermore, by linearity of the expectation and Jensen's inequality, we have 
\begin{align*}
||\kappa_{i, s}^{\mathcal{P}}||^2 &\leq \frac{1}{T^2} \sum_{t=1}^T \Big|\Big| \ex (\Delta \mathbf{X}_{it}\Delta \varepsilon_{it}|\mathcal{I}_{i, t-s}) -\ex (\Delta \mathbf{X}_{it, s}^\prime\Delta \varepsilon_{it, s}^\prime|\mathcal{I}_{i, t-s}) \Big|\Big|^2 \leq \\
&\leq \frac{1}{T^2} \sum_{t=1}^T \Big|\Big| \Delta \mathbf{X}_{it}\Delta \varepsilon_{it} -\Delta \mathbf{X}_{it, s}^\prime\Delta \varepsilon_{it, s}^\prime\Big|\Big|^2 =\\
& = \frac{1}{T^2} \sum_{t=1}^T \Big|\Big| \Delta \mathbf{H}_i(v_t, \mathcal{U}_{it})  \Delta G_i(\mathcal{Z}_{it}) - \Delta \mathbf{H}_i(v_t, \mathcal{U}_{it, s}^\prime)  \Delta G_i(\mathcal{Z}_{it, s}^\prime)\Big|\Big|^2 =\\
& = \frac{1}{T^2} \sum_{t=1}^T \Big|\Big| \Delta \mathbf{U}_i(v_t, \mathcal{I}_{it})  - \Delta \mathbf{U}_i(v_t, \mathcal{I}_{it, s}^\prime) \Big|\Big|^2 \leq \\
&\leq \frac{1}{T^2} \sum_{i=1}^T \delta_2^2(\Delta \mathbf{U}, s) = \frac{1}{T}\delta_2^2(\Delta \mathbf{U}, s)
\end{align*}
with $\mathcal{U}_{it, s}^\prime = (\ldots, u_{it-s-1}, u^\prime_{it-s}, u_{it-s+1}, \ldots, u_{it})$, $\mathcal{Z}_{it, s}^\prime = (\ldots, \eta_{it-s-1}, \eta^\prime_{it-s}, \eta_{it-s+1}, \ldots, \eta_{it})$, $\zeta^\prime_{it} = (u_{it}^\prime, \eta_{it}^\prime)^T$ and $\mathcal{I}_{i,t,s}^\prime =(\ldots, \zeta_{it-s-1}, \zeta^\prime_{it-s}, \zeta_{it-s+1}, \ldots, \zeta_{it})$.

Moreover,
\begin{align*}
\kappa_i - \ex \kappa_i &= \frac{1}{T}\sum_{t=1}^T \Delta \mathbf{X}_{it}\Delta \varepsilon_{it} - \ex \kappa_i = \frac{1}{T}\sum_{t=1}^T \ex ( \Delta \mathbf{X}_{it}\Delta \varepsilon_{it}|\mathcal{I}_{i,t}) - \ex \kappa_i =\\
&= \frac{1}{T}\sum_{t=1}^T \big( \ex ( \Delta \mathbf{X}_{it}\Delta \varepsilon_{it}|\mathcal{I}_{i,t} ) - \ex ( \mathbf{X}_{it}\Delta \varepsilon_{it}) \big) = \\
&= \frac{1}{T}\sum_{t=1}^T \sum_{s=0}^\infty \big( \ex ( \Delta \mathbf{X}_{it}\Delta \varepsilon_{it}|\mathcal{I}_{i,t-s} ) - \ex ( \Delta \mathbf{X}_{it}\Delta \varepsilon_{it}|\mathcal{I}_{i,t-s-1} )  \big) =\\
&= \frac{1}{T}\sum_{t=1}^T \sum_{s=0}^\infty \mathcal{P}_{i, t-s} (\Delta \mathbf{X}_{it}\Delta \varepsilon_{it}) = \sum_{s=0}^\infty \kappa_{i, s}^{\mathcal{P}}.
\end{align*}
Thus, by {\color{red} Assumption ??}
\[ || \kappa_i - \ex \kappa_i || \leq \sum_{s=0}^\infty ||\kappa_{i, s}^{\mathcal{P}} || \leq \frac{1}{\sqrt{T}}\sum_{s=0}^\infty \delta_2(\Delta \mathbf{U}, s) = O\Big(\frac{1}{\sqrt{T}}\Big)
\]
Since $\ex \kappa_i = 0$, we conclude that
\[  \Big|\Big| \frac{1}{T}\sum_{t=1}^T  \Delta \mathbf{X}_{it}\Delta \varepsilon_{it} \Big|\Big| = O\Big(\frac{1}{\sqrt{T}}\Big).
\]
Therefore, the proposition follows.
\end{proof}



\begin{proof}[\textnormal{\textbf{Proof of Theorem \ref{thm-reg-1}}}]
Recall the differencing estimator $\widehat{\bm{\beta}}_i$:
\begin{align*}
\widehat{\bm{\beta}}_i &= \Big( \sum_{t=1}^T \Delta \mathbf{X}_{it} \Delta \mathbf{X}_{it}^T \Big)^{-1} \sum_{t=1}^T \Delta \mathbf{X}_{it} \Delta Y_{it} =\\
& =  \Big( \sum_{t=1}^T \Delta \mathbf{X}_{it} \Delta \mathbf{X}_{it}^T \Big)^{-1} \sum_{t=1}^T \Delta \mathbf{X}_{it} \bigg(\Delta \mathbf{X}_{it}^T \bm{\beta}_i + \Delta \varepsilon_{it} + O\Big(\frac{1}{T}\Big) \bigg) =\\
&= \bm{\beta}_i +  \Big( \sum_{t=1}^T \Delta \mathbf{X}_{it} \Delta \mathbf{X}_{it}^T \Big)^{-1} \sum_{t=1}^T \Delta \mathbf{X}_{it} \Delta \varepsilon_{it}
+  O\Big(\frac{1}{T}\Big) \Big( \sum_{t=1}^T \Delta \mathbf{X}_{it} \Delta \mathbf{X}_{it}^T \Big)^{-1} \sum_{t=1}^T \Delta \mathbf{X}_{it}. 
\end{align*}
This leads to
\begin{align*}
\sqrt{T}( \widehat{\bm{\beta}}_i - \bm{\beta}_i) &=  \Big(\frac{1}{T} \sum_{t=1}^T \Delta \mathbf{X}_{it} \Delta \mathbf{X}_{it}^T \Big)^{-1}\frac{1}{\sqrt{T}} \sum_{t=1}^T \Delta \mathbf{X}_{it} \Delta \varepsilon_{it} +\\
&+  O\Big(\frac{1}{\sqrt{T}}\Big) \Big( \frac{1}{T}\sum_{t=1}^T \Delta \mathbf{X}_{it} \Delta \mathbf{X}_{it}^T \Big)^{-1} \frac{1}{T}\sum_{t=1}^T \Delta \mathbf{X}_{it}. 
\end{align*}
\end{proof}

%\section{Clustering}
%\subsection{Clustering of time trends}\label{subsec-test-equality-clustering}
%
%
%Consider a situation in which the null hypothesis $H_0: m_1 = m_2 = \ldots = m_n$ is violated. Even though some of the trend functions are different in this case, part of them may still be the same. Put differently, there may be groups of time series which have the same time trend. Formally speaking, we define a group structure as follows: There exist sets or groups of time series $G_1,\ldots,G_N$ with $N \le n$ and $\{1,\ldots,n\} = \mathbin{\dot{\bigcup}}_{\ell=1}^{N} G_\ell$ such that for each $1 \le \ell \le N$,
%\[ m_i = g_\ell \quad \text{for all } i \in G_\ell, \]
%where $g_\ell$ are group-specific trend functions. Hence, the time series which belong to the group $G_\ell$ all have the same time trend $g_\ell$. Throughout the section, we suppose that the group-specific trend functions $g_\ell$ have the following properties: For each $\ell$, $g_\ell = g_{\ell,T}$ is a Lipschitz continuous function with $\int_0^1 g_{\ell,T}(w) dw = 0$. In particular, it holds that $|g_{\ell,T}(v) - g_{\ell,T}(w)| \le L |v-w|$ for all $v,w \in [0,1]$ and some constant $L < \infty$ that does not depend on $T$. Moreover, for any $\ell \ne \ell^\prime$, the trends $g_{\ell,T}$ and $g_{\ell^\prime,T}$ are assumed to differ in the following sense: There exists $(u,h) \in \mathcal{G}_T$ with $[u-h,u+h] \subseteq [0,1]$ such that $g_{\ell,T}(w) - g_{\ell^\prime,T}(w) \ge c_T \sqrt{\log T/(Th)}$ for all $w \in [u-h,u+h]$ or $g_{\ell^\prime,T}(w) - g_{\ell,T}(w) \ge c_T \sqrt{\log T/(Th)}$ for all $w \in [u-h,u+h]$, where $0 < c_T \rightarrow \infty$.
%
%
%In many applications, it is natural to suppose that there is a group structure in the data. In this case, a particular interest lies in estimating the unknown groups from the data at hand. In what follows, we combine our multiscale methods with a clustering algorithm to achieve this. More specifically, we use the multiscale statistics $\widehat{\Psi}_{ij,T}$ as distance measures which are fed into a hierarchical clustering algorithm. To describe the algorithm, we first need to introduce the notion of a dissimilarity measure: Let $S \subseteq \{1,\ldots,n\}$ and $S^\prime \subseteq \{1,\ldots,n\}$ be two sets of time series from our sample. We define a dissimilarity measure between $S$ and $S^\prime$ by setting 
%\begin{equation}\label{dissimilarity}
%\widehat{\Delta}(S,S^\prime) = \max_{\substack{i \in S, \\ j \in S^\prime}} \widehat{\Psi}_{ij,T}. 
%\end{equation}
%This is commonly called a complete linkage measure of dissimilarity. Alternatively, we may work with an average or a single linkage measure. We now combine the dissimilarity measure $\widehat{\Delta}$ with a hierarchical agglomerative clustering (HAC) algorithm which proceeds as follows: 
%\vspace{10pt}
%
%\noindent \textit{Step $0$ (Initialization):} Let $\widehat{G}_i^{[0]} = \{ i \}$ denote the $i$-th singleton cluster for $1 \le i \le n$ and define $\{\widehat{G}_1^{[0]},\ldots,\widehat{G}_n^{[0]} \}$ to be the initial partition of time series into clusters. 
%\vspace{5pt}
%
%\noindent \textit{Step $r$ (Iteration):} Let $\widehat{G}_1^{[r-1]},\ldots,\widehat{G}_{n-(r-1)}^{[r-1]}$ be the $n-(r-1)$ clusters from the previous step. Determine the pair of clusters $\widehat{G}_{\ell}^{[r-1]}$ and $\widehat{G}_{{\ell}^\prime}^{[r-1]}$ for which 
%
%\[ \widehat{\Delta}(\widehat{G}_{\ell}^{[r-1]},\widehat{G}_{{\ell}^\prime}^{[r-1]}) = \min_{1 \le k < k^\prime \le n-(r-1)} \widehat{\Delta}(\widehat{G}_{k}^{[r-1]},\widehat{G}_{k^\prime}^{[r-1]}) \]  
%and merge them into a new cluster. 
%\vspace{10pt}
%
%\noindent Iterating this procedure for $r = 1,\ldots,n-1$ yields a tree of nested partitions $\{\widehat{G}_1^{[r]},\ldots$ $\ldots,\widehat{G}_{n-r}^{[r]}\}$, which can be graphically represented by a dendrogram. Roughly speaking, the HAC algorithm merges the $n$ singleton clusters $\widehat{G}_i^{[0]} = \{ i \}$ step by step until we end up with the cluster $\{1,\ldots,n\}$. In each step of the algorithm, the closest two clusters are merged, where the distance between clusters is measured in terms of the dissimilarity $\widehat{\Delta}$. We refer the reader to Section 14.3.12 in \cite{HastieTibshiraniFriedman2009} for an overview of hierarchical clustering methods. 
%
%
%When the number of groups $N$ is known, we estimate the group structure $\{G_1,\ldots, G_N\}$ by the $N$-partition $\{\widehat{G}_1^{[n-N]},\ldots,\widehat{G}_{N}^{[n-N]}\}$ produced by the HAC algorithm. When $N$ is unknown, we estimate it by the $\widehat{N}$-partition $\{\widehat{G}_1^{[n-\widehat{N}]},\ldots,\widehat{G}_{\widehat{N}}^{[n-\widehat{N}]}\}$, where $\widehat{N}$ is an estimator of $N$. The latter is defined as 
%\[ \widehat{N} = \min \Big\{ r = 1,2,\ldots \Big| \max_{1 \le \ell \le r} \widehat{\Delta} \big( \widehat{G}_\ell^{[n-r]} \big) \le q_{n,T}(\alpha) \Big\}, \]
%where we write $\widehat{\Delta}(S) = \widehat{\Delta}(S,S)$ for short and $q_{n,T}(\alpha)$ is the $(1-\alpha)$-quantile of $\Phi_{n,T}$ defined in Section \ref{subsec-test-equality-test}. 
%
%
%\newpage
%The following proposition summarizes the theoretical properties of the estimators $\widehat{N}$ and $\{ \widehat{G}_1,\ldots,\widehat{G}_{\widehat{N}} \}$, where we use the shorthand $\widehat{G}_\ell = \widehat{G}_\ell^{[n-\widehat{N}]}$ for $1 \le \ell \le \widehat{N}$. 
%\begin{prop}\label{prop-clustering-1}
%Let the conditions of Theorem \ref{theo-stat-equality} be satisfied. Then 
%\[ \pr \Big( \big\{ \widehat{G}_1,\ldots,\widehat{G}_{\widehat{N}} \big\} = \{ G_1,\ldots,G_N \} \Big) \ge (1-\alpha) + o(1) \]
%and 
%\[ \pr \big( \widehat{N} = N \big) \ge (1-\alpha) + o(1). \]
%\end{prop}
%This result allows us to make statistical confidence statements about the estimated clusters $\{ \widehat{G}_1,\ldots,\widehat{G}_{\widehat{N}} \}$ and their number $\widehat{N}$. In particular, we can claim with asymptotic confidence $\ge 1 - \alpha$ that the estimated group structure is identical to the true group structure. Note that it is possible to let the significance level $\alpha$ depend on the sample size $T$ in Proposition \ref{prop-clustering-1}. In particular, we can allow $\alpha = \alpha_T$ to converge slowly to zero as $T \rightarrow \infty$, in which case we obtain that $\pr ( \{ \widehat{G}_1,\ldots,\widehat{G}_{\widehat{N}} \} = \{ G_1,\ldots,G_N \} ) \rightarrow 1$ and $\pr ( \widehat{N} = N ) \rightarrow 1$. The proof of Proposition \ref{prop-clustering-1} can be found in the Supplementary Material.   
%
%
%Our multiscale methods do not only allow us to compute estimators of the unknown groups $G_1,\ldots,G_N$. They also provide information on the locations where two group-specific trend functions $g_\ell$ and $g_{\ell^\prime}$ differ from each other. To turn this claim into a mathematically precise statement, we need to introduce some notation. First of all, note that the indexing of the estimators $\widehat{G}_1,\ldots,\widehat{G}_{\widehat{N}}$ is completely arbitrary. We could, for example, change the indexing according to the rule $\ell \mapsto \widehat{N} - \ell + 1$. In what follows, we suppose that the estimated groups are indexed such that $P( \widehat{G}_\ell = G_\ell \text{ for all } \ell ) \ge (1-\alpha) + o(1)$. Theorem \ref{prop-clustering-1} implies that this is possible without loss of generality. Keeping this convention in mind, we define the sets 
%\[ \mathcal{A}_{n,T}(\ell,\ell^\prime) = \Big\{ (u,h) \in \mathcal{G}_T: \Big| \frac{\widehat{\psi}_{ij,T}(u,h)}{(\widehat{\sigma}_i^2 + \widehat{\sigma}_j^2)^{1/2}} \Big| > q_{n,T}(\alpha) + \lambda(h) \text{ for some } i \in \widehat{G}_\ell, j \in \widehat{G}_{\ell^\prime} \Big\} \] 
%and  
%\[ \Pi_{n,T}(\ell,\ell^\prime) = \big\{ I_{u,h} = [u-h,u+h]: (u,h) \in \mathcal{A}_{n,T}(\ell,\ell^\prime) \big\} \]
%for $1 \le \ell < \ell^\prime \le \widehat{N}$. An interval $I_{u,h}$ is contained in $\Pi_{n,T}(\ell,\ell^\prime)$ if our multiscale test indicates a significant difference between the trends $m_i$ and $m_j$ on the interval $I_{u,h}$ for some $i \in \widehat{G}_\ell$ and $j \in \widehat{G}_{\ell^\prime}$. Put differently,  $I_{u,h} \in \Pi_{n,T}(\ell,\ell^\prime)$ if the test suggests a significant difference between the trends of the $\ell$-th and the $\ell^\prime$-th group on the interval $I_{u,h}$. We further let
%\[ E_{n,T}(\ell,\ell^\prime) = \Big\{ \forall I_{u,h} \in \Pi_{n,T}(\ell,\ell^\prime): g_\ell(v) \ne g_{\ell^\prime}(v) \text{ for some } v \in I_{u,h} = [u-h,u+h] \Big\} \]
%be the event that the group-specific time trends $g_\ell$ and $g_{\ell^\prime}$ differ on all intervals $I_{u,h} \in \Pi_{n,T}(\ell,\ell^\prime)$. With this notation at hand, we can make the following formal statement whose proof is given in the Supplementary Material.
%\begin{prop}\label{prop-clustering-2}
%Under the conditions of Theorem \ref{theo-stat-equality}, the event 
%\[ E_{n,T} = \Big\{ \bigcap_{1 \le \ell < \ell^\prime \le \widehat{N}} E_{n,T}(\ell,\ell^\prime) \Big\} \cap \Big\{ \widehat{N} = N \text{ and } \widehat{G}_\ell = G_\ell \text{ for all } \ell \Big\} \]
%asymptotically occurs with probability $\ge 1-\alpha$, that is, 
%\[ \pr \big( E_{n,T} \big) \ge (1 - \alpha) + o(1). \]
%\end{prop}
%The statement of Proposition \ref{prop-clustering-2} remains to hold true when the sets of intervals $\Pi_{n,T}(\ell,\ell^\prime)$ are replaced by the corresponding sets of minimal intervals. According to Proposition \ref{prop-clustering-2}, the sets $\Pi_{n,T}(\ell,\ell^\prime)$ allow us to locate, with a pre-specified confidence, time regions where the group-specific trend functions $g_\ell$ and $g_{\ell^\prime}$ differ from each other. In particular, we can claim with asymptotic confidence $\ge 1 - \alpha$ that the trend functions $g_\ell$ and $g_{\ell^\prime}$ differ on all intervals $I_{u,h} \in \Pi_{n,T}(\ell,\ell^\prime)$.


\newpage
\bibliographystyle{ims}
{\small
\setlength{\bibsep}{0.55em}
\bibliography{bibliography}}



\end{document}
