\section{State of the art and preliminary work}\label{sec-intro}


The comparison of nonparametric curves is a classic topic in econometrics and statistics. Depending on the specific application, the curves of interest are densities, distribution functions, time trends or regression curves. The problem of testing for equality of densities has been studied in \cite{Mammen1992}, \cite{Anderson1994} and \cite{Li2009} among others. Tests for equality of distribution functions can be found for example in \cite{Kiefer1959}, \cite{Anderson1962} and \cite{Finner2018}. Tests for equality of trend or regression curves have been developed in \cite{HaerdleMarron1990}, \cite{Hall1990}, \cite{Delgado1993}, \cite{DegrasWu2012}, \cite{Zhang2012} and \cite{Hidalgo2014} among many others. In the proposed project, we focus on the comparison of nonparametric trend curves.


The statistical problem of comparing trends has a wide range of applications in economics, finance and other fields such as climatology and biology. In economics, one may wish is to compare trends in real gross domestic product (GDP) across different countries \citep[cp.][]{Grier1989}. Another example concerns the dynamics of long-term interest rates. To better understand these dynamics, researchers aim to compare the yields of US Treasury bills at different maturities over time \citep[cp.][]{Park2009}. In finance, it is of interest to compare the volatility trends of different stocks \citep[cp.][]{Nyblom2000}. Finally, in climatology, researchers are interested in comparing the trending behaviour of temperature time series across different spatial locations \citep[cp.][]{KarolyWu2005}. 


Classically, time trends are modelled stochastically in econometrics; see e.g.\ \cite{Stock1988}. Recently, however, there has been a growing interest in econometric models with deterministic time trends; see \cite{Cai2007}, \cite{Atak2011}, \cite{Robinson2012} and \cite{ChenGaoLi2012} among others. Non- and semiparametric trend modelling has attracted particular interest in a panel data context. \cite{LiChenGao2010}, \cite{Atak2011}, \cite{Robinson2012} and \cite{ChenGaoLi2012} considered panel models where the observed time series have a common time trend. In many applications, however, the assumption of a common time trend is quite harsh. In particular when the number of observed time series is large, it is quite natural to suppose that the time trend may differ across time series. More flexible panel settings with heterogeneous trends have been studied, for example, in \cite{Zhang2012} and \cite{Hidalgo2014}. 

 
In what follows, we consider a general panel framework with heterogeneous trends which is useful for a number of economic and financial applications: Suppose we observe a panel of $n$ time series $\mathcal{Z}_i = \{ (Y_{it},\mathbf{X}_{it}): 1 \le t \le T \}$ for $1 \le i \le n$, where $Y_{it}$ are real-valued random variables and $\mathbf{X}_{it} = (X_{it,1},\ldots,X_{it,d})^\top$ are $d$-dimensional random vectors. Each time series $\mathcal{Z}_i$ is modelled by the equation
\begin{equation}\label{model}
Y_{it} = m_i \Big( \frac{t}{T} \Big) + \bm{\beta}_i^\top \mathbf{X}_{it} + \alpha_i + \varepsilon_{it}
\end{equation}
for $1 \le t \le T$, where $m_i: [0,1] \rightarrow \mathbb{R}$ is a nonparametric (deterministic) trend function, $\mathbf{X}_{it}$ is a vector of regressors or controls and $\bm{\beta}_i$ is the corresponding parameter vector. Moreover, $\alpha_i$ are so-called fixed effect error terms and $\varepsilon_{it}$ are standard regression errors with $\ex[\varepsilon_{it}|\mathbf{X}_{it}] = 0$ for all $t$. Model \eqref{model} nests a number of panel models which have recently been considered in the literature. Special cases of model \eqref{model} with a nonparametric trend specification are for example considered in \cite{Atak2011}, \cite{Zhang2012} and \cite{Hidalgo2014}. Versions of model \eqref{model} with a parametric trend are studied in \cite{Vogelsang2005}, \cite{Sun2011} and \cite{Xu2012} among others.

Within the general framework of model \eqref{model}, we can formulate a number of interesting statistical questions concerning the set of trend functions $\{ m_i:  1 \le i \le n \}$. 

\vspace{10pt}


\noindent \textbf{(a) Testing for equality of nonparametric trend curves} 
\vspace{10pt} 

 
\noindent In many application contexts, an important question is whether the time trends $m_i$ in model \eqref{model} are all the same. Put differently, the question is whether the observed time series
have a common trend. This question can formally be addressed by a statistical test of the null hypothesis 
\[ H_0: \text{There exists a function } m: [0,1] \rightarrow \mathbb{R} \text{ such that } m_i = m  \text{ for all } 1 \le i \le n. \]
A closely related question is whether all time trends have the same parametric form. To formulate the corresponding null hypothesis, let $m(\theta,\cdot): [0,1] \rightarrow \mathbb{R}$ be a function which is known up to the finite-dimensional parameter $\theta \in \Theta$, where $\Theta$ denotes the parameter space. The null hypothesis of interest now reads as follows:  
\[ H_{0,\text{para}}: \text{ There exists } \theta \in \Theta \text{ such that } m_i(\cdot) = m(\theta,\cdot) \text{ for all } 1 \le i \le n. \]  
If $m(\theta,w) = a + b w$ with $\theta = (a,b)$, for example, then $H_0$ is the hypothesis that all trends $m_i$ are linear with the same intercept $a$ and slope $b$. A somewhat simpler but yet important hypothesis is given by 
\[ H_{0,\text{const}}: m_i \equiv 0 \text{ for all } 1 \le i \le n. \]
Under this hypothesis, there is no time trend at all in the observed time series. Put differently, all the time trends $m_i$ are constant. (Note that under the normalization constraint $\int_0^1 m_i(w) dw = 0$, $m_i$ must be equal to zero if it is a constant function.) A major goal of our project is to develop new tests for the hypotheses $H_0$, $H_{0,\text{para}}$ and $H_{0,\text{const}}$ in model \eqref{model}. In order to keep the exposition as clear as possible, we focus attention on the hypothesis $H_0$ in what follows. Tests of $H_{0,\text{para}}$, $H_{0,\text{const}}$ and related hypotheses have for example been studied in \cite{Lyubchich2016} and \cite{ChenWu2018}. 


In recent years, a number of different approaches have been developed to test the hypothesis $H_0$. \cite{DegrasWu2012} consider the problem of testing $H_0$ within the model framework
\begin{equation}\label{model-degras}
Y_{it} = m_i \Big( \frac{t}{T} \Big) + \alpha_i + \varepsilon_{it} \qquad (1 \le t \le T, \, 1 \le i \le n), 
\end{equation}
where $\mathbb{E}[\varepsilon_{it}] = 0$ for all $i$ and $t$ and the terms $\alpha_i$ are assumed to be deterministic. Obviously, \eqref{model-degras} is a special case of \eqref{model} which does not include additional regressors. \cite{DegrasWu2012} construct an $L_2$-type statistic to test $H_0$. The statistic is based on the difference between estimators of the trend with and without imposing $H_0$. Let $\hat{m}_{i,h}$ be the estimator of $m_i$ and $\hat{m}_h$ the estimator of the common trend $m$ under $H_0$, where $h$ denotes the bandwidth parameter. With these estimators, the authors define the statistic
\begin{equation}\label{stat-degras}
\Delta_{n,T} = \sum_{i=1}^n \int_0^1 \big(\hat{m}_{i,h}(u) - \hat{m}_h(u)\big)^2 du, 
\end{equation} 
which measures the $L_2$-distance between $\hat{m}_{i, h}$ and $\hat{m}_h$. In the theoretical part of their paper, they derive the limit distribution of $\Delta_{n,T}$. \cite{ChenWu2018} develop theory for test statistics closely related to those from \cite{DegrasWu2012}, but under more general conditions on the error terms. 


\cite{Zhang2012} investigate the problem of testing the hypothesis $H_0$ in a slightly restricted version of model \eqref{model}, where $\beta_i = \beta$ for all $i$. The regression coefficients $\beta_i$ are thus assumed to be homogeneous in their setting. They construct a residual-based test statistic as follows: First, they obtain profile least squares estimators $\hat{\beta}$ and $\hat{m}_h(t/T)$ of the parameter vector $\beta$ and the common trend $m$ under $H_0$, where $h$ denotes the bandwidth. With these estimators, they compute the residuals $\hat{u}_{it} = Y_{it} - \hat{\beta}^T X_{it} - \hat{m}_h(t/T)$. These residuals are shown to have the form $\hat{u}_{it} = \Delta_i(t/T) + \eta_{it}$, where $\Delta_i$ is a deterministic function with the property that $\Delta_i \equiv 0$ under $H_0$ and $\eta_{it}$ denotes the error term. Testing $H_0$ is thus equivalent to testing the hypothesis $H_0^\prime: \Delta_i \equiv 0$ for all $1 \le i \le n$. The authors construct a test statistic for the hypothesis $H_0^\prime$ on the basis of nonparametric kernel estimators of the functions $\Delta_i$ and derive its limit distribution.  


The tests of \cite{Zhang2012}, \cite{DegrasWu2012} and \cite{ChenWu2018} are based on nonparametric estimators of the trend functions $m_i$ that depend on one or several bandwidth parameters. Unfortunately, it is far from clear how to choose these bandwidths in an appropriate way. This is a general problem concerning essentially all tests based on nonparametric curve estimators. There are of course many theoretical results on optimal bandwidth choice for estimation purposes. However, the optimal bandwidth for curve estimation is usually not optimal for testing. Optimal bandwidth choice for tests is indeed an open problem, and only little theory for simple cases is available \citep[cp.][]{GaoGijbels2008}. Since tests based on nonparametric curve estimators are commonly quite sensitive to the choice of bandwidth and theory for optimal bandwidth selection is not available, it appears preferable to work with bandwidth-free tests. 


A classical way to obtain a bandwidth-free test of the hypothesis $H_0$ is to use CUSUM-type statistics which are based on partial sum processes. This approach is taken in \cite{Hidalgo2014}. A more modern approach to obtain a bandwidth-free test is to employ multiscale methods. These methods avoid the need to choose a bandwidth by considering a large collection of bandwidths simultaneously. More specifically, the basic idea is as follows: Let $S_h$ be a test statistic for the null hypothesis of interest, which depends on the bandwidth $h$. Rather than considering only a single statistic $S_h$ for a specific bandwidth $h$, a multiscale approach simultaneously considers a whole family of statistics $\{S_h: h \in \mathcal{H} \}$, where $\mathcal{H}$ is a set of bandwidth values. The multiscale test then proceeds as follows: For each bandwidth or scale $h$, one checks whether $S_h > q_h(\alpha)$, where $q_h(\alpha)$ is a bandwidth-dependent critical value (for given significance level $\alpha$). The multiscale test rejects if $S_h > q_h(\alpha)$ for at least one scale $h$. The main theoretical difficulty in this approach is of course to derive appropriate critical values $q_h(\alpha)$. Specifically, the critical values $q_h(\alpha)$ need to be determined such that the multiscale test has the correct (asymptotic) level, that is, such that $\pr (S_h > q_h(\alpha) \text{ for some } h \in \mathcal{H} ) = (1-\alpha) + o(1)$. 


Multiscale methods have been developed for a variety of different test problems in recent years. \cite{ChaudhuriMarron1999, ChaudhuriMarron2000} introduced the so-called SiZer method which has been extended in various directions; see for example \cite{HannigMarron2006} and \cite{Rondonotti2007}. \cite{HorowitzSpokoiny2001} proposed a multiscale test for the parametric form of a regression function. \cite{DuembgenSpokoiny2001} constructed a multiscale approach which works with additively corrected supremum statistics. This general approach has been very influential in recent years and has been further developed in numerous ways; see for example \cite{Duembgen2002}, \cite{Rohde2008} and \cite{ProkschWernerMunk2018} for multiscale methods in the regression context and \cite{DuembgenWalther2008}, \cite{RufibachWalther2010}, \cite{SchmidtHieber2013} and \cite{EckleBissantzDette2017} for methods in the context of density estimation. Importantly, all of these studies are restricted to the case of independent data. It turns out that it is highly non-trivial to extend the multiscale approach of \cite{DuembgenSpokoiny2001} to the case of dependent data. A first step into this direction has recently been made in \cite{KhismatullinaVogt2018}. They developed multiscale methods to test for local increases/decreases of the nonparametric trend function $m$ in the univariate time series model $Y_t = m(t/T) + \varepsilon_t$.  


To the best of our knowledge, multiscale tests of the hypotheses $H_0$, $H_{0,\text{para}}$ and $H_{0,\text{const}}$ in model \eqref{model} are not available in the literature. The only exception is \cite{Park2009} who developed SiZer methods for the comparison of nonparametric trend curves in a strongly simplified version of model \eqref{model}. Their analysis, however, is mainly methodological and not fully backed up by theory. Indeed, theory has only been derived for the special case $n=2$, that is, for the case that only two time series are observed. 
\vspace{10pt}


\noindent \textbf{(b) Clustering of nonparametric trend curves} 
\vspace{10pt} 


\noindent Consider the situation that the null hypothesis $H_0: m_1 = \ldots = m_n$ is violated in the general panel data model \eqref{model}. Even though some of the trend functions $m_i$ are different in this case, there may still be groups of time series with the same time trend. Formally, a group stucture can be defined as follows within the framework of model \eqref{model}: There exist sets or groups of time series $G_1,\ldots,G_{K_0}$ with $\{1,\ldots,n\} = \dot\bigcup_{k=1}^{K_0} G_k$ such that for each $1 \le k \le K_0$, 
\begin{equation}\label{model-groups}
m_i = m_j \quad \text{for all } i,j \in G_k. 
\end{equation}
According to \eqref{model-groups}, the time series of a given group $G_k$ all have the same time trend. In many applications, it is very natural to suppose that there is such a group structure in the data. An interesting statistical problem which we aim to investigate in our project is how to estimate the unknown groups $G_1,\ldots,G_{K_0}$ and their unknown number $K_0$ from the data. 


Several approaches to this problem have been proposed in the context of models closely related to \eqref{model}. \cite{DegrasWu2012} used a repeated testing procedure based on $L_2$-type test statistics of the form \eqref{stat-degras} in order to estimate the unknown group structure in model \eqref{model-degras}. \cite{Zhang2013} developed a clustering method within the same model framework which makes use of an extended Bayesian information criterion. \cite{VogtLinton2017} constructed a thresholding method to estimate the unknown group structure in the panel model $Y_{it} = m_i(X_{it}) + u_{it}$, where $X_{it}$ are random regressors and $u_{it}$ are general error terms that may include fixed effects. Their approach can also be adapted to the case of fixed regressors $X_{it} = t/T$.  As an alternative to a group structure, factor-type structures have been imposed on the trend and regression functions in panel models. Such factor-type structures are studied in \cite{Kneip2012}, \cite{LintonVogt2015} and \cite{BonevaLintonVogt2016} among others. 


The problem of estimating the unknown groups $G_1,\ldots,G_{K_0}$ and their unknown number $K_0$ in model \eqref{model} has close connections to functional data clustering. There, the aim is to cluster smooth random curves that are functions of (rescaled) time and that are observed with or without noise. A number of different clustering approaches have been proposed in the context of functional data models; see for example \cite{Abraham2003}, \cite{Tarpey2003} and \cite{Tarpey2007} for procedures based on $k$-means clustering, \cite{James2003} and \cite{Chiou2007} for model-based clustering approaches and \cite{Jacques2014} for a recent survey. 


The problem of finding the unknown group structure in model \eqref{model} is also closely related to a developing literature in econometrics which aims to identify unknown group structures in parametric panel regression models. In its simplest form, the panel regression model under consideration is given by the equation $Y_{it} = \beta_i^\top X_{it} + u_{it}$ for $1 \le t \le T$ and $1 \le i \le n$, where the coefficient vectors $\beta_i$ are allowed to vary across individuals $i$ and the error terms $u_{it}$ may include fixed effects. Similar to the trend functions in model \eqref{model}, the coefficients $\beta_i$ are assumed to belong to a number of groups: there are $K_0$ groups $G_1,\ldots,G_{K_0}$ such that $\beta_i = \beta_j$ for all $i,j \in G_k$ and all $1\le k \le K_0$. The problem of estimating the unknown groups and their unknown number has been studied in different versions of this modelling framework; cp.\ \cite{Su2016}, \cite{Su2018} and \cite{Wang2018} among others. \cite{Bonhomme2015} considered a related model where the group structure is not imposed on the regression coefficients but rather on some unobserved time-varying fixed effect components of the panel model. 


Virtually all the proposed procedures to cluster nonparametric curves in panel and functional data models related to \eqref{model} depend on a number of bandwidth or smoothing parameters required to estimate the nonparametric functions $m_i$. In general, nonparametric curve estimators are strongly affected by the chosen bandwidth parameters. A clustering procedure which is based on such estimators can be expected to be strongly influenced by the choice of bandwidths as well. Moreover, as in the context of statistical testing, there is no theory available on how to pick the bandwidths optimally for the clustering problem. Hence, as in the context of testing, it is desirable to construct a clustering procedure which is free of bandwidth or smoothing parameters that need to be selected. 


One way to obtain a clustering method which does not require to select any bandwidth parameter is to use multiscale methods. This approach has recently been taken in \cite{VogtLinton2018}. They develop a clustering approach in the context of the panel model $Y_{it} = m_i(X_{it}) + u_{it}$, where $X_{it}$ are random regressors and $u_{it}$ are general error terms that may include fixed effects. Imposing the same group structure as in \eqref{model-groups} on their  model, they construct estimators of the unknown groups and their unknown number as follows: In a first step, they develop bandwidth-free multiscale statistics $\hat{d}_{ij}$ which measure the distance between pairs of functions $m_i$ and $m_j$. To construct them, they make use of the multiscale testing methods described in part (a) of this section. In a second step, the statistics $\hat{d}_{ij}$ are employed as dissimilarity measures in a hierarchical clustering algorithm. 
