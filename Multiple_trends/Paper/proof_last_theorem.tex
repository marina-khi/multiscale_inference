\documentclass[a4paper,12pt]{article}
\usepackage{amsmath, bm}
\usepackage{amssymb,amsthm,graphicx}
\usepackage{enumitem}
\usepackage{color}
%\usepackage{epsfig}
%\usepackage{graphics}
%\usepackage{pdfpages}
%\usepackage{subcaption}
%\usepackage[font=small]{caption}
%\usepackage[hang,flushmargin]{footmisc} 
\usepackage{float}
%\usepackage{booktabs}
\usepackage[mathscr]{euscript}
\usepackage{natbib}
%\usepackage{setspace}
%\usepackage{mathrsfs}
\usepackage{bibentry}
\usepackage[left=2.7cm,right=2.7cm,bottom=2.7cm,top=2.7cm]{geometry}
\parindent0pt 

\newcommand{\doublehat}[1]{\skew{5.5}\widehat{\widehat{#1}}}
\newcommand{\doublehattwo}[1]{\widehat{\widehat{#1}}}


% General

\newcommand{\reals}{\mathbb{R}}
\newcommand{\integers}{\mathbb{Z}}
\newcommand{\naturals}{\mathbb{N}}

\newcommand{\pr}{\mathbb{P}}        % probability
\newcommand{\ex}{\mathbb{E}}        % expectation
\newcommand{\var}{\textnormal{Var}} % variance
\newcommand{\cov}{\textnormal{Cov}} % covariance

\newcommand{\law}{\mathcal{L}} % law of X
\newcommand{\normal}{N}        % normal distribution 

\newcommand{\argmax}{\textnormal{argmax}}
\newcommand{\argmin}{\textnormal{argmin}}

\newcommand{\ind}{\boldsymbol{1}} % indicator function
\renewcommand{\ker}{W} % kernel function

\newcommand{\X}{X}
\newcommand{\pairs}{\mathcal{S}}
\newcommand{\countries}{\mathcal{C}}
\newcommand{\intervals}{\mathcal{F}}
\newcommand{\indexset}{\mathcal{M}}

% Convergence

\newcommand{\convd}{\stackrel{d}{\longrightarrow}}              % convergence in distribution
\newcommand{\convp}{\stackrel{P}{\longrightarrow}}              % convergence in probability
\newcommand{\convas}{\stackrel{\textrm{a.s.}}{\longrightarrow}} % convergence almost surely
\newcommand{\convw}{\rightsquigarrow}                           % weak convergence


% Theorem-like declarations

\theoremstyle{plain}

\newtheorem{theorem}{Theorem}[section]
\newtheorem{prop}{Proposition}[section]
\newtheorem{corollary}{Corollary}[section]
\newtheorem{lemma}{Lemma}[section]
\newtheorem{definition}{Definition}[section]
\newtheorem{remark}{Remark}[section]
\newtheorem{algo}{Algorithm}
\newtheorem{theoremA}{Theorem}[section]
\newtheorem{propA}{Proposition}[section]
\newtheorem{corollaryA}{Corollary}[section]
\newtheorem{lemmaA}{Lemma}[section]
\renewcommand{\thetheoremA}{A.\arabic{theoremA}}
\renewcommand{\thepropA}{A.\arabic{propA}}
\renewcommand{\thecorollaryA}{A.\arabic{corollaryA}}
\renewcommand{\thelemmaA}{A.\arabic{lemmaA}}
\newtheorem{lemmaS}{Lemma}[section]
\renewcommand{\thelemmaS}{S.\arabic{lemmaS}}


% Theorem numbering to the left

\makeatletter
\newcommand{\lefteqno}{\let\veqno\@@leqno}
\makeatother


% Heading

\newcommand{\heading}[3]
{  \setcounter{page}{1}
   \begin{center}

   %\phantom{Distance to upper boundary}
   %\vspace{0.5cm}

   {\LARGE \textbf{#1}}
   \vspace{0.25cm}

   {\LARGE \textbf{#2}}
   \vspace{0.25cm}

   {\LARGE \textbf{#3}}
   \end{center}
}

\newcommand{\headingsupplement}[4]
{  \setcounter{page}{1}
   \begin{center}

   %\phantom{Distance to upper boundary}
   %\vspace{0.5cm}

   {\LARGE \textbf{#1}}
   \vspace{0.25cm}

   {\LARGE \textbf{#2}}
   \vspace{0.25cm}

   {\LARGE \textbf{#3}}
   \vspace{0.25cm}

   {\LARGE \textbf{#4}}
   \end{center}
}


% Authors

\newcommand{\authors}[4]
{  %\parindent0pt
   \begin{center}
      \phantom{-----------------}
      \begin{minipage}[c][1.5cm][c]{5.5cm}
      \begin{center} 
      {\large #1}  
      \vspace{0.1cm}      

      #2 
      \end{center}
      \end{minipage}
      \begin{minipage}[c][1.5cm][c]{5.5cm}
      \begin{center} 
      {\large #3} 
      \vspace{0.1cm}

      #4 \phantom{-}
      \end{center}
      \end{minipage}
      \phantom{---------}
   \end{center}
}

\newcommand{\authorssupplement}[4]
{  %\parindent0pt
   \begin{center}
      \phantom{-----------------}
      \begin{minipage}[c][1.5cm][c]{5.5cm}
      \begin{center} 
      {\large #1} \\[0.01cm]            
      #2 
      \end{center}
      \end{minipage}
      \begin{minipage}[c][1.5cm][c]{5.5cm}
      \begin{center} 
      {\large #3} \\[0.01cm]      
      #4 
      \end{center}
      \end{minipage}
      \phantom{---------}
   \end{center}
}


% Version

\newcommand{\version}[1]
{  \begin{center}
   {\large #1}
   \end{center}   
} 










\begin{document}



\heading{Multiscale Testing for Equality}{of Nonparametric Trend Curves}

\vspace{-0.5cm}

\authors{Marina Khismatullina\renewcommand{\thefootnote}{1}\footnotemark[1]}{University of Bonn}{Michael Vogt\renewcommand{\thefootnote}{2}\footnotemark[2]}{University of Bonn} 
\footnotetext[1]{Address: Bonn Graduate School of Economics, University of Bonn, 53113 Bonn, Germany. Email: \texttt{marina.k@uni-bonn.de}.}
\renewcommand{\thefootnote}{2}
\footnotetext[2]{Corresponding author. Address: Department of Economics and Hausdorff Center for Mathematics, University of Bonn, 53113 Bonn, Germany. Email: \texttt{michael.vogt@uni-bonn.de}.}
\renewcommand{\thefootnote}{\arabic{footnote}}
\setcounter{footnote}{0}

%\vspace{-0.5cm}

%\version{\today}




\renewcommand{\abstractname}{}


\enlargethispage{0.25cm}
\renewcommand{\baselinestretch}{1.2}\normalsize



\numberwithin{equation}{section}
\allowdisplaybreaks[1]




\begin{proof}[\textnormal{\textbf{Proof of Theorem \ref{theo-regs}}}]
Define $\Delta m_{it} = m_i \left( \frac{t}{T} \right) - m_i \left(\frac{t-1}{T}\right)$.

Recall the differencing estimator $\widehat{\bm{\beta}}_i$:
\begin{align*}
\widehat{\bm{\beta}}_i &= \Big( \sum_{t=1}^T \Delta \mathbf{X}_{it} \Delta \mathbf{X}_{it}^\top \Big)^{-1} \sum_{t=1}^T \Delta \mathbf{X}_{it} \Delta Y_{it} =\\
& =  \Big( \sum_{t=1}^T \Delta \mathbf{X}_{it} \Delta \mathbf{X}_{it}^\top \Big)^{-1} \sum_{t=1}^T \Delta \mathbf{X}_{it} \bigg(\Delta \mathbf{X}_{it}^\top \bm{\beta}_i + \Delta \varepsilon_{it} +  \Delta m_{it}\bigg) =\\
&= \bm{\beta}_i +  \Big( \sum_{t=1}^T \Delta \mathbf{X}_{it} \Delta \mathbf{X}_{it}^\top \Big)^{-1} \sum_{t=1}^T \Delta \mathbf{X}_{it} \Delta \varepsilon_{it}
+   \Big( \sum_{t=1}^T \Delta \mathbf{X}_{it} \Delta \mathbf{X}_{it}^\top \Big)^{-1} \sum_{t=1}^T \Delta \mathbf{X}_{it} \Delta m_{it}. 
\end{align*}
This leads to
\begin{align}\label{theo-regs-proof-1}
\begin{split}
\big| \sqrt{T}( \widehat{\bm{\beta}}_i - \bm{\beta}_i) \big| \leq & \bigg| \Big(\frac{1}{T} \sum_{t=1}^T \Delta \mathbf{X}_{it} \Delta \mathbf{X}_{it}^\top \Big)^{-1}\frac{1}{\sqrt{T}} \sum_{t=1}^T \Delta \mathbf{X}_{it} \Delta \varepsilon_{it} \bigg|+\\
&+  \bigg|\Big( \frac{1}{T}\sum_{t=1}^T \Delta \mathbf{X}_{it} \Delta \mathbf{X}_{it}^\top \Big)^{-1} \frac{1}{\sqrt{T}}\sum_{t=1}^T \Delta \mathbf{X}_{it} \Delta m_{it}\bigg|.
\end{split}
\end{align}
First, we take a closer look at the second summand in \eqref{theo-regs-proof-1}. By the assumption in Theorem \ref{theo-regs}, $m_i(\cdot)$ is Lipschitz continuous, that is,  $|\Delta m_{it}| = \left|m_i \left( \frac{t}{T} \right) - m_i \left(\frac{t-1}{T}\right) \right| \leq C \frac{1}{T}$ for all $t \in \{1, \ldots, T\}$ and some constant $C > 0$. Hence, 
\begin{align}\label{theo-regs-proof-2}
\begin{split}
	\left|\frac{1}{\sqrt{T}}\sum_{t=1}^T \Delta \mathbf{X}_{it} \Delta m_{it} \right| &= \left| \frac{1}{\sqrt{T}}\sum_{t=1}^T \Delta  \mathbf{H}_i (\mathcal{U}_{it})\Delta m_{it}\right| \leq \\
	&\leq \frac{1}{\sqrt{T}}\sum_{t=1}^T \left|\Delta  \mathbf{H}_i (\mathcal{U}_{it})\right| \cdot \left| \Delta m_{it} \right| \leq \\
	&\leq \frac{1}{\sqrt{T}} \sum_{t=1}^T \left|\Delta  \mathbf{H}_i (\mathcal{U}_{it})\right| \cdot \left| \Delta m_{it} \right|  \leq \frac{C}{\sqrt{T}} \cdot \frac{1}{T} \sum_{t=1}^T \left|\Delta  \mathbf{H}_i (\mathcal{U}_{it})\right|.
\end{split}
\end{align}

Now, in order to show that the whole vector $\frac{1}{T} \sum_{t=1}^T \left|\Delta  \mathbf{H}_i (\mathcal{U}_{it})\right|$ is $O_P(1)$, we will do that for every element $\frac{1}{T} \sum_{t=1}^T \left|\Delta  H_{ij} (\mathcal{U}_{it})\right|$ of this vector separately.

Fix $j \in {1, \ldots, d}$. By Chebyshev's inequality we have
\begin{align}\label{theo-regs-proof-3}
\pr \left(\frac{1}{T} \sum_{t=1}^T \left|\Delta  H_{ij} (\mathcal{U}_{it})\right| > a \right) \leq \frac{\ex \left[ \left(\frac{1}{T} \sum_{t=1}^T \left|\Delta  H_{ij} (\mathcal{U}_{it})\right|\right)^2 \right]}{a^2}
\end{align}
and 
\begin{align}\label{theo-regs-proof-4}
\begin{split}
\ex &\left[ \left(\frac{1}{T} \sum_{t=1}^T \left|\Delta  H_{ij} (\mathcal{U}_{it})\right|\right)^2 \right] = \frac{1}{T^2}\ex \left[ \left(\sum_{t=1}^T \left|\Delta  H_{ij} (\mathcal{U}_{it})\right|\right)^2 \right]  =\\
&=  \frac{1}{T^2}\sum_{t=1}^T \ex \left[ \Delta  H^2_{ij} (\mathcal{U}_{it})  \right] + \frac{1}{T^2}\sum_{t=1, s = 1, t\neq s}^T\ex \big[ \left|\Delta  H_{ij} (\mathcal{U}_{it}) \Delta  H_{ij} (\mathcal{U}_{is})\right| \big].
\end{split}
\end{align}
Note that by the Cauchy-Scwarz inequality for all $t$ and $s$ we have
\begin{align*}
0 \leq \ex \big[ \left| H_{ij}(\mathcal{U}_{it}) H_{ij}(\mathcal{U}_{is})\right| \big] \leq \sqrt{\ex\big[ H^2_{ij}(\mathcal{U}_{it})\big]} \sqrt{ \ex \big[H^2_{ij}(\mathcal{U}_{is})\big]} = \ex \left[ H^2_{ij}(\mathcal{U}_{i0}) \right] 
\end{align*}
and 
\begin{align}\label{theo-regs-proof-5}
0 \leq  \left|\ex \big[ H_{ij}(\mathcal{U}_{it}) H_{ij}(\mathcal{U}_{is}) \big]\right|\leq \ex \big[ \left| H_{ij}(\mathcal{U}_{it}) H_{ij}(\mathcal{U}_{is})\right| \big] \leq  \ex \left[ H^2_{ij}(\mathcal{U}_{i0}) \right].
\end{align}

Hence, 
\begin{align*}
0 \leq  \ex \left[ \Delta  H^2_{ij} (\mathcal{U}_{it})  \right]  &= \ex \left[ \left(H_{ij} (\mathcal{U}_{it}) - H_{ij}(\mathcal{U}_{it-1}) \right)^2  \right] =\\
&=  \ex \left[ H^2_{ij} (\mathcal{U}_{it}) \right] - 2\ex \left[ H_{ij} (\mathcal{U}_{it}) H_{ij}(\mathcal{U}_{it-1}) \right]  + \ex \left[ H^2_{ij}(\mathcal{U}_{it-1}) \right] \leq \\
& \leq \ex \left[ H^2_{ij} (\mathcal{U}_{i0}) \right] + 2\ex \left[ H^2_{ij} (\mathcal{U}_{i0}) \right]  + \ex \left[ H^2_{ij}(\mathcal{U}_{i0}) \right] = \\
&= 4 \ex \left[ H^2_{ij} (\mathcal{U}_{i0}) \right]
\end{align*}
and the first summand in \eqref{theo-regs-proof-4} can be bounded by $\frac{4}{T} \ex \left[ H^2_{ij} (\mathcal{U}_{i0}) \right]$.

Now to the second summand in \eqref{theo-regs-proof-4}:
\begin{align*}
0 &\leq \ex \big[ \left|\Delta  H_{ij} (\mathcal{U}_{it}) \Delta  H_{ij} (\mathcal{U}_{is})\right| \big] =  \ex \Big[ \big| \left(H_{ij} (\mathcal{U}_{it}) - H_{ij} (\mathcal{U}_{it-1}) \right) \left(H_{ij} (\mathcal{U}_{is}) - H_{ij} (\mathcal{U}_{is-1})\right)\big| \Big] \leq \\
&\leq \ex \big[ \left| H_{ij} (\mathcal{U}_{it}) H_{ij} (\mathcal{U}_{is}) \right|\big] + \ex \big[ \left| H_{ij} (\mathcal{U}_{it-1}) H_{ij} (\mathcal{U}_{is}) \right|\big] +\\
&\quad + \ex \big[ \left| H_{ij} (\mathcal{U}_{it}) H_{ij} (\mathcal{U}_{is-1}) \right|\big] + \ex \big[ \left| H_{ij} (\mathcal{U}_{it-1}) H_{ij} (\mathcal{U}_{is-1}) \right|\big] \leq \\
&\leq  4\ex \left[ H^2_{ij}(\mathcal{U}_{i0}) \right],
\end{align*}
where in the last inequality we used \eqref{theo-regs-proof-5}. This means that the second summand in \eqref{theo-regs-proof-4} can be bounded by $\frac{4T(T-1)}{T^2} \ex \left[ H^2_{ij} (\mathcal{U}_{i0}) \right]= \frac{4(T-1)}{T} \ex \left[ H^2_{ij} (\mathcal{U}_{i0}) \right] $.

Plugging these bounds in \eqref{theo-regs-proof-4}, we get
\begin{align*}
\ex &\left[ \left(\frac{1}{T} \sum_{t=1}^T \left|\Delta  H_{ij} (\mathcal{U}_{it})\right|\right)^2 \right] \leq \frac{4}{T} \ex \left[ H^2_{ij} (\mathcal{U}_{i0}) \right] + \frac{4(T-1)}{T} \ex \left[ H^2_{ij} (\mathcal{U}_{i0}) \right] = 4 \ex \left[ H^2_{ij} (\mathcal{U}_{i0}) \right],
\end{align*}
which together with \eqref{theo-regs-proof-3} leads to $\frac{1}{T}\sum_{t=1}^T \big|\Delta H_{ij}(\mathcal{U}_{it})\big| = O_P(1)$. Since it holds for each $j\in\{1, \ldots, d\}$, we can establish that
\begin{align}\label{theo-regs-proof-6}
\frac{1}{T}\sum_{t=1}^T \big| \Delta \mathbf{H}_i (\mathcal{U}_{it}) \big| =  \frac{1}{T}\sum_{t=1}^T \big|\Delta \mathbf{X}_{it}\big| = O_P(1).
\end{align}


Similarly, by Proposition \ref{prop-reg-1} and Chebyshev's inequality, we have that for each $j, k\in\{1, \ldots, d\}$
\[  \Big|\frac{1}{T}\sum_{t=1}^T \Delta H_{ij}(\mathcal{U}_{it}) \Delta H_{ik}(\mathcal{U}_{it})\Big| = O_P(1),
\]
which leads to 
\begin{align*}
\Big| \frac{1}{T}\sum_{t=1}^T \Delta \mathbf{H}_i (\mathcal{U}_{it})\Delta \mathbf{H}_i (\mathcal{U}_{it})^\top \Big| =\Big|\frac{1}{T}\sum_{t=1}^T\Delta \mathbf{X}_{it} \Delta \mathbf{X}_{it}^\top\Big| = O_P(1),
\end{align*}
where $|A|$ with $A$ being a matrix is any matrix norm.

By Assumption \ref{C-reg2}, we know that $\ex [\Delta \mathbf{X}_{it} \Delta \mathbf{X}_{it}^\top] = \ex [\Delta \mathbf{X}_{i0} \Delta \mathbf{X}_{i0}^\top]$ is invertible, thus, 
\begin{align}\label{theo-regs-proof-7}
\Bigg|  \Big(\frac{1}{T}\sum_{t=1}^T\Delta \mathbf{X}_{it} \Delta \mathbf{X}_{it}^\top\Big)^{-1}\Bigg| = O_P(1).
\end{align}
Plugging \eqref{theo-regs-proof-6} into \eqref{theo-regs-proof-2} and combining it with \eqref{theo-regs-proof-7}, we get that the second summand in \eqref{theo-regs-proof-1} is $O_P(1/\sqrt{T})$.

Furthermore, we can apply the Proposition \ref{prop-reg-4} together with \eqref{theo-regs-proof-7} to get that the first summand in \eqref{theo-reg-proof-1} is $O_P(1)$. And the statement of the theorem follows.
\end{proof}


%\[\var \Big[ \frac{1}{T}\sum_{t=1}^T \Delta H_{ij}(\mathcal{U}_{it})\Big] \leq \frac{4}{T^2} \ex\big[H_{ij}^2(\mathcal{U}_{it})\big],\]
%by Chebyshev's inequality we have that $\Big|\frac{1}{T}\sum_{t=1}^T \Delta H_{ij}(\mathcal{U}_{it})\Big| = O_P(1)$. for each $j\in\{1, \ldots, d\} $. 
%
%The latter is true because
%\begin{align*}
%E\big[(X_{it} - \bar{X}_{i})^2\big] = E X_{it}^2 - 2 E (X_{it}\bar{X}_i) + E \bar{X}_i^2.
%\end{align*}
%Now according to \ref{C-reg1}, we have $ E X_{it}^2 = E X_{i0}^2$. Moreover, 
%and
%\begin{align*}
%E \bar{X}_i^2 &= \frac{1}{T^2} \sum_{t=1}^T \sum_{s=1}^T E (X_{it} X_{is})  \leq \\
%&\leq \frac{1}{T^2} \sum_{t=1}^T \sum_{s=1}^T \sqrt{E X_{it}^2 E X_{is}^2} = \frac{1}{T^2} \sum_{t=1}^T \sum_{s=1}^T E X_{i0}^2  = E X_{i0}^2.
%\end{align*}
%Hence,, 
%\begin{align*}
%E\big[(X_{it} - \bar{X}_{i})^2\big] \leq 4 E X_{i0}^2 < \infty,
%\end{align*}
%which proves \eqref{sec-moment}.
%together with the Cauchy-Schwarz inequality to obtain
%\begin{align*}
%\Big| \sum_{t=1}^T w_{t,T}(u,h) (X_{it} - \bar{X}_{j})  \Big| &= \Big| \sum_{t=1}^T w_{t,T}(u,h) \mathbf{1}_{\{T(u-h) \le t \le T(u+h)\}} (X_{it} - \bar{X}_{i})  \Big| \le \\
%&\le  \Big( \sum_{t=1}^T w^2_{t,T}(u,h) \Big)^{1/2} \Big( \sum_{t=1}^T \mathbf{1}_{\{T(u-h) \le t \le T(u+h)\}} (X_{it} - \bar{X}_{i})^2  \Big)^{1/2} = \\
%& =  \Big( \sum_{t=1}^T w^2_{t,T}(u,h) \Big)^{1/2} \Big( \sum_{t=\lfloor T(u-h) \rfloor}^{\lceil T(u+h) \rceil} (X_{it} - \bar{X}_{i})^2  \Big)^{1/2}.
%\end{align*}








\end{document}