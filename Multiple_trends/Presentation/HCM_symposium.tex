\documentclass[10pt, handout]{beamer}

\usetheme[progressbar=frametitle]{metropolis}

\usepackage{appendixnumberbeamer}

\usepackage{booktabs}
\usepackage{blkarray}
%\usepackage{ccicons}
\usepackage{graphicx}
\usepackage{color}

\definecolor{UniBlue}{RGB}{7,82,154}
\definecolor{UniYellow}{RGB}{234,185,12}

%\setbeamercolor{title}{fg=UniBlue, bg = UniYellow}
%\setbeamercolor{frametitle}{fg=UniBlue, bg= UniYellow}
%\setbeamercolor{structure}{fg=UniBlue, bg= UniYellow}
%\setbeamercolor{progress bar}{fg=UniBlue, bg= UniYellow}
\usepackage{xspace}

\title{Multiscale inference for nonparametric time trends}
\date{August 24, 2021\\ HCM Symposium}
\author{Marina Khismatullina \and Michael Vogt}
\setbeamertemplate{frame footer}{Multiscale inference for nonparametric time trends}
\metroset{block=fill}
% \titlegraphic{\hfill\includegraphics[height=1.5cm]{logo.pdf}}

\newcommand{\Prob}{\mathrm{P}}
\newcommand{\E}{\mathbb{E}}
\newcommand{\Eps}{\mathcal{E}}
\newcommand{\eps}{\varepsilon}
\newcommand{\reals}{\mathbb{R}}
\newcommand{\X}{\boldsymbol{X}}
\newcommand{\bfbeta}{\boldsymbol{\beta}}
\newcommand{\Var}{\mathrm{Var}}
\newcommand{\Cov}{\mathrm{Cov}}
\newcommand{\Corr}{\mathrm{Corr}}
\newcommand{\sgn}{\text{sgn}}
\newtheorem{prop}{Proposition}
\newcommand{\ind}{\boldsymbol{1}\Big( \frac{t}{T} \in \mathcal{I}_k \Big)} % indicator function
\newcommand{\indsmall}{\boldsymbol{1}\big( \frac{t}{T} \in \mathcal{I}_k \big)} % indicator function

\begin{document}

\maketitle

\begin{frame}{Table of contents}
  \setbeamertemplate{section in toc}[sections numbered]
  \tableofcontents[hideallsubsections]
\end{frame}

\section{Introduction}


\begin{frame}{Motivation}

{\onslide<1-> \begin{block}{Aim of the paper}
	To develop new inference methods that allow to \textit{identify} and \textit{locate} differences between nonparametric trend curves with dependent errors.
\end{block}}
	{\onslide<2>\begin{figure}
    		\centering
    		\includegraphics[height=0.45\textheight]{plots/indices.pdf}
  	\end{figure}}
	{\onslide<3>
	\vspace{-46,81mm}
	\begin{figure}
    		\centering
    		\includegraphics[height=0.45\textheight]{plots/indices_1.pdf}
  	\end{figure}}	
	{\onslide<4->
	\vspace{-46,81mm}
	\begin{figure}
    		\centering
    		\includegraphics[height=0.45\textheight]{plots/indices_2.pdf}
  	\end{figure}}	
\vspace{-3mm}
{\onslide<5->\textbf{Research question:}
	Out of many given intervals, how to find those where the trends are significantly different?}


%\textbf{Aim:} to compare the underlying trends on different intervals simultaneously.}
\end{frame}

\begin{frame}{Motivation}
\vspace{-4mm}
\textbf{Why is it relevant?}

Finding systematic differences between trends = basis for further research

%$\quad \quad \Rightarrow$ understanding which government policies are more effective.\pause

\vspace{3mm}

\textbf{Why is it difficult?}	

Testing many hypotheses at the same time = multiple testing problem

$\quad \quad \Rightarrow$ large probability of one true null hypothesis being rejected.\pause

\vspace{3mm}

\textbf{Is it limited to one application?}

No! Our method = general method for comparing nonparametric trends

$\quad \quad \Rightarrow$ new statistical test for equality of nonparametric trend curves.
 
\end{frame}

\begin{frame}{Literature}
	Comparison of deterministic trends:
	\begin{itemize}
		\item {\color<2>{mLightBrown}Park et al. (2009)}, Degras et al. (2012), Zhang et al. (2012), Hidalgo and Lee (2014), Chen and Wu (2019).
	\end{itemize}\pause\pause
	Multiscale tests:
	\begin{itemize}
		\item Chaudhuri and Marron (1999, 2000), Hall and Heckman (2000), D{\"u}mbgen and Spokoiny (2001), {\color<4>{mLightBrown}Park et al. (2009)}.
	\end{itemize}\pause\pause
	Comparison of volatility trends:
	\begin{itemize}
		\item Nyblom and Harvey (2000), ...
	\end{itemize}
\end{frame}


\section{Model}

\begin{frame}{Model}
We observe a panel of $n$ time series $\mathcal{Z}_i = \{(Y_{it}, \X_{it}): 1 \le t \le T \}$ of length $T$, where $Y_{it} \in \reals$ and $\X_{it}\in \reals^{d}$.\pause

We assume the following model:
\begin{equation*}
Y_{it} = m_i \Big( \frac{t}{T} \Big) + \bfbeta_i^T\X_{it}+ \alpha_i + \varepsilon_{it},
\end{equation*}\pause
\vspace{-3mm}
where
\begin{itemize}
\item $m_i$ are unknown trend functions on $[0,1]$;\pause
\item $\bfbeta_i$ is $d\times 1$ vector of unknown parameters;\pause
\item $\alpha_i$ are so-called fixed effect error terms;
\item $\Eps_i = \{\eps_{it}: 1\leq t \leq T\}$ is a zero-mean stationary and causal error process.
\end{itemize}
\end{frame}

\begin{frame}{Model, part 2}
\begin{equation*}
Y_{it} = m_i \Big( \frac{t}{T} \Big) + \bfbeta_i^T\X_{it}+ \alpha_i + \varepsilon_{it},
\end{equation*}\pause
If we knew $\alpha_i$ and $\bfbeta_i$, then the model becomes much simpler:
\begin{align*}
Y_{it} - \alpha_i - \bfbeta_i^\top \X_{it} & =: Y_{it}^\circ\\
					& = m_i \Big( \frac{t}{T} \Big) + \varepsilon_{it}. 
\end{align*}\pause
In reality the variables $Y_{it}^\circ$ are \textbf{not} observed. 

But given $\widehat{\alpha}_i$ and $\widehat{\bfbeta}_i$, we can consider
\begin{align*}
	\widehat{Y}_{it} := Y_{it} -\widehat{\alpha}_i - \widehat{\bfbeta}_i^\top \X_{it} =(\bfbeta_i - \widehat{\bfbeta}_i)^\top \X_{it} + m_i \Big( \frac{t}{T} \Big) + \big( \alpha_i - \widehat{\alpha}_i \big) + \varepsilon_{it}. 
\end{align*}
\end{frame}


\begin{frame}{Model, part 3}
1. We estimate $\bfbeta_i$:
\begin{align*}
\widehat{\bfbeta}_i = \Big( \sum_{t=2}^T \Delta \X_{it} \Delta \X_{it}^\top \Big)^{-1} \sum_{t=2}^T \Delta \X_{it} \Delta Y_{it}
\end{align*}\pause
\vspace{-3mm}
\begin{block}{Theorem}
Under certain regularity assumptions, $\widehat{\bfbeta}_i$ is a consistent estimator of $\bfbeta_i$ with the property $\bfbeta_i - \widehat{\bfbeta}_i = O_P(T^{-1/2})$.
\end{block}\pause
2. We estimate the fixed effects $\alpha_i$:
\begin{align*}
\widehat{\alpha}_i &= \frac{1}{T}\sum_{t=1}^T \big(Y_{it} - \widehat{\bfbeta}_i^\top \X_{it}\big)
\end{align*}\pause
We then work with the augmented time series \color{mLightBrown}{$\widehat{Y}_{it} = Y_{it} - \widehat{\alpha}_i - \widehat{\bfbeta}_i^\top \X_{it}$}.
\end{frame}

\section{Testing procedure}

\begin{frame}{Testing problem}

\begin{align*}
H_0: m_1 = m_2 = \ldots = m_n
\end{align*}\pause

\vspace{-4mm}
\textbf{Question}: if we reject the global null, how to locate the differences between the trends? \pause

Consider a grid $\mathcal{G}_T = \{(u, h): [u-h, u+h] \subseteq [0, 1]\}$ of location-bandwidth parameters. \pause For each pair $(i, j)$ and for each interval $[u-h, u+h]$ we consider the null hypothesis 
\[ H_0^{[i,j]}(u,h): m_i(w) = m_j(w) \text{ for all } w \in [u-h,u+h]. \]\pause
Then the global null $H_0: m_1 = m_2 = \ldots = m_n$ can be reformulated as
\begin{align*}
H_0:  &\text{The hypotheses } H_0^{[i,j]}(u,h) \text{ hold true for all intervals } \\ &[u- h , u+h], (u, h) \in \mathcal{G}_T, \text{ and for all } 1 \le i < j \le n. 
\end{align*} 
\end{frame}


\begin{frame}{Test statistic}
For a given location $u \in [0,1]$ and bandwidth $h$ and a given pair $(i, j)$ we construct the kernel averages
\begin{equation*}
\widehat{\psi}_{ij, T}(u,h) = \sum\limits_{t=1}^T w_{t,T}(u,h) \big(\widehat{Y}_{it} - \widehat{Y}_{jt}), 
\end{equation*}\pause
\vspace{-3mm}
where 
\begin{align*}
w_{t,T}(u,h) &= \frac{\Lambda_{t,T}(u,h)}{ \{\sum\nolimits_{t=1}^T \Lambda_{t,T}^2(u,h)\}^{1/2} } ,\\
\Lambda_{t,T}(u,h) &= K\Big(\frac{t/T-u}{h}\Big) \Big[ S_{T,2}(u,h)  - S_{T,1}(u,h)\Big(\frac{t/T-u}{h}\Big) \Big], \\
S_{T,\ell}(u,h) &= \frac{1}{Th} \sum\nolimits_{t=1}^T K\Big(\frac{t/T-u}{h}\Big) \Big(\frac{t/T-u}{h}\Big)^\ell
\end{align*}
for $\ell = 1,2$ and $K$ is a kernel function.
\end{frame}

\begin{frame}[label = frame_teststatistic]{Test statistic, part 2}
The kernel averages $\widehat{\psi}_{ij, T}(u,h)$ measure the distance between two trend curves $m_i$ and $m_j$ on $[u - h, u + h]$. \pause

Instead with working directly with $\widehat{\psi}_{ij, T}(u,h)$, we replace them by
\begin{equation*}
\widehat{\psi}^0_{ij, T}(u,h) = \bigg\{ \bigg|\frac{\widehat{\psi}_{ij, T}(u,h)}{\big(\widehat{\sigma}_i^2 + \widehat{\sigma}_j^2\big)^{1/2}}\bigg| - \lambda(h) \bigg\}, 
\end{equation*}\pause
\vspace{-3mm}
%Test statistic is defined as follows
%\begin{equation*}
%\widehat{\Psi}_T = \max_{(u,h) \in \mathcal{G}_T} \Big\{ \Big|\frac{\widehat{\psi}_T(u,h)}{\widehat{\sigma}}\Big| - \lambda(h) \Big\}, 
%\end{equation*} \pause
where 
\begin{itemize}
\item $\widehat{\sigma}_i^2$ is an appropriate estimator of the long-run variance $\sigma^2_i$;
\item $\lambda(h) = \sqrt{2 \log \{ 1/(2h) \}}$ is an additive correction term (D{\"u}mbgen and Spokoiny (2001)). \hyperlink{frame_lambda}{\beamerbutton{Explanation}}
\end{itemize}
\end{frame}

\begin{frame}{Test statistic, part 3}

To test the global null, we aggregate the individual test statistics \linebreak for all $(i, j)$ and all location-bandwidth pairs $(u, h) \in \mathcal{G}_T$:
\begin{equation*}
\widehat{\Psi}_{n, T} = \max_{1 \leq i < j \leq n} \max_{(u,h) \in \mathcal{G}_T} \widehat{\psi}^0_{ij, T}(u,h). 
\end{equation*}\pause
\vspace{-3mm}
\begin{block}{Note}
Under certain conditions and under the null, $\widehat{\psi}^0_{ij, T}(u,h)$ and $\widehat{\Psi}_{n, T}$ can be approximated by the corresponding Gaussian versions of the test statistics.
\end{block}
\end{frame}

\begin{frame}{Test procedure}
Gaussian version of the individual test statistics:
\begin{align*}
{\phi}^0_{ij, T}(u,h) = \max_{(u,h) \in \mathcal{G}_T} \bigg\{ \bigg|\frac{\phi_T(u,h)}{\big(\sigma^2_i + \sigma^2_j\big)^{1/2}}\bigg| - \lambda(h) \bigg\},
\end{align*} \pause
\vspace{-3mm}
where
\begin{itemize}
\item $\phi_T(u,h) = \sum\nolimits_{t=1}^T w_{t,T}(u,h) \, {\color<3>{mLightBrown}\big\{\sigma_i(Z_{it} - \bar{Z}_i) - \sigma_j(Z_{jt} - \bar{Z}_j)\big\}}$;\pause
\item $Z_{it}$ are independent standard normal random variables;\pause
\item $\bar{Z}_i$ is the empirical average of $Z_{i1}, \ldots, Z_{iT}$.
\end{itemize}\pause


Aggregated Gaussian test statistics:
\begin{equation*}
\Phi_{n, T} = \max_{1 \leq i < j \leq n} \max_{(u,h) \in \mathcal{G}_T} \phi^0_{ij, T}(u,h). 
\end{equation*}\pause

\end{frame}

\begin{frame}[label = frame_test]{Test procedure, part 2}

\begin{enumerate}
	\item Consider the Gaussian test statistic 
	\vspace{-2mm} \[ \Phi_{n, T} = \max_{1 \leq i < j \leq n} \max_{(u,h) \in \mathcal{G}_T} \phi^0_{ij, T}(u,h), \] where $\phi^0_{ijk}$ are weighted averages of the differences of standard normal random variables.\pause
	\item Compute a $(1-\alpha)$-quantile $q_{n, T} (\alpha)$ of $\Phi_{n,T}$ by Monte Carlo simulations.\pause
	\item Perform the test for the global hypothesis $H_0$: reject $H_0$ if $\widehat{\Psi}_{n, T} > q_{n, T}(\alpha)$.\pause
	\item For each $i, j$, and each $(u, h) \in \mathcal{G}_T$, carry out the test for the local null hypothesis $H^{[i,j]}_0(u, h)$: reject $H^{[i,j]}_0(u, h)$ if $\widehat{\psi}^0_{ij, T}(u, h) > q_{n, T}(\alpha)$.
	\end{enumerate}
\end{frame}


\section{Theoretical properties}




\begin{frame}{Assumptions}
\begin{itemize}
\item[$\mathcal{C}1$] \label{C-err1} The variables $\varepsilon_t$ are weakly dependent.\pause
\item[$\mathcal{C}2$] \label{C-err2} It holds that $\| \varepsilon_t \|_q < \infty$ for some $q > 4$.\pause
\item[$\mathcal{C}3$] \label{C-ker} Standard assumptions on the kernel function $K$. \pause
\item[$\mathcal{C}4$] \label{C-grid} $|\mathcal{G}_T| = O(T^\theta)$ for some arbitrarily large but fixed constant $\theta > 0$.\pause
\only<5>{\begin{align*}
\mathcal{G}_T = \big\{ & (u,h): u = t/T \text{ for some } 1 \le t \le T \text{ and } h \in [h_{\min},h_{\max}] \\ & \text{ with } h = t/T \text{ for some } 1 \le t \le T  \big\},
\end{align*}}\pause
\item[$\mathcal{C}5$] \label{C-h} $h_{\min} \gg T^{-(1-\frac{2}{q})} \log T$ and $h_{\max} = o(1)$.\pause
\item[$\mathcal{C}6$] Assume that  $\widehat{\sigma}^2 = \sigma^2 + o_p(\rho_T)$ with $\rho_T = o(1/\log T)$.
\end{itemize}
\end{frame}

\begin{frame}{Assumptions}
\begin{itemize}
\onslide<1->\item[$\mathcal{C}1$] \label{C1} The functions $\lambda_i$ are uniformly Lipschitz continuous: $|\lambda_i(u) - \lambda_i(v)| \le L |u-v|$ for all $u, v \in [0,1]$.
\onslide<2->\item[$\mathcal{C}2$] \label{C2} $0 < \lambda_{\min} \le \lambda_i(w) \le \lambda_{\max} < \infty$ for all $w \in [0, 1]$ and all $i$. 
\onslide<3->\item[$\mathcal{C}3$] \label{C3} $\eta_{it}$ are independent both across $i$ and $t$.
\onslide<4->\item[$\mathcal{C}4$] \label{C4} $\E[\eta_{it}] = 0$, $\E[\eta_{it}^2] = 1$ and $\E[|\eta_{it}|^\theta] \le C_\theta < \infty$ for some $\theta > 4$. 
\onslide<5->\item[$\mathcal{C}5$] \label{C5} $h_{\max} = o(1/\log T)$ and $h_{\min} \ge CT^{-b}$ for some $b \in (0,1)$.
\onslide<6>\item[$\mathcal{C}6$] \label{C6} $p := \{ \# (i, j, k) \} = O(T^{(\theta/2)(1-b)-(1+\delta)})$ for some small $\delta > 0$.
\end{itemize}
\end{frame}

\begin{frame}{Theoretical properties}
\begin{prop}\label{prop1}
Let $\mathcal{M}_0$ be the set of triplets $(i, j, k)$ for which $H_0^{(ijk)}$ holds true. Then under $\mathcal{C}1 - \mathcal{C}6$, it holds that 
\vspace{-2mm}
\begin{align*}
 \Prob\Big( \forall (i,j,k) \in \mathcal{M}_0: |\hat{\psi}_{ijk}| \le c_{\textnormal{Gauss}}(\alpha,h_k) \Big) \ge 1 - \alpha + o(1)
\end{align*}
\end{prop}\pause
\begin{prop}\label{prop2}
Consider a sequence of functions $\lambda_{i} = \lambda_{i,T}$, $\lambda_{j} = \lambda_{j, T}$ such that 
\begin{equation}\label{eq:localt}
\exists \, \mathcal{I}_{k}:  \lambda_{i}(w) - \lambda_{j}(w) \ge c_T \sqrt{\log T / (T h_{k})} \,\, \forall w \in \mathcal{I}_{k},
\end{equation} and $c_T \rightarrow \infty$ faster than $\frac{\sqrt{\log T}\sqrt{\log \log T}}{\log \log \log T}$.  Let $\mathcal{M}_1$ be the set of triplets $(i, j, k)$ for which \eqref{eq:localt} holds true. Then under $\mathcal{C}1 - \mathcal{C}6$, it holds that
\vspace{-2mm}
\begin{equation*}
\Prob\Big( \forall (i,j,k) \in \mathcal{M}_1: |\hat{\psi}_{ijk}| > c_{\textnormal{Gauss}}(\alpha,h_k) \Big) = 1 - o(1)
\end{equation*}
\end{prop}
\end{frame}

\begin{frame}{Strategy of the proof}
\begin{itemize}
\item Replace the statistic $\widehat{\Psi}_T$ under $H_0: m = 0$ by a statistic $\widetilde{\Phi}_T$ with the same distribution and the property that 
\begin{equation*}\label{eq-theo-stat-strategy-step1}
\big| \widetilde{\Phi}_T - \Phi_T \big| = o_p(\delta_T),
\end{equation*}
where $\delta_T = o(1)$. To do so, we make use of strong approximation theory for dependent processes as derived in Berkes et al. (2014)\pause
\vspace{2mm}
\item Using the anti-concentration results for Gaussian random vectors (Chernozhukov et al. 2015), prove that $\Phi_T$ does not concentrate too strongly in small regions of the form $[x-\delta_T,x+\delta_T]$, i.e.
\begin{equation*}\label{eq-theo-stat-strategy-step2}
\sup_{x \in \mathbb{R}} \Prob \big( |\Phi_T - x| \le \delta_T \big) = o(1).
\end{equation*}\pause
\vspace{-2mm}
\item Show that 
\begin{equation*}\label{eq-theo-stat-strategy-claim}
\sup_{x \in \mathbb{R}} \big| \Prob(\widetilde{\Phi}_T \le x) - \Prob(\Phi_T \le x) \big| = o(1). 
\end{equation*}
\end{itemize}
\end{frame}



%\begin{frame}[label = frame_critval]{Critical values}
%How to construct critical values $c_{ijk}(\alpha)$?\pause
%\begin{itemize} 
%\item Traditional approach: $c_{ijk}(\alpha) = c(\alpha)$ for all $(i,j,k)$. \pause
%\item More modern approach: $c_{ijk}(\alpha)$ depend on the length $h_k$ of the time interval (D{\"u}mbgen and Spokoiny (2001))\pause:
%\[c_{ijk}(\alpha) = c(\alpha,h_k) := b_k + q(\alpha)/a_k,\] where $a_k$ and $b_k$ are scale-dependent constants and $q(\alpha)$ is chosen such that we control FWER.
%\hyperlink{frame_scaleconstants}{\beamerbutton{Details}}
%\end{itemize}
%\end{frame}
%
%
%\begin{frame}{Critical values, part 2}
%We want to control FWER. \pause Let $\mathcal{M}_0: = \big\{(i, j, k) | H_0^{(ijk)} \text{ is true} \big\}$, then
%\begin{align*}
%\text{FWER}(\alpha) &= \Prob \Big( \exists (i,j,k) \in \mathcal{M}_0:  |\widehat{\psi}_{ijk} | > c_{ijk}(\alpha) \Big) \\
%\onslide<3->{&= 1 - \Prob \Big( \forall (i,j,k) \in \mathcal{M}_0: |\widehat{\psi}_{ijk} | \le c_{ijk}(\alpha) \Big)}\\
%\onslide<4->{& =  1 - \Prob \Big( \forall (i,j,k) \in \mathcal{M}_0: a_k \big(|\hat{\psi}_{ijk}| - b_k\big) \le q(\alpha) \Big)}\\
%\onslide<5->{& = 1 - \Prob\Big( \max_{(i,j,k) \in \mathcal{M}_0} a_k \big( |\hat{\psi}_{ijk}| - b_k \big) \le q(\alpha) \Big)}\\
%\onslide<6->{ & \leq 1 - \Prob\Big( \max_{(i,j,k)} a_k \big( |{\color<7>{mLightBrown}\hat{\psi}_{ijk}^0}| - b_k \big) \le q(\alpha) \Big)}
%\end{align*}
%\onslide<8->{Hence, we choose $q(\alpha)$ as the $(1-\alpha)$-quantile of the statistic 
%\[ \hat{\Psi}_T = \max_{(i,j,k)} a_k \big( |\hat{\psi}_{ijk}^0| - b_k \big), \]
%where $\hat{\psi}_{ijk}^0$ is equal to $\hat{\psi}_{ijk}$ under the null.}
%\end{frame}
%
%\begin{frame}{Critical values, part 3}
%
%But we do not know the distribution of $\hat{\Psi}_T$ in practice!\pause
%
%$\Rightarrow$ the quantiles $q(\alpha)$ are also not known. How to approximate them?\pause
%
%Under our assumptions, 
%\[ \hat{\psi}_{ijk}^0 \approx \frac{1}{\sqrt{2 T h_k}} \sum\limits_{t=1}^T \ind  (\eta_{it} - \eta_{jt} ), \] \pause
%which can be approximated by a Gaussian version of the test statistic:
%\begin{align*}
%\phi_{ijk} = \frac{1}{\sqrt{2 T h_k}} \sum\limits_{t=1}^T \ind (Z_{it} - Z_{jt}), 
%\end{align*}
%where $Z_{it}$ are independent standard normal random variables.
%
%
%\end{frame}




\section{Application}

\begin{frame}{Graphical representation}
How to represent the results of the test? \pause

Plot the results of pairwise comparison $\mathcal{F}_{\text{reject}}(i, j)$:
\begin{align*}
 \Prob\Big( \forall (i,j,k) \in \mathcal{M}_0: \mathcal{I}_k \notin \mathcal{F}_{\text{reject}}(i, j) \Big) \ge 1 - \alpha + o(1)
\end{align*}

\pause
\begin{block}{Minimal intervals}
An interval $\mathcal{I}_k \in \mathcal{F}_{\text{reject}}(i, j)$ is called \textbf{minimal} if there is no other interval $\mathcal{I}_{k^\prime} \in \mathcal{F}_{\text{reject}}(i, j)$ with $\mathcal{I}_{k^\prime} \subset \mathcal{I}_k$.

The set of minimal intervals is denoted $\mathcal{F}_{\text{reject}}^{\min} (i, j)$.
\end{block}\pause
We can make similar confidence statements about minimal intervals:
\begin{align*}
 \Prob\Big( \forall (i,j,k) \in \mathcal{M}_0: \mathcal{I}_k \notin \mathcal{F}_{\text{reject}}^{\min} (i, j) \Big) \ge 1 - \alpha + o(1)
\end{align*}
\end{frame}

%\begin{frame}{Application setting}
%\begin{itemize}
%\item Five countries: Germany, Italy, Spain, France and the UK.
%\item $T = 150$ days. 
%\item The data is aligned by weekdays: first Monday after reaching $100$ cases as $t=1$.
%\item Lengths of time intervals $7, 14, 21, 28$ days. The intervals start at days $1, 8, 15, \ldots$ and $4, 11, 19, \ldots$
%\item $\alpha = 0.05$.
%\item $5000$ Monte Carlo simulation runs to produce critical values.
%\end{itemize}
%\end{frame}
%
%\begin{frame}{Application results}
%	\begin{figure}
%		\includegraphics[width=0.49\textwidth]{plots/DEU_vs_ITA_presentation}
%   		%\caption{Observed new cases per day in Germany and Italy}    			\label{fig:DEUvsITA}
%		\hfill
%		\includegraphics[width=0.49\textwidth]{plots/DEU_vs_ESP_presentation}
%	\end{figure}
%\end{frame}
%
%\begin{frame}{Application results, part 2}
%	\begin{figure}
%		\includegraphics[width=0.49\textwidth]{plots/DEU_vs_GBR_presentation}
%   		%\caption{Observed new cases per day in Germany and Italy}    			\label{fig:DEUvsITA}
%		\hfill
%		\includegraphics[width=0.49\textwidth]{plots/DEU_vs_FRA_presentation}
%	\end{figure}
%\end{frame}


\begin{frame}{Discussion}
We can claim, with confidence of about $95\%$, that the null hypothesis is violated for all intervals (and all pairs of time series) for which our test rejects the null. \pause

However, we can not say anything about the causes of such differences. This question requires further (probably not purely statistical) analysis.\pause

Further possible extensions:
\vspace{-1mm}
\begin{itemize}
	\item introduce scaling factor in the trend function, that will allow to adjust for the size of the country (population, density, testing regimes, etc.);\pause
	\item include the dependence between covariates and error terms;
	\item cluster the time series based on the trends they exhibit.
\end{itemize}
\end{frame}

\begin{frame}[standout]
  Thank you!
\end{frame}


\appendix

\begin{frame}{Family of time intervals}
	\begin{figure}
		\includegraphics[width=0.95\textwidth]{plots/all_intervals}
	\end{figure}
\end{frame}

\begin{frame}{Simulation results for the size of the test}
\begin{figure}[t!]
	\includegraphics[height = 0.4\textheight]{plots/lambda_fct}
\end{figure}
\vspace{-2mm}
\scriptsize{\begin{table}[t]
\begin{center}
\caption{Size of the multiscale test}
\label{tab:size_shape}
% latex table generated in R 3.6.1 by xtable 1.8-4 package
% 
\begin{tabular}{cccccccccc}
  \hline
  \hline
$T = 100$ & 0.012 & 0.046 & 0.105 & 0.009 & 0.039 & 0.088 & 0.008 & 0.033 & 0.069 \\ 
  $T = 250$ & 0.012 & 0.051 & 0.100 & 0.007 & 0.046 & 0.090 & 0.008 & 0.037 & 0.085 \\ 
  $T = 500$ & 0.009 & 0.046 & 0.098 & 0.010 & 0.044 & 0.092 & 0.009 & 0.042 & 0.077 \\ 
   \hline
\end{tabular}

\end{center}
\end{table}}
\end{frame}

\begin{frame}{Simulation results for the power of the test}
\begin{figure}[t!]
	\onslide<1->\includegraphics[width = 0.49\textwidth, height = 0.4\textheight]{plots/lambda_fcts_height}
	\onslide<2->\includegraphics[width = 0.49\textwidth, height = 0.4\textheight]{plots/lambda_fcts_shift}	
\end{figure}\pause
\vspace{-5mm}
{\onslide<1>\scriptsize{\begin{table}[t]
\begin{center}
\caption{Power of the multiscale test for scenario A}
\label{tab:size_shape}
% latex table generated in R 3.6.1 by xtable 1.8-4 package
% 
\begin{tabular}{cccccccccc}
  \hline
  \hline
$T = 100$ & 0.335 & 0.518 & 0.597 & 0.306 & 0.474 & 0.545 & 0.212 & 0.352 & 0.418 \\ 
  $T = 250$ & 0.615 & 0.790 & 0.836 & 0.580 & 0.764 & 0.800 & 0.470 & 0.648 & 0.705 \\ 
  $T = 500$ & 0.736 & 0.905 & 0.917 & 0.738 & 0.884 & 0.890 & 0.636 & 0.799 & 0.830 \\ 
   \hline
\end{tabular}

\end{center}
\end{table}}}
{\onslide<2>
\vspace{-39.5mm}
\scriptsize{\begin{table}[t]
\begin{center}
\caption{Power of the multiscale test for scenario B}
\label{tab:size_shape}
% latex table generated in R 3.6.1 by xtable 1.8-4 package
% 
\begin{tabular}{cccccccccc}
  \toprule
 & \multicolumn{3}{c}{$n = 5$} & \multicolumn{3}{c}{$n = 10$} & \multicolumn{3}{c}{$n = 50$} \\
\cmidrule[0.4pt]{2-4} \cmidrule[0.4pt]{5-7} \cmidrule[0.4pt]{8-10}
 & \multicolumn{3}{c}{significance level $\alpha$} &\multicolumn{3}{c}{significance level $\alpha$} & \multicolumn{3}{c}{significance level $\alpha$} \\
 & 0.01 & 0.05 & 0.1  &  0.01 & 0.05 & 0.1  &  0.01 & 0.05 & 0.1 \\
\cmidrule[0.4pt]{1-10}
$T = 100$ & 0.824 & 0.910 & 0.903 & 0.812 & 0.893 & 0.890 & 0.738 & 0.847 & 0.857 \\ 
  $T = 250$ & 0.991 & 0.972 & 0.941 & 0.991 & 0.960 & 0.920 & 0.991 & 0.965 & 0.933 \\ 
  $T = 500$ & 0.997 & 0.973 & 0.949 & 0.995 & 0.961 & 0.923 & 0.996 & 0.969 & 0.932 \\ 
   \hline
\end{tabular}

\end{center}
\end{table}}}
\end{frame}

\begin{frame}[label = frame_sigma]{Estimator of ${\sigma}^2$}
We estimate the overdispersion paramter $\sigma^2$ by \[\widehat{\sigma}^2= \frac{1}{n} \sum_{i = 1}^n \hat{\sigma}_i^2 \text{ and } \hat{\sigma}_i^2 = \frac{\sum_{t=2}^T (X_{it}-X_{it-1})^2}{2 \sum_{t=1}^T X_{it}}\] \pause
We assume that $\lambda_i$ is Lipschitz continuous. Then
\[ X_{it} - X_{it-1} = \sigma \sqrt{\lambda_i\Big(\frac{t}{T}\Big)} (\eta_{it} - \eta_{it-1}) + r_{it}, \]
where $|r_{it}| \le C(1+|\eta_{it-1}|)/T$ with a sufficiently large $C$.\pause \, Hence,
\[ \frac{1}{T} \sum_{t=2}^T (X_{it} - X_{it-1})^2 = 2 \sigma^2 \Big\{ \frac{1}{T} \sum_{t=2}^T \lambda_i(t/T) \Big\} + o_p(1)\] \pause
Together with \[ \frac{1}{T} \sum_{t=1}^T X_{it} = \frac{1}{T} \sum_{t=1}^T \lambda_i(t/T) + o_p(1), \] we get that $\hat{\sigma}_i^2 = \sigma^2 + o_p(1)$ for any $i$ and thus $\hat{\sigma}^2 = \sigma^2 + o_p(1)$. \hyperlink{frame_teststatistic<4>}{\beamerbutton{Go back}}
\end{frame}



\begin{frame}[label = frame_scaleconstants]{Idea behind $a_k$ and $b_k$}

D{\"u}mbgen and Spokoiny (2001): the critical values $c_{ijk}(\alpha)$ depend on the scale of the testing problem, i.e. the length $h_k$ of the time interval.\pause 

Specifically, 
\[c_{ijk}(\alpha) = c(\alpha,h_k) := b_k + q(\alpha)/a_k,\] where $a_k = \{\log(e/h_k)\}^{1/2} / \log \log(e^e / h_k)$ and $b_k = \sqrt{2 \log(1/h_k)}$ are scale-dependent constants and $q(\alpha)$ is chosen such that we control FWER.
\end{frame}

\begin{frame}{Idea behind $a_k$ and $b_k$, part 2}

This choice of scale-dependent constants helps us balance the significance of hypotheses between the time intervals of different lengths $h_k$:

\begin{figure}
    		\centering
	\includegraphics[width=0.95\textwidth]{plots/size_with_correction}
\end{figure}


\hyperlink{frame_critval<4>}{\beamerbutton{Go back}}
\end{frame}



\begin{frame}{Idea behind the additive correction}
Consider the uncorrected Gaussian statistic
\begin{align*}
\Phi^{\text{uncor}} = \max_{(i,j,k)} |\phi_{ijk}|
\end{align*}\pause
and let the family of intervals be \[\mathcal{F} = \big\{[(m-1) h_l, m h_l] \text{ for } 1\le m \le 1/h_l, 1 \le l \le L\big\}\]\pause
Then we can rewrite the uncorrected test statistic as
\begin{align*}
\Phi^{\text{uncor}} = \max_{i, j} \max_{\substack{1 \le l \le L, \\ 1\le m \le 1/h_l}} \Big|\frac{1}{\sqrt{2 T h_l}} \sum\limits_{t=1}^T 1 \Big( \frac{t}{T} \in [(m-1) h_l, m h_l] \Big) (Z_{it} - Z_{jt})\Big|
\end{align*}\pause
$\Rightarrow \quad \max_m \ldots =\sqrt{2\log(1/h_l)} + o_P(1) \to \infty$ as $h \to 0$ and the stochastic behavior of $\Phi^{\text{uncor}}$ is dominated by the elements with small bandwidths $h_l$. \hyperlink{frame_test<4>}{\beamerbutton{Go back}}
\end{frame}



\end{document}
