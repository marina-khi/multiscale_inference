\documentclass[a4paper,12pt]{article}
\usepackage{amsmath}
\usepackage{amssymb,amsthm,graphicx}
\usepackage{enumitem}
\usepackage{color}
\usepackage{epsfig}
\usepackage{graphics}
\usepackage{pdfpages}
\usepackage{subcaption}
\usepackage[font=small]{caption}
\usepackage[hang,flushmargin]{footmisc} 
\usepackage{float}
\usepackage{booktabs}
\usepackage[mathscr]{euscript}
\usepackage{natbib}
\usepackage{setspace}
\usepackage{mathrsfs}
\usepackage[Q=yes]{examplep}
\usepackage[T1]{fontenc}
%\usepackage{hanging}
\usepackage[left=3cm,right=3cm,bottom=3cm,top=3cm]{geometry}
\renewcommand{\baselinestretch}{1.05}\normalsize
\parindent0pt




\begin{document}



\begin{center}
{\LARGE \bf Code documentation}
\end{center}
\vspace{0.5cm}



\section*{Summary}


\setlength{\parskip}{0.2cm} 
This document describes the R code that can be used to replicate the empirical results reported in the paper \textit{Multiscale inference for nonparametric time trends}. The overall structure of the code is as follows. There are four main files each of which produces a specific part of the simulations and applications:
\vspace{0.2cm}

\everypar{\hangafter=1\hangindent=1.45cm\relax}
\verb|main_shape_simulation.R| \hspace{1pt} produces the simulation results for the test of the hypothesis $H_0: m^\prime = 0$ from Section 4. It can also be used to run the simulations for testing the hypothesis $H_0: m = 0$ from Section $3$.

\verb|main_equality_simulation.R| \hspace{1pt} produces the simulation results for the test of the hypothesis $H_0: m_1 = \ldots = m_n$ from Section 5.

\verb|main_shape_application.R| \hspace{1pt} produces the application results from Section 8.1, where the test of the hypothesis $H_0: m^\prime = 0$ is used.

\verb|main_equality_application.R| \hspace{1pt} produces the application results from Section 8.2, where the test of the hypothesis $H_0: m_1 = \ldots = m_n$ is used. 
\vspace{0.2cm}

\everypar{\hangafter=0\relax}
These main files read in a number of functions which are collected in the two folders \verb|Shape| and \verb|Equality|. The folder \verb|Shape| contains the code for the test methods from Section 4 and \verb|Equality| the code for the methods from Section 5. The simulation and application results are stored either as figures or as \verb|.tex| files (for tables) in the folder \verb|Plots|. 
\vspace{0.2cm}


All programs are written in R with some functions in C. They are all quite self-explanatory and commented. The code is self-sufficient, the only thing that is OS-specific and may need to be changed is the extension of dynamic libraries. Dynamic libraries are defined in the beginning of each main file. At the moment everything is configured for running on Windows, that is, the \verb|.dll| extension is used. In order to make the programs Linux or MacOS compatible, this extension should be changed to \verb|.so|. For example, 
\begin{verbatim}
dyn.load("Shape/C_code/estimating_sigma.dll")
\end{verbatim} should be replaced by
\begin{verbatim}
dyn.load("Shape/C_code/estimating_sigma.so").
\end{verbatim}
All the dynamic libraries are already provided, additional  compilation is not neccessary.



\newpage
\section*{Code for the test methods from Section 4}


\setlength{\parskip}{0.3cm}
\everypar{\hangafter=1\hangindent=1.45cm\relax}

\verb|main_shape_simulation.R| \hspace{1pt} produces the simulation results for the test method from Section 4, that are reported in Section $7$, in particular Tables 1 and 2. If you use Linux or MacOS, please change the extension of dynamic libraries in the very beginning of the main file to \verb|.so|. The size and power results in Tables 1 and 2 are calculated from $1000$ simulated samples. This number can be set as \verb|N_rep| in \verb|main_shape_simulation.R|. The running time for $1000$ simulations is quite small, amounting to a few minutes. 

\verb|main_shape_application.R| \hspace{1pt} produces the application results from Section 8.1, in particular Figure 1. As above, please change the extension of dynamic libraries if you use Linux or MacOS.

\verb|Shape| \hspace{1pt} is a directory that contains all the necessary functions, data and simulated distributions to run \verb|main_shape_simulation.R| and \verb|main_shape_application.R|.

\verb|Shape\C_code| \hspace{1pt} is a folder that contains the dynamic libraries and R-wrappers of the C functions.

\verb|Shape\functions.R| \hspace{1pt} is a file that consists of auxiliary R functions.

\verb|Shape\data| \hspace{1pt} is a folder with the file \verb|cetml1659on.dat| which contains the monthly and yearly mean central England temperature from 1659 up to 2017. It was downloaded from \verb|https://www.metoffice.gov.uk/hadobs/hadcet/data| \verb|/download.html|. A full description of the data can be accessed on the website of the UK Met Office \verb|https://www.metoffice.gov.uk|. 

\verb|Shape\distribution| \hspace{1pt} is a folder that  contains simulated distributions of the Gaussian statistic $\Phi'_{T}$ for various time series lengths $T$. In order to recalculate these distributions, the stored files need to be deleted or removed; the computation is then done automatically while running the main file. The distribution of the Gaussian statistic $\Phi'_T$ used to calculate the critical values of the test in the application in Section 8.1 is stored in \verb|Shape\distribution\distr_T_359_testing_constant_type_ll.RData|. \newline As before, the distribution of $\Phi'_T$ can be recalculated by removing or deleting this file before running \verb|N_rep| in \verb|main_equality_simulation.R|.  
\vspace{0.2cm}



\section*{Code for the test methods from Section 5}


\everypar{\hangafter=1\hangindent=1.45cm\relax}

\verb|main_equality_simulation.R| \hspace{1pt} produces the simulation results for the methods from Section 5, that are reported in Section 7, in particular Tables 3, 4 and 5. If you use Linux or MacOS, please change the extension of dynamic libraries in the very beginning of the main file. The size and power results in Tables 3 and 4 are calculated from $1000$ simulated samples. This number can be set as \verb|N_rep| in \verb|main_equality_simulation.R|, though rising this parameter is not advisable due to computational tractability. 

\verb|main_equality_application.R| \hspace{1pt} produces the application results from Section 8.2. As above, please change the extension of dynamic libraries if you use Linux or MacOS.

\verb|Equality| \hspace{1pt} is a directory that contains all the necessary functions, data and simulated distributions to run \verb|main_equality_simulation.R| and \verb|main_equality| \linebreak \verb|_application.R|.

\verb|Equality\C_code| \hspace{1pt} is a folder that contains the dynamic libraries and R-wrappers of the C functions.

\verb|Equality\functions.R| \hspace{1pt} is a file that includes all the auxiliary R functions.

\verb|Equality\data| \hspace{1pt} is a folder that contains historic station data downloaded from the website
{\verb|https://www.metoffice.gov.uk/public/weather/climate-historic/#?| \linebreak \verb|tab=climateHistoric|
and the description file \verb|.\Equality\| \verb|Description_| \verb|file.xlsx|. The data consist of: \\[0.2cm]
-- mean daily maximum temperature (\textit{tmax}) \\
-- mean daily minimum temperature (\textit{tmin}) \\
-- days of air frost (\textit{af}) \\
-- total rainfall (\textit{rain}) \\
-- total sunshine duration (\textit{sun}). \\[0.2cm]
The monthly mean temperature used in the paper is calculated from the average of the mean daily maximum and the mean daily minimum temperature, i.e.\ $(tmax+tmin)/2$. A full description of the data can be accessed on the official website of The UK Met Office \verb|https://www.metoffice.gov.uk|.}

\verb|Equality\distribution| \hspace{1pt} is a folder that contains the simulated distributions of the Gaussian statistic $\Phi_{n, T}$ for $n= 15$ time series (the number that was used for the simulation study) and different time series lengths $T$. If you want to recalculate these distributions, the stored files need to be deleted or removed; recomputation is then done automatically while running the main files. However, note that with increasing time series length $T$, the running time increases significantly. For example, for $T = 1000$ and $n = 15$, the calculation of $\Phi_{n, T}$ may take several hours. Running time may be substantially improved by parallel computations which is not implemented in the current version. The distribution of the Gaussian statistic $\Phi'_{n, T}$ that was used for calculating quantiles for the application in Section 8.2 is stored in the file
\verb|.\Equality\distribution\distr_for_application_T_386_and_N_ts_| \linebreak \verb|25_and_method_ll.RData|.


\end{document}
