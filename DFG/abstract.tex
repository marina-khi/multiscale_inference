\documentclass[a4paper,12pt]{article}
\usepackage{amsmath}
\usepackage{amssymb,amsthm,graphicx}
\usepackage{titlesec}
\usepackage{textcomp}
\usepackage{enumitem}
\usepackage{color}
\usepackage{epsfig}
\usepackage{graphics}
\usepackage{pdfpages}
\usepackage{subcaption}
\usepackage[font=small]{caption}
\usepackage[hang,flushmargin]{footmisc} 
\usepackage{float}
\usepackage{rotating,tabularx}
\usepackage{booktabs}
\usepackage[mathscr]{euscript}
\usepackage{natbib}
\usepackage{setspace}
\usepackage{mathrsfs}
\usepackage[official]{eurosym}
\usepackage[left=2.8cm,right=2.8cm,bottom=2.8cm,top=2.5cm]{geometry}
\parindent0pt

\setcounter{secnumdepth}{4}
\renewcommand{\baselinestretch}{1.1}


% General

\newcommand{\reals}{\mathbb{R}}
\newcommand{\integers}{\mathbb{Z}}
\newcommand{\naturals}{\mathbb{N}}

\newcommand{\pr}{\mathbb{P}}        % probability
\newcommand{\ex}{\mathbb{E}}        % expectation
\newcommand{\var}{\textnormal{Var}} % variance
\newcommand{\cov}{\textnormal{Cov}} % covariance

\newcommand{\law}{\mathcal{L}} % law of X
\newcommand{\normal}{N}        % normal distribution 

\newcommand{\argmax}{\textnormal{argmax}}
\newcommand{\argmin}{\textnormal{argmin}}

\newcommand{\ind}{\mathbbm{1}} % indicator function
\newcommand{\kernel}{K} % kernel function
\newcommand{\wght}{W} % kernel weight
\newcommand{\thres}{\pi} % threshold parameter


% Convergence

\newcommand{\convd}{\stackrel{d}{\longrightarrow}}              % convergence in distribution
\newcommand{\convp}{\stackrel{P}{\longrightarrow}}              % convergence in probability
\newcommand{\convas}{\stackrel{\textrm{a.s.}}{\longrightarrow}} % convergence almost surely
\newcommand{\convw}{\rightsquigarrow}                           % weak convergence


% Theorem-like declarations

\theoremstyle{plain}

\newtheorem{theorem}{Theorem}[section]
\newtheorem{prop}[theorem]{Proposition}
\newtheorem{lemma}[theorem]{Lemma}
\newtheorem{corollary}[theorem]{Corollary}
\newtheorem*{theo}{Theorem}
\newtheorem{propA}{Proposition}[section]
\newtheorem{lemmaA}[propA]{Lemma}
\newtheorem{definition}{Definition}[section]
\newtheorem{remark}{Remark}[section]
\renewcommand{\thelemmaA}{A.\arabic{lemmaA}}
\renewcommand{\thepropA}{A.\arabic{propA}}
\newtheorem*{algo}{Clustering Algorithm}


% Theorem numbering to the left

\makeatletter
\newcommand{\lefteqno}{\let\veqno\@@leqno}
\makeatother


% Heading

\newcommand{\heading}[2]
{  \setcounter{page}{1}
   \begin{center}

   \phantom{Distance to upper boundary}
   \vspace{0.5cm}

   {\LARGE \textbf{#1}}
   \vspace{0.4cm}
 
   {\LARGE \textbf{#2}}
   \end{center}
}


% Authors

\newcommand{\authors}[4]
{  \parindent0pt
   \begin{center}
      \begin{minipage}[c][2cm][c]{5cm}
      \begin{center} 
      {\large #1} 
      \vspace{0.05cm}
      
      #2 
      \end{center}
      \end{minipage}
      \begin{minipage}[c][2cm][c]{5cm}
      \begin{center} 
      {\large #3}
      \vspace{0.05cm}

      #4 
      \end{center}
      \end{minipage}
   \end{center}
}

%\newcommand{\authors}[2]
%{  \parindent0pt
%   \begin{center}
%   {\large #1} 
%   \vspace{0.1cm}
%      
%   #2 
%   \end{center}  
%}


% Version

\newcommand{\version}[1]
{  \begin{center}
   {\large #1}
   \end{center}
   \vspace{3pt}
} 










\begin{document}



\begin{center}
{\large \textbf{Abstract (English)}} 
\end{center}


The main purpose of the project is to develop new methods and theory for the analysis
%comparison and clustering 
of nonparametric time trend curves. Recently, there has been a growing interest in econometric models with non- and semiparametric time trends. 
Non- and semiparametric trend modelling has attracted particular interest in a panel data context. 
Important questions are whether the observed time series in the panel all have the same trend or whether they can be clustered into groups with the same trend. A number of test and clustering methods have been developed in the literature to approach these questions, which are relevant in a variety of economic and financial applications. Most of the proposed methods, however, depend on a number of bandwidth or smoothing parameters whose optimal choice is a notoriously difficult problem. In our project, we tackle the challenge of developing new test and clustering methods which are free of classic bandwidth para\-meters and thus avoid the issue of bandwidth selection. To achieve this, we will build on techniques from statistical multiscale testing which have recently been introduced into the literature. The methodological and theoretical analysis of the project will be complemented by simulations and empirical applications. In particular, we intend to apply the developed methods to an empirical question of recent interest in macroeconomics, that is, the question of whether real GDP growth has been faster in some countries than in others. 
\vspace{5pt}


\begin{center}
{\large \textbf{Abstract (Deutsch)}} 
\end{center}


Das Hauptziel des Projekts besteht in der Entwicklung neuer statistischer Methoden zur Analyse nichtparametrischer Zeittrends. In den letzten Jahren ist ein wachsendes Interesse an \"okonometrischen Modellen mit nicht- und semiparametrischen Zeittrends aufgekommen. Nicht- und semiparametrische Trendmodellierung hat vor allem im Kontext von Paneldatenmodellen neues Interesse geweckt. 
Wichtige Fragen in diesem Zusammenhang sind, ob die im Panel beobachteten Zeitreihen alle den gleichen Trend haben oder ob es zumindest Gruppen von Zeitreihen mit gleichem Trend gibt. In der Literatur wurden verschiedene Test- und Clustering-Verfahren entwickelt, um diese Fragen anzugehen, die f\"ur eine Vielzahl \"okonomischer und finanzwirtschaftlicher Anwendungen relevant sind. Die meisten der entwickelten Methoden h\"angen allerdings von gewissen Bandweiten- bzw.\ Gl\"attungsparametern ab, deren optimale Wahl ein notorisch schwieriges Problem ist. In unserem Projekt gehen wir die Herausforderung an, neue statistische Methoden zum Vergleich und Clustern nichtparametrischer Trendkurven zu entwickeln, die frei von klassischen Bandweitenparametern sind und damit das Problem der Bandweitenwahl vermeiden. Dazu werden wir auf Multiskalentechniken aus der statistischen Testtheorie zur\"uckgreifen. Die methodische und theoretische Analyse des Projekts wird durch Simulationsstudien und Anwendungsbeispiele erg\"anzt werden. Insbesondere werden wir die entwickelten Methoden verwenden, um das \"okonomische Wachstum in verschiedenen L\"andern miteinander zu vergleichen. 
%auf eine empirische Frage aus der Makro\"okonomik anwenden, und zwar auf die Frage, ob das reelle Wachstum in manchen \"Okonomien schneller vor sich gegangen ist als in anderen. 





\end{document}
