\documentclass[a4paper,12pt]{article}
\usepackage{amsmath}
\usepackage{amssymb,amsthm,graphicx}
\usepackage{titlesec}
\usepackage{textcomp}
\usepackage{enumitem}
\usepackage{color}
\usepackage{epsfig}
\usepackage{graphics}
\usepackage{pdfpages}
\usepackage{subcaption}
\usepackage[font=small]{caption}
\usepackage[hang,flushmargin]{footmisc} 
\usepackage{float}
\usepackage{rotating,tabularx}
\usepackage{booktabs}
\usepackage[mathscr]{euscript}
\usepackage{natbib}
\usepackage{setspace}
\usepackage{mathrsfs}
\usepackage[official]{eurosym}
\usepackage[left=3cm,right=3cm,bottom=3cm,top=2.5cm]{geometry}

\setcounter{secnumdepth}{4}
\renewcommand{\baselinestretch}{1.2}


% General

\newcommand{\reals}{\mathbb{R}}
\newcommand{\integers}{\mathbb{Z}}
\newcommand{\naturals}{\mathbb{N}}

\newcommand{\pr}{\mathbb{P}}        % probability
\newcommand{\ex}{\mathbb{E}}        % expectation
\newcommand{\var}{\textnormal{Var}} % variance
\newcommand{\cov}{\textnormal{Cov}} % covariance

\newcommand{\law}{\mathcal{L}} % law of X
\newcommand{\normal}{N}        % normal distribution 

\newcommand{\argmax}{\textnormal{argmax}}
\newcommand{\argmin}{\textnormal{argmin}}

\newcommand{\ind}{\mathbbm{1}} % indicator function
\newcommand{\kernel}{K} % kernel function
\newcommand{\wght}{W} % kernel weight
\newcommand{\thres}{\pi} % threshold parameter


% Convergence

\newcommand{\convd}{\stackrel{d}{\longrightarrow}}              % convergence in distribution
\newcommand{\convp}{\stackrel{P}{\longrightarrow}}              % convergence in probability
\newcommand{\convas}{\stackrel{\textrm{a.s.}}{\longrightarrow}} % convergence almost surely
\newcommand{\convw}{\rightsquigarrow}                           % weak convergence


% Theorem-like declarations

\theoremstyle{plain}

\newtheorem{theorem}{Theorem}[section]
\newtheorem{prop}[theorem]{Proposition}
\newtheorem{lemma}[theorem]{Lemma}
\newtheorem{corollary}[theorem]{Corollary}
\newtheorem*{theo}{Theorem}
\newtheorem{propA}{Proposition}[section]
\newtheorem{lemmaA}[propA]{Lemma}
\newtheorem{definition}{Definition}[section]
\newtheorem{remark}{Remark}[section]
\renewcommand{\thelemmaA}{A.\arabic{lemmaA}}
\renewcommand{\thepropA}{A.\arabic{propA}}
\newtheorem*{algo}{Clustering Algorithm}


% Theorem numbering to the left

\makeatletter
\newcommand{\lefteqno}{\let\veqno\@@leqno}
\makeatother


% Heading

\newcommand{\heading}[2]
{  \setcounter{page}{1}
   \begin{center}

   \phantom{Distance to upper boundary}
   \vspace{0.5cm}

   {\LARGE \textbf{#1}}
   \vspace{0.4cm}
 
   {\LARGE \textbf{#2}}
   \end{center}
}


% Authors

\newcommand{\authors}[4]
{  \parindent0pt
   \begin{center}
      \begin{minipage}[c][2cm][c]{5cm}
      \begin{center} 
      {\large #1} 
      \vspace{0.05cm}
      
      #2 
      \end{center}
      \end{minipage}
      \begin{minipage}[c][2cm][c]{5cm}
      \begin{center} 
      {\large #3}
      \vspace{0.05cm}

      #4 
      \end{center}
      \end{minipage}
   \end{center}
}

%\newcommand{\authors}[2]
%{  \parindent0pt
%   \begin{center}
%   {\large #1} 
%   \vspace{0.1cm}
%      
%   #2 
%   \end{center}  
%}


% Version

\newcommand{\version}[1]
{  \begin{center}
   {\large #1}
   \end{center}
   \vspace{3pt}
} 










\begin{document}

 

\noindent {\Large \bf Project Description} 
\vspace{0.25cm}

\noindent \hrulefill
\vspace{0.5cm}

\noindent\begin{tabular}{ll}
\large{Applicant:} & \noindent {\large Michael Vogt, University of Bonn} \\[0.1cm]
\large{Project Title:} & \noindent {\large New Methods and Theory for the Comparison of} \\
 & \noindent {\large Nonparametric Trend Curves}
\end{tabular}
\vspace{0.5cm}

\noindent \hrulefill



\section{State of the art and preliminary work}\label{sec:stateofart}


The comparison of nonparametric curves is a classical topic in econometrics and statistics. Depending on the application context, the curves of interest are densities, distribution functions, time trends or regression curves. The problem of testing for equality of densities has been studied in \cite{Mammen1992}, \cite{Anderson1994} and \cite{Li2009} among others. Tests for equality of distribution functions can be found for example in \cite{Kiefer1959}, \cite{Anderson1962} and \cite{Finner2018}. Tests for equality of trend and regression curves have been developed in \cite{HaerdleMarron1990}, \cite{Hall1990}, \cite{Delgado1993}, \cite{DegrasWu2012}, \cite{Zhang2012} and \cite{Hidalgo2014} among many others. In the proposed project, we focus on the comparison of nonparametric trend curves.


The statistical problem of comparing trends has a wide range of applications in economics, finance and other fields such as climatology and biology. In economics, a common issue is to compare trends in real gross domestic product (GDP) across different countries \citep[cp.][]{Grier1989}. 
%It is highly debated whether real GDP growth is faster in economies in transition (such as Brazil, China or India) than in developed countries.
Another example concerns the dynamics of long-term interest rates. To better understand these dynamics, researchers aim to compare the yields of US Treasury bills at different maturities over time \citep[cp.][]{Park2009}. In finance, it is of interest to compare the volatility trends of different stocks \citep[cp.][]{Nyblom2000}. Finally, in climatology, researchers are very much interested in comparing the trending behaviour of temperature time series across different spatial locations \citep[cp.][]{KarolyWu2005}. 
%A final example comes from climatology. In recent years, large spatial data sets have been collected which comprise long temperature time series for many different locations. Climatologists are very much interested in analyzing the trending behaviour of these time series (cp.\ \cite{Mudelsee2018}). In particular, they would like to know how the temperature trend varies across locations. 


Classically, time trends are modelled stochastically in econometrics; see e.g.\ \cite{Stock1988}. Recently, however, there has been a growing interest in econometric models with deterministic time trends; see \cite{Cai2007}, \cite{Atak2011}, \cite{Robinson2012} and \cite{ChenGaoLi2012} among others. Non- and semiparametric trend modelling has attracted particular interest in a panel data context. \cite{LiChenGao2010}, \cite{Atak2011}, \cite{Robinson2012} and \cite{ChenGaoLi2012} considered panel models where the observed time series have a common time trend. In many applications, however, the assumption of a common time trend is quite harsh. In particular when the number of observed time series is large, it is quite natural to suppose that the time trend may differ across time series. More flexible panel settings with heterogeneous trends have been studied, for example, in \cite{Zhang2012} and \cite{Hidalgo2014}. 


%In the proposed project, 
In what follows, we consider a general panel framework with heterogeneous trends which is useful for a number of economic and financial applications: 
Suppose we observe a panel of $n$ time series $\mathcal{Z}_i = \{ (Y_{it},X_{it}): 1 \le t \le T \}$ for $1 \le i \le n$, where $Y_{it}$ are real-valued random variables and $X_{it} = (X_{it,1},\ldots,X_{it,d})^\top$ are $d$-dimensional random vectors. Each time series $\mathcal{Z}_i$ is modelled by the equation
\begin{equation}\label{model}
Y_{it} = m_i \Big( \frac{t}{T} \Big) + \beta_i^\top X_{it} + \alpha_i + \varepsilon_{it}
\end{equation}
for $1 \le t \le T$, where $m_i: [0,1] \rightarrow \mathbb{R}$ is a nonparametric (deterministic) trend function, $X_{it}$ is a vector of regressors or controls and $\beta_i$ is the corresponding parameter vector. Moreover, $\alpha_i$ are so-called fixed effect error terms and $\varepsilon_{it}$ are standard regression errors with $\ex[\varepsilon_{it}|X_{it}] = 0$ for all $t$. Model \eqref{model} nests a number of panel settings which have recently been considered in the literature. Special cases of model \eqref{model} with a nonparametric trend specification are for example considered in \cite{Atak2011}, \cite{Zhang2012} and \cite{Hidalgo2014}. Versions of model \eqref{model} with a parametric trend are studied in \cite{Vogelsang2005}, \cite{Sun2011} and \cite{Xu2012} among others.


As usual in nonparametric regression, the trend functions $m_i$ in model \eqref{model} depend on rescaled time $t/T$ rather than on real time $t$; cp.\ \cite{Robinson1989}, \cite{Dahlhaus1997} and \cite{VogtLinton2014} for the use and some discussion of the rescaled time argument. The functions $m_i$ are only identified up to an additive constant in model \eqref{model}: One can reformulate the model as $Y_{it} = [m_i(t/T) + c_i] + \beta_i^\top X_{it} + [\alpha_i - c_i] + \varepsilon_{it}$, that is, one can freely shift additive constants $c_i$ between the trend $m_i(t/T)$ and the error component $\alpha_i$. In order to obtain identification, one may impose different normalization constraints on the trends $m_i$. One possibility is to normalize them such that $\int_0^1 m_i(u) du = 0$ for all $i$. In what follows, we take for granted that the trends $m_i$ satisfy this constraint. 
%Model \eqref{model} nests a number of panel settings which have recently been considered in the literature. Special cases of model \eqref{model} with a nonparametric trend specification are for example considered in \cite{Atak2011}, \cite{Zhang2012} and \cite{Hidalgo2014}. Versions of model \eqref{model} with a parametric trend are studied in \cite{Vogelsang2005}, \cite{Sun2011} and \cite{Xu2012} among others. 


Within the general framework of model \eqref{model}, we can formulate a number of interesting statistical questions concerning the trend functions $m_i$ for $1 \le i \le n$. 

\vspace{10pt}


\noindent \textbf{(a) Testing for equality of nonparametric trend curves } 
\vspace{10pt} 

 
%\noindent Large number of approaches developed in the literature on deterministic trends in time series relies on the critical assumption of the common trend structure. This assumption means that each individual in the panel exhibits the same trend behavior. Most of the works on this topic can not be easily generalized to the setting where the trend functions are not the same for different individuals. Therefore, it is vital to be able to test the assumption of the common trend before imposing it. 


\noindent In many application contexts, an important question is whether the time trends $m_i$ in model \eqref{model} are all the same. Put differently, the question is whether the observed time series
have a common trend. This question can formally be adressed by a statistical test of the null hypothesis 
\[ H_0: \text{There exists a function } m: [0,1] \rightarrow \mathbb{R} \text{ such that } m_i = m  \text{ for all } 1 \le i \le n. \]
A closely related question is whether all time trends have the same parametric form. To formulate the corresponding null hypothesis, let $m(\theta,\cdot): [0,1] \rightarrow \mathbb{R}$ be a function which is known up to the finite-dimensional parameter $\theta \in \Theta$, where $\Theta$ denotes the parameter space. The null hypothesis of interest now reads as follows:  
\[ H_{0,\text{para}}: \text{ There exists } \theta \in \Theta \text{ such that } m_i(\cdot) = m(\theta,\cdot) \text{ for all } 1 \le i \le n. \]  
If $m(\theta,w) = a + b w$ with $\theta = (a,b)$, for example, then $H_0$ is the hypothesis that all trends $m_i$ are linear with the same intercept $a$ and slope $b$. A somewhat simpler but yet important hypothesis is given by 
\[ H_{0,\text{const}}: m_i \equiv 0 \text{ for all } 1 \le i \le n. \]
Under this hypothesis, there is no time trend at all in the observed time series. Put differently, all the time trends $m_i$ are constant. (Note that under the normalization constraint $\int_0^1 m_i(w) dw = 0$, $m_i$ must be equal to zero if it is a constant function.) A major aim of our project is to develop new tests for the hypotheses $H_0$, $H_{0,\text{para}}$ and $H_{0,\text{const}}$ in model \eqref{model}. In order to keep the exposition as clear as possible, we focus attention to the hypothesis $H_0$ in what follows. Tests of $H_{0,\text{para}}$, $H_{0,\text{const}}$ and related hypotheses were for example studied in \cite{Lyubchich2016} and \cite{ChenWu2018}. 
%A closely related question is whether there is no time trend at all in the observed time series. Put differently, the question is whether the time trends $m_i$ are constant for all $i$. The corresponding null hypothesis can be formulated as 
%\[ H_{0, \text{const}}: \text{There exists some constant } c \text{ such that } m_i \equiv c \text{ for all } 1 \le i \le n. \]
%There are also other interesting hypotheses that one may test in this context. For example, one may be interested in the hypothesis whether the trends $m_i$ are all linear or, more generally, whether they all have the same parametric form. Tests of such hypotheses were for example studied in \cite{Lyubchich2016} and \cite{ChenWu2018}. In order to keep the exposition focused, we restrict attention to the hypothesis $H_0$ in what follows.  


In recent years, a number of different approaches have been developed to test the hypothesis $H_0$. \cite{DegrasWu2012} considered the problem of testing $H_0$ within the model framework
\begin{equation}\label{model-degras}
Y_{it} = m_i \Big( \frac{t}{T} \Big) + \alpha_i + \varepsilon_{it} \qquad (1 \le t \le T, \, 1 \le i \le n), 
\end{equation}
where $\mathbb{E}[\varepsilon_{it}] = 0$ for all $i$ and $t$ and the terms $\alpha_i$ are assumed to be deterministic. Obviously, \eqref{model-degras} is a special case of \eqref{model} which does not include additional regressors. \cite{DegrasWu2012} construct an $L_2$-type statistic to test $H_0$. This statistic is based on nonparametric kernel estimators $\hat{m}_{i,h}$ and $\hat{m}_h$ of the functions $m_i$ and $m$, where $h$ denotes the bandwidth parameter. With the help of these estimators, the authors define the statistic
\[ \Delta_{n,T} = \sum_{i=1}^n \int_0^1 \big(\hat{m}_{i,h}(u) - \hat{m}_h(u)\big)^2 du, \] 
which measures the $L_2$-distance between the estimators $\hat{m}_{i, h}$ and $\hat{m}_h$. In the theoretical part of their paper, they derive the limit distribution of $\Delta_{n,T}$. 
%In particular, they prove that $T h^{1/2} (n-1)^{-1/2} [ \Delta_{n,T} - \mathbb{E} \Delta_{n,T} ]$ is asymptotically normal. 
\cite{ChenWu2018} develop theory for test statistics closely related to those from \cite{DegrasWu2012}, but under more general conditions on the error terms. 
%error processes $\mathcal{E}_i = \{ \varepsilon_{it}: 1 \le t \le T \}$. 


\cite{Zhang2012} investigate the problem of testing the hypothesis $H_0$ in a slightly restricted version of model \eqref{model}, where $\beta_i = \beta$ for all $i$. The regression coefficients $\beta_i$ are thus assumed to be homogeneous in their setting. They construct a residual-based test statistic as follows: First, they obtain profile least squares estimators $\hat{\beta}$ and $\hat{m}_h(t/T)$ of the parameter vector $\beta$ and the common trend $m$ under $H_0$, where $h$ denotes the bandwidth. With these estimators, they compute the residuals $\hat{u}_{it} = Y_{it} - \hat{\beta}^T X_{it} - \hat{m}_h(t/T)$. These residuals are shown to have the form $\hat{u}_{it} = \Delta_i(t/T) + \eta_{it}$, where $\Delta_i$ is a deterministic function with the property that $\Delta_i \equiv 0$ under $H_0$ and $\eta_{it}$ denotes the error term. Testing $H_0$ is thus equivalent to testing the hypothesis $H_0^\prime: \Delta_i \equiv 0$ for all $1 \le i \le n$. The authors construct a test statistic for the hypothesis $H_0^\prime$ on the basis of nonparametric kernel estimators of the functions $\Delta_i$ and derive its limit distribution.  
%Afterwards, the authors calculate the non-parametric goodness-of-fit $R_i^2$ for the local linear regression of $\hat{u}_{it}$ on $t/T$. The test statistic is then defined by averaging the non-parametric goodness-of-fit $R^2_i$, which is proved to be asymptotically normal both under the null hypothesis and under a sequence of local alternatives.


The tests of \cite{Zhang2012}, \cite{DegrasWu2012} and \cite{ChenWu2018} are based on nonparametric estimators of the trend functions $m_i$. They thus depend on one or several bandwidth parameters. It is however far from clear how to choose these bandwidths in an appropriate way. This is a quite general problem which concerns essentially all tests that are based on nonparametric curve estimators. There are of course many theoretical results on the optimal choice of bandwidth for estimation purposes. However, the optimal bandwidth for curve estimation is usually not optimal for testing. Optimal bandwidth choice for tests is indeed a quite open problem, and only little theory for simple cases is available (cp.\ \cite{GaoGijbels2008}). Since tests based on nonparametric curve estimators are commonly quite sensitive to the choice of bandwidth and theory for optimal bandwidth selection is not available, it appears preferable to work with bandwidth-free tests. 


A classical way to obtain a bandwidth-free test of the hypothesis $H_0$ is to use CUSUM-type statistics which are based on partial sum processes. This approach is taken in \cite{Hidalgo2014}. 
%The model considered by the authors is the same as in \cite{Zhang2012}, which is a slightly restricted version of the model \eqref{model} with $\beta_i = \beta$ for all $i$. The authors construct the test statistic as follows. First, they remove the trend and the fixed effects from the model under $H_0$ by a certain transformation of the model and calculate the least squares estimator $\hat{\beta}$ of $\beta$ from the augmented model. Then, instead of $Y_{it}$ they consider the obtained residuals 
%\[\hat{v}_{it}:= Y_{it} - \hat{\beta}^T X_{it} = (\beta - \hat{\beta})^T X_{it} + m_i(t/T) + \alpha_i + \varepsilon_{it}.\]
%Afterwards, they construct CUSUM test statistics based on certain transformations of $\hat{v}_{it}$ and prove that the normalized supremum of these test statistics converges to a random variable with a known distribution. However, the theory derived in \cite{Hidalgo2014} relies on the assumption that the errors $\varepsilon_{it}$ are independent across $t$ which is usually not true in practice. Furthermore, their test statistic is only applicable in the case when the number of cross-sectional units grows which is also a heavy restriction on the possible applications.
%
%
A more modern way to obtain a test statistic which is free of classical bandwidth parameters is to use multiscale methods. The general idea is as follows: Let $S_h$ be a test statistic for the null hypothesis of interest, which depends on the bandwidth $h$. Rather than considering only a single statistic $S_h$ for a specific bandwidth $h$, a multiscale approach simultaneously considers a whole family of statistics $\{S_h: h \in \mathcal{H} \}$, where $\mathcal{H}$ is a set of bandwidth values. The multiscale test then proceeds as follows: For each bandwidth or scale $h$, one checks whether $S_h > q_h(\alpha)$, where $q_h(\alpha)$ is a bandwidth-dependent critical value (for given significance level $\alpha$). The multiscale test rejects if $S_h > q_h(\alpha)$ for at least one scale $h$. The main theoretical difficulty in this approach is of course to derive appropriate critical values $q_h(\alpha)$. 


%The general idea of multiscale tests is to simultaneously consider a whole family of test statistics which correspond to different bandwidths. A more modern way to obtain a test statistic which is free of classical bandwidth parameters is to use multiscale methods. To illustrate this idea, we consider the simple trend model $Y_{it} = m_i(t/T) + \varepsilon_{it}$ for $1 \le t \le T$ and $1 \le i \le n$. For simplicity of exposition, we suppose that $n=2$ and that the errors $\varepsilon_{it}$ are i.i.d.\ with zero mean and unit variance. We want to test the null hypothesis $H_0: m_1 = m_2$. Let $H_0(u,h)$ be the hypothesis that $m_1$ and $m_2$ are the same on the interval $[u-h,u+h]$. Obviously, $H_0$ holds true if and only if $H_0(u,h)$ is fulfilled for any $u \in [0,1]$ and $h > 0$.  We now attempt to construct a procedure which tests the hypothesis $H_0(u,h)$ simultaneously for all $u$ and $h$.\footnote{In practice, we can of course not consider all $u \in [0,1]$ and $h > 0$ but have to take a finite subset thereof.} To achieve this, we proceed in two steps: 
%\begin{enumerate}
%\item We construct a test statistic $\Delta_T(u,h)$ for the hypothesis $H_0(u,h)$ for given $u$ and $h$. In particular, we let $\Delta_T(u,h) = \sqrt{Th} |\hat{m}_{1,h}(u) - \hat{m}_{2,h}(u)|$, where $\hat{m}_{i,h}(u)$ is a standard (local constant or local linear) kernel estimator of $m_i$ at location $u$ with bandwidth $h$. 
%\item We aggregate the individual test statistics $\Delta_T(u,h)$ into one overall statistic $\Delta_T$. As an example, we may simple aggregate the statistics $\Delta_T(u,h)$ by taking the supremum $\sup_{u,h} \Delta_T(u,h)$. In recent years, more intriciate aggregation schemes have been developed which are superior to a simple supremum, cp.\ for example \cite{DuembgenSpokoiny2001}.  
%\end{enumerate}
%$\Delta_T$ is a so-called multiscale statistic as it simulteneously takes into accoiun multple locations $u$ and bandwidths or scales $h$. The  ain theoretical chalnnege is to appropriate critical values for the multiscale statistic. 


One of the first multiscale methods proposed in the literature is the SiZer approach of \cite{ChaudhuriMarron1999, ChaudhuriMarron2000}. In recent years, this approach has been extended in various directions; see \cite{Park2004} and \cite{HannigMarron2006} among others. \cite{Park2009} developed SiZer methods for the comparison of nonparametric trend curves in a simplified version of model \eqref{model}. Their analysis, however, is mainly methodological and only partly backed up by theory. Indeed, theory is only derived for the special case $n=2$, that is, for the case that only two time series are observed. Moreover, the theoretical results are only valid under very severe restrictions on the set of bandwidths $\mathcal{H}$ that is taken into account. In particular, the bandwidths in the set $\mathcal{H}$ are assumed to be bounded away from zero. Put differently, they are not allowed to converge to zero as the sample size grows, which is obviously a very severe limitation. 


A major aim of our project is to develop novel multiscale tests of the hypo\-thesis $H_0$ in the general model \eqref{model} which do not have the limitations of the SiZer methods discussed above. Importantly, we do not only intend to develop new test methodology but also to back up the methods by a general asymptotic distribution theory. To achieve this, we plan to build on a multiscale approach pioneered by \cite{DuembgenSpokoiny2001}. This general approach has been very influential in recent years and has been further developed in numerous directions; see for example \cite{Duembgen2002}, \cite{Rohde2008} and \cite{ProkschWernerMunk2018} for multiscale methods in the regression context and \cite{DuembgenWalther2008}, \cite{RufibachWalther2010}, \cite{SchmidtHieber2013} and \cite{EckleBissantzDette2017} for methods in the context of density estimation. Importantly, all of these studies are limited to the case of independent data. It turns out that it is highly non-trivial to extend the methods to the case of dependent data. To do so, markedly different technical tools are needed. A first step to provide such tools has recently been made in \cite{KhismatullinaVogt2018}. They developed multiscale methods for testing shape restrictions of the nonparametric trend function $m$ in the univariate time series model $Y_t = m(t/T) + \varepsilon_t$. In our project, we aim to extend the techniques and methods from \cite{KhismatullinaVogt2018} to approach the problem of testing $H_0$ in the panel model \eqref{model}. More details are given in Section \ref{sec:objectives} on objectives. 
\vspace{15pt}


\noindent \textbf{(b) Clustering of nonparametric trend curves} 
\vspace{10pt} 


\noindent Consider the situation that the null hypothesis $H_0: m_1 = \ldots = m_n$ is violated in the general panel data model \eqref{model}. 
%and suppose that the null hypothesis $H_0: m_1 = \ldots = m_n$ is violated in this model. 
Even though some of the trend functions $m_i$ are different in this case, there may still be groups of time series with the same time trend. Formally, a group stucture can be defined as follows within the framework of model \eqref{model}: There exist sets or groups of time series $G_1,\ldots,G_{K_0}$ with $\{1,\ldots,n\} = \dot\bigcup_{k=1}^{K_0} G_k$ such that for each $1 \le k \le K_0$, 
\begin{equation}\label{model-groups}
m_i = m_j \quad \text{for all } i,j \in G_k. 
\end{equation}
According to \eqref{model-groups}, the time series of a given group $G_k$ all have the same time trend. In many applications, it is very natural to suppose that there is such a group structure in the data. An interesting statistical problem is how to estimate the unknown groups $G_1,\ldots,G_{K_0}$ and their unknown number $K_0$ from the data. 


There are several approaches to this problem in the context of models closely related to \eqref{model}. \cite{DegrasWu2012} used a repeated testing procedure based on the methods described in part (a) of this section to estimate the unknown group structure in model \eqref{model-degras}. \cite{Zhang2013} developed a clustering method within the same model framework which makes use of an extended Bayesian information criterion. 
% (EBIC). EBIC is calculated for a particular partioning of the time series from the residual sum of squares across all subgrouos and depends on the tuning parameter. The optimal partitioning is then achieved by minimizing EBIC for a fixed tuning parameter. 
\cite{VogtLinton2017} constructed a thresholding method to estimate the unknown group structure in the panel model $Y_{it} = m_i(X_{it}) + u_{it}$, where $X_{it}$ are random regressors and $u_{it}$ are general error terms that may include fixed effects. Their approach can also be adapted to the case of fixed regressors $X_{it} = t/T$.  As an alternative to a group structure, factor-type structures may be imposed on the trend and regression functions in panel models. Such factor-type structures have been studied in \cite{Kneip2012}, \cite{LintonVogt2015} and \cite{BonevaLintonVogt2016} among others. 


The problem of estimating the unknown groups $G_1,\ldots,G_{K_0}$ and their unknown number $K_0$ in model \eqref{model} has close connections to functional data clustering. There, the aim is to cluster smooth random curves that are functions of (rescaled) time and that are observed with or without noise. A number of different clustering approaches have been proposed in the context of functional data models; see for example \cite{Abraham2003}, \cite{Tarpey2003} and \cite{Tarpey2007} for procedures based on $k$-means clustering, \cite{James2003} and \cite{Chiou2007} for model-based clustering approaches and \cite{Jacques2014} for a recent survey. 


\enlargethispage{0.2cm}
The problem of finding the unknown group structure in model \eqref{model} is also closely related to a developing literature in econometrics which aims to identify unknown group structures in parametric panel regression models. In its simplest form, the panel regression model under consideration is given by the equation $Y_{it} = \beta_i^\top X_{it} + u_{it}$ for $1 \le t \le T$ and $1 \le i \le n$, where the coefficient vectors $\beta_i$ are allowed to vary across individuals $i$ and the error terms $u_{it}$ may include fixed effects. Similar to the trend functions in model \eqref{model}, the coefficients $\beta_i$ are assumed to belong to a number of groups: there are $K_0$ groups $G_1,\ldots,G_{K_0}$ such that $\beta_i = \beta_j$ for all $i,j \in G_k$ and all $1\le k \le K_0$. The problem of estimating the unknown groups and their unknown number has been studied in different versions of this modelling framework; cp.\ \cite{Su2016}, \cite{Su2018} and \cite{Wang2018} among others. \cite{Bonhomme2015} considered a related model where the group structure is not imposed on the regression coefficients but rather on the unobserved time-varying fixed effects. 


%Alternatively to a group structure, factor-type structures are often considered for the regression coefficients or regression functions in panel models. Such factor-type structures have been studied in \cite{Kneip2012}, \cite{LintonVogt2015} and \cite{BonevaLintonVogt2016} among others. 
%Whether it is meaningful to impose a group structure on the panel model under consideration depends of course on the application context. In some applications, a group structure is very natural to assume; in others, one may impose other structures on the panel. A prominent examples are factor-type structure. To be more specific, consider the $Y_{it} = m_i(X_{it}) + u_{it}$, where $X_{it}$ are random or deterministic design points and $u_{it}$ are generic error terms. As a special case, one may have $X_{it} = t/T$ in this model. Rather than imposing a group structure on the functions $m_i$, one may work with a factor-structure of the following kind: The functions $m_i(w)$ have the form $m_i(w) = \beta_i^\top g(w)$, where $\beta_i = (\beta_{i1},\ldots,\beta_{iK_0})^\top$ are coefficient vectors and $g = (g_1,\ldots,g_{K_0})^\top$ is a vector of functions. Here, the functions $g$ can be interpreted as common factors and $\beta_i$ play the role of the factor loadings. Such factor-type structures have been considered in ??, ?? and ?? in a panel context.  


Virtually all of the proposed procedures to cluster nonparametric curves in panel and functional data models related to \eqref{model} have the following drawback: they depend on a number of bandwidths or smoothing parameters required to estimate the nonparametric functions $m_i$. 
%A common approach is to approximate the functions $m_i$ by a series expansion $m_i(x) \approx \sum_{j=1}^{L} \beta_{ij} \phi_j(x)$, where $\{ \phi_j: j =1,2,\ldots \}$ is a function basis and $L$ is the number of basis elements taken into account for the estimation of $m_i$. Here, $L$ plays the role of the smoothing parameter and may vary across $i$, that is, $L = L_i$. To estimate the classes $G_1,\ldots,G_{K_0}$, estimators $\hat{\beta}_i$ of the coefficient vectors $\beta_i = (\beta_{i1},\ldots,\beta_{iL})^\top$ are clustered into groups by a standard clustering algorithm. Variants of this approach have for example been investigated in \cite{Abraham2003}, \cite{Chiou2007} and \cite{Tarpey2007}. Another approach is to compute nonparametric estimators $\hat{m}_{i} = \hat{m}_{i,h}$ of the functions $m_i$ for some smoothing parameter $h$ (which may differ across $i$) and to calculate distances $\hat{\rho}_{ij} = \rho(\hat{m}_{i},\hat{m}_{j})$ between the estimates $\hat{m}_{i}$ and $\hat{m}_{j}$, where $\rho(\cdot,\cdot)$ is a distance measure such as a supremum or an $L_2$-distance. A distance-based clustering algorithm is then applied to the distances $\hat{\rho}_{ij}$. This strategy has for example been used in \cite{VogtLinton2017}. 
In general, nonparametric curve estimators strongly depend on the chosen bandwidth parameters. A clustering procedure which is based on such estimators can be expected to be strongly influenced by the choice of bandwidths as well. Moreover, as in the context of statistical testing, there is no theory available on how to pick the bandwidths optimally for the clustering problem. Hence, as in the context of testing, it is desirable to construct a clustering procedure which is free of classical bandwidth parameters. 


There are different ways to move into the direction of a bandwidth-free clustering algorithm. One possibility is to employ Wavelet methods. A Bayesian Wavelet-based method to cluster nonparametric curves has been developed in \cite{Ray2006}. There, the simple model $Y_{it} = m_i(t/T) + \varepsilon_{it}$ is considered, where $m_i$ are smooth functions of rescaled time $t/T$ and the error terms $\varepsilon_{it}$ are restricted to be i.i.d.\ Gaussian noise. 


Another possibility is to use multiscale methods. This approach has recently been taken in \cite{VogtLinton2018}. They develop a clustering approach in the context of the panel model $Y_{it} = m_i(X_{it}) + u_{it}$, where $X_{it}$ are random regressors and $u_{it}$ are general error terms that may include fixed effects. Imposing the same group structure as in \eqref{model-groups} on their  model, they construct estimators of the unknown groups and their unknown number as follows: In a first step, they develop bandwidth-free multiscale statistics $\hat{d}_{ij}$ which measure the distance between pairs of functions $m_i$ and $m_j$. To construct them, they make use of the multiscale testing methods described in part (a) of this section. In a second step, they employ the statistics $\hat{d}_{ij}$ as dissimilarity measures in a hierarchical clustering algorithm. In the theoretical part of their paper, they derive consistency results for their estimators. Letting $\hat{K}_0$ be the estimator of $K_0$ and $\{ \hat{G}_1,\ldots,\hat{G}_{\hat{K}_0} \}$ the estimator of the group structure $\{ G_1,\ldots,G_{K_0} \}$, they in particular show that under appropriate regularity conditions, 
\begin{equation}\label{consistency-res-VogtLinton2018}
\pr \big( \hat{K}_0 = K_0 \big) \rightarrow 1 \quad \text{and} \quad \pr \Big( \{ \hat{G}_1,\ldots,\hat{G}_{\hat{K}_0} \} = \{ G_1,\ldots,G_{K_0} \} \Big) \rightarrow 1 
\end{equation}
as the sample size goes to infinity. %According to \eqref{consistency-res-VogtLinton2018}, the estimators are identical to their true counterparts with probability tending to $1$. 
Even though promising, the consistency result \eqref{consistency-res-VogtLinton2018} is only a first step into the direction of a complete asymptotic theory. A more refined theory would comprise results on convergence rates and confidence statements about the estimators. 


Building on the work of \cite{VogtLinton2018}, we aim to develop multiscale clustering methods in model \eqref{model}. We in particular aim to go beyond the basic theory derived in \cite{VogtLinton2018} and to provide results on convergence rates and confidence statements. We give more details on these objectives in Section \ref{sec:objectives}.  


\subsection{Project-related publications}


\subsubsection{Articles published by outlets with scientific quality assurance, book publications, and works accepted for publication but not yet published}


{\small

\hangindent=0.4cm \textsc{Boneva}, L. and \textsc{Linton}, O. and \textsc{Vogt}, M. (2015). A semiparametric model for heterogeneous panel data with fixed effects. \textit{Journal of Econometrics}, \textbf{188} 327-345.

\vspace{5pt}

\noindent \hangindent=0.4cm \textsc{Boneva}, L. and \textsc{Linton}, O. and \textsc{Vogt}, M. (2016). The effect of fragmentation in trading on market quality in the UK equity market. \textit{Journal of Applied Econometrics}, \textbf{31} 192-213.

\vspace{5pt}

\noindent \hangindent=0.4cm \textsc{Vogt}, M. and \textsc{Linton}, O. (2017). Classification of non-parametric regression functions in longitudinal data models. \textit{Journal of the Royal Statistical Society: Series B}, \textbf{79} 5-27.

}


\subsubsection{Other publications}


{\small

\hangindent=0.4cm \textsc{Khismatullina}, M. and \textsc{Vogt}, M. (2018). Multiscale inference and long-run variance estimation in nonparametric regression with time series errors. \textit{Preprint}.

\vspace{5pt}

\noindent \hangindent=0.4cm \textsc{Vogt}, M. and \textsc{Linton}, O. (2018). Multiscale clustering of nonparametric regression curves. \textit{Preprint}. 

}



\section{Objectives and work programme}


\subsection{Anticipated total duration of the project}


2 years from October 1, 2019 to September 30, 2021.


\subsection{Objectives}\label{sec:objectives}


The main aim of the project is to develop new methods and theory for the comparison and clustering of nonparametric trend curves. As a modelling framework, we will consider the general panel setting \eqref{model} which has already been introduced in Section \ref{sec:stateofart}. We briefly summarize the model setting once again for convenience: Suppose we observe a panel of $n$ time series $\mathcal{Z}_i = \{(Y_{it},X_{it}): 1 \le t \le T\}$ for $ 1 \le i \le n$, where $Y_{it}$ are real-valued random variables and $X_{it}$ are $d$-dimensional random vectors. Each time series $\mathcal{Z}_i$ is modelled by the equation 
\begin{equation}\label{model-objectives}
Y_{it} = m_i \Big( \frac{t}{T} \Big) + \beta_i^\top X_{it} + \alpha_i + \varepsilon_{it} 
\end{equation}
for $ 1 \le t \le T$, where $m_i$ is a nonparametric time trend curve, $X_{it}$ is a vector of regressor or control variables, $\alpha_i$ are unobserved fixed effects and $\varepsilon_{it}$ are idiosyncratic error terms with $\mathbb{E}[\varepsilon_{it}|X_{it} ] = 0$. For each $i$, $\mathcal{P}_i = \{(X_{it},\varepsilon_{it}): 1 \le t \le T\}$ is assumed to be a general time series process which fulfills some weak dependence conditions (e.g.\ conditions formulated in terms of strong mixing coefficients or in terms of the physical dependence measure introduced by \cite{Wu2005}). We will not only allow for time series dependence in the data, but also for some forms of cross-sectional dependence. To derive our theoretical results, we will assume that the time series length $T$ tends to infinity. The number of time series $n$, in contrast, may either be bounded or diverging. 
\vspace{15pt}


\noindent \textbf{(a) Contributions to statistical multiscale testing} 
\vspace{10pt} 


\noindent The first main contribution of the project is to develop a novel multiscale test for the comparison of the trend curves $m_i$ in model \eqref{model-objectives}. More specifically, we aim to develop multiscale tests for the hypothesis $H_0: m_1 = \ldots = m_n$ as well as for related hypotheses such as $H_{0,\text{para}}$ and $H_{0,\text{const}}$. To keep the exposition focused, we restrict attention to $H_0$ in what follows. For any interval $[u-h,u+h] \subseteq [0,1]$, consider the hypothesis
\[ H_0^{[i,j]}(u,h): m_i(w) = m_j(w) \text{ for all } w \in [u-h,u+h]. \] 
Obviously, $H_0$ can be reformulated as
\begin{align*}
H_0: \ & \text{The hypothesis } H_0^{[i,j]}(u,h) \text{ holds true for all intervals } [u-h,u+h] \subseteq [0,1] \\ & \text{ and for all } 1 \le i < j \le n. 
\end{align*} 
We aim to develop a multiscale method which simultaneously tests the hypothesis $H_0^{[i,j]}(u,h)$ for all possible points $(u,h)$ and all pairs $(i,j)$ with $i < j$.\footnote{Obviously, in practice, we cannot consider all points $u \in (0,1)$ and all $h > 0$ but have to restrict attention to a finite subset of points. We ignore this in our presentation for simplicity.} Our strategy to derive such a method can be outlined as follows:
\vspace{10pt}


\noindent \textit{Step 1: Construction of the test statistic.}
\begin{enumerate}[label=(\roman*),leftmargin=0.75cm]

\item Construct nonparametric estimators $\hat{m}_{i,h}$ of the trend functions $m_i$, where $h$ denotes the bandwidth parameter.   

\item For each given $(u,h)$ and $(i,j)$, construct a test statistic $\hat{S}_{ij}(u,h)$ of the hypothesis $H_0^{[i,j]}(u,h)$. A simple choice is a statistic of the form $\hat{S}_{ij}(u,h) = \sqrt{Th} (\hat{m}_{i,h}(u) - \hat{m}_{j,h}(u)) / \hat{\nu}_{ij,h}(u)$, where $\hat{\nu}_{ij,h}(u)$ is chosen to normalize the asymptotic variance of the statistic to $1$. 

\item Aggregate the statistics $\hat{S}_{ij}(u,h)$ for all possible $(u,h)$ and $(i,j)$ into a multiscale statistic. As already discussed in Section \ref{sec:stateofart}, we will use the aggregation scheme of \cite{DuembgenSpokoiny2001} to do so. The resulting multiscale statistic has the form 
\[ \hat{\Psi}_{n,T} = \max_{1 \le i < j \le n} \sup_{u,h} \big\{ |\hat{S}_{ij}(u,h)| - \lambda(h) \big\},  \]
where $\lambda(h)$ are (appropriately chosen) additive correction terms. As one can see, the multiscale statistic $\hat{\Psi}_{n,T}$ is not obtained by simply taking the supremum of the individual statistics $\hat{S}_{ij}(u,h)$. We rather take the supremum of the additively corrected statistics $|\hat{S}_{ij}(u,h)| - \lambda(h)$ as first suggested in \cite{DuembgenSpokoiny2001}. 

\end{enumerate}


\noindent \textit{Step 2: Construction of the test procedure.}
\begin{enumerate}[label=(\roman*),leftmargin=0.75cm]

\item Suppose for a moment we could compute the $(1-\alpha)$-quantile $q_{n,T}^*(\alpha)$ of the multiscale statistic $\hat{\Psi}_{n,T}$ under the null $H_0$. Then our multiscale test would be carried out as follows: 
\begin{itemize}[leftmargin=1cm]

\item[(T$^*$)] Reject the overall null hypothesis $H_0$ if $\hat{\Psi}_{n,T} > q_{n,T}^*(\alpha)$. 
\end{itemize}
By construction, the decision rule (T$^*$) is a rigorous level-$\alpha$-test, which means that $\mathbb{P}(\hat{\Psi}_{n,T} > q_{n,T}^*(\alpha)) = 1-\alpha$ under $H_0$. 
\item The quantile $q_{n,T}^*(\alpha)$ is a highly complicated quantity which is not known in practice. Hence, it cannot be used to set up the test. The main theoretical challenge is to come up with an (asymptotic) approximation $q_{n,T}(\alpha)$ of this quantile which is computable in practice. In particular, $q_{n,T}(\alpha)$ should be ensured to have the theoretical property that 
\[ \mathbb{P} (\hat{\Psi}_{n,T} > q_{n,T}(\alpha)) = (1-\alpha) + o(1). \]
Once we have successfully derived the approximation $q_{n,T}(\alpha)$, the multiscale test can be carried out as follows: 
\begin{itemize}[leftmargin=0.8cm]
\item[(T)] Reject the overall null hypothesis $H_0$ if $\hat{\Psi}_{n,T} > q_{n,T}(\alpha)$. 
\end{itemize}

\item By using the decision rule (T), we regard the multiscale method as a test of the overall hypothesis $H_0$. Alternatively, one may view it as a simultaneous test of the family of hypotheses $H_0^{[i,j]}(u,h)$ for all points $(u,h)$ and pairs $(i,j)$. Looking at the method this way, one may proceed as follows: 
\begin{itemize}[leftmargin=1.5cm]
\item[(T$_\text{mult}$)] 
For each interval $[u-h,u+h]$, reject the hypothesis $H_0^{[i,j]}(u,h)$ if the corrected test statistic $|\hat{S}_{ij}(u,h)| - \lambda(h)$ is above the critical value $q_{n,T}(\alpha)$, that is, if $|\hat{S}_{ij}(u,h)| - \lambda(h) > q_{n,T}(\alpha)$. 
\end{itemize}
We conjecture that it is possible to prove the following theoretical result on the multiple testing procedure $(T_{\text{mult}})$ under appropriate regularity conditions: With asymptotic probability at least $1-\alpha$, the hypothesis $H_0^{[i,j]}(u,h)$ is violated for all pairs $(i,j)$ and for all intervals $[u-h,u+h]$ for which $|\hat{S}_{ij}(u,h)| - \lambda(h) > 0$. According to this result, we can make the following simultaneous confidence statement: We can claim, with (asymptotic) confidence at least $1-\alpha$, that the hypothesis $H_0^{[i,j]}(u,h)$ is violated for all pairs of time series $(i,j)$ and for all intervals $[u-h,u+h]$ for which our test rejects. Hence, the multiscale test does not only give us information on whether the overall null hypothesis $H_0$ is violated. It also allows us to make rigorous statistical confidence statements about (i) which pairs of time series $(i,j)$ have different trends and (ii) in which time regions $[u-h,u+h]$ these trends differ. This is valuable information in many applications. 

\end{enumerate}
In order to derive the multiscale methods outlined above, we will build on the methods and theory developed in \cite{KhismatullinaVogt2018}. However, since the data structure in the panel model \eqref{model-objectives} differs in various important respects from that of a univariate time series, the results from \cite{KhismatullinaVogt2018} do not carry over directly. Hence, a substantial amount of work is needed to develop the above multiscale methods and to derive theory for them. 


Compared to existing test procedures, the multiscale test proposed above has the following main advantages: 
\begin{enumerate}[label=(\roman*),leftmargin=0.75cm]
\item Unlike many other methods, it does not depend on a specific bandwidth para\-meter $h$. It rather takes into account multiple scales or bandwidths $h$ simultaneously and can thus be regarded as bandwidth-free. (Cp.\ Section \ref{sec:stateofart} for a more detailed discussion of this advantage.)
\item It is much more informative than non-multiscale tests: As explained above, it does not only allow to test whether the overall null hypothesis $H_0$ is violated. It also allows to make rigorous statistical confidence statements about which time series have a different trend and in which time regions these trends differ. 
\end{enumerate}
%. The only exception is ?? who have developed theory for the case $n=2$. However, the theory is developed under severe restrictions: ??.   
%We do not only aim to develop methodology but also derive a complete asymptotic theory for the proposed multiscale test. In particular, we will derive the limit distribution and analyse the behaviour under (local) alternatives. 
\vspace{5pt}


\noindent \textbf{(b) Contributions to curve clustering} 
\vspace{10pt} 


\noindent The second main contribution of the project is to develop a multiscale clustering approach which is based on the test methods from the first part of the project. 
%The only multiscale clustering method available in the literature is \cite{VogtLinton2018}. 
%More specifically, our objectives are as follows:
To the best of our knowledge, the only multiscale clustering method available in the literature is due to \cite{VogtLinton2018}. Building on their approach, our strategy to construct a multiscale clustering algorithm in model \eqref{model-objectives} is as follows: (i) We use the multiscale test statistics from the first part of the project to construct distance measures between pairs of trends $m_i$ and $m_j$. (ii) From these distance measures, we obtain so-called dissimilarity measures which form the basis of a hierarchical clustering algorithm. 


As already discussed at the end of Section \ref{sec:stateofart}, \cite{VogtLinton2018} derived some basic consistency statements for their clustering algorithm. The main challenge is to develop a refined theory which goes beyond these basic statements. To achieve this, we plan to make use of the theoretical results developed for the multiscale test in the first part of the project. We conjecture that with the help of these results, we should be able to derive much more precise theoretical statements than those in \cite{VogtLinton2018}. Let us consider the problem of estimating the unknown number of groups $K_0$ to illustrate this point. The estimator of $K_0$ in \cite{VogtLinton2018} depends on a tuning parameter $\pi_{n,T}$. Only a heuristic rule is available for the choice of this parameter. In contrast to this, we plan to derive rigorous theory for the choice of this tuning parameter. In particular, we intend to choose $\pi_{n,T} = q_{n,T}(\alpha)$, where $q_{n,T}(\alpha)$ is the approximate quantile of the multiscale statistic introduced above. With this choice, our estimator $\hat{K}_0$ of $K_0$ implicitly depends on the significance level $\alpha$, that is, $\hat{K}_0 = \hat{K}_0(\alpha)$. For a given $\alpha$, we aim to prove that  
\begin{equation}\label{res-K0}
\mathbb{P}( \hat{K}_0 = K_0 ) \ge (1-\alpha) + O(r_{n,T}), 
\end{equation}
where $r_{n,T}$ is the rate of the lower order terms. This statement can be interpreted as follows: For given $\alpha$, $\hat{K}_0$ is equal to the true number of groups $K_0$ with (asymptotic) probability at least $1-\alpha$. Hence, we can tune the clustering algorithm in such a way that the probability of misestimating the number of groups $K_0$ is (asymptotically) controlled. \eqref{res-K0} can thus be understood as an asymptotic confidence statement about the estimator $\hat{K}_0$. 
\vspace{15pt}


\noindent \textbf{(c) Empirical applications} 
\vspace{10pt}


\noindent The methodological and theoretical analysis of the project will be complemented by simulations and empirical applications. First of all, we will carry out a detailed simulation study to examine the finite sample performance of the proposed test and clustering methods. Moreover, we intend to demonstrate their usefulness by empirical data examples. 


Our test and clustering methods have a wide range of potential applications in economics and finance. Among other things, they can be used to compare the volatility trends of different stocks \citep{Nyblom2000}, short-term risk-free interest rates \citep{Fan2008,Park2009} or long-term rates across countries \citep{Park2009}.
Another potential application is concerned with economic growth, which has been a key topic in macroeconomics for many decades. Economists are very much interested in the question whether gross domestic product (GDP) growth has been faster in some countries than in others. A suitable econometric framework to investigate this question is the panel data model \eqref{model-objectives}. \cite{Zhang2012} used a special case of this model to analyze data from 16 OECD countries. For each of the $n=16$ countries, quarterly time series data on gross domestic product ($GDP$), capital stock ($K$), labour input ($L$) and human capital ($H$) were available. The data were assumed to follow the model 
\[ \Delta \log GDP_{it} = m_i\Big(\frac{t}{T}\Big) + \beta_1 \Delta \log L_{it} + \beta_2 \Delta \log K_{it} + \beta_3 \Delta \log H_{it} + \alpha_i + \varepsilon_{it} \]
with $1 \le t \le T = 140$ and $1 \le i \le n = 16$, where $m_i$ is the time trend of country $i$, $\beta_k$ are unknown regression coefficients and $\Delta \log Z_{it} = \log Z_{it} - \log Z_{it-1}$ for $Z_{it} = GDP_{it}$, $L_{it}$, $K_{it}$, $H_{it}$. \cite{Zhang2012} tested the widely used common trends hypothesis $H_0: m_1 = \ldots = m_n$ in this framework. Their analysis provided evidence against $H_0$. Specifically, their test rejected $H_0$ at the 10\% confidence level. However, even if the common trends hypothesis is violated, there may still be groups of countries with the same time trend. It may thus be interesting to cluster the OECD countries into groups. We intend to use the multiscale methods developed in the project to produce such a clustering and, more generally, to analyze an updated version of the data sample from \cite{Zhang2012}. 


Another application we would like to explore deals with the analysis of temperature data, which has attracted some attention in econometrics in recent years; see e.g.\ \cite{Gao2006}, \cite{Atak2011} and \cite{Davidson2016}. Over the last decades, large panel data sets have become available which contain long temperature time series $\mathcal{Z}_i = \{ Y_{it}: 1 \le t \le T \}$ for a huge number of different spatial locations $i$; see the Berkeley Earth project at \texttt{http://berkeleyearth.org} for examples of such big data sets. A simple trend model for the time series $\mathcal{Z}_i$ is given by the equation
\[ Y_{it} = m_i\Big(\frac{t}{T}\Big) + \alpha_i + \varepsilon_{it} \]
with $\ex[\varepsilon_{it}] = 0$, where $m_i$ is the temperature trend at location $i$. %; cp.\ for example \cite{Ghil1991} and \cite{Mudelsee2018}. 
If covariates are available, we could also work with the more general model \eqref{model-objectives}. Climatologists are very much interested in analyzing the trending behaviour of temperature time series. Information on the trending behaviour is needed to better understand long-term climate variability.  Among other things, they would like to know whether the time trends $m_i$ are the same across locations or whether they can be clustered into groups. We aim to investigate these questions by the test and clustering methods developed in the project. 


%Comparison of trends has practically limitless range of applications in finance and economics. Possible examples of the time series include and are not limited to volatility of different stocks, short-term risk-free interest rates (\cite{Fan2008}, \cite{Park2009}) or long-term rates across countries (\cite{Park2009}).
%%Short-term risk-free interest rates are one of the main topics of interest in the financial markets. For example, it is a key component of the capital asset pricing model, which describes the relationship between risk and return. Furthermore, the risk-free rate is also a required input in financial calculations regarding the pricing of bonds. US Treasury bills are the real-world investment that serve as proxy for these rates. \cite{Park2009} analyze the yields of the 3-month, 6-month, and 12-month US Treasury bills in the context of comparing nonparametric curves by applying the Si{Z}er method to the data. The authors could not find any significant difference between any pair of the time periods, which concides with the results from applying other methods, see, for example, \cite{Fan2008}.
%%As for comparing different market rates across countries, \cite{Park2009} study the long-term rates for US, Canada, and Japan from January 1980 to December 2000. The authors perform pairwise comparison of the curves as well as comparison of the three time series at the same time using the proposed Si{Z}er method. In both cases their method was able to detect significant differences and indicate ``suspicious'' regions.
%%the long-term interest rates for the US and Canada moved quite closely together from approximately 1993 to 1995, despite different business cycle positions at those times. We also can confirm in the SiZer map the events of the fall of the Canadian rates to just below the US rates for the first time in over a decade around 1996.
%we can see that in the period from 1982 to mid 1984 the US rates rose as the Japanese rates were falling, believed by the authors to be caused in part by the effects of US fiscal expansion in raising the demand for domestic savings relative to its supply indicating this early 1980s time period. we can see significant divergences in the interest rates in the late 1980s as US rates begin to fall back, rates in Canada and Japan are increasing. the larger values of Canada and Japan cause a significant negative difference. We can see this short-term similarity between Canada and Japan, however, the graph is clearly dominated by the more rapid descent of the Canadian rates through the overall decrease of both countries.
%%It would also be interesting to compare three yields of three countries at the same time, as in Examples 1 and 2, respectively. To save space we only report the result of Example 2 for multiple comparison. We can see that in Fig. 9, there are differences that occur within each SiZer map, denoting that there are present. We have seen in Fig. 8 that there existed pairwise differences between all of the countries. The presence of these differences are also correctly detected when we compare each set of residuals from each country's individual estimated function to the residuals from the overall estimation.
%Moreover, economic growth has been a key topic in marcoeconomics over many decades. Economists are very much interested in the question whether gross domestic product (GDP) growth  has been faster in some countries than in others. One of the ways to model the source of economic growth is to incorporate a nonparametric deterministic time trend in the model. For example, \cite{Zhang2012} consider such a model for the OECD economic growth data.
%%Specifically, they investigate the following model for growth rates:
%%\begin{equation}\label{model-zhang}
%%\Delta \ln GDP_{it} = \beta_1 \Delta \log L_{it} +\beta_2 \Delta \log K_{it} +\beta_3 \Delta \log H_{it} +f_i(t/T) +\alpha_i + \varepsilon_{it},
%%\end{equation}
%%where $i = 1,\ldots, n$, $t = 1, \ldots, T = 140$, $GDP$ is gross domestic product, $K$ is capital stock, $L$ is labour input, $H$ is human capital, $\alpha_i$ is a fixed effect, $f_i(\cdot)$ is an unknown smooth time trend furnction and $\varepsilon_{it}$ are idiosyncratic errors. The errors are allowed to be dependent cross-sectionally, but not serially over $t$. The data comes from $n = 16$ OECD countries.
%The authors estimate the common component of time trends which appears to be significantly different from zero over a wide range of its support. Moreover, they test the null hypothesis that there are no significant differences in the time trends for the 16 OECD countries. Based on the bootstrap $p$-values the authors are able to reject the null hypothesis of all the trends being equal at the 10\% confidence level. Hence, it can be interesting to be able to further cluster the OECD countries based on their economic growth rates. We intend to apply our test and clustering method to investighate this in more detail. 


%\begin{example}
%The issue of global warming has been a vital topic for many scientists over the last few decades. Since the late 1970, different models that describe the global temperature have been published. In the current literature it is common to assume that the temperate time series (global as well as local) follow a model that can be decomposed into a deterministic trend component and a noise component, see, for example, \cite{Ghil1991} and \cite{Mudelsee2018}. In order to estimate and attribute the trends in climate variables, a variety of econometric methods have been employed, starting from the simple linear (\cite{Yue2013}) regression to the empirical mode decompoistion (\cite{Wu2011}), spectrum analysis (\cite{Ghil1991}) and semi- and fully non-parametric methods (\cite{Gao2006}). And while parametric and change points methods are mostly suited to quantify the magnitude of the warming trend or to determine the change points, nonparametric methods are best designed to describe the trend over the full time interval without imposing any additional structure on it.

%However, most of the current statistical paper either are concerned with analyzing only one time series or the authors assume that the trend function is common for different time series. To our knowledge, only a few papers regarding the comparison of warming trends in different cities or countries have been published. One of the examples is a paper by \cite{Zhang2012}, where the authors analyze the data on the temperatures in different UK cities. Based on the obtained $p$-values, \cite{Zhang2012} reject the null hypothesis of common trend at $5\%$ level for the monthly mean maximum temperature and the monthly mean minimum temperature. It may be interesting to further cluster the UK stations based on the common trend behavior. Moreover, it can also be of particular interest to see in which time regions the trends are significantly different from each other in order to be able to detect the causes of the warming trend.

%\cite{Atak2011} propose the following semiparametric panel model for unbalanced data to describe the trend in UK regional temperatures:
%\begin{equation}\label{model-atak}
%y_{it} = \alpha_i + \beta_i^\prime D_t + \gamma_i^\prime X_{it} + g(t/T) + \varepsilon_{it},
%\end{equation}
%where $y_{it}$ are the monthly mean temperature at a station $i, i =1, \ldots, n$ in month $t, t=t_i, \ldots, T$, $D_t$ is a vector of seasonal dummy variables, $X_{it}$ are a vector of observed covariates, $\alpha_i$ is a fixed effect for station $i$, $g(\cdot)$ is an unknown single common trend and $\varepsilon_{it}$ are idiosyncratic errors.

%\cite{Zhang2012} propose the following semiparametric panel model for unbalanced data to describe the trend in UK regional temperatures:
%\begin{equation}\label{model-atak}
%y_{it} =\beta_i^{T}D_t + m_i(t/T) + \alpha_i + \varepsilon_{it},\quad i =1, \ldots, n, \quad t=1, \ldots, T
%\end{equation}
%where $y_{it}$ are the monthly mean maximum temperature, monthly mean minimum temperature or total rainfall in millimeters at a station $i$ in month $t$, $D_t$ is a $11$-dimensional vector of monthly dummy variables, $\alpha_i$ is the fixed effect for station $i$, $m_i(\cdot)$ is an unknown trend function and $\varepsilon_{it}$ are idiosyncratic errors. The dataset used is the balanced panel data set for $n=26$ stations in UK for $T=382$ months from October 1978 to July 2010. This model is a special case of our proposed model \eqref{model-objectives} with dummy variables as covariates.

%Over the past decade, substantial efforts have gone into establishing reliable and accurate records of surface air temperatures for periods of a century or more. The Intergovernmental Panel on Climate Change (IPCC) Report 10 provides an excellent review of remaining problems. We have chosen for the present analysis the time series of annually-averaged temperatures from 1854 to 1988 produced by the Climate Research Unit (CRU) of the University of East Anglia, and verified the results against the IPCC consensus time series (1856-1989). Only the annual means for the Northern Hemisphere (NH), Southern Hemisphere (SH) and the entire globe were used here.
%An ever-growing body of evidence regarding observed changes in the climate system has been gathered over the last three decades, and large modeling efforts have been carried to explore how climate may evolve during the present century. The impacts from both observed weather and climate endured during the twentieth century and the magnitude of the potential future impacts of climate change have made this phenomenon of high interest for the policy-makers and the society at large. Two fundamental questions arise for understanding the nature of this problem and the appropriate strategies to address it: Is there a long-term warming signal in the observed climate, or is it the product of natural variability alone? If so, how much of this warming signal can be attributed to anthropogenic activities? As discussed in this review, these questions are intrinsically related to the study of the time-series properties of climate and radiative forcing variables and of the existence of common features such as secular co-movements. This paper presents a brief summary of how detection and attribution studies have evolved in the climate change literature and an overview of the time-series and econometric methods that have been applied for these purposes.
%Significant advances have been made in documenting how global and hemispheric temperatures have evolved and in learning about the causes of these changes. On the one hand, large efforts have been devoted to investigate the time series properties of temperature and radiative forcing variables \cite{Gay-Garcia2009}; \cite{Kaufmann2006}; \cite{Mills2013}; \cite{Tol1993}.
%In addition,, including features such as breaks and nonlinearities \cite{Estrada2013}; \cite{Gallagher2013}; \cite{Harvey2002}; \cite{Karl2000}; \cite{Pretis2015}; \cite{Reeves2007}; \cite{Seidel2004}; \cite{Stocker2013}; Tom´e and Miranda, 2004). Multivariate models of temperature and radiative forcing series provide strong evidence for a common secular trend between these variables, and help to evaluate the relative importance of its natural and anthropogenic drivers (Estrada, Perron and Mart´ınez-L´opez, 2013; Estrada, Perron, Gay-Garc´ıa and Mart´ınez-L´opez, 2013; Kaufmann et al., 2006; Tol and Vos, 1998). The methodological contributions of the econometrics literature to this field have been notable; e.g., Dickey and Fuller (1979), Engle and Granger (1987), Johansen (1991), Perron (1989, 1997), Bierens (2000), Ng and Perron (2001), Kim and Perron (2009), Perron and Yabu (2009), among many others, see Estrada and Perron (2014) for a review. Regardless of the differences in assumptions and methods (statisticalor physical), there is a general consensus about the existence of a common secular trend between temperatures and radiative forcing variables.
%\end{example}


%The main purpose of the research project is to propose a new multiscale testing and inference approach for the model which consists of multiple time series with time series error structure and in the presence of generated regressors. 

%When several time series $\mathcal{Y}_i = \{ Y_{it}: 1 \le t \le T \}$ are observed for $1 \le i \le n$, we model each time series $\mathcal{Y}_i$ by the equation
%\begin{equation}\label{model2-intro}
%Y_{it} = m_i \Big( \frac{t}{T} \Big) + \alpha_i + \varepsilon_{it}
%\end{equation}
%for $1 \le t \le T$, where $m_i$ is a nonparametric time trend, $\alpha_i$ is a (random or deterministic) intercept and $\varepsilon_{it}$ are time series errors with $\ex[\varepsilon_{it}] = 0$ for all $t$.

%An important question in many applications is whether the time trends $m_i$ are the same for all $i$. When some of the trends are different, there may still be groups of time series with the same trend. In this case, it is often of interest to estimate the unknown groups from the data. In addition, when two trends $m_i$ and $m_j$ are not the same, it may also be relevant to know in which time regions they differ from each other. In Section \ref{sec-test-equality}, we construct statistical methods to approach these questions. In particular, we develop a test of the hypothesis that all time trends in model \eqref{model2-intro} are the same, that is, $m_1 = m_2 = \ldots = m_n$. Similar as before, our method does not only allow to test whether the null hypothesis is violated. It also allows to detect, with a given statistical confidence, which time trends are different and in which time regions they differ from each other. Based on our test method, we further construct an algorithm which clusters the observed time series into groups with the same trend. 


\subsection{Work programme incl. proposed research methods}


%All phases of the research will be conducted in close collaboration with the partners in Bonn.
The first phase of the project will be devoted to derive the multiscale test methods described in part (a) of Section \ref{sec:objectives}. The second phase will focus on the multiscale clustering methods described in part (b) of Section \ref{sec:objectives}. In each of the two phases, we will proceed as follows: We will first develop the statistical methodology, then derive the theoretical properties thereof, evaluate the finite sample performance by simulations, and finally illustrate the methods by the applications discussed in part (c) of Section \ref{sec:objectives}. A timetable of the two main phases of the project is included below. 

\begin{center}
\begin{tabular}{r c c c}
{\bf Milestone} & {\bf 2019} & {\bf 2020} & {\bf 2021} \\
Multiscale testing & Oct--Dec & Jan--Dec & \\
Multiscale clustering & & & Jan--Oct
\end{tabular}
\end{center}


%\subsection{Data handling}

%[Text]

%\subsection{Other information}
%Please use this section for any additional information you feel is relevant which has not been provided elsewhere.

%[Text]

%\subsection{Descriptions of proposed investigations involving experiments on humans, human materials or animals as well as dual use research of concern}

%[Text]

%\subsection{Information on scientific and financial involvement of international cooperation partners}

%[Text]



\section{Bibliography}

\vspace{-1.25cm}

\renewcommand\refname{}
\bibliographystyle{ims}
{\small
\setlength{\bibsep}{0.55em}
\bibliography{bibliography}}



\section{Requested modules/funds}
%Explain each item for each applicant (stating last name, first name).


\subsection{Basic Module}


\subsubsection{Funding for Staff}

\begin{center}
\begin{tabular}{c l c c c}
No. & Position & 2019 & 2020 & 2021 \\
\hline 
1.  & Doctoral researcher & ??\euro{} & ??\euro{} & ??\euro{} \\
    & (Bes.Gr. E13 Stufe 2 / E14 Stufe 1, 75\%) & & & \\
2.  & Student Assistant   & ??\euro{} & ??\euro{} & ??\euro{} \\
\hline
    & Total Amount        & ??\euro{} & ??\euro{} & ??\euro{} \\
\end{tabular}
\end{center}

\noindent Job description for requested staff: 
\begin{enumerate}[leftmargin=0.5cm]
\item As research staff, a doctoral student is required who already possesses a tho\-rough expertise in statistical multiscale methods. Marina Khismatullina who is a member of the Bonn Graduate School of Economics fits these requirements very well. Ms.\ Khismatullina and the applicant have already collaborated on a project which is concerned with statistical multiscale techniques; cp.\ \cite{KhismatullinaVogt2018}. Ms.\ Khismatullina would thus bring in the expertise needed for the project. Moreover, with her very advanced programming skills, she will be able to develop the computational software required for simulations and empirical applications. 
\item At the onset of the project, a student assistent position should be available for support with exploratory data analysis, data mining, auxiliary programming and organisational issues. The prerequisites are strong analytical and programming skills.
\end{enumerate}

\newpage
\noindent \textcolor{red}{Wie bestimme ich sinnvoll die Zahlen in obiger Tabelle? Soll ich einfach die Zahlen aus dem DFG-Merkblatt f\"ur Personalmittels\"atze \"ubernehmen (also $66 \, 300$ \euro{} pro Jahr f\"ur die Doktorandenstelle)? Wie bestimme ich die Hiwi-Entlohnung?} \\
\noindent \textcolor{red}{Ist es in Ordnung, dass ich mich in Punkt 1.\ explizit auf Frau Khismatullina beziehe oder sollte ich das sein lassen?}


\subsubsection{Direct Project Costs}

%\paragraph{Equipment up to Euro 10,000, Software and Consumables}

\paragraph{Travel Expenses}

\begin{center}
\begin{tabular}{l c c c}
 & 2019 & 2020 & 2021 \\
\hline 
International conferences & 500 \euro{} & 1000 \euro{} & 1000 \euro{} \\
\end{tabular}
\end{center}

\noindent \textcolor{red}{Sind diese Zahlen angemessen? Zu hoch/niedrig f\"ur die DFG angesetzt?}

%\paragraph{Visiting Researchers (excluding Mercator Fellows)}

%\paragraph{Expenses for Laboratory Animals}

%\paragraph{Other Costs}

\paragraph{Project-related publication expenses}

\begin{center}
\begin{tabular}{l c c c}
 & 2019 & 2020 & 2021 \\
\hline 
Journal submission fees & 0 \euro{} & 200 \euro{} & 200 \euro{} \\
\end{tabular}
\end{center}


%\subsubsection{Instrumentation}

%\paragraph{Equipment exceeding Euro 10,000}

%\paragraph{Major Instrumentation exceeding Euro 50,000}

%\subsection{Module Temporary Position for Principal Investigator}

%\subsection{Module Replacement Funding}

%\subsection{Module Temporary Clinician Substitute}

%\subsection{Module Mercator Fellows}

%\subsection{Module Workshop Funding}

%\subsection{Module Public Relations Funding}



\section{Project requirements}

\subsection{Employment status information}

%For each applicant, state the last name, first name, and employment status (including duration of contract and funding body, if on a fixed-term contract).
Vogt, Michael, Professor, tenured position

\subsection{First-time proposal data}

%Only if applicable: Last name, first name of first-time applicant
Vogt, Michael

\subsection{Composition of the project group}
%List only those individuals who will work on the project but will not be paid out of the project funds. State each person’s name, academic title, employment status, and type of funding.
Vogt, Michael, Professor, tenured position

\noindent \textcolor{red}{Ist die Liste so vollst\"andig? Soll ich hier mehr Details zu meiner Person, Forschung etc.\ anf\"uhren?}

\subsection{Cooperation with other researchers}

\subsubsection{Researchers with whom you have agreed to cooperate on this project}

\textcolor{red}{Wie ist dies gemeint?}

\subsubsection{Researchers with whom you have collaborated scientifically within the past three years}

\noindent Holger Dette -- University of Bochum, Germany 

\noindent Oliver Linton -- University of Cambridge, UK

\noindent Mar\'ia Dolores Mart\'inez-Miranda -- University of Granada, Spain

\noindent Enno Mammen -- University of Heidelberg, Germany

\noindent Jens Perch Nielsen -- CASS Business School, London, UK 

\noindent Matthias Schmid -- University of Bonn, Germany

\noindent Christopher Walsh -- University of Dortmund, Germany 


\subsection{Scientific equipment}
The University of Bonn has a sufficient infrastructure in hard- and software to carry out the project. Personal computers are available and can be used within the project. Equipment like printers and copiers can be used as well.

%\subsection{Project-relevant cooperation with commercial enterprises}
%If applicable, please note the EU guidelines on state aid or contact your research institution in this regard.

%\subsection{Project-relevant participation in commercial enterprises}
%Information on connections between the project and the production branch of the enterprise



\section{Additional information}
%If applicable, please list proposals requesting major instrumentation and/or those previously submitted to a third party here.

A request for funding this project has not been submitted to any other addresses. The DFG liaison officer of the University of Bonn has been informed about this application.



\vspace{10pt}

\noindent \textcolor{red}{Allgemeine Frage: ``I'' oder ``We''?}



\end{document}
