%%%%%%%%%%%%%%%%%%%% author.tex %%%%%%%%%%%%%%%%%%%%%%%%%%%%%%%%%%%
%
% sample root file for your "abstract" to the IFCS 2024 Conference
%
% Use this file as a template for your own input.
%
%%%%%%%%%%%%%%%% Springer %%%%%%%%%%%%%%%%%%%%%%%%%%%%%%%%%%


% RECOMMENDED %%%%%%%%%%%%%%%%%%%%%%%%%%%%%%%%%%%%%%%%%%%%%%%%%%%
\documentclass[graybox]{IFCS2024}


\usepackage{type1cm}        % activate if the above 3 fonts are
                            % not available on your system
%
\usepackage{makeidx}         % allows index generation
\usepackage{multicol}        % used for the two-column index
\usepackage[bottom]{footmisc}% places footnotes at page bottom


\usepackage{newtxtext}       % 
\usepackage[varvw]{newtxmath}       % selects Times Roman as basic font

% DO NOT USE OTHER COMMANDS OR MACROS


\makeindex             % used for the subject index
                       % please use the style svind.ist with
                       % your makeindex program

%%%%%%%%%%%%%%%%%%%%%%%%%%%%%%%%%%%%%%%%%%%%%%%%%%%%%%%%%%%%%%%%%%%%%%%%%%%%%%%%%%%%%%%%%

\begin{document}

\title*{Guidelines for Authors of Abstracts \protect\linebreak
		Submitted to IFCS 2024 Book of Abstracts}
\titlerunning{Short Title} %for an abbreviated version of your contribution title if the original one is too long
\author{Name of First Author and Name of Second Author}
\authorrunning{Name of First Author and Name of Second Author} %If there are more than two authors, please, abbreviate the authors' list using 'et al'

\institute{Name of First Author \at Name of Institute, Address of Institute, \email{name@email.address}
\and Name of Second Author \at Name of Institute, Address of Institute \email{name@email.address}}
%
% Use the package "url.sty" to avoid
% problems with special characters
% used in your e-mail or web address
%
\maketitle


\abstract{
This template for submitting an abstract to the International Federation of Classification Societies (IFCS) 2024 Conference, that will take place in San Jos\'e, Costa Rica.
Since all texts should be compiled together, files that do not comply with this format will not be accepted.
Please, use only the \texttt{IFCS2024.cls} style file and standard fonts.
\textbf{DO NOT} include any user-defined commands and macros.
No figures nor tables should be included in the abstract.
}

\keywords{aaa, bbb, ccc}


%%%%%%% END OF ABSTRACT
%%%%%%% FURTHER INSTRUCTIONS

\medskip

For the heading, specify the following items:
\begin{itemize}
\item \verb@\title*@ to specify the title of your manuscript,
\item \verb@\titlerunning@ to specify the title in the running head (odd page numbers),  use an abbreviated version of your contribution title if the original one is too long.
\item \verb@\author@ to specify the authors. Authors are
separated by the \verb@\and@ command. Use the \verb@\inst{1}@,
\verb@\inst{2}@, \ldots\ commands to define the reference mark to your affiliation.
\item \verb@\authorrunning@ to specify the author names in the running heads (even page numbers).
If there are more than two authors, please, abbreviate the authors' list
(e.g., Brito et al.).
\item \verb@\institute@ to specify your affiliation, address and e-mail address.
Separate two or more different affiliations by the
\verb@\and@ command.
\end{itemize}

Specify 3 to 5 \textbf{keywords} in your abstract, using the command \verb@\keywords@.


Use \verb@\cite{<label>}@ for bibliographic references, in case you use them. 
	 References should be \textit{cited} in the text by number.  
The reference list should be arranged in ascending order.
Make sure that all references from the list are cited in the text. 
The \textbf{styling} of references is depicted in 
~\cite{science-contrib, science-online, science-mono, science-journal, science-DOI}, for, respectively, a Chapter in an edited book, an Online Document, a Book/Monograph, a Journal paper, a Journal paper by DOI. \\
Please pay attention to the \textbf{capitalization style} for each
type of reference.\\
Always use the standard abbreviation of a journal's name according to the ISSN \textit{List of Title Word Abbreviations}, see \url{http://www.issn.org/en/node/344}.\\ See examples below.


\begin{thebibliography}{99.}%
	%
	%
	% Use \bibitem to create references.
	%
	% Chapter in an edited book 
	\bibitem{science-contrib} Broy, M.: Software engineering --- from auxiliary to key technologies. In: Broy, M., Dener, E. (eds.) Software Pioneers, pp. 10-13. Springer, Heidelberg (2002)
	%
	% Online Document
	\bibitem{science-online} Dod, J.: Effective substances. In: The Dictionary of Substances and Their Effects. Royal Society of Chemistry (1999) Available via DIALOG. \\
	\url{http://www.rsc.org/dose/title of subordinate document. Cited 15 Jan 1999}
	%
	% Book/Monograph
	\bibitem{science-mono} Geddes, K.O., Czapor, S.R., Labahn, G.: Algorithms for Computer Algebra. Kluwer, Boston (1992) 
	%
	% Journal paper
	\bibitem{science-journal} Hamburger, C.: Quasimonotonicity, regularity and duality for nonlinear systems of partial differential equations. Ann. Mat. Pura. Appl. \textbf{169}, 321--354 (1995)
	%
	% Journal paper by DOI
	\bibitem{science-DOI} Slifka, M.K., Whitton, J.L.: Clinical implications of dysregulated cytokine production. J. Mol. Med. (2000) doi: 10.1007/s001090000086 
	
	
\end{thebibliography}


\end{document}
