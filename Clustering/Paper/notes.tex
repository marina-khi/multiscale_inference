\documentclass[a4paper,11pt]{article}
\usepackage{amsmath, bm}
\usepackage{amssymb,amsthm,graphicx}
\usepackage{enumitem}
\usepackage{color}
\usepackage{epsfig}
\usepackage{graphics}
\usepackage{pdfpages}
\usepackage{subcaption}
\usepackage[font=small]{caption}
\usepackage[hang,flushmargin]{footmisc} 
\usepackage{float}
\usepackage{rotating,tabularx}
\usepackage{booktabs}
\usepackage[mathscr]{euscript}
\usepackage{natbib}
\usepackage{setspace}
\usepackage{placeins}
\usepackage{ulem}
\usepackage[left=3cm,right=3cm,bottom=3cm,top=3cm]{geometry}
\numberwithin{equation}{section}
\allowdisplaybreaks[3]


% General

\newcommand{\reals}{\mathbb{R}}
\newcommand{\integers}{\mathbb{Z}}
\newcommand{\naturals}{\mathbb{N}}

\newcommand{\pr}{\mathbb{P}}        % probability
\newcommand{\ex}{\mathbb{E}}        % expectation
\newcommand{\var}{\textnormal{Var}} % variance
\newcommand{\cov}{\textnormal{Cov}} % covariance

\newcommand{\law}{\mathcal{L}} % law of X
\newcommand{\normal}{N}        % normal distribution 

\newcommand{\argmax}{\textnormal{argmax}}
\newcommand{\argmin}{\textnormal{argmin}}

\newcommand{\ind}{\mathbbm{1}} % indicator function
\newcommand{\kernel}{K} % kernel function
\newcommand{\wght}{W} % kernel weight
\newcommand{\thres}{\pi} % threshold parameter


% Convergence

\newcommand{\convd}{\stackrel{d}{\longrightarrow}}              % convergence in distribution
\newcommand{\convp}{\stackrel{P}{\longrightarrow}}              % convergence in probability
\newcommand{\convas}{\stackrel{\textrm{a.s.}}{\longrightarrow}} % convergence almost surely
\newcommand{\convw}{\rightsquigarrow}                           % weak convergence


% Theorem-like declarations

\theoremstyle{plain}

\newtheorem{theorem}{Theorem}[section]
\newtheorem{prop}[theorem]{Proposition}
\newtheorem{lemma}[theorem]{Lemma}
\newtheorem{corollary}[theorem]{Corollary}
\newtheorem*{theo}{Theorem}
\newtheorem{propA}{Proposition}[section]
\newtheorem{lemmaA}[propA]{Lemma}
\newtheorem{definition}{Definition}[section]
\newtheorem{remark}{Remark}[section]
\renewcommand{\thelemmaA}{A.\arabic{lemmaA}}
\renewcommand{\thepropA}{A.\arabic{propA}}
\newtheorem*{algo}{Clustering Algorithm}


% Theorem numbering to the left

\makeatletter
\newcommand{\lefteqno}{\let\veqno\@@leqno}
\makeatother


% Heading

\newcommand{\heading}[2]
{  \setcounter{page}{1}
   \begin{center}

   \phantom{Distance to upper boundary}
   \vspace{0.5cm}

   {\LARGE \textbf{#1}}
   \vspace{0.4cm}
 
   {\LARGE \textbf{#2}}
   \end{center}
}


% Authors

\newcommand{\authors}[4]
{  \parindent0pt
   \begin{center}
      \begin{minipage}[c][2cm][c]{5cm}
      \begin{center} 
      {\large #1} 
      \vspace{0.05cm}
      
      #2 
      \end{center}
      \end{minipage}
      \begin{minipage}[c][2cm][c]{5cm}
      \begin{center} 
      {\large #3}
      \vspace{0.05cm}

      #4 
      \end{center}
      \end{minipage}
   \end{center}
}

%\newcommand{\authors}[2]
%{  \parindent0pt
%   \begin{center}
%   {\large #1} 
%   \vspace{0.1cm}
%      
%   #2 
%   \end{center}  
%}


% Version

\newcommand{\version}[1]
{  \begin{center}
   {\large #1}
   \end{center}
   \vspace{3pt}
} 










\begin{document}



\heading{Clustering of the epidemic time trends:}{the case of COVID-19}

%\authors{Marina Khismatullina\renewcommand{\thefootnote}{1}\footnotemark[1]}{University of Bonn}{Michael Vogt\renewcommand{\thefootnote}{2}\footnotemark[2]}{Ulm University} 
%\footnotetext[1]{Corresponding author. Address: Bonn Graduate School of Economics, University of Bonn, 53113 Bonn, Germany. Email: \texttt{marina.k@uni-bonn.de}.}
%\renewcommand{\thefootnote}{2}
%\footnotetext[2]{Address: Institute of Statistics, Department of Mathematics and Economics, Ulm University, 89081 Ulm, Germany. Email: \texttt{m.vogt@uni-ulm.de}.}
%\renewcommand{\thefootnote}{\arabic{footnote}}
%\setcounter{footnote}{2}

%\vspace{-0.85cm}

%\renewcommand{\baselinestretch}{1.2}\normalsize

%\renewcommand{\abstractname}{}
%\begin{abstract}
%\noindent The COVID-19 pandemic is one of the most pressing issues at present. A question which is particularly important for governments and policy makers is the following: Does the virus spread in the same way in different countries? Or are there significant differences in the development of the epidemic? In this paper, we devise new inference methods that allow to detect differences in the development of the COVID-19 epidemic across countries in a statistically rigorous way. In our empirical study, we use the methods to compare the outbreak patterns of the epidemic in a number of European countries.
%\end{abstract}

%\noindent \textbf{Key words:} simultaneous hypothesis testing; multiscale test; time trend; panel data; COVID-19.

%\noindent \textbf{JEL classifications:} C12; C23; C54.

%\noindent \textbf{AMS 2010 subject classifications:} 62E20; 62G10; 62G15; 62G20.

\renewcommand{\baselinestretch}{1.5}\normalsize



%\section{Introduction}


%\section{Extensions}
We consider the following nonparametric regression equation:
\begin{equation*}
\X_{it} = c_i \lambda_i\Big(\frac{t}{T}\Big) + \varepsilon_{it} \quad \text{with} \quad \varepsilon_{it} = \sigma \sqrt{\lambda_i\Big(\frac{t}{T}\Big)} \eta_{it}, 
\end{equation*}
where $c_i$ is the country-specific scaling parameter that accounts for the size of the country or population density. We introduce this additional parameter in order to be able to compare countries that differ substantially in terms of the population, i.e. Luxembourg and Russia.  In what follows, we present a method that allows researchers to test the hypothesis that the time trends of new COVID-19 cases in different countries are the same up to some scaling parameter and to cluster the countries based on the differences.

For the identification purposes, we need to assume that for each $i \in \mathcal{C}$ we have $\int_0^1 \lambda_i(u)du = 1$. Only then we are able to estimate the scaling parameter $c_i$. Thus, the testing procedure is as follows.

\textit{Step 1}

First, we estimate the scaling parameter:
\begin{align*}
\widehat{c_i} &= \frac{1}{T}\sum_{t = 1}^T X_{it} \\
&= c_i \frac{1}{T}\sum_{t = 1}^T \lambda_i\Big(\frac{t}{T}\Big) + \sigma\frac{1}{T}\sum_{t = 1}^T \sqrt{\lambda_i\Big(\frac{t}{T}\Big)} \eta_{it}\\
& = c_i \frac{1}{T}\sum_{t = 1}^T \lambda_i\Big(\frac{t}{T}\Big) + o_P(1)\\
& = c_i + o_P(1),
\end{align*}
where in the last inequality we used the normalization $\int_0^1 \lambda_i(u)du = 1$. Hence, for any fixed $i \in \mathcal{C}$, $\widehat{c}_i$ is a consistent estimator of $c_i$.

\textit{Step 2}

Instead of working with $X_{it}$, we consider the following variables:
\begin{align*}
X^*_{it} &= \frac{X_{it}}{\frac{1}{T}\sum_{t = 1}^T X_{it}} \\
&= \frac{c_i}{\widehat{c}_i} \lambda_i \Big(\frac{t}{T}\Big) + \frac{\sigma}{\widehat{c}_i} \sqrt{\lambda_i\Big(\frac{t}{T}\Big)} \eta_{it}.
\end{align*}

A statistic to test the hypothesis $H_0^{(ijk)}$ for a given triple $(i,j,k)$ is then constructed as follows. We work with the following quantity
\[ \hat{s}_{ijk,T} = \frac{1}{\sqrt{Th_k}} \sum\limits_{t=1}^T \ind\Big(\frac{t}{T} \in \mathcal{I}_k\Big) (\X_{it}^* - \X_{jt}^*). \]
Then
\begin{align*}
\frac{\hat{s}_{ijk,T}}{\sqrt{Th_k}} =& \frac{1}{Th_k} \sum\limits_{t=1}^T \ind\Big(\frac{t}{T} \in \mathcal{I}_k\Big) (\X_{it}^* - \X_{jt}^*)\\
=& \frac{1}{Th_k} \sum\limits_{t=1}^T \ind\Big(\frac{t}{T} \in \mathcal{I}_k\Big) \bigg( \lambda_i \Big(\frac{t}{T}\Big)  - \lambda_j \Big(\frac{t}{T}\Big)\bigg) + R_1 + R_2,
\end{align*}
where
\begin{align*}
R_1 &= \frac{1}{Th_k} \sum\limits_{t=1}^T \ind\Big(\frac{t}{T} \in \mathcal{I}_k\Big) \bigg( \Big(\frac{c_i}{\widehat{c}_i} - 1 \Big) \lambda_i \Big(\frac{t}{T}\Big)  - \Big(\frac{c_j}{\widehat{c}_j} - 1 \Big) \lambda_j \Big(\frac{t}{T}\Big)\bigg),\\
R_2& =  \frac{1}{Th_k} \sum\limits_{t=1}^T \ind\Big(\frac{t}{T} \in \mathcal{I}_k\Big) \Big( \frac{\sigma}{\widehat{c}_i} \sqrt{\lambda_i\Big(\frac{t}{T}\Big)} \eta_{it} - \frac{\sigma}{\widehat{c}_j} \sqrt{\lambda_j\Big(\frac{t}{T}\Big)} \eta_{jt} \Big).
\end{align*}
Since $\widehat{c}_i = c_i + o_P(1)$ and $0 \leq  \sum\nolimits_{t=1}^T \ind\big(\frac{t}{T} \in \mathcal{I}_k\big) \lambda_i \big(\frac{t}{T}\big) \leq h_k \lambda_{max}$, we have
\begin{align}\label{eq:aux1}
|R_1| &\leq \Big|\frac{c_i}{\widehat{c}_i} - 1 \Big| \frac{1}{Th_k} \sum\limits_{t=1}^T \ind\Big(\frac{t}{T} \in \mathcal{I}_k\Big) \lambda_i \Big(\frac{t}{T}\Big)  + \Big|\frac{c_j}{\widehat{c}_j} - 1 \Big| \frac{1}{Th_k} \sum\limits_{t=1}^T \ind\Big(\frac{t}{T} \in \mathcal{I}_k\Big) \lambda_j \Big(\frac{t}{T}\Big),\nonumber \\
&\leq o_P(1) \cdot \frac{\lambda_{max}}{T} + o_P(1) \cdot \frac{\lambda_{max}}{T} = o_P\Big(\frac{1}{T}\Big).
\end{align}
Furthermore, applying the law of large numbers, we get:
\begin{align*}
 \frac{1}{Th_k} \sum\limits_{t=1}^T \ind\Big(\frac{t}{T} \in \mathcal{I}_k\Big) \sqrt{\lambda_i\Big(\frac{t}{T}\Big)} \eta_{it}  = o_P(1).
\end{align*}
Hence, if we uniformly bound the scaling parameters away from 0, i.e. $\exists \, c_{min}$ such that for all $i \in \mathcal{C}$ we have $0 < c_{min} \leq c_i$, we can use the fact that $\frac{\sigma}{\widehat{c}_i} = O_P(1)$ to get that
\begin{align}\label{eq:aux2}
R_2& =  \frac{\sigma}{\widehat{c}_i} \frac{1}{Th_k} \sum\limits_{t=1}^T \ind\Big(\frac{t}{T} \in \mathcal{I}_k\Big) \sqrt{\lambda_i\Big(\frac{t}{T}\Big)} \eta_{it} -\frac{\sigma}{\widehat{c}_j}\frac{1}{Th_k} \sum\limits_{t=1}^T \ind\Big(\frac{t}{T} \in \mathcal{I}_k\Big)  \sqrt{\lambda_j\Big(\frac{t}{T}\Big)} \eta_{jt}\nonumber\\
& = o_P(1).
\end{align}
Combining \eqref{eq:aux1} and \eqref{eq:aux2} together, we get $\hat{s}_{ijk,T}/\sqrt{Th_k} = (Th_k)^{-1} \sum_{t=1}^T \ind(t/T \in \mathcal{I}_k) \{\lambda_i(t/T) - \lambda_j(t/T)\} + o_p(1)$ for any fixed pair of countries $(i,j)$. Hence, the statistic $\hat{s}_{ijk,T}/\sqrt{Th_k}$ estimates the average distance between the functions $\lambda_i$ and $\lambda_j$ on the interval $\mathcal{I}_k$. The variance of $\hat{s}_{ijk,T}$ can not be easily calculated:
\begin{align*}
 \var(\hat{s}_{ijk,T})  =&\frac{1}{Th_k} \var \Big( \sum\limits_{t=1}^T \ind\Big(\frac{t}{T} \in \mathcal{I}_k\Big) (X_{it}^* - X_{jt}^*) \Big)\\
=&\frac{1}{Th_k} \var \Big( \sum\limits_{t=1}^T \ind\Big(\frac{t}{T} \in \mathcal{I}_k\Big) X_{it}^*\Big) + \frac{1}{Th_k} \var \Big( \sum\limits_{t=1}^T \ind\Big(\frac{t}{T} \in \mathcal{I}_k\Big) X_{jt}^*\Big)\\
 = &\frac{1}{Th_k} \var \bigg( \frac{\sum\nolimits_{t=1}^T \ind\big(\frac{t}{T} \in \mathcal{I}_k\big) X_{it}}{\frac{1}{T}\sum\nolimits_{t=1}^T X_{it}} \bigg) + \frac{1}{Th_k} \var \bigg( \frac{\sum\nolimits_{t=1}^T \ind\big(\frac{t}{T} \in \mathcal{I}_k\big) X_{jt}}{\frac{1}{T}\sum\nolimits_{t=1}^T X_{jt}} \bigg),
\end{align*}
hence, we 'normalize' $\hat{s}_{ijk,T}$ intuitively by dividing it by the following value:
\[ \hat{\nu}_{ijk,T}^2 = \frac{\hat{\sigma}^2}{Th_k} \sum\limits_{t=1}^T \ind\Big(\frac{t}{T} \in \mathcal{I}_k\Big) \{ \X^*_{it} + \X^*_{jt} \}.\]

{\color{red} Instead look at the bootstrap!}

Normalizing the statistic $\hat{s}_{ijk,T}$ by the estimator $\hat{\nu}_{ijk,T}$ yields the expression 
\begin{equation*}
\hat{\psi}_{ijk,T} := \frac{\hat{s}_{ijk,T}}{\hat{\nu}_{ijk,T}} = \frac{\sum\nolimits_{t=1}^T \ind(\frac{t}{T} \in \mathcal{I}_k) (\X^*_{it} - \X^*_{jt})}{ \hat{\sigma}\{ \sum\nolimits_{t=1}^T \ind(\frac{t}{T} \in \mathcal{I}_k) (\X^*_{it} + \X^*_{jt}) \}^{1/2}}, 
\end{equation*}
which serves as our test statistic of the hypothesis $H_0^{(ijk)}$. For later reference, we additionally introduce the statistic 
\begin{equation}\label{eq:stat0}
\hat{\psi}_{ijk,T}^{ 0} = \frac{\sum\nolimits_{t=1}^T \ind(\frac{t}{T} \in \mathcal{I}_k) \Big(\big( \frac{c_i}{\hat{c}_i} - \frac{c_j}{\hat{c}_j} \big)\overline{\lambda}_{ij} + \overline{\lambda}_{ij}^{1/2}(\frac{t}{T}) (\frac{\sigma}{\hat{c}_i} \eta_{it} - \frac{\sigma}{\hat{c}_j} \eta_{jt}) \Big)}{ \hat{\sigma} \{ \sum\nolimits_{t=1}^T \ind(\frac{t}{T} \in \mathcal{I}_k) (\X^*_{it} + \X^*_{jt}) \}^{1/2}}
\end{equation}
with $\overline{\lambda}_{ij}(u) = \{ \lambda_i(u) + \lambda_j(u) \}/2$, which is identical to $\hat{\psi}_{ijk,T}$ under $H_0^{(ijk)}$. 


%where $(\hat{\sigma}^*)^2$ is defined as follows: For each country $i$, let 
%\begin{align*}
%(\hat{\sigma}_i^*)^2 = \frac{\sum_{t=2}^T (\X^*_{it}-\X^*_{it-1})^2}{2 \sum_{t=1}^T \X^*_{it}} = \frac{\sum_{t=2}^T (\frac{\X_{it}}{\widehat{c}_i}-\frac{\X_{it-1}}{\widehat{c}_i})^2}{2 \sum_{t=1}^T \frac{\X_{it}}{\widehat{c}_i}} = \frac{\widehat{\sigma}_i^2}{\widehat{c}_i}
%\end{align*}
%and set $(\widehat{\sigma}^*)^2 = |\countries|^{-1} \sum_{i \in \countries} (\widehat{\sigma}^*_i)^2$. We have already shown that $\widehat{\sigma}_i^2 = \sigma^2 + o_p(1)$ and $\hat{c}_i = c_i + o_P(1)$ for any $i$ and thus $(\widehat{\sigma}_i^*)^2 = \frac{\sigma^2}{c_i} + o_p(1)$.


\subsection{Construction of the test} 


Our multiscale test is carried out as follows: For a given significance level $\alpha \in (0,1)$ and each $(i,j,k) \in \indexset$, we reject $H_0^{(ijk)}$ if 
\[ |\hat{\psi}_{ijk,T}| > c_{ijk,T}(\alpha), \]
where $c_{ijk,T}(\alpha)$ is the critical value for the $(i,j,k)$-th test problem. The critical values $c_{ijk,T}(\alpha)$ are chosen such that the familywise error rate (FWER) is controlled at level $\alpha$, which is defined as the probability of wrongly rejecting $H_0^{(ijk)}$ for at least one $(i,j,k)$. More formally speaking, for a given significance level $\alpha \in (0,1)$, the FWER is 
\begin{align*}
\text{FWER}(\alpha) 
 & = \pr \Big( \exists (i,j,k) \in \indexset_0: |\hat{\psi}_{ijk,T}| > c_{ijk,T}(\alpha) \Big) \\
 & =  1 - \pr \Big( \forall (i,j,k) \in \indexset_0: |\hat{\psi}_{ijk,T}| \le c_{ijk,T}(\alpha) \Big) \\
 & = 1 - \pr\Big( \max_{(i,j,k) \in \indexset_0} |\hat{\psi}_{ijk,T}| \le c_{ijk,T}(\alpha) \Big), 
\end{align*}
where $\indexset_0 \subseteq \indexset$ is the set of triples $(i,j,k)$ for which $H_0^{(ijk)}$ holds true. As before, the critical values are chosen as
\begin{equation*}
c_{ijk,T}(\alpha) = c_T(\alpha,h_k) := b_k + q_T(\alpha)/a_k, 
\end{equation*}
where $a_k = \{\log(e/h_k)\}^{1/2} / \log \log(e^e / h_k)$ and $b_k = \sqrt{2 \log(1/h_k)}$ are scale-dependent constants and the quantity $q_T(\alpha)$ is determined by the following consideration: Since 
\begin{align*}
\text{FWER}(\alpha)
  & = \pr \Big( \exists (i,j,k) \in \indexset_0: |\hat{\psi}_{ijk,T}| > c_T(\alpha,h_k) \Big)  \\
 & =  1 - \pr \Big( \forall (i,j,k) \in \indexset_0: |\hat{\psi}_{ijk,T}| \le c_T(\alpha,h_k) \Big) \\
 & =  1 - \pr \Big( \forall (i,j,k) \in \indexset_0: a_k \big(|\hat{\psi}_{ijk,T}| - b_k\big) \le q_T(\alpha) \Big) \\
 & = 1 - \pr\Big( \max_{(i,j,k) \in \indexset_0} a_k \big( |\hat{\psi}_{ijk,T}| - b_k \big) \le q_T(\alpha) \Big),
\end{align*}
we need to choose the quantity $q_T(\alpha)$ as the $(1-\alpha)$-quantile of the statistic 
\[ \hat{\Psi}_T = \max_{(i,j,k) \in \indexset} a_k \big( |\hat{\psi}_{ijk,T}^{0}| - b_k \big) \]
in order to ensure control of the FWER at level $\alpha$. As the quantiles $q_T(\alpha)$ are not known in practice, we cannot compute the critical values $c_T(\alpha,h_k)$ exactly in practice but need to approximate them. This can be achieved as follows: Under appropriate regularity conditions, it can be shown that \textcolor{red}{???}
\begin{align*}
\hat{\psi}_{ijk,T}^{0} &= \frac{\sum\nolimits_{t=1}^T \ind(\frac{t}{T} \in \mathcal{I}_k) \Big(\big( \frac{c_i}{\hat{c}_i} - \frac{c_j}{\hat{c}_j} \big)\overline{\lambda}_{ij} + \overline{\lambda}_{ij}^{1/2}(\frac{t}{T}) (\frac{\sigma}{\hat{c}_i} \eta_{it} - \frac{\sigma}{\hat{c}_j} \eta_{jt}) \Big)}{ \hat{\sigma} \{ \sum\nolimits_{t=1}^T \ind(\frac{t}{T} \in \mathcal{I}_k) (\X^*_{it} + \X^*_{jt}) \}^{1/2}}\\
 & \approx \frac{1}{\sqrt{2Th_k}} \sum\limits_{t=1}^T \ind\Big(\frac{t}{T} \in \mathcal{I}_k\Big) \Big\{ \frac{\eta_{it}}{\hat{c}_i} - \frac{\eta_{jt}}{\hat{c}_j} \Big\}.
\end{align*} 
In what follows, we will be using the Gaussian version $\phi_{ijk,T}$ of the statistic displayed above:
\begin{align*}
\phi_{ijk,T} = \frac{1}{\sqrt{2Th_k}} \sum\limits_{t=1}^T \ind\Big(\frac{t}{T} \in \mathcal{I}_k\Big) ( d_{i, T} Z_{it} - d_{j, T} Z_{jt} ),
\end{align*} 
where $Z_{it}$ are independent standard normal random variables for $1 \le t \le T$ and $1 \le i \le n$. However, since the variance of $\hat{\psi}_{ijk,T}^{0}$ is not easy to calculate and we need the variances of $\hat{\psi}_{ijk,T}^{0}$ and $\phi_{ijk,T}$ to be equal, we can not provide the exact formula for $d_{i,T}$. We regard $\phi_{ijk,T}$ as an auxiliary test statistic with unknown distribution, which will be then approximated by bootstrap. For now, the statistic
\[ \Phi_T = \max_{(i,j,k) \in \indexset} a_k \big( |\phi_{ijk,T}| - b_k \big) \]
can be regarded as a Gaussian version of the statistic $\hat{\Psi}_T$. Further in this section, we will show how that the critical values of $\Phi_T$ and the unknown quantiles $q_T(\alpha)$ can be approximated using a multiplier bootstrap by the $\Psi_{\text{bootstrap}, T}$ and its respective $(1-\alpha)$-quantile $\widehat{q}_{T,\text{bootstrap}}(\alpha)$.

To summarize, we propose the following procedure to simultaneously test the hypothesis $H_0^{(ijk)}$ for all $(i,j,k) \in \indexset$ at the significance level $\alpha \in (0,1)$: 
\begin{equation}\label{eq:test}
\text{For each } (i,j,k) \in \indexset, \text{ reject } H_0^{(ijk)} \text{ if } |\hat{\psi}_{ijk,T}| > c_{T,\text{bootstrap}}(\alpha,h_k),
\end{equation}
where $c_{T,\text{bootstrap}}(\alpha,h_k) = b_k + \widehat{q}_{T,\text{bootstrap}}(\alpha)/a_k$ with $a_k = \{\log(e/h_k)\}^{1/2} / \log \log(e^e / h_k)$ and $b_k = \sqrt{2 \log(1/h_k)}$. 


\subsection{Proof strategy}

Here we outline the proof strategy, which can be divided into several major steps.

\begin{enumerate}[label=\textit{Step \arabic*.}, leftmargin=0cm, itemindent=1.45cm]

\item Let $\hat{\Psi}_T = \max_{(i,j,k) \in \indexset} a_k (|\hat{\psi}_{ijk,T}^0| - b_k)$ with $\hat{\psi}_{ijk,T}^0$ introduced in \eqref{eq:stat0}
and define $\Psi_T = \max_{(i,j,k) \in \indexset} a_k (|\psi_{ijk,T}^0| - b_k)$ with 
\[ \psi_{ijk,T}^0 = \frac{1}{\sqrt{2Th_k}} \sum\limits_{t=1}^T \ind\Big(\frac{t}{T} \in \mathcal{I}_k\Big) \Big(\frac{\eta_{it}}{\hat{c}_i} - \frac{\eta_{jt}}{\hat{c}_j}\Big). \]
To start with, we prove that  
\begin{equation}\label{eq:approxerror1}
\big| \hat{\Psi}_T - \Psi_T \big| = o_p(r_T),
\end{equation}
where $\{r_T\}$ is any null sequence that converges more slowly to zero than {\color{red} ???

Need to prove that. It is possible that the exact formula of $\Psi_T$ will change.}
%$\rho_T = \sqrt{\log T} \{ \log p/\sqrt{Th_{\min}} + h_{\max} \sqrt{\log p} \}$, that is, $\rho_T/r_T \rightarrow 0$ as $T \rightarrow \infty$. Since the proof of \eqref{eq:approxerror1} is rather technical and lengthy, the details are provided in the Supplementary Material. 


\item We next prove that 
\begin{equation}\label{eq:kolmogorov-distance}
\sup_{q \in \reals} \Big| \pr \big( \Psi_T \le q \big) - \pr \big( \Phi_T \le q \big) \Big| = o(1).
\end{equation}
To do so, we rewrite the statistics $\Psi_T$ and $\Phi_T$ as follows: Define 
\begin{equation*}
V^{(ijk)}_t = V^{(ijk)}_{t,T} := \sqrt{\frac{T}{2Th_k}} \ind\Big(\frac{t}{T} \in \mathcal{I}_k\Big) \Big(\frac{\eta_{it}}{\hat{c}_i} - \frac{\eta_{jt}}{\hat{c}_j}\Big)
\end{equation*}
for $(i,j,k) \in \indexset$ and let $\boldsymbol{V}_t = (V_t^{(ijk)}: (i,j,k) \in \indexset)$ be the $p$-dimensional random vector with the entries $V_t^{(ijk)}$. With this notation, we get that $\psi_{ijk,T}^0 = T^{-1/2} \sum_{t=1}^T V^{(ijk)}_t$ and thus 
\begin{align*}
\Psi_T 
 & = \max_{(i,j,k) \in \indexset}  a_k \big( |\psi_{ijk,T}^0| - b_k \big) \\
 & = \max_{(i,j,k) \in \indexset} a_k \Big\{ \Big|\frac{1}{\sqrt{T}} \sum_{t=1}^T V^{(ijk)}_t\Big| - b_k \Big\}.
\end{align*} 
Analogously, we define 
\begin{equation*}
W^{(ijk)}_t = W^{(ijk)}_{t,T} := \sqrt{\frac{T}{2Th_k}} \ind\Big(\frac{t}{T} \in \mathcal{I}_k\Big) (d_{i, T}Z_{it} - d_{j, T}Z_{jt})
\end{equation*}
with $Z_{it}$ i.i.d.\ standard normal and let $\boldsymbol{W}_t = (W_t^{(ijk)}: (i,j,k) \in \indexset)$. The vector $\boldsymbol{W}_t$ is a Gaussian version of $\boldsymbol{V}_t$ with the same mean and variance. In particular, $\ex[\boldsymbol{W}_t] = \ex[\boldsymbol{V}_t] = 0$ and $\ex[\boldsymbol{W}_t \boldsymbol{W}_t^\top] = \ex[\boldsymbol{V}_t \boldsymbol{V}_t^\top]$ {\color{red}(this we achieve by introducing $d_{i, T}$)}. Similarly as before, we can write $\phi_{ijk,T} = T^{-1/2} \sum_{t=1}^T W^{(ijk)}_t$ and  
\begin{align*}
\Phi_T 
 & = \max_{(i,j,k) \in \indexset} a_k \big( |\phi_{ijk,T}| - b_k \big) \\
 & = \max_{(i,j,k) \in \indexset} a_k \Big\{ \Big|\frac{1}{\sqrt{T}} \sum_{t=1}^T W^{(ijk)}_t\Big| - b_k \Big\}.
\end{align*} 
For any $q \in \reals$, it holds that
\begin{align*}
\pr \big( \Psi_T \le q \big) 
 & = \pr \Big( \max_{(i,j,k) \in \indexset} a_k \Big\{ \Big|\frac{1}{\sqrt{T}} \sum_{t=1}^T V^{(ijk)}_t\Big| - b_k \Big\} \le q \Big) \\
 & = \pr \Big( \Big|\frac{1}{\sqrt{T}} \sum_{t=1}^T V^{(ijk)}_t\Big| \le c_{ijk}(q) \text{ for all } (i,j,k) \in \indexset \Big) \\
 & = \pr \Big( \Big|\frac{1}{\sqrt{T}} \sum_{t=1}^T \boldsymbol{V}_t\Big| \le \boldsymbol{c}(q) \Big),
\end{align*} 
where $\boldsymbol{c}(q) = (c_{ijk}(q): (i,j,k) \in \indexset)$ is the $\reals^p$-vector with the entries $c_{ijk}(q) = q/a_k + b_k$, we use the notation $|v| = (|v_1|,\ldots,|v_p|)^\top$ for vectors $v \in \reals^p$ and the inequality $v \le w$ is to be understood componentwise for $v,w \in \reals^p$. Analogously, we have  
\[ \pr \big( \Phi_T \le q \big) = \pr \Big( \Big|\frac{1}{\sqrt{T}} \sum_{t=1}^T \boldsymbol{W}_t\Big| \le \boldsymbol{c}(q) \Big). \]
With this notation at hand, we can make use of Proposition 2.1 from \cite{Chernozhukov2017}. In our context, this proposition can be stated as follows: 
\begin{propA}\label{prop:Chernozhukov}
Assume that 
\begin{enumerate}[label=(\alph*),leftmargin=0.7cm]
\item $T^{-1} \sum_{t=1}^T \ex (V^{(ijk)}_t)^2 \ge \delta > 0$ for all $(i,j,k) \in \indexset$.
\item $T^{-1} \sum_{t=1}^T \ex[ |V^{(ijk)}_t|^{2+r} ] \le B_T^r$ for all $(i,j,k) \in \indexset$ and $r=1,2$, where $B_T \ge 1$ are constants that may tend to infinity as $T \rightarrow \infty$.  
\item $\ex[ \{ \max_{(i,j,k) \in \indexset} |V^{(ijk)}_t| / B_T \}^\theta ] \le 2$ for all $t$ and some $\theta > 4$.  
\end{enumerate}
Then  
\begin{align}
\sup_{\boldsymbol{c} \in \reals^p} \Big| \pr \Big( \Big|\frac{1}{\sqrt{T}} \sum_{t=1}^T \boldsymbol{V}_t\Big| \le \boldsymbol{c} \Big) & - \pr \Big( \Big|\frac{1}{\sqrt{T}} \sum_{t=1}^T \boldsymbol{W}_t\Big| \le \boldsymbol{c} \Big) \Big| \nonumber \\ & \le C \Big\{ \Big( \frac{B_T^2 \log^7(pT)}{T} \Big)^{1/6} + \Big( \frac{B_T^2 \log^3(pT)}{T^{1-2/\theta}} \Big)^{1/3} \Big\}, \label{eq:Chernozhukov}
\end{align}
where $C$ depends only on $\delta$ and $\theta$. 
\end{propA}
{\color{red} We need to check that with our choice of the test statstics, the assumptions (a)--(c) are satisfied. For which $B_T$?}
% under the conditions of Theorem \ref{theo1} for sufficiently large $T$, where $B_T$ can be chosen as $B_T = C p^{1/\theta} h_{\min}^{-1/2}$ with $C$ sufficiently large. Moreover, it can be shown that the right-hand side of \eqref{eq:Chernozhukov} is $o(1)$ for this choice of $B_T$.

Hence, Proposition \ref{prop:Chernozhukov} yields that 
\[ \sup_{\boldsymbol{c} \in \reals^p} \Big| \pr \Big( \Big|\frac{1}{\sqrt{T}} \sum_{t=1}^T \boldsymbol{V}_t\Big| \le \boldsymbol{c} \Big) - \pr \Big( \Big|\frac{1}{\sqrt{T}} \sum_{t=1}^T \boldsymbol{W}_t\Big| \le \boldsymbol{c} \Big) \Big| = o(1), \]
which in turn implies \eqref{eq:kolmogorov-distance}. 

\item Now we have the problem that $d_{i, T}$ are unknown. This means that the quantile $q_T(\alpha)$ of $\Phi_T$ are not known and can not be approximated by usual Monte Carlo simulations. We need to find another way of approximating them, for example, multiplier bootstrap from \cite{Chernozhukov2017}.

We next prove that for $\alpha \in (0, e^{-1})$ we can claim with probability $1- \alpha$ that
\begin{equation}\label{eq:kolmogorov-distance:2}
\sup_{q \in \reals} \Big| \pr \big( \tilde{\Psi}_T \le q \big| X_{it}) - \pr \big( \Phi_T \le q \big) \Big| = o(1),
\end{equation}

where $\tilde{\Psi}_T$ will be defined further.


%\[ \sup_{\boldsymbol{c} \in \reals^p} \Big| \pr \Big( \Big| \frac{1}{\sqrt{T}} \sum_{t=1}^T e_t (\boldsymbol{V}_t - \bar{\boldsymbol{V}_t}) \Big| \leq \boldsymbol{c}(q) \Big| \big\{V_t^{(ijk)}\big\} \Big) - \pr \Big( \Big|\frac{1}{\sqrt{T}} \sum_{t=1}^T \boldsymbol{W}_t\Big| \le \boldsymbol{c} \Big) \Big| = o(1), \]



We have already rewritten the statistics $\Phi_T$ as follows: 
\begin{align*}
\Phi_T 
 & = \max_{(i,j,k) \in \indexset} a_k \big( |\phi_{ijk,T}| - b_k \big) \\
 & = \max_{(i,j,k) \in \indexset} a_k \Big\{ \Big|\frac{1}{\sqrt{T}} \sum_{t=1}^T W^{(ijk)}_t\Big| - b_k \Big\}
\end{align*} 
and for any $q \in \reals$, it holds that
\begin{align*}
\pr \big( \Phi_T \le q \big)  = \pr \Big( \Big|\frac{1}{\sqrt{T}} \sum_{t=1}^T \boldsymbol{W}_t\Big| \le \boldsymbol{c}(q) \Big),
\end{align*} 
where $\boldsymbol{c}(q) = (c_{ijk}(q): (i,j,k) \in \indexset)$ is the $\reals^p$-vector with the entries $c_{ijk}(q) = q/a_k + b_k$. 

In order to approximate the unknown distribution of $\Phi_T$ by the multiplier bootstrap, we introduce auxiliary test statistics
\begin{align*}
\tilde{\Psi}_T  & = \max_{(i,j,k) \in \indexset} a_k \big( |\tilde{\psi}_{ijk,T}| - b_k \big) %\\
% & = \max_{(i,j,k) \in \indexset} a_k \Big\{ \Big|\frac{1}{\sqrt{T}} \sum_{t=1}^T W^{(ijk)}_t\Big| - b_k \Big\}.
\end{align*} 
and 
\[ \tilde{\psi}_{ijk,T} = \frac{1}{\sqrt{2Th_k}} \sum\limits_{t=1}^T \ind\Big(\frac{t}{T} \in \mathcal{I}_k\Big) e_t \Bigg( \Big(\frac{\eta_{it}}{\hat{c}_i} - \frac{\eta_{jt}}{\hat{c}_j}\Big) - \frac{1}{T} \sum\limits_{t=1}^T \ind\Big(\frac{t}{T} \in \mathcal{I}_k\Big) \Big(\frac{\eta_{it}}{\hat{c}_i} - \frac{\eta_{jt}}{\hat{c}_j}\Big) \Bigg). \]
with $e_1, \ldots, e_T$ being i.i.d. standard normal random variables independent of $\{Z_{it}| i, t\}$ and $\{X_{it} | i, t\}$. We will use the conditional distribution of $\tilde{\Psi}_T$ given the data as approximation of the distribution of $\Phi_T$.

We proceed as follows. As before, 
\begin{equation*}
V^{(ijk)}_t = V^{(ijk)}_{t,T} := \sqrt{\frac{T}{2Th_k}} \ind\Big(\frac{t}{T} \in \mathcal{I}_k\Big)\Big\{ \frac{\eta_{it}}{\hat{c}_i} - \frac{\eta_{jt}}{\hat{c}_j} \Big\}
\end{equation*}
for $(i,j,k) \in \indexset$ and let $\boldsymbol{V}_t = (V_t^{(ijk)}: (i,j,k) \in \indexset)$ be the $p$-dimensional random vector with the entries $V_t^{(ijk)}$. Additionally, define 
\begin{equation*}
\bar{V}^{(ijk)} = \bar{V}^{(ijk)}_{T} := \frac{1}{T} \sum_{t=1}^T \sqrt{\frac{T}{2Th_k}} \ind\Big(\frac{t}{T} \in \mathcal{I}_k\Big)\Big\{ \frac{\eta_{it}}{\hat{c}_i} - \frac{\eta_{jt}}{\hat{c}_j} \Big\}
\end{equation*}
and let $\bar{\boldsymbol{V}} = (\bar{V}^{(ijk)}: (i,j,k) \in \indexset)$ be the $p$-dimensional random vector with the entries $\bar{V}^{(ijk)}$. We consider the following conditional probability:
\begin{align*}
\pr &\Big( \Big| \frac{1}{\sqrt{T}} \sum_{t=1}^T e_t (\boldsymbol{V}_t - \bar{\boldsymbol{V}}) \Big| \leq \boldsymbol{c}(q) \Big| \big\{V_t^{(ijk)}\big\} \Big) = \\
 & = \pr \Big( \Big|\frac{1}{\sqrt{T}} \sum_{t=1}^T e_t(V^{(ijk)}_t - \bar{V}^{(ijk)} )\Big| \le c_{ijk}(q) \text{ for all } (i,j,k) \in \indexset \Big| \big\{V_t^{(ijk)}\big\}\Big) \\
 & = \pr \Big( \max_{(i,j,k) \in \indexset} a_k \Big\{ \Big|\frac{1}{\sqrt{T}} \sum_{t=1}^Te_t(V^{(ijk)}_t - \bar{V}^{(ijk)} )\Big| - b_k \Big\} \le q\Big| \big\{V_t^{(ijk)}\big\} \Big)\\
 & = \pr \Big( \max_{(i,j,k) \in \indexset} a_k \Big( \big|\tilde{\psi}_{ijk, T} \big| - b_k \Big) \le q\Big| \big\{X_{it}\big\} \Big) \\
  & = \pr \Big( \tilde{\Psi}_T \le q\Big| \big\{X_{it}\big\} \Big) 
\end{align*}


{\color{red} We have already checked in the previous step that  $\ex[\boldsymbol{W}_t] = \ex[\boldsymbol{V}_t] = 0$ and $\ex[\boldsymbol{W}_t \boldsymbol{W}_t^\top] = \ex[\boldsymbol{V}_t \boldsymbol{V}_t^\top]$.}

With this notation at hand, we can make use of Corollary 4.2 from \cite{Chernozhukov2017}. In our context, this proposition can be stated as follows: 
\begin{propA}\label{prop:Chernozhukov}
Let $\alpha\in (0, e^{-1})$ be a constant and assume that 
\begin{enumerate}[label=(\alph*),leftmargin=0.7cm]
\item $T^{-1} \sum_{t=1}^T \ex (V^{(ijk)}_t)^2 \ge \delta > 0$ for all $(i,j,k) \in \indexset$.
\item $T^{-1} \sum_{t=1}^T \ex[ |V^{(ijk)}_t|^{2+r} ] \le B_T^r$ for all $(i,j,k) \in \indexset$ and $r=1,2$, where $B_T \ge 1$ are constants that may tend to infinity as $T \rightarrow \infty$.  
\item $\ex[ \{ \max_{(i,j,k) \in \indexset} |V^{(ijk)}_t| / B_T \}^\theta ] \le 2$ for all $t$ and some $\theta > 4$.  
\end{enumerate}
Then  we have with probability at least $1 - \alpha$,
\begin{align}
\sup_{\boldsymbol{c} \in \reals^p} \Big| \pr &\Big( \Big| \frac{1}{\sqrt{T}} \sum_{t=1}^T e_t (\boldsymbol{V}_t - \bar{\boldsymbol{V}_t}) \Big| \leq \boldsymbol{c}(q) \Big| \big\{V_t^{(ijk)}\big\} \Big)  - \pr \Big( \Big|\frac{1}{\sqrt{T}} \sum_{t=1}^T \boldsymbol{W}_t\Big| \le \boldsymbol{c} \Big) \Big| \nonumber \\ & \le C \Big\{ \Big( \frac{B_T^2 \log^5(pT)\log^2(1/\alpha)}{T} \Big)^{1/6} + \Big( \frac{B_T^2 \log^3(pT)}{\alpha^{2/\theta}T^{1-2/\theta}} \Big)^{1/3} \Big\}, \label{eq:Chernozhukov}
\end{align}
where $C$ depends only on $\delta$ and $\theta$. 
\end{propA}
{\color{red}
Again, assumptions (a)--(c) are satisfied if we checked that in the previous step. We just need to determine the rate of $B_T$ such that the RHS is $o(1)$. Hence, Proposition \ref{prop:Chernozhukov} yields that with probability at least $1- \alpha$
\[ \sup_{\boldsymbol{c} \in \reals^p} \Big| \pr \Big( \Big| \frac{1}{\sqrt{T}} \sum_{t=1}^T e_t (\boldsymbol{V}_t - \bar{\boldsymbol{V}_t}) \Big| \leq \boldsymbol{c}(q) \Big| \big\{V_t^{(ijk)}\big\} \Big) - \pr \Big( \Big|\frac{1}{\sqrt{T}} \sum_{t=1}^T \boldsymbol{W}_t\Big| \le \boldsymbol{c} \Big) \Big| = o(1), \]
which in turn implies \eqref{eq:kolmogorov-distance:2}. 
}


\item {\color{red} Need to rewrite that.

With the help of \eqref{eq:approxerror1} and \eqref{eq:kolmogorov-distance}, we now show that 
\begin{equation}\label{eq:kolmogorov-distance-hat}
\sup_{q \in \reals} \Big| \pr \big( \hat{\Psi}_T \le q \big) - \pr \big( \Phi_T \le q \big) \Big| = o(1).
\end{equation}
To start with, the above supremum can be bounded by 
\begin{align}
 & \sup_{q \in \reals} \Big| \pr \big( \hat{\Psi}_T \le q \big) - \pr \big( \Phi_T \le q \big) \Big| \nonumber \\
 & = \sup_{q \in \reals} \Big| \pr \Big( \Psi_T \le q + \big\{ \Psi_T - \hat{\Psi}_T \big\} \Big) - \pr \big( \Phi_T \le q \big) \Big| \nonumber \\
 & \le \sup_{q \in \reals} \max \Big\{ \Big| \pr \Big( \Psi_T \le q + \big| \Psi_T - \hat{\Psi}_T \big| \Big) - \pr \big( \Phi_T \le q \big) \Big|, \nonumber \\
 & \phantom{\le \sup_{q \in \reals} \max \Big\{ \ } \Big| \pr \Big( \Psi_T \le q - \big| \Psi_T - \hat{\Psi}_T \big| \Big) - \pr \big( \Phi_T \le q \big) \Big| \Big\} \nonumber \\
 & \le \sup_{q \in \reals} \max \Big\{ \Big| \pr \Big( \Psi_T \le q + r_T \Big) - \pr \big( \Phi_T \le q \big) \Big| + \pr \Big( \big| \Psi_T - \hat{\Psi}_T \big| > r_T \Big), \nonumber \\
 & \phantom{\le \sup_{q \in \reals} \max \Big\{ \ } \Big| \pr \Big( \Psi_T \le q - r_T \Big) - \pr \big( \Phi_T \le q \big) \Big| + \pr \Big( \big| \Psi_T - \hat{\Psi}_T \big| > r_T \Big) \Big\} \nonumber \\
 & \le \max_{\ell=0,1} \, \sup_{q \in \reals} \Big| \pr \Big( \Psi_T \le q + (-1)^\ell r_T \Big) - \pr \big( \Phi_T \le q \big) \Big| + \pr \Big( \big| \Psi_T - \hat{\Psi}_T \big| > r_T \Big) \nonumber \\
 & = \max_{\ell=0,1} \, \sup_{q \in \reals} \Big| \pr \Big( \Psi_T \le q + (-1)^\ell r_T \Big) - \pr \big( \Phi_T \le q \big) \Big| + o(1), \label{eq:step3:a}
\end{align}
where the last line is by \eqref{eq:approxerror1}. Moreover, for $\ell=0,1$, 
\begin{align}
 & \sup_{q \in \reals} \Big| \pr \Big( \Psi_T \le q + (-1)^\ell r_T \Big) - \pr \big( \Phi_T \le q \big) \Big| \nonumber \\
 & \le \sup_{q \in \reals} \Big| \pr \Big( \Psi_T \le q + (-1)^\ell r_T \Big) - \pr \Big( \Phi_T \le q + (-1)^\ell r_T \Big) \Big| \nonumber \\
 & \quad + \sup_{q \in \reals} \Big| \pr \Big( \Phi_T \le q + (-1)^\ell r_T \Big) - \pr \big( \Phi_T \le q \big) \Big| \nonumber \\
 & = \sup_{q \in \reals} \Big| \pr \Big( \Phi_T \le q + (-1)^\ell r_T \Big) - \pr \big( \Phi_T \le q \big) \Big| + o(1), \label{eq:step3:b}
\end{align}
the last line following from \eqref{eq:kolmogorov-distance}. Finally, by Nazarov's inequality (\citeauthor{Nazarov2003}, \citeyear{Nazarov2003} and Lemma A.1 in \citeauthor{Chernozhukov2017}, \citeyear{Chernozhukov2017}), we have that for $\ell = 0,1$,   
\begin{align} 
 & \sup_{q \in \reals} \Big| \pr \Big( \Phi_T \le q + (-1)^\ell r_T \Big) - \pr \big( \Phi_T \le q \big) \Big| \nonumber \\
 & = \sup_{q \in \reals} \Big| \pr \Big( \Big|\frac{1}{\sqrt{T}} \sum_{t=1}^T \boldsymbol{W}_t\Big| \le \boldsymbol{c}(q + (-1)^\ell r_T) \Big) - \pr \Big( \Big|\frac{1}{\sqrt{T}} \sum_{t=1}^T \boldsymbol{W}_t\Big| \le \boldsymbol{c}(q) \Big) \Big| \nonumber \\
 & \le C \frac{r_T \sqrt{\log(2 p)}}{\min_{1 \le k \le K} a_k} \le C r_T \sqrt{\log \log T} \sqrt{\log (2p)}, \label{eq:step3:c}
\end{align}
where $C$ is a constant that depends only on the parameter $\delta$ defined in condition (a) of Proposition \ref{prop:Chernozhukov} and we have used the fact that $\min_k a_k \ge c / \sqrt{\log \log T}$ for some $c > 0$. Inserting \eqref{eq:step3:b} and \eqref{eq:step3:c} into equation \eqref{eq:step3:a} completes the proof of \eqref{eq:kolmogorov-distance-hat}.


\item By definition of the quantile $q_{T,\text{Gauss}}(\alpha)$, it holds that $\pr(\Phi_T \le q_{T,\text{Gauss}}(\alpha)) \ge 1-\alpha$. As shown in the Supplementary Material, we even have that  
\begin{equation}\label{eq:quant-exact}
\pr(\Phi_T \le q_{T,\text{Gauss}}(\alpha)) = 1-\alpha
\end {equation} 
for any $\alpha \in (0,1)$. From this and \eqref{eq:kolmogorov-distance-hat}, it immediately follows that  
\begin{equation}\label{eq:probbound-Psihat}
\pr \big( \hat{\Psi}_T \le q_{T,\text{Gauss}}(\alpha) \big) = 1 - \alpha + o(1), 
\end{equation}
which in turn implies that 
\begin{align*}
\text{FWER}(\alpha)
 & = \pr \Big( \exists (i,j,k) \in \indexset_0: |\hat{\psi}_{ijk,T}| > c_{T,\text{Gauss}}(\alpha,h_k) \Big) \\
 & = \pr \Big( \max_{(i,j,k) \in \indexset_0} a_k \big( |\hat{\psi}_{ijk,T}| - b_k \big) > q_{T,\text{Gauss}}(\alpha) \Big) \\
 & = \pr \Big( \max_{(i,j,k) \in \indexset_0} a_k \big( |\hat{\psi}_{ijk,T}^0| - b_k \big) > q_{T,\text{Gauss}}(\alpha) \Big) \\
 & \le \pr \Big( \max_{(i,j,k) \in \indexset} a_k \big( |\hat{\psi}_{ijk,T}^0| - b_k \big) > q_{T,\text{Gauss}}(\alpha) \Big) \\
 & = \pr \big( \hat{\Psi}_T > q_{T,\text{Gauss}}(\alpha) \big) = \alpha + o(1).
\end{align*}
This completes the proof of Theorem \ref{theo1}. \qedhere}
\end{enumerate}


\bibliographystyle{ims}
{\small
\setlength{\bibsep}{0.35em}
\bibliography{bibliography}}



\end{document}
