%2multibyte Version: 5.50.0.2960 CodePage: 936

\documentclass[12pt,a4paper]{article}
%%%%%%%%%%%%%%%%%%%%%%%%%%%%%%%%%%%%%%%%%%%%%%%%%%%%%%%%%%%%%%%%%%%%%%%%%%%%%%%%%%%%%%%%%%%%%%%%%%%%%%%%%%%%%%%%%%%%%%%%%%%%%%%%%%%%%%%%%%%%%%%%%%%%%%%%%%%%%%%%%%%%%%%%%%%%%%%%%%%%%%%%%%%%%%%%%%%%%%%%%%%%%%%%%%%%%%%%%%%%%%%%%%%%%%%%%%%%%%%%%%%%%%%%%%%%
\usepackage{amssymb}
\usepackage{amsmath}
\usepackage{amsfonts}
%\usepackage{sw20elba}

\setcounter{MaxMatrixCols}{10}
%TCIDATA{OutputFilter=LATEX.DLL}
%TCIDATA{Version=5.50.0.2960}
%TCIDATA{Codepage=936}
%TCIDATA{<META NAME="SaveForMode" CONTENT="1">}
%TCIDATA{BibliographyScheme=Manual}
%TCIDATA{Created=Thursday, February 24, 2005 10:49:32}
%TCIDATA{LastRevised=Wednesday, August 05, 2020 10:05:27}
%TCIDATA{<META NAME="GraphicsSave" CONTENT="32">}
%TCIDATA{<META NAME="DocumentShell" CONTENT="Articles\SW\Elbert Walker's Article">}
%TCIDATA{Language=American English}
%TCIDATA{CSTFile=LaTeX article (bright).cst}

%\input tcilatex
\renewcommand{\baselinestretch}{1.2}
\setlength{\textwidth}{18cm}
\setlength{\oddsidemargin}{-10mm}
\setlength{\evensidemargin}{-20mm}

\begin{document}


\begin{quotation}
As we know, There are known knowns. There are things we know we know. We
also know There are known unknowns. That is to say, We know there are some
things We do not know. But there are also unknown unknowns, The ones we
don't know We don't know. Donald Rumsfeld, U.S. Secretary of Defence
\end{quotation}

Suppose that%
\begin{equation*}
y_{it}=m_{i}(t/T)+u_{it},
\end{equation*}%
where $i=1,\ldots ,n$ and $t=1,\ldots ,T;$ both $n$ and $T$ are large. We
suppose that there exists $J$ classes of functions%
\begin{equation*}
\mathcal{G}_{j}=\left\{ f:f(u)=cg_{j}((u-a)/b),\text{ }a,b,c\in \mathbb{R}%
_{+},\text{ }g_{j}\text{ a density}\right\} .
\end{equation*}%
We suppose that for any $i,$ $m_{i}(\cdot )\in \mathcal{G}_{j}$ for some $j,$
that is, for some $j$ there exists $a_{i},b_{i},c_{i}$ with%
\begin{equation*}
m_{i}(u)=c_{i}g_{j}((u-a_{i})/b_{i})/b_{i}
\end{equation*}%
for all $u\in \lbrack 0,1].$ 

Estimation. Suppose that there is only one group with unknown $g(.).$ Given
unrestricted estimates $\widehat{m}_{i}(\cdot ),$ we can estimate $c_{i}$ by 
$\int_{0}^{1}\widehat{m}_{i}(u)du$ and work with the ratio $\widehat{m}%
_{i}^{\ast }(u)=\widehat{m}_{i}(u)/\int_{0}^{1}\widehat{m}_{i}(u)du.$ We may
make different assumptions here about $a,b.$ For example, $a$ is the mean
and $b$ is the standard deviation of the density $g.$ In that case we can
estimate $a_{i}$ by $\int_{0}^{1}u\widehat{m}_{i}^{\ast }(u)du$ and $b_{i}$
by the square root of $\int_{0}^{1}u^{2}\widehat{m}_{i}^{\ast
}(u)du-(\int_{0}^{1}u\widehat{m}_{i}^{\ast }(u)du)^{2}.$ Alternatively,
median and interquartile range work. Then one can estimate $g(u)$ by 
\begin{equation*}
\frac{1}{n}\sum_{i=1}^{n}\widehat{b}_{i}\widehat{m}_{i}^{\ast }\left( u%
\widehat{b}_{i}+\frac{\widehat{a}_{i}}{\widehat{b}_{i}}\right) .
\end{equation*}

Now suppose that there are multiple $g^{\prime }s.$ The recovery of the
constants $a,b,c$ only uses the individual regression function. But now the
uncertainty is around which $i$ to average over. This can be addressed by
the clustering algorithms. 

The parameter estimates $\widehat{a}_{i},\widehat{b}_{i},\widehat{c}_{i}$
are $\sqrt{T}$ consistent, whereas the estimates of $g_{j}$ will be $\sqrt{%
nTh}$ consistent. 

\end{document}
