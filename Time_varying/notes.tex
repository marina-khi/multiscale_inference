\documentclass[a4paper,12pt]{article}
\usepackage{amsmath}
\usepackage{amssymb,amsthm,graphicx}
\usepackage{enumitem}
\usepackage{color}
\usepackage{epsfig}
\usepackage{graphics}
\usepackage{pdfpages}
\usepackage{subcaption}
\usepackage[font=small]{caption}
\usepackage[hang,flushmargin]{footmisc} 
\usepackage{float}
\usepackage{booktabs}
\usepackage[mathscr]{euscript}
\usepackage{natbib}
\usepackage{setspace}
\usepackage{mathrsfs}
\usepackage[left=2.7cm,right=2.7cm,bottom=2.7cm,top=2.7cm]{geometry}
\parindent0pt 


% General

\newcommand{\reals}{\mathbb{R}}
\newcommand{\integers}{\mathbb{Z}}
\newcommand{\naturals}{\mathbb{N}}

\newcommand{\pr}{\mathbb{P}}        % probability
\newcommand{\ex}{\mathbb{E}}        % expectation
\newcommand{\var}{\textnormal{Var}} % variance
\newcommand{\cov}{\textnormal{Cov}} % covariance

\newcommand{\law}{\mathcal{L}} % law of X
\newcommand{\normal}{N}        % normal distribution 

\newcommand{\argmax}{\textnormal{argmax}}
\newcommand{\argmin}{\textnormal{argmin}}

\newcommand{\ind}{\mathbbm{1}} % indicator function
\newcommand{\kernel}{K} % kernel function
\newcommand{\wght}{W} % kernel weight
\newcommand{\thres}{\pi} % threshold parameter


% Convergence

\newcommand{\convd}{\stackrel{d}{\longrightarrow}}              % convergence in distribution
\newcommand{\convp}{\stackrel{P}{\longrightarrow}}              % convergence in probability
\newcommand{\convas}{\stackrel{\textrm{a.s.}}{\longrightarrow}} % convergence almost surely
\newcommand{\convw}{\rightsquigarrow}                           % weak convergence


% Theorem-like declarations

\theoremstyle{plain}

\newtheorem{theorem}{Theorem}[section]
\newtheorem{prop}[theorem]{Proposition}
\newtheorem{lemma}[theorem]{Lemma}
\newtheorem{corollary}[theorem]{Corollary}
\newtheorem*{theo}{Theorem}
\newtheorem{propA}{Proposition}[section]
\newtheorem{lemmaA}[propA]{Lemma}
\newtheorem{definition}{Definition}[section]
\newtheorem{remark}{Remark}[section]
\renewcommand{\thelemmaA}{A.\arabic{lemmaA}}
\renewcommand{\thepropA}{A.\arabic{propA}}
\newtheorem*{algo}{Clustering Algorithm}


% Theorem numbering to the left

\makeatletter
\newcommand{\lefteqno}{\let\veqno\@@leqno}
\makeatother


% Heading

\newcommand{\heading}[2]
{  \setcounter{page}{1}
   \begin{center}

   \phantom{Distance to upper boundary}
   \vspace{0.5cm}

   {\LARGE \textbf{#1}}
   \vspace{0.4cm}
 
   {\LARGE \textbf{#2}}
   \end{center}
}


% Authors

\newcommand{\authors}[4]
{  \parindent0pt
   \begin{center}
      \begin{minipage}[c][2cm][c]{5cm}
      \begin{center} 
      {\large #1} 
      \vspace{0.05cm}
      
      #2 
      \end{center}
      \end{minipage}
      \begin{minipage}[c][2cm][c]{5cm}
      \begin{center} 
      {\large #3}
      \vspace{0.05cm}

      #4 
      \end{center}
      \end{minipage}
   \end{center}
}

%\newcommand{\authors}[2]
%{  \parindent0pt
%   \begin{center}
%   {\large #1} 
%   \vspace{0.1cm}
%      
%   #2 
%   \end{center}  
%}


% Version

\newcommand{\version}[1]
{  \begin{center}
   {\large #1}
   \end{center}
   \vspace{3pt}
} 










\begin{document}
\renewcommand{\baselinestretch}{1.2}\normalsize



\section{Nonparametric inference for the time-varying regression model}

Consider the time-varying regression model
\begin{align}\label{model}
Y_t = m_t(X_t) + e_t, \quad t = 1,\ldots, T,
\end{align}
where $Y_t$, $X_t$ and $e_t$ are the responses, the predictors and the errors, respectively, and $m_t(\cdot) = m(\cdot, t/T)$ is a time-varying regression function. Suppose that $X_t$ has compact support $\mathcal{X}\in \reals$. Here $m: \mathcal{X}\times [0, 1] \rightarrow \reals $ is a smooth function, and $t/T,\, t = 1,\ldots, T,$ represents the time rescaled to the unit interval.

Let $K_X(\cdot)$ be a (potentially in the future $d$-dimensional) kernel function, and let $K_T(\cdot)$ be a temporal kernel function. Assumptions on the kernel functions are provided in Assumption \ref{C-kerX} and \ref{C-kerT}.

Consider two bandwidths $h_x$ and $h_t$, a point $(u, s) \in \reals \times [0, 1]$ and the corresponding kernel average
\begin{equation*}
\widehat{\psi}_{h_x, h_t}(u, s) = \sum\limits_{t=1}^T w_{t, h_x, h_t}(u, s)  Y_{t}, 
\end{equation*}
where $w_{t, h_x, h_t}(u, s)$ is a kernel weight. In order to avoid boundary issues, we work with a local linear weighting scheme. We in particular set 
\begin{equation}\label{weights}
w_{t, h_x, h_t}(u, s) = \frac{\Lambda_{t,h_t}(s)K_X\big(\frac{X_t-u}{h_x}\big) }{\sum\nolimits_{t=1}^T \Lambda_{t,h_t}(s) K_X\big(\frac{X_t-u}{h_x}\big)  }, 
\end{equation}
where
\[ \Lambda_{t,h_t}(s) = K_T\Big(\frac{\frac{t}{T}-s}{h_t}\Big) \bigg[ S_{2}(s) - \Big(\frac{\frac{t}{T}-s}{h_t}\Big) S_{1}(s) \bigg], \]
and $S_{\ell}(s) = (Th_t)^{-1} \sum\nolimits_{t=1}^T K_T(\frac{\frac{t}{T}-s}{h_t}) (\frac{\frac{t}{T}-s}{h_t})^\ell$ for $\ell = 0,1,2$.

The kernel average $\widehat{\psi}_{h_x, h_t}(u, s)$ is nothing else than a rescaled local linear estimator of the function $m(\cdot, \cdot)$.

To allow nonstationary and dependent observations, we assume that the covariates $X_t$ have the following properties. 
\begin{enumerate}[label=(C\arabic*),leftmargin=1.05cm]
\item \label{C-reg1} The variables $X_{t}$ allow for the representation $X_{t} = H(t/T; \mathcal{G}_{t})$, where $\mathcal{G}_{t} = (\ldots, \xi_{t-1}, \xi_{t})$, the random variables $\xi_{t}$ are i.i.d.\ and $H: [0, 1] \times \reals^\infty \rightarrow \mathcal{X}$ is a measurable function such that $H(t/T; \mathcal{G}_{t})$ is well-defined for each $t$.%We use the notation $\X_{it} = \boldsymbol{h}_{i}(\mathcal{G}_{it})$ with $\boldsymbol{h}_i = (h_{i1}, \ldots, h_{id})^\top$ and $\mathcal{G}_{it} = (\mathcal{G}_{it,1}, \ldots, \mathcal{G}_{it, d})^\top$. It holds that $\ex [X_{it, j}]=0$ and $\| X_{it, j} \|_{q^\prime} <\infty$ for all $i$ and $j$, where $q^\prime > \max \{ 4, \theta q \}$ with $q$ from \ref{C-err1} and $\theta$ specified in \ref{C-grid} below.
\item \label{C-reg2} The value of $\ex[H^2 (t/T; \mathcal{G}_{0})]$ is bounded away from zero and infinity on $[0, 1]$.
%\item \label{C-reg3} For each $i$ and $j$, it holds that $\sum_{s \ge 0} \delta_{q^\prime}(h_{ij}, s)$ is finite and $\sum_{s \ge t} \delta_{q^\prime}(h_{ij}, s)= O(t^{-\alpha})$ for some $\alpha > 1/2 - 1/{q^\prime}$ with $q^\prime$ from \ref{C-reg1}.
\end{enumerate}
%Assumption \ref{C-reg1} guarantees that the process $\{ X_{t}: 1 \le t \le T \}$ is stationary and causal. Assumption \ref{C-reg3} restricts the serial dependence of the process $\{ \X_{it}: 1 \le t \le T \}$ for each $i$ in terms of the physical dependence measure.



For the error process, we assume that
\begin{align*}
e_t = \sigma_t(X_t)\eta_t = \sigma(X_t, t/T) \eta_t,
\end{align*}
where for now we consider i.i.d. $\eta_t$.


In order for the theory to work, we need the following assumptions:

\begin{enumerate}[label=(C\arabic*),leftmargin=1.05cm]
\setcounter{enumi}{2}
\item \label{C-kerX} The kernel $K_X$ is non-negative, symmetric about zero and integrates to one. Moreover, it has compact support $[-1,1]$ and is Lipschitz continuous, that is, $|K_X(v) - K_X(w)| \le C |v-w|$ for any $v,w \in \reals$ and some constant $C > 0$. 
\item \label{C-kerT} The kernel $K_T$ is non-negative, symmetric about zero and integrates to one. Moreover, it has compact support $[-1,1]$ and is Lipschitz continuous, that is, $|K_T(v) - K_T(w)| \le C |v-w|$ for any $v,w \in \reals$ and some constant $C > 0$. 
\end{enumerate} 



\section{Nonparametric inference for the time-varying regression model}

Consider the time-varying regression model
\begin{align}\label{model}
Y_t = m_t(X_t) + e_t, \quad t = 1,\ldots, T,
\end{align}
where $Y_t$, $X_t$ and $e_t$ are the responses, the predictors and the errors, respectively, and $m_t(\cdot) = m(\cdot, t/T)$ is a time-varying regression function. Suppose that $X_t$ has compact support $\mathcal{X}\in \reals$. Here $m: \mathcal{X}\times [0, 1] \rightarrow \reals $ is a smooth function, and $t/T,\, t = 1,\ldots, T,$ represents the time rescaled to the unit interval.

Let $K_X(\cdot)$ be a (potentially in the future $d$-dimensional) kernel function, and let $K_T(\cdot)$ be a temporal kernel function. Assumptions on the kernel functions are provided in Assumption \ref{C-kerX} and \ref{C-kerT}.

Consider two bandwidths $h_x$ and $h_t$, a point $(u, s) \in \reals \times [0, 1]$ and the corresponding kernel average
\begin{equation*}
\widehat{\psi}_{h_x, h_t}(u, s) = \sum\limits_{t=1}^T w_{t, h_x, h_t}(u, s)  Y_{t}, 
\end{equation*}
where $w_{t, h_x, h_t}(u, s)$ is a kernel weight. In order to avoid boundary issues, we work with a local linear weighting scheme. We in particular set 
\begin{equation}\label{weights}
w_{t, h_x, h_t}(u, s) = \frac{\Lambda_{t,h_t}(s)K_X\big(\frac{X_t-u}{h_x}\big) }{\sum\nolimits_{t=1}^T \Lambda_{t,h_t}(s) K_X\big(\frac{X_t-u}{h_x}\big)  }, 
\end{equation}
where
\[ \Lambda_{t,h_t}(s) = K_T\Big(\frac{\frac{t}{T}-s}{h_t}\Big) \bigg[ S_{2}(s) - \Big(\frac{\frac{t}{T}-s}{h_t}\Big) S_{1}(s) \bigg], \]
and $S_{\ell}(s) = (Th_t)^{-1} \sum\nolimits_{t=1}^T K_T(\frac{\frac{t}{T}-s}{h_t}) (\frac{\frac{t}{T}-s}{h_t})^\ell$ for $\ell = 0,1,2$.

The kernel average $\widehat{\psi}_{h_x, h_t}(u, s)$ is nothing else than a rescaled local linear estimator of the function $m(\cdot, \cdot)$.

To allow nonstationary and dependent observations, we assume that the covariates $X_t$ have the following properties. 
\begin{enumerate}[label=(C\arabic*),leftmargin=1.05cm]
\item \label{C-reg1} The variables $X_{t}$ allow for the representation $X_{t} = H(t/T; \mathcal{G}_{t})$, where $\mathcal{G}_{t} = (\ldots, \xi_{t-1}, \xi_{t})$, the random variables $\xi_{t}$ are i.i.d.\ and $H: [0, 1] \times \reals^\infty \rightarrow \mathcal{X}$ is a measurable function such that $H(t/T; \mathcal{G}_{t})$ is well-defined for each $t$.%We use the notation $\X_{it} = \boldsymbol{h}_{i}(\mathcal{G}_{it})$ with $\boldsymbol{h}_i = (h_{i1}, \ldots, h_{id})^\top$ and $\mathcal{G}_{it} = (\mathcal{G}_{it,1}, \ldots, \mathcal{G}_{it, d})^\top$. It holds that $\ex [X_{it, j}]=0$ and $\| X_{it, j} \|_{q^\prime} <\infty$ for all $i$ and $j$, where $q^\prime > \max \{ 4, \theta q \}$ with $q$ from \ref{C-err1} and $\theta$ specified in \ref{C-grid} below.
\item \label{C-reg2} The value of $\ex[H^2 (t/T; \mathcal{G}_{0})]$ is bounded away from zero and infinity on $[0, 1]$.
%\item \label{C-reg3} For each $i$ and $j$, it holds that $\sum_{s \ge 0} \delta_{q^\prime}(h_{ij}, s)$ is finite and $\sum_{s \ge t} \delta_{q^\prime}(h_{ij}, s)= O(t^{-\alpha})$ for some $\alpha > 1/2 - 1/{q^\prime}$ with $q^\prime$ from \ref{C-reg1}.
\end{enumerate}
%Assumption \ref{C-reg1} guarantees that the process $\{ X_{t}: 1 \le t \le T \}$ is stationary and causal. Assumption \ref{C-reg3} restricts the serial dependence of the process $\{ \X_{it}: 1 \le t \le T \}$ for each $i$ in terms of the physical dependence measure.



For the error process, we assume that
\begin{align*}
e_t = \sigma_t(X_t)\eta_t = \sigma(X_t, t/T) \eta_t,
\end{align*}
where for now we consider i.i.d. $\eta_t$.


In order for the theory to work, we need the following assumptions:

\begin{enumerate}[label=(C\arabic*),leftmargin=1.05cm]
\setcounter{enumi}{2}
\item \label{C-kerX} The kernel $K_X$ is non-negative, symmetric about zero and integrates to one. Moreover, it has compact support $[-1,1]$ and is Lipschitz continuous, that is, $|K_X(v) - K_X(w)| \le C |v-w|$ for any $v,w \in \reals$ and some constant $C > 0$. 
\item \label{C-kerT} The kernel $K_T$ is non-negative, symmetric about zero and integrates to one. Moreover, it has compact support $[-1,1]$ and is Lipschitz continuous, that is, $|K_T(v) - K_T(w)| \le C |v-w|$ for any $v,w \in \reals$ and some constant $C > 0$. 
\end{enumerate} 




\end{document}
