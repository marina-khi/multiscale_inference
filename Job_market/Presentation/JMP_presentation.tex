\documentclass[10pt]{beamer}

\usetheme[progressbar=frametitle]{metropolis}

\usepackage{appendixnumberbeamer}

\usepackage{booktabs}
\usepackage{blkarray}
%\usepackage{ccicons}
\usepackage{graphicx}
\usepackage{color}

\definecolor{UniBlue}{RGB}{7,82,154}
\definecolor{UniYellow}{RGB}{234,185,12}

%\setbeamercolor{title}{fg=UniBlue, bg = UniYellow}
%\setbeamercolor{frametitle}{fg=UniBlue, bg= UniYellow}
%\setbeamercolor{structure}{fg=UniBlue, bg= UniYellow}
%\setbeamercolor{progress bar}{fg=UniBlue, bg= UniYellow}
\usepackage{xspace}

\title{Simultaneous statistical inference for epidemic trends: the case of COVID-19}
\date{01/10/2020}
\author{Marina Khismatullina \and Michael Vogt}
\setbeamertemplate{frame footer}{Simultaneous statistical inference for epidemic trends}
\metroset{block=fill}
% \titlegraphic{\hfill\includegraphics[height=1.5cm]{logo.pdf}}

\newcommand{\Prob}{\mathrm{P}}
\newcommand{\E}{\mathbb{E}}
\newcommand{\Var}{\mathrm{Var}}
\newcommand{\Cov}{\mathrm{Cov}}
\newcommand{\Corr}{\mathrm{Corr}}
\newcommand{\sgn}{\text{sgn}}
\newtheorem{prop}{Proposition}
\newcommand{\ind}{\boldsymbol{1}\Big( \frac{t}{T} \in \mathcal{I}_k \Big)} % indicator function

\begin{document}

\maketitle

\begin{frame}{Table of contents}
  \setbeamertemplate{section in toc}[sections numbered]
  \tableofcontents[hideallsubsections]
\end{frame}

\section{Introduction}


\begin{frame}{Motivation}
	\textbf{Research question:}
	
	How do outbreak patterns of COVID-19 compare across countries?
	\begin{figure}
    		\centering
    		\includegraphics[height=0.45\textheight]{plots/Germany_and_Italy.pdf}
    		%\caption{Observed new cases per day in Germany and Italy}
    		\label{fig:DEUvsITA}
  	\end{figure}\pause
	\begin{block}{Aim of the paper}
	To develop new inference methods that allow to \textit{identify} and \textit{locate} differences between time trends.
	\end{block}
\end{frame}

\section{Model}
\begin{frame}{Model}
We observe $n$ time series $\mathcal{X}_i = \{X_{it}: 1 \le t \le T \}$ of length $T$ for $1 \le i \le n$
\begin{equation*}\label{model}
X_{it} = \lambda_i \Big( \frac{t}{T} \Big) + \sigma\sqrt{\lambda_i \Big( \frac{t}{T} \Big)} \eta_{it} 
\end{equation*}\pause
\vspace{-3mm}
where
\begin{itemize}
\item $\lambda_i$ are unknown trend functions on $[0,1]$;
\item $\sigma$ is an overdispersion parameter;
\item $\eta_{it}$ are error terms that are independent across $i$ and $t$ and have zero mean and unit variance.
\end{itemize}
\end{frame}

\begin{frame}{Literature}
	Curve comparisons
	\begin{itemize}
		\item 
	\end{itemize}\pause
	Studies of COVID-19
	\begin{itemize}
		\item SEIR models
	\end{itemize}
\end{frame}

\section{The multiscale method}
\subsection{Testing problem}
\begin{frame}{Testing}

Let $\mathcal{F} =\{ \mathcal{I}_k \subset [0, 1]: 1 \le k \le K\}$ be a family of intervals on $[0, 1]$, and for a given interval $\mathcal{I}_k$ we want to test whether the functions $\lambda_i$ and $\lambda_j$ are the same on this interval. Formally, the testing problem is then
\begin{align*}
H_0^{(ijk)}:\quad  \lambda_i(w) = \lambda_j(w) \text{ for all } w\in \mathcal{I}_k
\end{align*}\pause
We want to test these hypothesis $H_0^{(ijk)}$ simultaneously for all pairs of countries $i$ and $j$ and all intervals $\mathcal{I}_k$ in the family $\mathcal{F}$.
\end{frame} 


\begin{frame}{Test statistic}
For an interval $\mathcal{I}_k$ and a pair of time series $i$ and $j$ we construct the kernel averages
\begin{equation*}
\hat{s}_{ijk,T} = \frac{1}{\sqrt{T h_k}} \sum\limits_{t=1}^T \ind (X_{it} -X_{jt}), 
\end{equation*}
%\vspace{-3mm}
where $h_k$ is the length of the interval $\mathcal{I}_k$. \pause 

Under certain assumptions, 
\begin{align*}
\nu_{ijk,T}^2 := \Var(\hat{s}_{ijk,T})  & = \frac{\sigma^2}{Th_k} \sum\limits_{t=1}^T \ind \Big\{ \lambda_i\Big(\frac{t}{T}\Big) + \lambda_j\Big(\frac{t}{T}\Big) \Big\}. 
\end{align*}\pause
In order to normalize the variance of the statistic $\hat{s}_{ijk,T}$, we scale it by:
\[ \hat{\nu}_{ijk,T}^2 = \frac{\hat{\sigma}^2}{Th_k} \sum\limits_{t=1}^T \ind \{ X_{it} + X_{jt} \}, \]
with $\hat{\sigma}^2 = n^{-1} \sum_{i = 1}^n \hat{\sigma}_i^2$ and $\hat{\sigma}_i^2 = \frac{\sum_{t=2}^T (X_{it}-X_{it-1})^2}{2 \sum_{t=1}^T X_{it}}$.\hyperlink{frame_sigma}{\beamerbutton{Idea}}
\end{frame}


\begin{frame}[label = frame_teststatistic]{Test statistic}
Test statistic for the hypothesis $H_0^{(ijk)}$ is defined as follows
\begin{equation*}
\widehat{\psi}_{ijk, T} = \frac{\sum\nolimits_{t=1}^T \ind (X_{it} -X_{jt})}{\hat{\sigma} \big\{ \sum\nolimits_{t=1}^T \ind \{ X_{it} + X_{jt} \}\big\}^{1/2}}, 
\end{equation*} 
\end{frame}


\begin{frame}{Test procedure}
Testing problem:
\vspace{-3mm}
\begin{align*}
H_0^{(ijk)}: \quad \lambda_i(w) = \lambda_j(w) \text{ for all } w \in \mathcal{I}_k
\end{align*} \pause
\vspace{-2mm}
Gaussian version of the test statistic:
\begin{align*}
\phi_{ijk,T}(u,h) = \frac{1}{\sqrt{2 T h_k}} \sum\limits_{t=1}^T \ind (Z_{it} - Z_{jt}), 
\end{align*}
where 
\begin{itemize}
	\item $Z_t$ are independent standard normal random variables;
\end{itemize}
$q_{n,T}(\alpha)$ is  $(1 - \alpha)$ quantile of $\Phi_{n,T}$.\pause
\begin{block}{Test procedure}
For a given significance level $\alpha \in (0,1)$, we reject $H_0$ if $\widehat{\Psi}_{n,T} > q_{n,T}(\alpha)$.
\end{block}
\end{frame}


\begin{frame}{Theoretical properties}
\begin{prop}\label{prop-equality-1}
Supose that $\mathcal{E}_i$ are independent across $i$ and satisfy $\mathcal{C}1- \mathcal{C}2$ for each $i$. Under our remaining assumptions and under $H_0: m_1 = m_2 =\ldots = m_n$ it holds that 
\vspace{-3mm}
\begin{align*}
\Prob \big( \widehat{\Psi}_{n, T} \le q_{n,T}(\alpha) \big) = (1 - \alpha) + o(1).
\end{align*}
\end{prop}\pause

\begin{prop}\label{prop-equality-2}
Let the conditions of previous proposition be satisfied. Under local alternatives we have
\vspace{-3mm}
\begin{align*}
\Prob \big( \widehat{\Psi}_{n,T} \le q_{n,T}(\alpha) \big) = o(1).
\end{align*}
\end{prop}
\end{frame}

\begin{frame}{Test procedure}
Gaussian version of the test statistic:
\begin{align*}
\Phi_T = \max_{(u,h) \in \mathcal{G}_T} \Big\{ \Big|\frac{\phi_T(u,h)}{\sigma}\Big| - \lambda(h) \Big\},
\end{align*} 
\vspace{-3mm}
where
\begin{itemize}
\item $\phi_T(u,h) = \sum\nolimits_{t=1}^T w_{t,T}(u,h) \, \sigma Z_t$;
\item $Z_t$ are independent standard normal random variables;
\item $q_T(\alpha)$ is $(1 - \alpha)$ quantile of $\Phi_T$.
\end{itemize}\pause
\begin{block}{Test procedure}
For a given significance level $\alpha \in (0,1)$, we reject $H_0$ if $\widehat{\Psi}_T > q_T(\alpha)$.
\end{block}
\end{frame}

\section{Theoretical properties}
\begin{frame}{Assumptions}
\begin{itemize}
\onslide<1->\item[$\mathcal{C}1$] \label{C-err1} The variables $\varepsilon_t$ are weakly dependent.
\onslide<2->\item[$\mathcal{C}2$] \label{C-err2} It holds that $\| \varepsilon_t \|_q < \infty$ for some $q > 4$.
\onslide<3->\item[$\mathcal{C}3$] \label{C-ker} Standard assumptions on the kernel function $K$.
\onslide<4->\item[$\mathcal{C}4$] Assume that  $\widehat{\sigma}^2 = \sigma^2 + o_p(\rho_T)$ with $\rho_T = o(1/\log T)$.
\onslide<5->\item[$\mathcal{C}5$] \label{C-grid} $|\mathcal{G}_T| = O(T^\theta)$ for some arbitrarily large but fixed constant $\theta > 0$.
\begin{overprint}
\onslide<6>
\vspace{-3mm}
\begin{align*}
\mathcal{G}_T = \big\{ & (u,h): u = t/T \text{ for some } 1 \le t \le T \text{ and } h \in [h_{\min},h_{\max}] \\ & \text{ with } h = t/T \text{ for some } 1 \le t \le T  \big\},
\end{align*}
\onslide<7>
\vspace{-3mm}
\item[$\mathcal{C}6$] \label{C-h} $h_{\min} \gg T^{-(1-\frac{2}{q})} \log T$ and $h_{\max} = o(1)$.

\end{overprint}
\end{itemize}
\end{frame}

\begin{frame}{Theoretical properties}
\begin{prop}\label{prop-shape-1}
Under our assumptions and under $H_0: m^{\prime}= 0$ it holds that 
\vspace{-3mm}
\begin{align*}
\Prob \big( \widehat{\Psi}_T \le q_T(\alpha) \big) = (1 - \alpha) + o(1).
\end{align*}
\end{prop}\pause
\begin{prop}\label{prop-shape-2}
Under our assumptions and under local alternatives, we have 
\vspace{-3mm}
\begin{align*}
\Prob \big( \widehat{\Psi}_T \le q_T(\alpha) \big) = o(1).
\end{align*}
\end{prop}
\end{frame}


\begin{frame}{Strategy of the proof}
\begin{itemize}
\item Replace the statistic $\widehat{\Psi}_T$ under $H_0: m = 0$ by a statistic $\widetilde{\Phi}_T$ with the same distribution and the property that 
\begin{equation*}\label{eq-theo-stat-strategy-step1}
\big| \widetilde{\Phi}_T - \Phi_T \big| = o_p(\delta_T),
\end{equation*}
where $\delta_T = o(1)$. To do so, we make use of strong approximation theory for dependent processes as derived in Berkes et al. (2014)\pause
\vspace{2mm}
\item Using the anti-concentration results for Gaussian random vectors (Chernozhukov et al. 2015), prove that $\Phi_T$ does not concentrate too strongly in small regions of the form $[x-\delta_T,x+\delta_T]$, i.e.
\begin{equation*}\label{eq-theo-stat-strategy-step2}
\sup_{x \in \mathbb{R}} \Prob \big( |\Phi_T - x| \le \delta_T \big) = o(1).
\end{equation*}\pause
\vspace{-2mm}
\item Show that 
\begin{equation*}\label{eq-theo-stat-strategy-claim}
\sup_{x \in \mathbb{R}} \big| \Prob(\widetilde{\Phi}_T \le x) - \Prob(\Phi_T \le x) \big| = o(1). 
\end{equation*}
\end{itemize}
\end{frame}


\begin{frame}{Theoretical properties}
Define
\begin{align*}
\Pi_T^+ = \big\{ I_{u,h} = [u-h,u+h]: (u,h) \in \mathcal{A}_T^+ \text{ and } I_{u, h} \subseteq [0,1] \big\}\\
\onslide<2->{\Pi_T^- = \big\{ I_{u,h} = [u-h,u+h]: (u,h) \in \mathcal{A}_T^- \text{ and } I_{u, h} \subseteq [0,1] \big\}}
\end{align*}
\vspace{-5mm}
with
\begin{align*} 
&\mathcal{A}_T^+ = \Big\{ (u,h) \in \mathcal{G}_T: \frac{\widehat{\psi}_T(u,h)}{\widehat{\sigma}} > q_T(\alpha)  + \lambda(h)  \Big\}\\
&\onslide<2->{\mathcal{A}_T^- = \Big\{ (u,h) \in \mathcal{G}_T: -\frac{\widehat{\psi}_T(u,h)}{\widehat{\sigma}} > q_T(\alpha)  + \lambda(h)  \Big\}}
\end{align*}
\end{frame}

\begin{frame}{Theoretical properties}
\begin{prop}\label{prop-shape-3}
Under our assumptions, for events $E_T^+ = \Big\{ \forall I_{u,h} \in \Pi_T^+: m^{\prime}(v) > 0 \text{ for some } v \in I_{u,h}\Big\}$
\only<1>{ it holds that}
\onslide<2->{and $E_T^{-} = \Big\{ \forall I_{u,h} \in \Pi_T^-: m^{\prime}(v) < 0 \text{ for some } v \in I_{u,h}\Big\}$ it holds that } 
\begin{align*}
\Prob \big( E_T^+ \big) \ge (1-\alpha) + o(1)\\
\onslide<2->{\Prob \big( E_T^- \big) \ge (1-\alpha) + o(1)}
\end{align*} 
\end{prop}
\end{frame}

\begin{frame}{Graphical representation}
\begin{block}{Minimal intervals}
An interval $I_{u, h} \in \Pi^+_T$ is called \textbf{minimal} if there is no other interval $I_{u^\prime, h^\prime} \in \Pi^+_T$ with $I_{u^\prime, h^\prime} \subset I_{u, h}$.
\end{block}\pause
Define
\begin{align*}
\Pi^{min, +}_T &= \text{ set of minimal intervals from }\Pi^+_T,\\
E_T^{min, +} &= \Big\{ \forall I_{u,h} \in \Pi_T^{min, +}: m'(v) > 0 \text{ for some } v \in I_{u,h}\Big\}
\end{align*}\pause
Since $E_T^{min, +} = E_T^{+}$, we have
\begin{align*}
\Prob \big( E_T^{min, +} \big) \ge (1-\alpha) + o(1).
\end{align*}
\end{frame}



\section{Conclusion}
\begin{frame}{Conclusion}
We developed multiscale methods to test qualitative hypotheses about nonparametric time trends:
\begin{itemize}
\item whether the trend is present at all;
\item whether the trend function is constant;
\item in which time regions there is an upward/downward movement in the trend.
\end{itemize}
We derived asymptotic theory for the proposed tests.

As an application of our method, we analyzed the behavior of the yearly mean temperature in Central England from 1659 to 2017.
\end{frame}


\begin{frame}[standout]
  Thank you!
\end{frame}




\appendix

%\begin{frame}{Simulation results of the multiscale test for constant trend function}
%\scriptsize{\begin{table}[t]
%\begin{center}
%\caption{Size of the multiscale test.}
%\label{tab:size_shape}
%% latex table generated in R 3.4.3 by xtable 1.8-2 package
% 
\begin{tabular}{cccc}
  \hline
  & \multicolumn{3}{c}{nominal size $\alpha$} \\
 $T$ & 0.01 & 0.05 & 0.1 \\
 \hline
250 & 0.006 & 0.033 & 0.069 \\ 
  350 & 0.011 & 0.045 & 0.083 \\ 
  500 & 0.005 & 0.047 & 0.092 \\ 
  1000 & 0.012 & 0.058 & 0.092 \\ 
   \hline
\end{tabular}

%\end{center}
%\end{table}}
%\begin{center}
%\normalsize{Power of the multiscale test for different slope parameters $\beta$.}
%\end{center}
%\vspace{-5mm}
%\scriptsize{\begin{columns}
%\begin{column}[b]{0.33\textwidth}
%\begin{table}[t]
%\centering
%\caption{$\beta = 1.25$}\label{tab:power_050_ll_shape}
%% latex table generated in R 3.4.3 by xtable 1.8-2 package
% 

\renewcommand{\arraystretch}{1.2}
\begin{tabular}{cccc}
\hline
    & \multicolumn{3}{c}{nominal size $\alpha$} \\
$T$ & 0.01 & 0.05 & 0.1 \\ 
\hline
250 & 0.153 & 0.265 & 0.399 \\ 
350 & 0.192 & 0.333 & 0.517 \\ 
500 & 0.355 & 0.571 & 0.694 \\ 
\hline
\end{tabular}

%\end{table}
%\end{column}
%\begin{column}[b]{0.33\textwidth}
%\begin{table}[t]
%\centering
%\caption{$\beta = 1.875$}\label{tab:power_075_ll_shape}
%% latex table generated in R 3.4.3 by xtable 1.8-2 package
% 
\begin{tabular}{cccc}
  \hline
  & \multicolumn{3}{c}{nominal size $\alpha$} \\
 $T$ & 0.01 & 0.05 & 0.1 \\
 \hline
250 & 0.365 & 0.582 & 0.709 \\ 
  350 & 0.644 & 0.779 & 0.845 \\ 
  500 & 0.784 & 0.942 & 0.976 \\ 
  1000 & 0.997 & 1.000 & 1.000 \\ 
   \hline
\end{tabular}

%\end{table}
%\end{column}
%\begin{column}[b]{0.33\textwidth}
%\begin{table}[t]
%\centering
%\caption{$\beta = 2.5$}\label{tab:power_100_ll_shape}
%% latex table generated in R 3.4.3 by xtable 1.8-2 package
% 
\begin{tabular}{cccc}
  \hline
  & \multicolumn{3}{c}{nominal size $\alpha$} \\
 $T$ & 0.01 & 0.05 & 0.1 \\
 \hline
250 & 0.740 & 0.890 & 0.910 \\ 
  350 & 0.890 & 0.970 & 0.990 \\ 
  500 & 0.990 & 0.990 & 1.000 \\ 
  1000 & 1.000 & 1.000 & 1.000 \\ 
   \hline
\end{tabular}

%\end{table}
%\end{column}
%\end{columns}}
%\end{frame}

\begin{frame}
\thispagestyle{empty}
\begin{center}
\Large{\textbf{Long-run error variance estimator}}
\end{center}
\end{frame}

\begin{frame}{Setting}
Estimate the long-run error variance $\sigma^2 = \sum\nolimits_{\ell=-\infty}^{\infty} \Cov(\varepsilon_0,\varepsilon_{\ell})$ of the error terms $\{\varepsilon_t\}$ in the model 
\begin{equation*}
Y_t = m \Big( \frac{t}{T} \Big) + \varepsilon_t, 
\end{equation*}
where $\{\varepsilon_t\}$ is a stationary and causal AR($p$) process of the form 
\begin{equation*}
\varepsilon_t = \sum_{j=1}^p a_j \varepsilon_{t-j} + \eta_t. 
\end{equation*} \pause
\begin{itemize}
	\item $\boldsymbol{a} = (a_1,\ldots,a_p)$ is a vector of the unknown parameters;\pause
	\item $\eta_t$ are i.i.d.\ innovations with $\E[\eta_t] = 0$ and $\E[\eta_t^2] = \nu^2$;\pause
	\item $p$ is known.
\end{itemize}
\end{frame}

\begin{frame}{Estimator, first stage}
Yule-Walker equations yield 
\begin{equation*}\label{YU-eq}
\boldsymbol{\Gamma}_q \boldsymbol{a} = \boldsymbol{\gamma}_q + \nu^2 \boldsymbol{c}_q,  
\end{equation*} 
where
\vspace{-3mm}
\begin{itemize}
	\item $\boldsymbol{c}_q = (c_{q-1},\dots,c_{q-p})^\top$ are the coefficients from the MA($\infty$) expansion of $\{ \varepsilon_t \}$;\pause
	\item $\boldsymbol{\gamma}_q = (\gamma_q(1),\dots,\gamma_q(p))^\top$ with $\gamma_q(\ell) = \Cov(\Delta_q \varepsilon_t,$ $\Delta_q \varepsilon_{t-\ell})$;\pause
	\item and $\boldsymbol{\Gamma}_q$ is the $p \times p$ covariance matrix $\boldsymbol{\Gamma}_q = (\gamma_q(i-j): 1 \le i,j \le p)$.
\end{itemize}\pause
\begin{block}{Note}
\vspace{-3mm}
\begin{center}
$\boldsymbol{\Gamma}_q \boldsymbol{a} \approx \boldsymbol{\gamma}_q$ for large values of $q$.
\end{center}\vspace{-3mm}\end{block}\pause
\vspace{-3mm}

We construct the first-stage estimator by
\begin{equation*}
\widetilde{\boldsymbol{a}}_q = \widehat{\boldsymbol{\Gamma}}_q^{-1} \widehat{\boldsymbol{\gamma}}_q, 
\end{equation*}
where $\widehat{\boldsymbol{\Gamma}}_q$ and $\widehat{\boldsymbol{\gamma}}_q$ are constructed from the sample autocovariances $\widehat{\gamma}_q(\ell) = (T-q)^{-1} \sum_{t=q+\ell+1}^T \Delta_q Y_{t,T} \Delta_q Y_{t-\ell,T}$. 
\end{frame}

\begin{frame}{Estimator, second stage}
\begin{block}{Problem}
If the trend $m$ is pronounced, the estimator $\widetilde{\boldsymbol{a}}_q$ will have a strong bias.
\end{block}\pause
\vspace{-2mm}
Solution:
\begin{itemize}
	\item \vspace{-2mm} Compute estimators $\widetilde{c}_k$ of $c_k$ based on $\widetilde{\boldsymbol{a}}_q$.\pause
	\item Estimate the innovation variance $\nu^2$ by $\widetilde{\nu}^2 = (2T)^{-1} \sum_{t=p+1}^T \widetilde{r}_{t,T}^2$, where $\widetilde{r}_{t,T} = \Delta_1 Y_{t,T} - \sum_{j=1}^p \widetilde{a}_j \Delta_1 Y_{t-j,T}$.\pause
	\item Estimate $\boldsymbol{a}$ by 
\begin{equation*}\label{est-AR-SS} 
\widehat{\boldsymbol{a}}_r = \widehat{\boldsymbol{\Gamma}}_r^{-1} (\widehat{\boldsymbol{\gamma}}_r + \widetilde{\nu}^2 \widetilde{\boldsymbol{c}}_r).
\end{equation*}\pause
	\item \vspace{-2mm} Average the estimators $\widehat{\boldsymbol{a}}_r$: $\widehat{\boldsymbol{a}} = \frac{1}{\overline{r}} \sum\limits_{r=1}^{\overline{r}} \widehat{\boldsymbol{a}}_r$.\pause
	\item Estimate the long-run variance $\sigma^2$ by 
\begin{equation*} \label{est-lrv}
\widehat{\sigma}^2 = \frac{\widehat{\nu}^2}{(1 - \sum_{j=1}^p \widehat{a}_j)^2}. 
\end{equation*}
\end{itemize}
\end{frame}



\begin{frame}{Motivation for the estimator}
If $\{\varepsilon_t\}$ is an AR($p$) process, then the time series $\{ \Delta_q \varepsilon_t \}$ of the differences $\Delta_q \varepsilon_t = \varepsilon_t - \varepsilon_{t-q}$ is an ARMA($p,q$) process of the form 
\begin{equation*}
\Delta_q \varepsilon_t - \sum_{j=1}^p a_j \Delta_q \varepsilon_{t-j} = \eta_t - \eta_{t-q}. 
\end{equation*}\pause

Then $\Delta_q Y_{t, T} = Y_{t, T} - Y_{t-q, T}$ is approximately an ARMA($p,q$) process.
\end{frame}

\begin{frame}{Theoretical properties of the estimator}
Performance:
\begin{itemize}
\item Our estimator $\widehat{\boldsymbol{a}}$ produces accurate estimation results even when the AR polynomial $A(z) = 1 - \sum_{j=1}^p a_j z^j$ has a root close to the unit circle.\pause
\item Our pilot estimator $\widetilde{\boldsymbol{a}}_q$ tends to have a substantial bias when the trend $m$ is pronounced. Our estimator $\widehat{\boldsymbol{a}}$ reduces this bias considerably.\pause
\end{itemize}
\begin{prop}{}
Our estimators $\widetilde{\boldsymbol{a}}_q$, $\widehat{\boldsymbol{a}}$ and $\widehat{\sigma}^2$ are $\sqrt{T}$-consistent. 
\end{prop}
\end{frame}



\begin{frame}{Clustering, group structure}
\begin{itemize}
\item The null hypothesis $H_0: m_1 = m_2 = \ldots = m_n$ is violated.\pause
\item There exist sets or groups of time series $G_1,\ldots,G_N$ with $N \le n$ and $\{1,\ldots,n\} = \mathbin{\dot{\bigcup}}_{\ell=1}^{N} G_\ell$ such that for each $1 \le \ell \le N$ we have $m_i = g_\ell \quad \text{for all } i \in G_\ell$, where $g_\ell$ are group-specific trend functions.\pause
\item For any $\ell \ne \ell^\prime$, the trends $g_{\ell,T}$ and $g_{\ell^\prime,T}$ differ in the following sense: There exists $(u,h) \in \mathcal{G}_T$ with $[u-h,u+h] \subseteq [0,1]$ such that $g_{\ell,T}(w) - g_{\ell^\prime,T}(w) \ge c_T \sqrt{\log T/(Th)}$ for all $w \in [u-h,u+h]$ or $g_{\ell^\prime,T}(w) - g_{\ell,T}(w) \ge c_T \sqrt{\log T/(Th)}$ for all $w \in [u-h,u+h]$, where $0 < c_T \rightarrow \infty$.
\end{itemize}
\end{frame}

\begin{frame}{Clustering, algorithm}
Dissimilarity measure between two sets of time series $S$ and $S^{\prime}$:
\begin{equation*}\label{dissimilarity}
\widehat{\Delta}(S,S^\prime) = \max_{\substack{i \in S, \\ j \in S^\prime}} \widehat{\Psi}_{ij,T}. 
\end{equation*}
\begin{center}
\textbf{Clustering algorithm}
\end{center}

\textit{Step $0$ (Initialization):} Let $\widehat{G}_i^{[0]} = \{ i \}$ denote the $i$-th singleton cluster for $1 \le i \le n$ and define $\{\widehat{G}_1^{[0]},\ldots,\widehat{G}_n^{[0]} \}$ to be the initial partition of time series into clusters. 

\noindent \textit{Step $r$ (Iteration):} Let $\widehat{G}_1^{[r-1]},\ldots,\widehat{G}_{n-(r-1)}^{[r-1]}$ be the $n-(r-1)$ clusters from the previous step. Determine the pair of clusters $\widehat{G}_{\ell}^{[r-1]}$ and $\widehat{G}_{{\ell}^\prime}^{[r-1]}$ for which 
\begin{align*}
\widehat{\Delta}(\widehat{G}_{\ell}^{[r-1]},\widehat{G}_{{\ell}^\prime}^{[r-1]}) = \min_{1 \le k < k^\prime \le n-(r-1)} \widehat{\Delta}(\widehat{G}_{k}^{[r-1]},\widehat{G}_{k^\prime}^{[r-1]})
\end{align*}
and merge them into a new cluster. 
\end{frame}

\begin{frame}{Clustering, theoretical properties}
The estimator of the number of groups is
\begin{align*}
\widehat{N} = \min \Big\{ r = 1,2,\ldots \Big| \max_{1 \le \ell \le r} \widehat{\Delta} \big( \widehat{G}_\ell^{[n-r]} \big) \le q_{n,T}(\alpha) \Big\}.
\end{align*}\pause
\begin{prop}\label{prop-clustering-1}
Let the conditions of previous propositions be satisfied. Then 
\begin{align*}
\Prob \Big( \big\{ \widehat{G}_1,\ldots,\widehat{G}_{\widehat{N}} \big\} = \{ G_1,\ldots,G_N \} \Big) \ge (1-\alpha) + o(1)
\end{align*}
and 
\begin{align*}
\Prob \big( \widehat{N} = N \big) \ge (1-\alpha) + o(1).
\end{align*}
\end{prop}
\end{frame}

\begin{frame}[label = frame_lambda]{Idea behind the additive correction}
\onslide<1->{Consider the uncorrected statistic
\begin{align*}
\widehat{\Psi}_{T, \text{uncorrected}} = \max_{(u,h) \in \mathcal{G}_T} \Big|\frac{\widehat{\psi}_T(u,h)}{\widehat{\sigma}}\Big|
\end{align*}
under the null hypothesis $H_0: m = 0$ and under simplifying assumptions:
\begin{itemize}
\item the errors $\varepsilon_i$ are i.i.d. normally distributed;
\item $\widehat{\sigma} = \sigma$;
\item $\mathcal{G}_T = \{(u_k, h_l) | u_k = (2k - 1)h_l \text{ for } 1\le k \le 1/2h_l, 1 \le l \le L\}$.\pause
\end{itemize}}
\onslide<2->{\begin{align*}
\widehat{\Psi}_{T, \text{uncorrected}} = \max_{1 \le l \le L} \max_{1\le k \le 1/2h_l} \Big|{\color<3>{mLightBrown}\frac{\widehat{\psi}_T(u_k,h_l)}{\sigma}}\Big|
\end{align*}}
\onslide<4->{$\Rightarrow \quad \max_k \frac{\widehat{\psi}_T(u_k,h_l)}{\sigma} ={\color<5->{mLightBrown}\sqrt{2\log(1/2h_l)}} + o_P(1) \to \infty$ as $h \to 0$ and the stochastic behavior of $\widehat{\Psi}_{T, \text{uncorrected}}$ is dominated by $\frac{\widehat{\psi}_T(u_k,h_l)}{\sigma}$ for small bandwidths $h_l$. \hyperlink{frame_teststatistic}{\beamerbutton{Go back}}}
\end{frame}

\begin{frame}[label = frame_sigma]{Idea behind $\hat{\sigma}$}
\hyperlink{frame_teststatistic}{\beamerbutton{Go back}}
\end{frame}

\end{document}
